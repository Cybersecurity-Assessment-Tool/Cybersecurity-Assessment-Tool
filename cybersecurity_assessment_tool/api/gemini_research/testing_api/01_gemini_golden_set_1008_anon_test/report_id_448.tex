```latex
\documentclass[12pt]{article}

% Preamble: Required Packages
\usepackage[margin=1in]{geometry}
\usepackage{pifont} % For checkmarks and crosses
\usepackage{booktabs} % For professional tables
\usepackage{hyperref} % For hyperlinks
\usepackage{url} % For URL formatting
\usepackage{seqsplit} % For splitting long strings
\usepackage{graphicx} % For logo (placeholder)
\usepackage{xcolor} % For colors

% Document Information
\title{Cybersecurity Posture Assessment Report}
\author{Cybersecurity Analyst Group}
\date{\today}

% Define colors for severity
\definecolor{critical}{HTML}{990000}
\definecolor{high}{HTML}{D14302}
\definecolor{medium}{HTML}{E5A000}
\definecolor{low}{HTML}{339900}

% Hyperref Setup
\hypersetup{
    colorlinks=true,
    linkcolor=blue,
    filecolor=magenta,      
    urlcolor=cyan,
    pdftitle={Cybersecurity Posture Assessment Report},
    pdfpagemode=FullScreen,
}

\begin{document}

\begin{titlepage}
    \centering
    \vspace*{1cm}
    \Huge\textbf{Cybersecurity Posture Assessment Report}
    \vspace{1.5cm}
    \large
    \textbf{Prepared for:} \\
    \vspace{0.2cm}
    \textbf{[Organization Name]}
    \vspace{2cm}
    
    \textbf{Date of Report:} \\
    \today
    
    \vfill
    
    \large
    \textbf{CONFIDENTIAL} \\
    \vspace{0.5cm}
    \small
    This document contains sensitive information and is intended solely for the use of the designated recipient. Unauthorized review, use, disclosure, or distribution is prohibited.
\end{titlepage}

\tableofcontents
\newpage

% --- Section 1: Executive Overview ---
\section{Executive Overview}
This report details the findings of a cybersecurity posture assessment conducted for \textbf{[Organization Name]}. The assessment combined a review of organizational security controls via a questionnaire with an external network vulnerability scan.

The key finding of this assessment is the presence of several \textbf{critical administrative and policy-based control gaps}. While the external network scan of the target IP address revealed no open ports—a positive indicator of a hardened network perimeter—the organizational policies and practices present significant risks. 

Specifically, the lack of Multi-Factor Authentication (MFA) for email and computer access, the absence of an employee acceptable use policy, and incomplete security awareness training create substantial vulnerabilities to common cyberattacks such as phishing, business email compromise, and unauthorized access.

Immediate remediation of these identified gaps is strongly recommended to reduce the organization's risk exposure and improve its overall security posture.

% --- Section 2: Organizational Information ---
\section{Organizational Information}
The following information was used as the basis for this assessment. As the provided data was anonymized, placeholders have been used.

\begin{itemize}
    \item \textbf{Organization Name:} \textbf{[Organization Name]}
    \item \textbf{Primary Email Domain:} \texttt{[Domain]}
    \item \textbf{Assessed External IP:} \texttt{[Client IP]}
\end{itemize}

% --- Section 3: Security Control Review ---
\section{Security Control Review (Questionnaire)}
The following table summarizes the organization's responses to the security controls questionnaire. A \textcolor{red}{\ding{55}} indicates a negative response, which often corresponds to a security gap that increases risk.

\begin{table}[h!]
\centering
\caption{Security Controls Questionnaire Analysis}
\label{tab:controls}
\begin{tabular}{p{0.6\linewidth} c p{0.25\linewidth}}
\toprule
\textbf{Control Question} & \textbf{Response} & \textbf{Analyst Notes} \\
\midrule
Do you require MFA to access email? & \textcolor{red}{\ding{55}} & \textbf{Critical Gap.} Lack of MFA on email is a primary vector for account compromise. \\
\addlinespace
Do you require MFA to log into computers? & \textcolor{red}{\ding{55}} & \textbf{Critical Gap.} Increases risk of unauthorized access from stolen credentials. \\
\addlinespace
Do you require MFA to access sensitive data systems? & \textcolor{green}{\ding{51}} & Good Practice. Protects the most critical assets. \\
\addlinespace
Does your organization have an employee acceptable use policy? & \textcolor{red}{\ding{55}} & \textbf{High Risk.} Lack of a formal policy creates ambiguity and legal/operational risk. \\
\addlinespace
Does your organization do security awareness training for new employees? & \textcolor{red}{\ding{55}} & \textbf{High Risk.} New hires are a common target and remain untrained on security policies. \\
\addlinespace
Does your organization do security awareness training for all employees at least once per year? & \textcolor{green}{\ding{51}} & Good Practice. Reinforces security concepts annually. \\
\bottomrule
\end{tabular}
\end{table}

% --- Section 4: Technical Scan Results ---
\section{Technical Scan Results}
An external network scan was performed to identify accessible services and potential vulnerabilities.

\begin{itemize}
    \item \textbf{Target IP Address:} \texttt{[Target IP]}
    \item \textbf{Scan Date:} Not provided in scan data.
    \item \textbf{Scanner:} Nmap
\end{itemize}

\subsection{Summary of Findings}
The scan confirmed that the target host was online and responsive. However, \textbf{no open TCP/IP ports were discovered}. All 1000 of the most common ports scanned were reported as `closed`.

\paragraph{Analyst Interpretation:}
This is a positive finding. It indicates a strong network perimeter configuration, likely due to a well-configured firewall that denies all unsolicited inbound traffic. This significantly reduces the external attack surface of the scanned asset. No further technical vulnerabilities were identified from this scan.

% --- Section 5: Risk Assessment ---
\section{Risk Assessment}
This section synthesizes findings from all data sources into a prioritized list of identified risks. As no pre-existing or technical vulnerabilities were found, the risks below are derived entirely from the security control gaps identified in Section 3.

\begin{table}[h!]
\centering
\caption{Summary of Identified Risks}
\label{tab:risks}
\begin{tabular}{p{0.25\linewidth} p{0.15\linewidth} p{0.5\linewidth}}
\toprule
\textbf{Risk Name} & \textbf{Severity} & \textbf{Overview} \\
\midrule
\addlinespace
Inadequate Access Control & \textcolor{critical}{\textbf{Critical}} & The absence of MFA for email and computer access exposes the organization to a high likelihood of account takeover, data breach, and ransomware via stolen or weak credentials. \\
\addlinespace
Lack of Formal Security Policies & \textcolor{high}{\textbf{High}} & Without an Acceptable Use Policy, employees lack clear guidelines on the secure use of company assets. This increases the risk of insider threat (both malicious and accidental) and complicates incident response. \\
\addlinespace
Incomplete Security Training Program & \textcolor{high}{\textbf{High}} & Failing to train new employees upon hiring leaves a critical window of vulnerability. New staff are often targeted by social engineering and may be unaware of organizational security procedures. \\
\addlinespace
\bottomrule
\end{tabular}
\end{table}

% --- Section 6: Recommendations ---
\section{Recommendations}
The following actions are recommended to mitigate the identified risks and strengthen the organization's security posture.

\subsection{Immediate Actions (0-30 Days)}
\begin{enumerate}
    \item \textbf{Implement Multi-Factor Authentication (MFA):}
    \begin{itemize}
        \item \textbf{Priority 1:} Enforce MFA for all user access to email systems (e.g., Microsoft 365, Google Workspace). This is the single most effective control to prevent business email compromise.
        \item \textbf{Priority 2:} Deploy MFA for all remote access (VPN) and local computer logins, especially for privileged administrator accounts.
    \end{itemize}
    
    \item \textbf{Establish New Hire Security Training:}
    \begin{itemize}
        \item Integrate a mandatory security awareness training module into the new employee onboarding process. This training should occur before a new hire is granted access to sensitive systems.
    \end{itemize}
\end{enumerate}

\subsection{Strategic Actions (30-90 Days)}
\begin{enumerate}
    \setcounter{enumi}{2} % Continue numbering
    \item \textbf{Develop and Implement an Acceptable Use Policy (AUP):}
    \begin{itemize}
        \item Draft a formal AUP that clearly defines the rules for using company networks, computers, email, and internet access.
        \item Require all current and new employees to read and formally acknowledge the policy.
        \item This policy provides a foundation for enforcing security standards and taking disciplinary action if necessary.
    \end{itemize}
\end{enumerate}

\end{document}
```