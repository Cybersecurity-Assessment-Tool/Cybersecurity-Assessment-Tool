```latex
\documentclass[12pt, a4paper]{article}

% Preamble: Required Packages
\usepackage[margin=1in]{geometry}
\usepackage{pifont} % For \ding
\usepackage{booktabs} % For professional tables
\usepackage{hyperref} % For hyperlinks
\usepackage{url} % For URL formatting
\usepackage{seqsplit} % For splitting long strings in tt font
\usepackage{xcolor} % For colors

% Document Information
\title{Cybersecurity Posture Assessment Report}
\author{Cybersecurity Analyst}
\date{\today}

% Hyperref Setup
\hypersetup{
    colorlinks=true,
    linkcolor=blue,
    filecolor=magenta,      
    urlcolor=cyan,
    pdftitle={Cybersecurity Posture Assessment Report},
    pdfpagemode=FullScreen,
}

\begin{document}

\maketitle

\begin{abstract}
This report provides a comprehensive analysis of the cybersecurity posture for \textbf{[Organization Name]}. The assessment is based on the correlation of an external network scan, a security controls questionnaire, and a review of pre-existing risks. The analysis identified several critical vulnerabilities, including an exposed and outdated FTP server with a known backdoor vulnerability and a systemic lack of Multi-Factor Authentication (MFA) across all critical services. These findings indicate a high-risk profile requiring immediate remediation.
\end{abstract}

\tableofcontents
\newpage

% ===================================================================
\section{Overview and Executive Summary}
% ===================================================================

This assessment was conducted to evaluate the external security posture and internal security controls of \textbf{[Organization Name]}. The findings reveal critical deficiencies that expose the organization to significant threats, such as unauthorized access, data breach, and ransomware.

\begin{itemize}
    \item \textbf{Critical Technical Vulnerability:} An external scan of the public IP address \texttt{[Client IP]} identified an open FTP service running a vulnerable version of \texttt{vsftpd} (2.3.4). This specific version is associated with a well-known backdoor (CVE-2011-2523), which could grant an attacker remote command execution. Furthermore, the service allows anonymous login, posing a severe and immediate risk.
    
    \item \textbf{Critical Control Gaps:} The security questionnaire revealed a complete absence of Multi-Factor Authentication (MFA) for email, computer logins, and access to sensitive data. This lack of a fundamental security control dramatically increases the risk of account compromise and subsequent lateral movement within the network.
    
    \item \textbf{Existing Medium Risk:} The organization is already tracking a medium-severity risk related to outdated Windows 7 workstations, which are no longer supported and do not receive security updates.
\end{itemize}

\noindent The combination of a publicly accessible, vulnerable service and weak authentication controls creates a high-likelihood path for a security breach. Priority must be given to remediating the external FTP service and implementing a comprehensive MFA policy.

% ===================================================================
\section{Organizational Information}
% ===================================================================

The following information was used as the basis for this assessment. As per the template mode, placeholders are used where data was not provided.

\begin{itemize}
    \item \textbf{Organization Name:} \textbf{[Organization Name]}
    \item \textbf{Primary Domain:} \texttt{[Domain]}
    \item \textbf{External IP Scanned:} \texttt{[Client IP]}
    \item \textbf{Target IP in Scan Data:} \texttt{[Target IP]}
\end{itemize}

% ===================================================================
\section{Security Control Review}
% ===================================================================

The following table summarizes the organization's responses to the security controls questionnaire. "No" answers indicate significant gaps in the security framework and are highlighted as high-risk findings.

\begin{table}[h!]
\centering
\caption{Security Controls Questionnaire Analysis}
\label{tab:controls}
\begin{tabular}{p{0.6\linewidth} c l}
\toprule
\textbf{Control Question} & \textbf{Status} & \textbf{Assessment} \\
\midrule
Do you require MFA to access email? & \ding{55} & \textcolor{red}{\textbf{Critical Gap}} \\
Do you require MFA to log into computers? & \ding{55} & \textcolor{red}{\textbf{Critical Gap}} \\
Do you require MFA to access sensitive data systems? & \ding{55} & \textcolor{red}{\textbf{Critical Gap}} \\
\midrule
Does your organization have an employee acceptable use policy? & \ding{51} & Control in Place \\
Does your organization do security awareness training for new employees? & \ding{51} & Control in Place \\
Does your organization do security awareness training for all employees at least once per year? & \ding{51} & Control in Place \\
\bottomrule
\end{tabular}
\end{table}

\paragraph{Analysis:} The lack of MFA across all key access points (email, endpoints, data systems) is a critical failure. A single compromised password could grant an attacker widespread access. While the organization's security awareness and policy controls are positive, they are insufficient to mitigate the risks posed by weak authentication.

% ===================================================================
\section{Technical Scan Results}
% ===================================================================

An Nmap scan was performed on the target IP address \texttt{[Target IP]}. The scan identified one open port with a critically vulnerable service.

\begin{table}[h!]
\centering
\caption{Open Port Analysis}
\label{tab:scan}
\begin{tabular}{l l l l p{0.3\linewidth}}
\toprule
\textbf{Port} & \textbf{State} & \textbf{Service} & \textbf{Version} & \textbf{Notes} \\
\midrule
21/tcp & Open & ftp & \seqsplit{\texttt{vsftpd 2.3.4}} & \textcolor{red}{\textbf{CRITICAL:}} This version is vulnerable to a backdoor (CVE-2011-2523). Anonymous FTP login is also allowed, permitting unauthenticated access. \\
\bottomrule
\end{tabular}
\end{table}

\paragraph{Analysis:} The presence of an open FTP port is discouraged due to its unencrypted nature. The specific version, \texttt{vsftpd 2.3.4}, contains a backdoor that was inserted into the source code, allowing a remote attacker to execute arbitrary commands. This represents a direct and immediate path to system compromise.

% ===================================================================
\section{Consolidated Risk Assessment}
% ===================================================================

The following table consolidates findings from the technical scan, control review, and pre-existing risk list. Risks are prioritized based on severity and potential impact.

\begin{table}[h!]
\centering
\caption{Summary of Identified Risks}
\label{tab:risks}
\begin{tabular}{p{0.25\linewidth} p{0.45\linewidth} l}
\toprule
\textbf{Risk Name} & \textbf{Overview} & \textbf{Severity} \\
\midrule
\textbf{Vulnerable External FTP Service} & An outdated FTP server (vsftpd 2.3.4) is exposed to the internet. It is vulnerable to a known backdoor (CVE-2011-2523) and allows anonymous login. & \textbf{Critical} \\
\addlinespace
\textbf{No Multi-Factor Authentication} & MFA is not enforced for email, computer logins, or sensitive systems, making accounts highly susceptible to takeover via password compromise. & \textbf{Critical} \\
\addlinespace
\textbf{Outdated Windows Policy} & Workstations are running Windows 7, which is an unsupported End-of-Life (EOL) operating system that no longer receives security updates. & Medium \\
\bottomrule
\end{tabular}
\end{table}

% ===================================================================
\section{Recommendations}
% ===================================================================

Based on the consolidated risk assessment, the following actions are recommended. They are prioritized to address the most critical threats first.

\subsection{Immediate Priority (0-7 Days)}
\begin{enumerate}
    \item \textbf{Remediate Vulnerable FTP Service:} Immediately take the FTP service on \texttt{[Client IP]} offline. 
    \begin{itemize}
        \item If the service is not required for business operations, disable it permanently and ensure the firewall blocks port 21.
        \item If the service is business-critical, upgrade \texttt{vsftpd} to the latest stable version, disable anonymous login, and configure it to use secure protocols (SFTP/FTPS).
    \end{itemize}
\end{enumerate}

\subsection{High Priority (1-3 Months)}
\begin{enumerate}
    \setcounter{enumi}{1}
    \item \textbf{Implement Multi-Factor Authentication (MFA):} Begin a phased rollout of MFA across the organization.
    \begin{itemize}
        \item \textbf{Phase 1:} Enforce MFA for all email accounts (e.g., Office 365, Google Workspace).
        \item \textbf{Phase 2:} Enforce MFA for all remote access systems (e.g., VPN) and privileged user accounts.
        \item \textbf{Phase 3:} Extend MFA to all computer logins and access to sensitive data systems.
    \end{itemize}
\end{enumerate}

\subsection{Medium Priority (3-6 Months)}
\begin{enumerate}
    \setcounter{enumi}{2}
    \item \textbf{Upgrade End-of-Life Systems:} Execute the plan to upgrade all workstations from Windows 7 to a supported operating system, such as Windows 10 or 11. This will ensure systems receive critical security patches.
\end{enumerate}

\end{document}
```