```latex
\documentclass[12pt]{article}

% --- PACKAGE IMPORTS ---
\usepackage[margin=1in]{geometry} % Set page margins
\usepackage{pifont}               % For checkmark and X symbols (\ding)
\usepackage{booktabs}             % For professional-looking tables
\usepackage{hyperref}             % For hyperlinks (e.g., in the footer)
\usepackage{url}                  % For formatting URLs
\usepackage{seqsplit}             % To split long strings in \texttt
\usepackage{graphicx}             % For including logos (optional)
\usepackage{xcolor}               % For custom colors

% --- DOCUMENT METADATA & STYLING ---
\hypersetup{
    colorlinks=true,
    linkcolor=blue,
    filecolor=magenta,      
    urlcolor=cyan,
    pdftitle={Cybersecurity Assessment Report},
    pdfauthor={Cybersecurity Analyst},
    pdfsubject={Security Assessment},
    pdfkeywords={Security, Analysis, Report},
}

% Define colors for severity
\definecolor{criticalred}{HTML}{D10000}
\definecolor{highorange}{HTML}{E67E22}
\definecolor{mediumyellow}{HTML}{F1C40F}
\definecolor{lowblue}{HTML}{3498DB}

\newcommand{\severitycritical}[1]{\textcolor{criticalred}{\textbf{#1}}}
\newcommand{\severityhigh}[1]{\textcolor{highorange}{\textbf{#1}}}

\begin{document}

% --- TITLE PAGE ---
\begin{titlepage}
    \centering
    \vspace*{1cm}
    \Huge\textbf{Cybersecurity Assessment Report}
    \vspace{1.5cm}
    \Large
    \textbf{Prepared for:}\\
    \vspace{0.5cm}
    \textbf{[Organization Name]}
    \vspace{2cm}
    \large
    \textbf{Date of Report:}\\
    \today
    \vfill
    \textit{This report contains sensitive and confidential information intended for the exclusive use of the recipient organization. Unauthorized distribution is strictly prohibited.}
\end{titlepage}

\tableofcontents
\newpage

% --- EXECUTIVE OVERVIEW ---
\section{Executive Overview}

This report details the findings of a cybersecurity assessment conducted for \textbf{[Organization Name]}. The analysis combines a review of organizational security controls, an external network scan, and an evaluation of pre-existing risks.

The assessment reveals several \severitycritical{critical} and \severityhigh{high-risk} gaps in foundational security controls. The most significant finding is the complete absence of Multi-Factor Authentication (MFA) for email, computer logins, and access to sensitive data systems. This deficiency exposes the organization to a high likelihood of account compromise, data breach, and unauthorized access through credential theft or phishing attacks.

Furthermore, significant procedural weaknesses were identified, including the lack of a formal Acceptable Use Policy (AUP) and the absence of security awareness training for new employees during their critical onboarding period.

The external network scan of the target IP address \texttt{[Target IP]} did not detect any open ports. While this suggests a potentially well-configured perimeter firewall for the scanned host, it does not mitigate the severe internal and policy-related risks identified. The organization's primary threats currently stem from policy gaps rather than external technical vulnerabilities.

Immediate remediation efforts should be prioritized to address the lack of MFA and to establish core security policies and training procedures.

% --- ORGANIZATIONAL INFORMATION ---
\section{Organizational Information}

The following details were used as the basis for this assessment. The information provided was anonymized.

\begin{itemize}
    \item \textbf{Organization Name:} \textbf{[Organization Name]}
    \item \textbf{Primary Domain:} \texttt{[Domain]}
    \item \textbf{Scanned Public IP:} \texttt{[Client IP]}
\end{itemize}

% --- SECURITY CONTROL REVIEW ---
\section{Security Control Review (Questionnaire Analysis)}

A review of the organization's security controls was conducted via a standardized questionnaire. The responses highlight significant gaps in access control and security governance. A summary of the findings is presented in Table \ref{tab:controls}. The symbol \ding{51} indicates a positive control is in place, while \ding{55} indicates a control gap.

\begin{table}[h!]
    \centering
    \caption{Security Control Questionnaire Results}
    \label{tab:controls}
    \begin{tabular}{p{0.8\linewidth} c}
        \toprule
        \textbf{Control Question} & \textbf{Response} \\
        \midrule
        Do you require MFA to access email? & \ding{55} \\
        Do you require MFA to log into computers? & \ding{55} \\
        Do you require MFA to access sensitive data systems? & \ding{55} \\
        Does your organization have an employee acceptable use policy? & \ding{55} \\
        Does your organization do security awareness training for new employees? & \ding{55} \\
        Does your organization do security awareness training for all employees at least once per year? & \ding{51} \\
        \bottomrule
    \end{tabular}
\end{table}

The lack of MFA across all critical systems is a \severitycritical{critical} weakness. The absence of an Acceptable Use Policy and security training for new hires represent \severityhigh{high} risks to the organization's security posture.

% --- TECHNICAL SCAN RESULTS ---
\section{Technical Scan Results}

An external network vulnerability scan was performed to identify exposed services and potential vulnerabilities on the organization's perimeter.

\begin{itemize}
    \item \textbf{Target IP Address:} \texttt{[Target IP]}
    \item \textbf{Scan Date:} Scan date not provided in the input data.
\end{itemize}

\subsection{Summary of Findings}
The network scan did not identify any open TCP or UDP ports on the specified target. This indicates that the host is likely protected by a restrictive firewall policy that drops or rejects unsolicited incoming traffic.

\textbf{Conclusion:} While the lack of open ports is a positive finding for this specific host, it should not be interpreted as a guarantee of overall network security. Other hosts may have different configurations, and this result does not account for vulnerabilities in web applications or risks from internal threats.

% --- RISK ASSESSMENT ---
\section{Risk Assessment}

This section synthesizes findings from the security control review, technical scan, and pre-existing risk data. The primary risks identified are procedural and policy-based, originating from the control gaps noted in Section 3. No pre-existing vulnerabilities were provided for this assessment.

\begin{table}[h!]
    \centering
    \caption{Identified Risks and Severity}
    \label{tab:risks}
    \begin{tabular}{p{0.25\linewidth} p{0.5\linewidth} p{0.15\linewidth}}
        \toprule
        \textbf{Risk Name} & \textbf{Overview} & \textbf{Severity} \\
        \midrule
        \textbf{Lack of Multi-Factor Authentication (MFA)} & No MFA is enforced for email, computer logins, or sensitive data systems. This exposes the organization to severe risk from credential theft, phishing, and brute-force attacks, potentially leading to a full-scale data breach. & \severitycritical{Critical} \\
        \addlinespace
        \textbf{Missing Acceptable Use Policy (AUP)} & The absence of a formal AUP means employees lack clear guidelines on the safe and appropriate use of company systems. This increases the risk of insider threats, accidental data exposure, and non-compliance with regulations. & \severityhigh{High} \\
        \addlinespace
        \textbf{Inadequate New Employee Onboarding} & New hires do not receive security awareness training upon joining. This creates a critical window of vulnerability where new employees are more susceptible to social engineering and policy violations before they are included in the annual training cycle. & \severityhigh{High} \\
        \bottomrule
    \end{tabular}
\end{table}

% --- RECOMMENDATIONS ---
\section{Recommendations}

The following prioritized, actionable recommendations are provided to mitigate the identified risks and strengthen the overall security posture of \textbf{[Organization Name]}.

\subsection{Priority 1: Implement Multi-Factor Authentication (Critical)}
\begin{itemize}
    \item \textbf{Action:} Immediately deploy and enforce MFA for all users across all critical platforms, including email (e.g., Office 365, Google Workspace), VPN access, and any systems hosting sensitive data.
    \item \textbf{Justification:} This is the single most effective control to prevent unauthorized access resulting from compromised credentials. It directly mitigates the highest-rated risk identified in this assessment.
\end{itemize}

\subsection{Priority 2: Develop and Implement Core Security Policies (High)}
\begin{itemize}
    \item \textbf{Action:} Draft, approve, and disseminate a formal Acceptable Use Policy (AUP). This policy should clearly define the rules for using company networks, devices, and data.
    \item \textbf{Justification:} An AUP establishes a baseline for secure employee behavior, provides a clear framework for enforcement, and is a foundational component of any security or compliance program.
\end{itemize}

\subsection{Priority 3: Enhance Security Awareness Program (High)}
\begin{itemize}
    \item \textbf{Action:} Integrate mandatory security awareness training into the new employee onboarding process. This training must be completed before a new hire is granted access to sensitive systems.
    \item \textbf{Justification:} Addressing security awareness at the point of entry reduces the likelihood of new employees falling victim to common attacks and ensures they understand their security responsibilities from day one.
\end{itemize}

\end{document}
```