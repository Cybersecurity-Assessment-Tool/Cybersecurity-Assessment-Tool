```latex
\documentclass[12pt]{article}

% Preamble: Required Packages
\usepackage[margin=1in]{geometry}
\usepackage{pifont} % For checkmarks and crosses
\usepackage{booktabs} % For professional tables
\usepackage{hyperref} % For hyperlinks and PDF metadata
\usepackage{url} % For formatting URLs
\usepackage{seqsplit} % For splitting long strings in tt font
\usepackage{graphicx} % For potential logos
\usepackage{xcolor} % For colors

% Document Information and Hyperlink Setup
\hypersetup{
    colorlinks=true,
    linkcolor=blue,
    filecolor=magenta,      
    urlcolor=cyan,
    pdftitle={Cybersecurity Assessment Report},
    pdfauthor={Cybersecurity Analyst},
    pdfsubject={Security Assessment},
    pdfkeywords={Cybersecurity, Nmap, Risk, Assessment},
    bookmarks=true,
}

% Custom Commands
\newcommand{\yes}{\ding{51}} % Green checkmark
\newcommand{\no}{\ding{55}}  % Red X

\begin{document}

% --- TITLE PAGE ---
\begin{titlepage}
    \centering
    \vspace*{1cm}
    \Huge\textbf{Cybersecurity Assessment Report}
    \vspace{1.5cm}
    \par
    \Large
    Prepared for: \textbf{[Organization Name]} \\
    \vspace{1cm}
    Date: \today
    \vfill
    \par
    \large
    \textbf{CONFIDENTIAL} \\
    \small This document contains sensitive information. Distribution is restricted to authorized personnel only.
\end{titlepage}

\tableofcontents
\newpage

% --- EXECUTIVE SUMMARY ---
\section{Executive Summary}
This report details the findings of a cybersecurity assessment conducted for \textbf{[Organization Name]}. The assessment combined a technical network scan, a review of existing risks, and an analysis of organizational security controls based on a questionnaire.

The overall security posture presents a significant contrast. On one hand, the technical network scan of the external IP address \texttt{[Client IP]} revealed a strong network perimeter with no open ports detected. This indicates a well-configured firewall and is a commendable security strength.

On the other hand, the review of administrative and policy-based controls revealed several critical gaps. The organization currently lacks mandatory Multi-Factor Authentication (MFA) for sensitive data systems, does not have a formal employee acceptable use policy, and has no security awareness training program for new or existing employees. These deficiencies expose the organization to a high risk of social engineering, insider threats, and policy violations, which could bypass the strong technical perimeter controls.

Immediate action is recommended to address these policy and training gaps to build a defense-in-depth security strategy and mitigate the significant human-element risks identified.

% --- ORGANIZATIONAL INFORMATION ---
\section{Organizational Information}
The following details were used as the basis for this assessment. As per the provided data, placeholder values are used where specific information was not available.

\begin{itemize}
    \item \textbf{Organization Name:} \textbf{[Organization Name]}
    \item \textbf{Primary Domain:} \texttt{[Domain]}
    \item \textbf{Scanned External IP:} \texttt{[Client IP]}
\end{itemize}

% --- SECURITY CONTROL REVIEW ---
\section{Security Control Review}
The following table summarizes the organization's responses to the security controls questionnaire. "No" answers indicate significant gaps in the current security framework and are correlated with the risks identified in Section 5.

\begin{table}[h!]
\centering
\caption{Security Controls Questionnaire Analysis}
\begin{tabular}{p{0.6\linewidth} c p{0.2\linewidth}}
\toprule
\textbf{Control Question} & \textbf{Response} & \textbf{Assessment} \\
\midrule
Do you require MFA to access email? & \yes & Control Implemented \\
Do you require MFA to log into computers? & \yes & Control Implemented \\
Do you require MFA to access sensitive data systems? & \no & \textbf{Critical Gap} \\
Does your organization have an employee acceptable use policy? & \no & \textbf{High Risk} \\
Does your organization do security awareness training for new employees? & \no & \textbf{High Risk} \\
Does your organization do security awareness training for all employees at least once per year? & \no & \textbf{High Risk} \\
\bottomrule
\end{tabular}
\end{table}

% --- TECHNICAL SCAN RESULTS ---
\section{Technical Scan Results}
An external network scan was performed to identify exposed services and potential vulnerabilities on the organization's perimeter.

\begin{itemize}
    \item \textbf{Target IP Address:} \texttt{[Target IP]}
    \item \textbf{Scan Date:} [Scan Date]
    \item \textbf{Scanner Used:} Nmap
\end{itemize}

\subsection{Summary of Findings}
The scan results were positive. No open ports were discovered on the target host. All other scanned ports were found to be in a \texttt{closed} state, meaning they are accessible but have no application listening on them.

\textbf{Conclusion:} This result indicates a strong firewall configuration at the network perimeter. It effectively minimizes the external attack surface, which is a foundational security best practice.

% --- RISK ASSESSMENT ---
\section{Risk Assessment}
This section synthesizes findings from the security control review, technical scan, and pre-existing risk data. Since no pre-existing vulnerabilities were provided, all risks listed below are new findings from this assessment. The primary areas of concern are administrative and policy-related.

\begin{table}[h!]
\centering
\caption{Identified Risks and Severity}
\begin{tabular}{p{0.1\linewidth} p{0.25\linewidth} p{0.4\linewidth} l}
\toprule
\textbf{Risk ID} & \textbf{Risk Name} & \textbf{Overview} & \textbf{Severity} \\
\midrule
RISK-001 & No MFA for Sensitive Data & The absence of MFA on systems containing sensitive data significantly increases the risk of unauthorized access via compromised credentials. & \textbf{Critical} \\
\addlinespace
RISK-002 & Lack of Acceptable Use Policy (AUP) & Without a formal AUP, there are no clear guidelines for employees on the acceptable use of company assets, increasing the risk of misuse and insider threats. & High \\
\addlinespace
RISK-003 & No Onboarding Security Training & New employees are not trained on security best practices, making them highly susceptible to phishing, social engineering, and other common attacks. & High \\
\addlinespace
RISK-004 & No Annual Security Training & The lack of ongoing training means employees' security awareness is not maintained, and they are not kept up-to-date on emerging threats. & High \\
\bottomrule
\end{tabular}
\end{table}

% --- RECOMMENDATIONS ---
\section{Recommendations}
The following actions are recommended to mitigate the identified risks and improve the overall security posture of \textbf{[Organization Name]}. Recommendations are prioritized based on risk severity.

\subsection{Critical Priority}
\begin{itemize}
    \item \textbf{RISK-001: Implement MFA for Sensitive Data Systems} \\
    Immediately enforce mandatory Multi-Factor Authentication (MFA) for all user accounts (including administrative and service accounts) that have access to systems storing or processing sensitive company or customer data.
\end{itemize}

\subsection{High Priority}
\begin{itemize}
    \item \textbf{RISK-002: Develop and Implement an Acceptable Use Policy (AUP)} \\
    Create a formal AUP that clearly defines the rules and responsibilities for all employees regarding the use of company networks, systems, and data. This policy should be reviewed by legal counsel, communicated to all staff, and formally acknowledged via signature.

    \item \textbf{RISK-003: Establish a New Hire Security Training Program} \\
    Develop a mandatory security awareness training module as part of the employee onboarding process. This training should cover core topics such as phishing identification, password hygiene, data handling, and the new AUP.

    \item \textbf{RISK-004: Institute an Annual Security Training Program} \\
    Implement a recurring, mandatory security awareness training program for all employees to be completed at least once per year. Supplement this with periodic phishing simulation campaigns to test and reinforce the training.
\end{itemize}

\end{document}
```