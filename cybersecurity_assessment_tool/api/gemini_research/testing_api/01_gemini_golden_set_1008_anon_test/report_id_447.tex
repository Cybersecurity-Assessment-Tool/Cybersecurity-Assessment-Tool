```latex
\documentclass[12pt]{article}

% --- PACKAGE IMPORTS ---
\usepackage[margin=1in]{geometry} % Set page margins
\usepackage{pifont}               % For checkmarks and crosses (\ding)
\usepackage{booktabs}             % For professional-looking tables
\usepackage{hyperref}             % For hyperlinks (e.g., in the footer)
\usepackage{url}                  % For formatting URLs
\usepackage{seqsplit}             % For splitting long strings in \texttt
\usepackage{graphicx}             % For potential logos
\usepackage{xcolor}               % For custom colors

% --- DOCUMENT METADATA & STYLING ---
\hypersetup{
    colorlinks=true,
    linkcolor=blue,
    filecolor=magenta,      
    urlcolor=cyan,
    pdftitle={Cybersecurity Posture Report},
    pdfpagemode=FullScreen,
}

\newcommand{\yes}{\ding{51}} % Green checkmark
\newcommand{\no}{\ding{55}}  % Red X

% --- BEGIN DOCUMENT ---
\begin{document}

% --- TITLE PAGE ---
\begin{titlepage}
    \centering
    \vspace*{2cm}
    
    \Huge \textbf{Cybersecurity Posture Report}
    
    \vspace{1.5cm}
    
    \Large \textbf{Prepared for:} \\
    \vspace{0.5cm}
    \huge \textbf{[Organization Name]}
    
    \vfill
    
    \large \textbf{Date of Report:} \\
    \today
    
    \vspace{1cm}
    
    \large \textbf{Generated by:} \\
    Cybersecurity Analysis Division
    
\end{titlepage}

\tableofcontents
\newpage

% --- EXECUTIVE SUMMARY ---
\section*{1. Executive Summary}
This report provides a comprehensive analysis of the cybersecurity posture of \textbf{[Organization Name]}, based on technical network scans, a review of organizational security controls, and an assessment of pre-existing risks.

The assessment identified several critical and high-risk vulnerabilities. The most severe finding is an externally exposed MySQL database (\texttt{[Target IP]}:3306) running an outdated and unsupported version (MySQL 5.7.33), which reached its End-of-Life in October 2023. This service is no longer receiving security patches and is a prime target for attackers.

Furthermore, significant gaps were identified in administrative and procedural controls. The lack of mandatory Multi-Factor Authentication (MFA) for computer logins presents a high risk for unauthorized access and lateral movement within the network. This is compounded by the absence of a formal employee acceptable use policy and a structured security awareness training program, increasing the susceptibility to social engineering and insider threats.

Immediate remediation is required to address the exposed database and implement foundational security controls to mitigate these risks and improve the organization's overall defensive posture.

% --- ORGANIZATIONAL INFORMATION ---
\section*{2. Organizational Information}
This report pertains to the following entity and associated assets.
\begin{itemize}
    \item \textbf{Organization Name:} \textbf{[Organization Name]}
    \item \textbf{Primary Domain:} \texttt{[Domain]}
    \item \textbf{Scanned External IP:} \texttt{[Client IP]}
\end{itemize}

% --- SECURITY CONTROL REVIEW ---
\section*{3. Security Control Review}
A review of administrative and procedural security controls was conducted via a standardized questionnaire. The results highlight significant gaps in employee policy and access control enforcement. A "No" indicates a deviation from security best practices and a potential area of high risk.

\begin{table}[h!]
\centering
\caption{Organizational Security Control Status}
\begin{tabular}{@{}lc@{}}
\toprule
\textbf{Control Question} & \textbf{Status} \\
\midrule
Do you require MFA to access email? & \yes \\
Do you require MFA to log into computers? & \no \\
Do you require MFA to access sensitive data systems? & \yes \\
Does your organization have an employee acceptable use policy? & \no \\
Does your organization do security awareness training for new employees? & \no \\
Does your organization do security awareness training for all employees at least once per year? & \no \\
\bottomrule
\end{tabular}
\end{table}

% --- TECHNICAL SCAN RESULTS ---
\section*{4. Technical Scan Results}
A network scan was performed on the client's external-facing infrastructure to identify open ports and exposed services.

\subsection*{4.1. Target Information}
\begin{itemize}
    \item \textbf{Target IP Address:} \texttt{[Target IP]}
    \item \textbf{Scan Status:} Host is Up
\end{itemize}

\subsection*{4.2. Open Ports and Services}
The following table details the services discovered to be accessible from the public internet.

\begin{table}[h!]
\centering
\caption{Discovered Open Ports}
\begin{tabular}{@{}lllll@{}}
\toprule
\textbf{Port} & \textbf{State} & \textbf{Service} & \textbf{Product} & \textbf{Version} \\
\midrule
3306/tcp & open & mysql & MySQL & 5.7.33 \\
\bottomrule
\end{tabular}
\end{table}

\subsection*{4.3. Critical Technical Finding}
\textbf{Unsupported Software Version:} The discovered MySQL service is version \textbf{5.7.33}. According to the official Oracle Lifetime Support policy, MySQL 5.7 reached its End-of-Life (EOL) in \textbf{October 2023}. Systems running EOL software do not receive security updates, patches for newly discovered vulnerabilities, or technical support. This exposes the database to a wide range of known exploits and significantly increases the risk of a data breach.

% --- RISK ASSESSMENT ---
\section*{5. Risk Assessment}
The following table synthesizes findings from the security control review, technical scans, and pre-existing risk data. Risks are prioritized based on their potential impact and likelihood of exploitation.

\begin{table}[h!]
\centering
\caption{Synthesized Risk Summary}
\begin{tabular}{@{}p{0.2\textwidth}p{0.5\textwidth}p{0.2\textwidth}@{}}
\toprule
\textbf{Risk Title} & \textbf{Description} & \textbf{Severity} \\
\midrule
\textbf{Exposed \& Outdated Database Service} & A MySQL database (v5.7.33) is publicly accessible on port 3306. This version is End-of-Life and no longer receives security patches, making it highly vulnerable to known exploits. This confirms and elevates the pre-existing "Database Exposure" risk. & \textbf{Critical} \\
\addlinespace
\textbf{Lack of Endpoint MFA} & The absence of mandatory MFA for computer logins creates a significant risk. A single compromised password could grant an attacker full access to an employee's workstation, enabling data theft and lateral movement across the network. & \textbf{High} \\
\addlinespace
\textbf{Deficient Security Awareness Program} & The organization lacks a formal Acceptable Use Policy and does not conduct security awareness training. This leaves employees unprepared to identify and resist phishing, social engineering, and other common cyberattacks. & \textbf{High} \\
\bottomrule
\end{tabular}
\end{table}

% --- RECOMMENDATIONS ---
\section*{6. Recommendations}
Based on the analysis, the following actions are recommended to mitigate the identified risks and strengthen the overall security posture.

\subsection*{6.1. Immediate Actions (To be completed within 72 hours)}
\begin{enumerate}
    \item \textbf{Restrict Database Access:} Immediately implement firewall rules to block all public access to port 3306 on \texttt{[Target IP]}. Access should be restricted to a whitelist of trusted IP addresses only.
    \item \textbf{Develop Upgrade Plan:} Formulate an urgent plan to migrate the MySQL 5.7.33 database to a currently supported version (e.g., MySQL 8.x). This plan should include data backup, testing, and a scheduled migration window.
\end{enumerate}

\subsection*{6.2. High-Priority Actions (To be completed within 30 days)}
\begin{enumerate}
    \item \textbf{Implement Endpoint MFA:} Enforce Multi-Factor Authentication for all employee computer and laptop logins using a reputable solution.
    \item \textbf{Establish Security Awareness Training:} Implement a mandatory security awareness training program for all new and existing employees. Training should be conducted annually and cover topics such as phishing, password security, and data handling.
    \item \textbf{Develop Acceptable Use Policy (AUP):} Draft and enforce an AUP that clearly defines the rules for using company IT assets, data, and network resources. All employees must read and acknowledge this policy.
\end{enumerate}

\subsection*{6.3. Strategic Long-Term Recommendations}
\begin{enumerate}
    \item \textbf{Implement VPN for Database Access:} For any required remote database administration, establish a Virtual Private Network (VPN). This ensures that all traffic to the database is encrypted and authenticated, removing the need for any public-facing exposure.
    \item \textbf{Establish Vulnerability Management Program:} Institute a regular schedule of internal and external vulnerability scanning to proactively identify and remediate security weaknesses before they can be exploited.
\end{enumerate}

\end{document}
```