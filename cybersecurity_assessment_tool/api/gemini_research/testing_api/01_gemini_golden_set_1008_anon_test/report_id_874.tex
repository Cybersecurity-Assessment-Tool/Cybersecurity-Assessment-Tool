```latex
\documentclass[12pt]{article}

% --- PACKAGE IMPORTS ---
\usepackage[margin=1in]{geometry}
\usepackage{pifont} % For checkmarks and crosses
\usepackage{booktabs} % For professional tables
\usepackage[hidelinks]{hyperref} % For clickable links
\usepackage{url} % For URL formatting
\usepackage{seqsplit} % For splitting long strings
\usepackage{graphicx} % For logo
\usepackage{fancyhdr} % For header/footer

% --- DOCUMENT METADATA ---
\title{Cybersecurity Posture Assessment Report}
\author{Cybersecurity Analysis Division}
\date{\today}

% --- HEADER & FOOTER ---
\pagestyle{fancy}
\fancyhf{} % Clear all header and footer fields
\fancyhead[L]{\textbf{[Organization Name]} \textbar{} Cybersecurity Report}
\fancyfoot[C]{Page \thepage}
\renewcommand{\headrulewidth}{0.4pt}
\renewcommand{\footrulewidth}{0.4pt}

\begin{document}

\maketitle
\thispagestyle{empty}
\tableofcontents
\newpage

% --- 1. EXECUTIVE SUMMARY ---
\section{Executive Summary}
This report provides a comprehensive analysis of the cybersecurity posture of \textbf{[Organization Name]}, conducted on \today. The assessment synthesizes data from an external network scan, a review of internal security controls via a questionnaire, and an analysis of pre-existing risks.

The assessment identified several critical and high-risk gaps in the organization's security controls, primarily related to identity and access management and foundational security policies. While the external network scan of the designated target did not reveal any open ports, this should not be interpreted as a sign of robust security. The policy-based vulnerabilities represent a significant threat to the organization.

Key findings include the absence of Multi-Factor Authentication (MFA) for email and computer access, the lack of a formal employee acceptable use policy, and incomplete security awareness training. These gaps expose the organization to substantial risks, including unauthorized access, data breaches, and ransomware attacks. Immediate remediation of these issues is strongly recommended to reduce the attack surface and improve the overall security posture.

% --- 2. ORGANIZATIONAL INFORMATION ---
\section{Organizational Information}
This section details the information provided about the organization. The placeholders indicate that this data was not available at the time of the assessment.

\begin{itemize}
    \item \textbf{Organization Name:} \textbf{[Organization Name]}
    \item \textbf{Primary Email Domain:} \texttt{[Domain]}
    \item \textbf{Assessed External IP:} \texttt{[Client IP]}
\end{itemize}

% --- 3. SECURITY CONTROL REVIEW ---
\section{Security Control Review}
The following table summarizes the responses to the security controls questionnaire. A "No" response indicates a potential gap in security that increases risk. Each gap has been assessed and is addressed in the Risk Assessment section.

\begin{table}[h!]
\centering
\caption{Security Controls Questionnaire Analysis}
\begin{tabular}{p{8cm} c l}
\toprule
\textbf{Control Question} & \textbf{Response} & \textbf{Assessment} \\
\midrule
Do you require MFA to access email? & \ding{55} & \textbf{Critical Gap} \\
Do you require MFA to log into computers? & \ding{55} & \textbf{Critical Gap} \\
Do you require MFA to access sensitive data systems? & \ding{51} & Best Practice Met \\
Does your organization have an employee acceptable use policy? & \ding{55} & High Risk \\
Does your organization do security awareness training for new employees? & \ding{55} & High Risk \\
Does your organization do security awareness training for all employees at least once per year? & \ding{51} & Best Practice Met \\
\bottomrule
\end{tabular}
\end{table}

% --- 4. TECHNICAL SCAN RESULTS ---
\section{Technical Scan Results}
An external network vulnerability scan was performed to identify open ports and exposed services.

\begin{itemize}
    \item \textbf{Target IP Address:} \texttt{[Target IP]}
    \item \textbf{Scan Date:} \today
\end{itemize}

\subsection{Summary of Findings}
The network scan completed successfully but did not identify any open TCP or UDP ports on the target system. 

\textbf{Analysis:} While the absence of open ports is a positive finding, it typically indicates the presence of a well-configured firewall or network access control list (ACL). This prevents external attackers from directly discovering and exploiting services. However, this result does not provide insight into the security of internal systems or the risks associated with phishing, malware, or credential theft, which can bypass perimeter defenses. The policy gaps identified in Section 3 remain the primary source of risk.

% --- 5. RISK ASSESSMENT ---
\section{Risk Assessment}
This section synthesizes findings from all data sources into a consolidated list of identified risks. The severity level is determined by the potential impact on the organization's confidentiality, integrity, and availability. No pre-existing vulnerabilities were reported.

\begin{table}[h!]
\centering
\caption{Consolidated Risk Summary}
\begin{tabular}{p{3.5cm} p{7cm} l}
\toprule
\textbf{Risk Name} & \textbf{Overview} & \textbf{Severity} \\
\midrule
\textbf{Lack of MFA for Critical Systems} & The absence of MFA on email and computer logins exposes the organization to account takeover attacks. A single compromised password could grant an attacker full access to sensitive communications and endpoint devices. & \textbf{Critical} \\
\addlinespace
\textbf{Missing Acceptable Use Policy (AUP)} & Without a formal AUP, employees lack clear guidelines on the secure and acceptable use of company assets. This increases the risk of insider threats, data leakage, and non-compliance with regulations. & \textbf{High} \\
\addlinespace
\textbf{Inadequate Security Awareness Training} & Failing to train new hires on security best practices from day one leaves a critical window of vulnerability. New employees are often prime targets for social engineering and may be unaware of company security policies. & \textbf{High} \\
\bottomrule
\end{tabular}
\end{table}

% --- 6. RECOMMENDATIONS ---
\section{Recommendations}
The following actions are recommended to mitigate the identified risks and strengthen the overall security posture of \textbf{[Organization Name]}.

\subsection{Immediate Actions (0-30 Days)}
\begin{enumerate}
    \item \textbf{Implement and Enforce MFA:}
    \begin{itemize}
        \item \textbf{Action:} Enable MFA for all user accounts across all email platforms (e.g., Microsoft 365, Google Workspace) and for all remote and local computer logins.
        \item \textbf{Priority:} Prioritize implementation for administrative and executive accounts, followed by a full rollout to all employees.
        \item \textbf{Justification:} This is the single most effective control to prevent unauthorized access resulting from stolen credentials.
    \end{itemize}
\end{enumerate}

\subsection{Short-Term Actions (30-90 Days)}
\begin{enumerate}
    \item \textbf{Develop and Implement an Acceptable Use Policy (AUP):}
    \begin{itemize}
        \item \textbf{Action:} Draft a formal AUP that clearly defines the rules for using company networks, devices, and data.
        \item \textbf{Priority:} Require all current employees to read and formally acknowledge the policy. Incorporate this step into the new-hire onboarding process.
        \item \textbf{Justification:} An AUP establishes a baseline for secure employee behavior and is a foundational component of any security program.
    \end{itemize}
    \item \textbf{Establish Onboarding Security Training:}
    \begin{itemize}
        \item \textbf{Action:} Create a mandatory security awareness training module for all new employees to be completed during their first week of employment.
        \item \textbf{Priority:} The training should cover key topics such as phishing, password security, data handling, and the new AUP.
        \item \textbf{Justification:} This ensures that security is ingrained in the company culture from the very beginning of an employee's tenure, reducing unforced errors.
    \end{itemize}
\end{enumerate}

\end{document}
```