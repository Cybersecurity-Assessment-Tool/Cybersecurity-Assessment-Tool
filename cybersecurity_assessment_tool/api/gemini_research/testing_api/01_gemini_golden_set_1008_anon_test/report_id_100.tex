```latex
\documentclass[12pt]{article}

% Preamble: Required Packages
\usepackage[margin=1in]{geometry}
\usepackage{pifont} % For \ding symbols
\usepackage{booktabs} % For professional tables
\usepackage{hyperref} % For clickable links
\usepackage{url} % For URL formatting
\usepackage{seqsplit} % For splitting long strings
\usepackage{graphicx}
\usepackage{fancyhdr}
\usepackage{lastpage}
\usepackage{xcolor}

% --- Document Setup ---

% Define colors for severity levels
\definecolor{critical}{HTML}{990000}
\definecolor{high}{HTML}{D14302}
\definecolor{medium}{HTML}{E5A000}
\definecolor{low}{HTML}{339900}

% Setup Hyperlinks
\hypersetup{
    colorlinks=true,
    linkcolor=blue,
    filecolor=magenta,      
    urlcolor=cyan,
    pdftitle={Cybersecurity Posture Assessment Report},
    pdfpagemode=FullScreen,
}

% --- Header and Footer ---
\pagestyle{fancy}
\fancyhf{} % Clear all header and footer fields
\fancyhead[L]{Cybersecurity Postuasion Assessment}
\fancyhead[R]{\textbf{[Organization Name]}}
\fancyfoot[C]{Page \thepage\ of \pageref{LastPage}}
\renewcommand{\headrulewidth}{0.4pt}
\renewcommand{\footrulewidth}{0.4pt}

% --- Document Body ---
\begin{document}

% --- Title Page ---
\begin{titlepage}
    \centering
    \vspace*{2cm}
    
    \Huge{\textbf{Cybersecurity Posture Assessment Report}}
    
    \vspace{1.5cm}
    
    \Large{\textbf{Prepared For:}}
    
    \vspace{0.5cm}
    
    \Huge{\textbf{[Organization Name]}}
    
    \vspace{2cm}
    
    \includegraphics[width=0.4\textwidth]{example-image-a} % Placeholder for a logo
    
    \vfill
    
    \Large{\textbf{Date of Report: \today}}
    
\end{titlepage}

\newpage
\tableofcontents
\newpage

% --- Section 1: Executive Summary ---
\section{Executive Summary}
This report provides a cybersecurity posture assessment for \textbf{[Organization Name]}, based on a review of organizational security controls, an external network scan, and an analysis of known risks. The assessment was conducted to identify security gaps and provide actionable recommendations to enhance the organization's defensive capabilities.

\paragraph{Key Findings:}
The assessment revealed a mixed security posture. On a positive note, the external network scan of the provided IP address did not identify any open ports, suggesting a strong firewall configuration for that specific asset. The organization also has foundational policies in place, such as an Acceptable Use Policy and security training for new hires.

However, several critical and high-risk gaps were identified in core security controls. The most significant concerns are the absence of Multi-Factor Authentication (MFA) for email and workstation access. This exposes the organization to a high risk of account compromise through phishing and credential theft. Furthermore, the lack of mandatory, annual security awareness training for all employees perpetuates a high susceptibility to social engineering attacks.

\paragraph{Overall Risk Level:}
The overall risk level is assessed as \textbf{High}. While the external perimeter appears hardened, the identified policy and identity management weaknesses could allow an attacker to bypass perimeter defenses and gain significant internal access. Immediate action is recommended to address the critical findings outlined in this report.

% --- Section 2: Organizational Information ---
\section{Organizational Information}
This section details the information provided for the assessment.
\begin{itemize}
    \item \textbf{Organization Name:} \textbf{[Organization Name]}
    \item \textbf{Primary Email Domain:} \texttt{[Domain]}
    \item \textbf{External IP Scanned:} \texttt{[Client IP]}
\end{itemize}

% --- Section 3: Security Control Review ---
\section{Security Control Review}
The following table summarizes the responses from the organizational security questionnaire. A green checkmark (\textcolor{green}{\ding{51}}) indicates a positive control is in place, while a red 'X' (\textcolor{red}{\ding{55}}) indicates a potential security gap.

\begin{table}[h!]
\centering
\caption{Security Controls Questionnaire Analysis}
\label{tab:controls}
\begin{tabular}{p{0.7\linewidth} c c}
\toprule
\textbf{Control Question} & \textbf{Response} & \textbf{Status} \\
\midrule
Do you require MFA to access email? & No & \textcolor{red}{\ding{55}} \\
Do you require MFA to log into computers? & No & \textcolor{red}{\ding{55}} \\
Do you require MFA to access sensitive data systems? & Yes & \textcolor{green}{\ding{51}} \\
Does your organization have an employee acceptable use policy? & Yes & \textcolor{green}{\ding{51}} \\
Does your organization do security awareness training for new employees? & Yes & \textcolor{green}{\ding{51}} \\
Does your organization do security awareness training for all employees at least once per year? & No & \textcolor{red}{\ding{55}} \\
\bottomrule
\end{tabular}
\end{table}

% --- Section 4: Technical Scan Results ---
\section{Technical Scan Results}
An external network vulnerability scan was performed to identify potential weaknesses visible from the public internet.

\subsection{Scan Details}
\begin{itemize}
    \item \textbf{Target IP:} \texttt{[Target IP]}
    \item \textbf{Scan Date:} [Scan Date]
    \item \textbf{Scan Type:} Nmap TCP Port Scan
\end{itemize}

\subsection{Findings Summary}
The scan confirmed that the target host was online and responsive. However, the scan reported that all 1000 of the most common TCP ports were in a \texttt{closed} state. 

\textbf{Conclusion:} No open ports were discovered on the target system. This is a positive finding, indicating that the network firewall is correctly configured to deny unsolicited inbound traffic, effectively reducing the external attack surface of this asset.

% --- Section 5: Risk Assessment ---
\section{Risk Assessment}
This section synthesizes findings from the security control review and technical scan into a prioritized list of identified risks. No pre-existing vulnerabilities were provided for this assessment.

\begin{table}[h!]
\centering
\caption{Identified Risks}
\label{tab:risks}
\begin{tabular}{p{0.15\linewidth} p{0.25\linewidth} p{0.4\linewidth} p{0.1\linewidth}}
\toprule
\textbf{Risk ID} & \textbf{Risk Name} & \textbf{Description} & \textbf{Severity} \\
\midrule
RISK-001 & Lack of MFA for Email Access & Failure to implement MFA on email accounts exposes the organization to a high likelihood of account takeover via phishing or credential stuffing. Compromised email is a primary vector for financial fraud and further internal network intrusion. & \textcolor{critical}{Critical} \\
\addlinespace
RISK-002 & Lack of MFA for Workstation Login & The absence of MFA on computer logins significantly increases the risk of unauthorized access if an employee's credentials are stolen. This could lead to data theft, ransomware deployment, or lateral movement across the network. & \textcolor{high}{High} \\
\addlinespace
RISK-003 & Insufficient Security Awareness Training & By not providing mandatory, annual security training for all employees, the organization's "human firewall" is weakened. This increases susceptibility to evolving social engineering and phishing attacks. & \textcolor{high}{High} \\
\bottomrule
\end{tabular}
\end{table}

% --- Section 6: Recommendations ---
\section{Recommendations}
Based on the risks identified in Section 5, the following actions are recommended to improve the security posture of \textbf{[Organization Name]}.

\begin{description}
    \item[\textbf{Recommendation for RISK-001 (Critical):}] \hfill \\
    \textbf{Implement MFA for Email:} Immediately prioritize the enforcement of MFA for all user accounts accessing the email system, both internally and externally. This is the single most effective control to prevent email account takeovers.
    
    \item[\textbf{Recommendation for RISK-002 (High):}] \hfill \\
    \textbf{Deploy MFA for Endpoints:} Roll out an MFA solution for all workstation and laptop logins, especially for users with administrative privileges or access to sensitive data. This should also be enforced for any remote access solutions (e.g., VPN).
    
    \item[\textbf{Recommendation for RISK-003 (High):}] \hfill \\
    \textbf{Establish Annual Security Training:} Develop and implement a mandatory security awareness training program for all employees, to be completed annually. The training should cover current threats such as phishing, ransomware, and proper data handling.
\end{description}

\end{document}
```