```latex
\documentclass[12pt]{article}

% Preamble: Required Packages
\usepackage[margin=1in]{geometry}
\usepackage{pifont} % For \ding symbols (checkmarks and crosses)
\usepackage{booktabs} % For professional-looking tables
\usepackage{hyperref} % For clickable links and references
\usepackage{url}      % For formatting URLs
\usepackage{seqsplit} % For breaking long text sequences like hashes or URLs
\usepackage{graphicx}
\usepackage{xcolor}
\usepackage{array}

% Hyperlink Setup
\hypersetup{
    colorlinks=true,
    linkcolor=blue,
    filecolor=magenta,
    urlcolor=cyan,
}

% Custom Commands for Readability
\newcommand{\yes}{\textcolor{green}{\ding{51}}}
\newcommand{\no}{\textcolor{red}{\ding{55}}}
\newcolumntype{L}[1]{>{\raggedright\let\newline\\\arraybackslash\hspace{0pt}}m{#1}}

% Document Information
\title{Cybersecurity Posture Assessment Report}
\author{Cybersecurity Analysis Division}
\date{\today}

\begin{document}

\maketitle
\thispagestyle{empty}
\newpage
\tableofcontents
\thispagestyle{empty}
\newpage
\setcounter{page}{1}

% ==============================================================================
% 1. EXECUTIVE SUMMARY
% ==============================================================================
\section{Executive Summary}

This report details the findings of a cybersecurity assessment conducted for \textbf{[Organization Name]}. The analysis is based on a synthesis of network scan data, a security controls questionnaire, and a review of pre-existing risks.

The assessment revealed several critical and high-severity risks that require immediate attention. Key findings include:
\begin{itemize}
    \item \textbf{Critical FTP Vulnerability:} An externally facing FTP server on \texttt{[Client IP]} is running a dangerously outdated version of \texttt{vsftpd} (2.3.4), which is known to contain a critical backdoor vulnerability (CVE-2011-2523). Furthermore, the server is misconfigured to allow anonymous logins, significantly increasing the risk of unauthorized access and data compromise.
    \item \textbf{Critical Lack of Multi-Factor Authentication (MFA):} MFA is not enforced for accessing email or other sensitive data systems. This represents a major security gap, leaving critical assets vulnerable to compromise via stolen credentials.
    \item \textbf{High Risk from Inadequate Training:} The organization does not conduct mandatory annual security awareness training for all employees. This oversight cultivates a high-risk environment where employees are more susceptible to phishing and social engineering attacks.
    \item \textbf{Pre-existing Medium Risk:} The organization is aware of an existing risk related to outdated Windows 7 workstations, which are no longer supported and do not receive security updates.
\end{itemize}

The combination of these findings indicates a weak security posture. Immediate and decisive action is required to remediate these vulnerabilities and mitigate the substantial risk of a security breach. This report provides specific, actionable recommendations to address each identified issue.

% ==============================================================================
% 2. ORGANIZATIONAL INFORMATION
% ==============================================================================
\section{Organizational Information}

The following details were used as the basis for this assessment. Where information was not provided, placeholders have been used.

\begin{itemize}
    \item \textbf{Organization Name:} \textbf{[Organization Name]}
    \item \textbf{Primary Domain:} \texttt{[Domain]}
    \item \textbf{External IP Scanned:} \texttt{[Client IP]}
\end{itemize}


% ==============================================================================
% 3. SECURITY CONTROL REVIEW (QUESTIONNAIRE)
% ==============================================================================
\section{Security Control Review}

The following table summarizes the organization's self-reported security controls based on the provided questionnaire. A green checkmark (\yes) indicates a positive control is in place, while a red cross (\no) indicates a control gap.

\begin{table}[h!]
\centering
\caption{Security Controls Questionnaire Results}
\begin{tabular}{L{0.75\textwidth} c}
\toprule
\textbf{Control Question} & \textbf{Status} \\
\midrule
Do you require MFA to access email? & \no \\
Do you require MFA to log into computers? & \yes \\
Do you require MFA to access sensitive data systems? & \no \\
Does your organization have an employee acceptable use policy? & \yes \\
Does your organization do security awareness training for new employees? & \yes \\
Does your organization do security awareness training for all employees at least once per year? & \no \\
\bottomrule
\end{tabular}
\end{table}

The identified gaps in MFA enforcement for email and sensitive systems, along with the lack of recurring security training, are significant contributors to the overall risk profile.


% ==============================================================================
% 4. TECHNICAL SCAN RESULTS
% ==============================================================================
\section{Technical Scan Results}

An external network scan was performed on the target IP address. The following table details the findings for the host that was found to be active.

\begin{itemize}
    \item \textbf{Target IP Address:} \texttt{[Target IP]}
    \item \textbf{Host Status:} Up
\end{itemize}

\begin{table}[h!]
\centering
\caption{Open Ports and Services on \texttt{[Target IP]}}
\begin{tabular}{c c l l}
\toprule
\textbf{Port} & \textbf{State} & \textbf{Service} & \textbf{Version} \\
\midrule
21/tcp & Open & ftp & vsftpd 2.3.4 \\
\bottomrule
\end{tabular}
\end{table}

\subsection{Analysis of Findings}
The scan identified one open port, which presents a critical security risk:
\begin{itemize}
    \item \textbf{Port 21 (FTP):} The FTP service is running \textbf{vsftpd version 2.3.4}. This specific version is widely known to be vulnerable to a critical backdoor vulnerability (\textbf{CVE-2011-2523}). An attacker can exploit this flaw to gain a command shell on the underlying server, leading to a full system compromise.
    \item \textbf{Anonymous FTP Enabled:} The scan confirmed that \textbf{anonymous FTP login is allowed}. This misconfiguration permits anyone on the internet to connect to the server and potentially access, upload, or download files, creating a high risk of data leakage or malware distribution.
\end{itemize}


% ==============================================================================
% 5. CONSOLIDATED RISK ASSESSMENT
% ==============================================================================
\section{Consolidated Risk Assessment}

The following table synthesizes all identified risks from the questionnaire, technical scan, and pre-existing risk data into a consolidated view.

\begin{table}[h!]
\centering
\caption{Summary of Identified Risks}
\begin{tabular}{p{0.25\linewidth} p{0.45\linewidth} c}
\toprule
\textbf{Risk Name} & \textbf{Overview} & \textbf{Severity} \\
\midrule
\textbf{Vulnerable FTP Service} & An outdated and misconfigured FTP server (vsftpd 2.3.4) with a known backdoor vulnerability is exposed to the internet. Anonymous login is enabled. & \textbf{Critical} \\
\addlinespace
\textbf{MFA Not Enforced} & Multi-Factor Authentication is not required for accessing email or sensitive data systems, leaving them vulnerable to credential theft. & \textbf{Critical} \\
\addlinespace
\textbf{Inadequate Security Training} & Employees do not receive security awareness training on an annual basis, increasing susceptibility to phishing and social engineering. & \textbf{High} \\
\addlinespace
\textbf{Outdated Windows Policy} & Workstations are running the unsupported Windows 7 operating system, which no longer receives security updates from Microsoft. & \textbf{Medium} \\
\bottomrule
\end{tabular}
\end{table}


% ==============================================================================
% 6. RECOMMENDATIONS
% ==============================================================================
\section{Recommendations}

The following actions are recommended to mitigate the identified risks. They are prioritized by severity.

\subsection{Immediate Actions (Critical Risks)}

\subsubsection{Remediate Vulnerable FTP Server}
The exposed FTP server represents an immediate and severe threat.
\begin{enumerate}
    \item \textbf{Decommission:} If the FTP service is not essential for business operations, shut it down and block port 21 at the firewall immediately.
    \item \textbf{Upgrade \& Secure:} If the service is required, take the following steps:
    \begin{itemize}
        \item Upgrade \texttt{vsftpd} to the latest stable version to patch the backdoor vulnerability.
        \item Disable anonymous FTP access in the configuration file.
        \item Implement strong user passwords and enforce access controls.
        \item Place the server behind a firewall and restrict access to only known, trusted IP addresses.
    \end{itemize}
\end{enumerate}

\subsubsection{Enforce Multi-Factor Authentication (MFA)}
\begin{enumerate}
    \item \textbf{Prioritize Email:} Immediately enforce MFA for all user access to the email system (e.g., Office 365, Google Workspace). This is the most common vector for account takeovers.
    \item \textbf{Sensitive Systems:} Develop a plan to roll out MFA for all systems identified as containing sensitive data within the next 30 days.
\end{enumerate}

\subsection{High-Priority Actions}

\subsubsection{Implement Annual Security Awareness Training}
\begin{enumerate}
    \item \textbf{Establish a Program:} Procure and implement a security awareness training solution.
    \item \textbf{Mandate Participation:} Make annual training mandatory for all employees, including management.
    \item \textbf{Track Completion:} Track completion rates to ensure 100\% participation. The training should cover key topics such as phishing, password hygiene, and acceptable use.
\end{enumerate}

\subsection{Medium-Priority Actions}

\subsubsection{Address Outdated Windows Policy}
\begin{enumerate}
    \item \textbf{Execute Upgrade Plan:} Continue with the existing recommendation to upgrade all Windows 7 workstations to a modern, supported operating system like Windows 10 or 11.
    \item \textbf{Isolate Legacy Systems:} If certain systems cannot be upgraded due to hardware or software constraints, they should be isolated on a separate network segment with restricted access to prevent them from being used as a pivot point in an attack.
\end{enumerate}

\end{document}
```