```latex
\documentclass[12pt]{article}

% 1. DOCUMENT SETUP & PACKAGES
\usepackage[margin=1in]{geometry}
\usepackage{pifont} % For checkmarks and crosses
\usepackage{booktabs} % For professional tables
\usepackage{hyperref} % For clickable links
\usepackage{url} % For URL formatting
\usepackage{seqsplit} % To split long strings in tt font
\usepackage{xcolor} % For colors
\usepackage{graphicx} % For logo (placeholder)

% --- Custom Commands & Colors ---
\newcommand{\yes}{\ding{51}}
\newcommand{\no}{\ding{55}}
\definecolor{sev_critical}{HTML}{940000}
\definecolor{sev_high}{HTML}{D14000}
\definecolor{sev_medium}{HTML}{E49B00}
\newcommand{\sevcritical}[1]{\textcolor{sev_critical}{\textbf{#1}}}
\newcommand{\sevhigh}[1]{\textcolor{sev_high}{\textbf{#1}}}
\newcommand{\sevmedium}[1]{\textcolor{sev_medium}{\textbf{#1}}}

% --- Hyperref Setup ---
\hypersetup{
    colorlinks=true,
    linkcolor=blue,
    filecolor=magenta,      
    urlcolor=cyan,
    pdftitle={Cybersecurity Posture Assessment Report},
    pdfpagemode=FullScreen,
}

\begin{document}

% 2. TITLE PAGE
\begin{titlepage}
    \centering
    \vspace*{1cm}
    
    \Huge
    \textbf{Cybersecurity Posture Assessment Report}
    
    \vspace{1.5cm}
    
    \Large
    Prepared for:
    
    \vspace{0.5cm}
    
    \textbf{[Organization Name]}
    
    \vspace{2cm}
    
    \Large
    \textbf{Date of Report:} \today
    
    \vfill
    
    \Large
    \textbf{Generated by:} \\
    Expert Cybersecurity Analyst
    
\end{titlepage}

\tableofcontents
\newpage

% 3. EXECUTIVE SUMMARY
\section*{Executive Summary}

This report provides a comprehensive analysis of the cybersecurity posture for \textbf{[Organization Name]}, based on a synthesis of organizational data, technical network scans, and a review of current risks. The assessment was conducted on \today.

The analysis reveals several \sevcritical{critical} and \sevhigh{high-risk} deficiencies that require immediate attention. The most significant findings are the complete absence of Multi-Factor Authentication (MFA) across all critical services (email, computer logins, and sensitive data systems) and the lack of foundational security policies and training programs for new employees. These gaps create a high probability of successful account compromise through common attacks like phishing.

Furthermore, a technical scan of the external network perimeter identified an exposed SSH (Secure Shell) management port on a key asset. When combined with the lack of MFA, this exposed service presents a direct and significant vector for unauthorized access to the internal network.

Immediate remediation should focus on implementing MFA, developing and enforcing an Acceptable Use Policy, mandating security training for new hires, and securing the exposed SSH service. Addressing these core issues will substantially improve the organization's resilience against prevalent cyber threats.

% 4. ORGANIZATIONAL INFORMATION
\section{Organizational Information}
This section details the information provided by the client for this assessment. The data has been anonymized as requested.

\begin{tabular}{@{}ll}
    \toprule
    \textbf{Attribute} & \textbf{Value} \\
    \midrule
    Organization Name & \textbf{[Organization Name]} \\
    Primary Email Domain & \texttt{[Domain]} \\
    External IP Address Scanned & \seqsplit{\texttt{[Client IP]}} \\
    \bottomrule
\end{tabular}

% 5. SECURITY CONTROL REVIEW (FROM QUESTIONNAIRE)
\section{Security Control Review}
The following table summarizes the organization's responses to a security controls questionnaire. This review provides insight into the maturity of the existing security program. "No" responses indicate significant control gaps.

\begin{center}
\begin{tabular}{p{0.75\linewidth} c}
    \toprule
    \textbf{Control Question} & \textbf{Response} \\
    \midrule
    Do you require MFA to access email? & \no \\
    Do you require MFA to log into computers? & \no \\
    Do you require MFA to access sensitive data systems? & \no \\
    Does your organization have an employee acceptable use policy? & \no \\
    Does your organization do security awareness training for new employees? & \no \\
    Does your organization do security awareness training for all employees at least once per year? & \yes \\
    \bottomrule
\end{tabular}
\end{center}

\subsection*{Analysis of Control Gaps}
The questionnaire reveals critical deficiencies in identity and access management and foundational security governance.
\begin{itemize}
    \item \textbf{Lack of MFA:} The absence of MFA for email, computer logins, and sensitive systems is a \sevcritical{critical} vulnerability. A single compromised password could grant an attacker widespread access.
    \item \textbf{Policy and Training Gaps:} The lack of an Acceptable Use Policy (AUP) and security training for new hires are \sevhigh{high-risk} findings. Without an AUP, employees lack clear guidelines on security responsibilities. New employees are often prime targets for social engineering, and the absence of onboarding training leaves them unprepared.
\end{itemize}

% 6. TECHNICAL SCAN RESULTS (FROM NMAP)
\section{Technical Scan Results}
An external network scan was performed to identify exposed services and potential vulnerabilities on the public-facing infrastructure.

\subsection*{Nmap Scan of Target: \seqsplit{\texttt{[Target IP]}}}
The scan revealed the following open port, indicating a service accessible from the public internet.

\begin{center}
\begin{tabular}{l l l l}
    \toprule
    \textbf{Port} & \textbf{State} & \textbf{Service} & \textbf{Analysis} \\
    \midrule
    22/tcp & Open & SSH & Secure Shell (SSH) is a remote management protocol. \\
    & & & Its exposure to the entire internet is a significant risk. \\
    & & & It is a primary target for brute-force and credential stuffing attacks. \\
    \bottomrule
\end{tabular}
\end{center}
\textbf{Note:} No service version information was available from the scan. The absence of this information does not diminish the risk associated with exposing a management protocol.

% 7. RISK ASSESSMENT SUMMARY
\section{Risk Assessment Summary}
This section correlates the findings from the security control review and the technical scan into a prioritized list of identified risks. No pre-existing vulnerabilities were reported.

\begin{center}
\begin{tabular}{p{0.1\linewidth} p{0.25\linewidth} p{0.45\linewidth} p{0.1\linewidth}}
    \toprule
    \textbf{ID} & \textbf{Risk Name} & \textbf{Description} & \textbf{Severity} \\
    \midrule
    RISK-001 & Widespread Lack of Multi-Factor Authentication & The absence of MFA on all critical systems allows for account takeover with only a compromised password. This is the most critical risk identified. & \sevcritical{Critical} \\
    \addlinespace
    RISK-002 & Inadequate Security Policies and Training & The lack of an Acceptable Use Policy and new-hire security training creates an uninformed user base, increasing susceptibility to phishing and insider threats. & \sevhigh{High} \\
    \addlinespace
    RISK-003 & Exposed SSH Management Service & The SSH service on \seqsplit{\texttt{[Target IP]}} is open to the public internet, creating a direct vector for attackers to attempt unauthorized access to internal systems. & \sevmedium{Medium} \\
    \bottomrule
\end{tabular}
\end{center}
\textbf{Note:} The severity of RISK-003 is elevated to \sevhigh{High} when considered in conjunction with RISK-001. An attacker with stolen credentials could gain immediate network access.

% 8. RECOMMENDATIONS
\section{Recommendations}
The following actions are recommended to mitigate the identified risks and improve the overall security posture of \textbf{[Organization Name]}.

\subsection*{Immediate Actions (0-30 Days)}
\begin{enumerate}
    \item \textbf{Implement MFA (Mitigates RISK-001):}
    \begin{itemize}
        \item Immediately enable and enforce MFA for all users on the primary email system.
        \item Prioritize the rollout of MFA for remote access (VPN) and all systems containing sensitive data.
    \end{itemize}
    
    \item \textbf{Secure Exposed SSH Service (Mitigates RISK-003):}
    \begin{itemize}
        \item Implement a firewall rule to restrict access to the SSH port (22/tcp) on \seqsplit{\texttt{[Target IP]}} to only trusted, whitelisted IP addresses.
        \item If remote access is required, mandate the use of a VPN with MFA.
        \item Internally, enforce the use of public key authentication for SSH and disable password-based logins.
    \end{itemize}
\end{enumerate}

\subsection*{Strategic Actions (30-90 Days)}
\begin{enumerate}
    \setcounter{enumi}{2} % Continue numbering
    \item \textbf{Develop and Implement Security Policies (Mitigates RISK-002):}
    \begin{itemize}
        \item Draft, approve, and communicate an organization-wide Acceptable Use Policy (AUP). This policy should cover password complexity, data handling, and acceptable use of company assets.
    \end{itemize}
    
    \item \textbf{Establish Security Training Program (Mitigates RISK-002):}
    \begin{itemize}
        \item Develop a security awareness training module and make it a mandatory part of the onboarding process for all new employees.
        \item Continue the annual security awareness training for all staff to reinforce best practices.
    \end{itemize}
\end{enumerate}

\end{document}
```