```latex
\documentclass[12pt]{article}

% ----------------------------------------------------------------------
% PREAMBLE
% ----------------------------------------------------------------------
\usepackage[margin=1in]{geometry}
\usepackage{pifont} % Required for \ding
\usepackage{booktabs} % For professional-looking tables
\usepackage{hyperref} % For clickable links and references
\usepackage{url}      % For formatting URLs
\usepackage{seqsplit} % For splitting long strings without spaces
\usepackage{graphicx}
\usepackage{xcolor}
\usepackage{array}

% --- Color and Hyperlink Setup ---
\definecolor{darkblue}{rgb}{0.0, 0.0, 0.55}
\hypersetup{
    colorlinks=true,
    linkcolor=darkblue,
    filecolor=darkblue,
    urlcolor=darkblue,
    citecolor=darkblue,
}

% --- Custom Commands ---
\newcommand{\yes}{\textcolor{green!80!black}{\ding{51}}}
\newcommand{\no}{\textcolor{red}{\ding{55}}}
\newcommand{\reporttitle}{Cybersecurity Posture Assessment Report}
\newcommand{\reportauthor}{Cybersecurity Analysis Division}

% ----------------------------------------------------------------------
% DOCUMENT START
% ----------------------------------------------------------------------
\begin{document}

\title{\reporttitle}
\author{\reportauthor}
\date{\today}
\maketitle

\hrule\vspace{1em}

% ----------------------------------------------------------------------
% EXECUTIVE SUMMARY
% ----------------------------------------------------------------------
\section*{Executive Summary}

This report details the findings of a cybersecurity posture assessment for \textbf{[Organization Name]}. The analysis combines a review of organizational security controls, an external network scan, and a summary of pre-existing risks.

The assessment identified a \textbf{critical-risk vulnerability}: the direct exposure of Remote Desktop Protocol (RDP) on port 3389 to the public internet. This configuration is a primary target for ransomware gangs and other malicious actors. This technical finding is correlated with a known risk and represents an immediate and severe threat to the organization's operational integrity.

Furthermore, significant gaps were identified in foundational security controls. The lack of mandatory Multi-Factor Authentication (MFA) for email access is a critical weakness that severely heightens the risk of account compromise and subsequent data breaches. Policy-level deficiencies, including the absence of an employee Acceptable Use Policy and security training for new hires, indicate a need for strengthening the overall security culture.

Immediate remediation of the exposed RDP service and enforcement of MFA on email are the highest priorities. Strategic recommendations are provided to address the identified policy and training gaps to build a more resilient long-term security posture.

% ----------------------------------------------------------------------
% ORGANIZATIONAL INFORMATION
% ----------------------------------------------------------------------
\section{Organizational Information}

The following details were used as the basis for this assessment. As per the provided data, placeholder values are used where specific information was not available.

\begin{center}
\begin{tabular}{@{} >{\bfseries}l l}
\toprule
Attribute & Value \\
\midrule
Organization Name & \textbf{[Organization Name]} \\
Primary Email Domain & \texttt{[Domain]} \\
External IP Scanned & \texttt{[Client IP]} \\
Assessment Date & \today \\
\bottomrule
\end{tabular}
\end{center}

% ----------------------------------------------------------------------
% SECURITY CONTROL REVIEW
% ----------------------------------------------------------------------
\section{Security Control Review}

A review of the organization's security controls was conducted based on a questionnaire. The responses highlight critical areas for improvement, particularly concerning user access and security policies. "No" answers indicate a gap in controls and are marked with \no.

\begin{center}
\begin{tabular}{p{0.75\textwidth} c}
\toprule
\textbf{Control Question} & \textbf{Status} \\
\midrule
Do you require MFA to access email? & \no \\
Do you require MFA to log into computers? & \yes \\
Do you require MFA to access sensitive data systems? & \yes \\
Does your organization have an employee acceptable use policy? & \no \\
Does your organization do security awareness training for new employees? & \no \\
Does your organization do security awareness training for all employees at least once per year? & \yes \\
\bottomrule
\end{tabular}
\end{center}

\subsection*{Analysis of Control Gaps}
\begin{itemize}
    \item \textbf{No MFA for Email:} This is a critical vulnerability. Email accounts are high-value targets for attackers seeking to gain an initial foothold, conduct phishing campaigns, or access sensitive data.
    \item \textbf{No Acceptable Use Policy (AUP):} An AUP is a foundational document that sets clear expectations for employees regarding the use of company assets. Its absence can lead to inconsistent security practices and insider threats.
    \item \textbf{No Security Training for New Employees:} Failing to train new hires on security best practices from the start leaves the organization vulnerable, as new staff may be unaware of policies and common threats.
\end{itemize}

% ----------------------------------------------------------------------
% TECHNICAL SCAN RESULTS
% ----------------------------------------------------------------------
\section{Technical Scan Results}

An external network scan was performed against the target IP address \texttt{[Target IP]}. The scan identified the following open port, which confirms the pre-existing risk documented in internal records.

\begin{center}
\begin{tabular}{lllll}
\toprule
\textbf{Port} & \textbf{State} & \textbf{Service Name} & \textbf{Product/Version} & \textbf{Notes} \\
\midrule
3389/tcp & open & ms-wbt-server & N/A & \textbf{Critical Risk}: RDP Exposure \\
\bottomrule
\end{tabular}
\end{center}

\subsection*{Analysis of Technical Findings}
The discovery of an open TCP port 3389 indicates that Microsoft Remote Desktop Protocol (RDP) is directly accessible from the internet. RDP is a primary vector for ransomware attacks, where attackers brute-force credentials or use stolen passwords to gain complete control over the exposed system. This finding requires immediate attention.

% ----------------------------------------------------------------------
% CONSOLIDATED RISK ASSESSMENT
% ----------------------------------------------------------------------
\section{Consolidated Risk Assessment}

This section synthesizes findings from the security control review, technical scan, and pre-existing risk data into a prioritized list.

\begin{center}
\begin{tabular}{p{0.25\textwidth} p{0.55\textwidth} l}
\toprule
\textbf{Risk / Vulnerability} & \textbf{Description} & \textbf{Severity} \\
\midrule
\textbf{Public RDP Exposure} & Port 3389 (RDP) is open to the internet, allowing attackers to attempt brute-force or credential-stuffing attacks to gain remote control of a server. & \textbf{Critical} \\
\addlinespace
\textbf{No MFA for Email} & The absence of MFA on email accounts makes them highly susceptible to compromise via phishing or password spraying, leading to potential data breaches. & \textbf{Critical} \\
\addlinespace
\textbf{Missing Acceptable Use Policy} & No formal policy exists to govern the use of IT assets, creating ambiguity and increasing the risk of misuse or accidental data exposure by employees. & \textbf{High} \\
\addlinespace
\textbf{Inadequate Onboarding Training} & New employees do not receive mandatory security awareness training, leaving them unprepared to identify and respond to threats like phishing. & \textbf{High} \\
\bottomrule
\end{tabular}
\end{center}

% ----------------------------------------------------------------------
% RECOMMENDATIONS
% ----------------------------------------------------------------------
\section{Recommendations}

The following actions are recommended to mitigate the identified risks and improve the overall security posture of \textbf{[Organization Name]}.

\subsection*{Immediate Actions (Priority 1)}
\begin{itemize}
    \item \textbf{Remediate RDP Exposure:} Immediately close port 3389 on the external firewall for IP \texttt{[Target IP]}. If remote access is required, implement a secure Virtual Private Network (VPN) with MFA. Direct RDP access from the internet should be strictly prohibited.
    \item \textbf{Enforce MFA for Email:} Mandate and enforce the use of MFA for all user email accounts without exception. This single control significantly reduces the risk of account takeovers.
\end{itemize}

\subsection*{Strategic Improvements (Priority 2)}
\begin{itemize}
    \item \textbf{Develop and Implement an Acceptable Use Policy (AUP):} Create a formal AUP that clearly defines the rules and responsibilities for all employees when using company technology and data. Ensure all employees read and acknowledge the policy.
    \item \textbf{Establish Onboarding Security Training:} Develop a mandatory security awareness training module for all new hires. This program should be part of the standard onboarding process and cover topics such as phishing, password hygiene, and the new AUP.
\end{itemize}

\end{document}
% ----------------------------------------------------------------------
% DOCUMENT END
% ----------------------------------------------------------------------
```