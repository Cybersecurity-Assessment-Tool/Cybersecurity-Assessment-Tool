```latex
\documentclass[12pt]{article}

% 1. Document Setup and Packages
\usepackage[margin=1in]{geometry}
\usepackage{pifont} % For checkmarks and crosses
\usepackage{booktabs} % For professional tables
\usepackage{hyperref} % For hyperlinks
\usepackage{url} % For URL formatting
\usepackage{seqsplit} % To split long text sequences
\usepackage{graphicx}
\usepackage{xcolor}

% --- Document Metadata ---
\title{Cybersecurity Posture Assessment Report}
\author{Cybersecurity Analyst}
\date{\today}

% --- Hyperref Setup ---
\hypersetup{
    colorlinks=true,
    linkcolor=blue,
    filecolor=magenta,      
    urlcolor=cyan,
    pdftitle={Cybersecurity Posture Assessment Report},
    pdfpagemode=FullScreen,
}

\begin{document}

\maketitle
\thispagestyle{empty}
\newpage

\tableofcontents
\newpage

% 2. Executive Overview
\section{Executive Overview}
This report details the findings of a cybersecurity posture assessment conducted for \textbf{[Organization Name]}. The analysis combines a review of organizational security controls, an external network scan, and a summary of identified risks.

The assessment revealed critical gaps in fundamental security controls. The lack of Multi-Factor Authentication (MFA) for computer and sensitive data system access represents a significant and immediate risk to the organization. Furthermore, the absence of a formal security awareness training program for employees leaves the organization highly susceptible to social engineering and phishing attacks.

On a positive note, the external network scan of the provided IP address, \texttt{[Client IP]}, did not identify any open ports or exposed services. This suggests a potentially well-configured perimeter firewall.

Overall, while the external network perimeter appears hardened, critical deficiencies in internal access controls and employee security training present a high level of risk. Immediate remediation of these identified gaps is strongly recommended to improve the organization's security posture.

% 3. Organizational Information
\section{Organizational Information}
The following details were used as the basis for this assessment. Due to missing data in the provided inputs, placeholders have been used.

\begin{itemize}
    \item \textbf{Organization Name:} \textbf{[Organization Name]}
    \item \textbf{Primary Domain:} \texttt{[Domain]}
    \item \textbf{External IP Assessed:} \texttt{[Client IP]}
\end{itemize}

% 4. Security Control Review (Questionnaire)
\section{Security Control Review}
A review of the organization's security controls was conducted via a standardized questionnaire. The responses highlight significant areas for improvement, particularly concerning identity and access management and security awareness.

\begin{table}[h!]
\centering
\caption{Security Controls Questionnaire Results}
\begin{tabular}{p{0.6\linewidth} c l}
\toprule
\textbf{Control Question} & \textbf{Response} & \textbf{Assessment} \\
\midrule
Do you require MFA to access email? & \ding{51} & Compliant \\
\addlinespace
Do you require MFA to log into computers? & \ding{55} & \textbf{Critical Gap} \\
\addlinespace
Do you require MFA to access sensitive data systems? & \ding{55} & \textbf{Critical Gap} \\
\addlinespace
Does your organization have an employee acceptable use policy? & \ding{51} & Compliant \\
\addlinespace
Does your organization do security awareness training for new employees? & \ding{55} & \textbf{High Risk} \\
\addlinespace
Does your organization do security awareness training for all employees at least once per year? & \ding{55} & \textbf{High Risk} \\
\bottomrule
\end{tabular}
\end{table}

% 5. Technical Scan Results
\section{Technical Scan Results}
An external network vulnerability scan was performed to identify exposed services and potential vulnerabilities on the organization's public-facing infrastructure.

\begin{itemize}
    \item \textbf{Target IP Address:} \texttt{[Target IP]}
    \item \textbf{Scan Date:} \today
\end{itemize}

\subsection{Summary of Findings}
\textbf{No open ports or services were identified on the target system.}

This result indicates that the host was not responsive to the scan probes, which typically suggests a well-configured firewall is in place that drops or rejects unsolicited traffic. While this is a positive finding from an external perspective, it should be internally verified to ensure this is the intended configuration and not the result of a network issue or the host being offline during the scan.

% 6. Consolidated Risk Assessment
\section{Consolidated Risk Assessment}
This section synthesizes findings from the security control review and the technical scan. No pre-existing risks were provided for this assessment. The following new risks have been identified based on the data collected.

\begin{table}[h!]
\centering
\caption{Identified Risks}
\begin{tabular}{p{0.25\linewidth} p{0.5\linewidth} l}
\toprule
\textbf{Risk Name} & \textbf{Overview} & \textbf{Severity} \\
\midrule
\addlinespace
Lack of MFA on Endpoints & The absence of MFA for computer logins significantly increases the risk of unauthorized access from compromised credentials. An attacker with a valid password can gain direct access to an endpoint and the corporate network. & \textbf{Critical} \\
\addlinespace
Lack of MFA on Sensitive Systems & Sensitive data systems are not protected by MFA, making them prime targets for data exfiltration. A single compromised password could lead to a major data breach. & \textbf{Critical} \\
\addlinespace
No Security Awareness Training Program & The organization lacks a formal security training program for both new and existing employees. This makes staff highly vulnerable to phishing, social engineering, and other human-targeted attacks, turning them into an unintentional insider threat. & \textbf{High} \\
\bottomrule
\end{tabular}
\end{table}

% 7. Recommendations
\section{Recommendations}
Based on the findings of this assessment, the following actions are recommended to mitigate the identified risks and strengthen the overall security posture of \textbf{[Organization Name]}.

\begin{enumerate}
    \item \textbf{Implement Comprehensive MFA (Critical):}
    \begin{itemize}
        \item \textbf{Immediate Priority:} Deploy MFA on all systems containing sensitive or critical data.
        \item \textbf{Secondary Priority:} Enforce MFA for all employee computer/endpoint logins.
        \item This single control will drastically reduce the risk of unauthorized access resulting from stolen or weak passwords.
    \end{itemize}
    \item \textbf{Establish a Security Awareness Training Program (High):}
    \begin{itemize}
        \item Develop and implement a mandatory security awareness training module for all new employees as part of their onboarding process.
        \item Institute a mandatory annual security refresher course for all staff to keep them informed about evolving threats, such as new phishing techniques.
        \item Conduct periodic, unannounced phishing simulations to test and reinforce employee awareness.
    \end{itemize}
    \item \textbf{Internal Network Verification (Informational):}
    \begin{itemize}
        \item We recommend that your IT or security team internally confirms that the external IP \texttt{[Target IP]} is correctly configured to block all inbound traffic. This will validate the external scan results and ensure the perimeter defense is working as intended.
    \end{itemize}
\end{enumerate}

\end{document}
```