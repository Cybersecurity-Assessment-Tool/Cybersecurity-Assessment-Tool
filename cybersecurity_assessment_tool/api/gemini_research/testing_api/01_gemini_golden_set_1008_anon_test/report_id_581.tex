```latex
\documentclass[12pt, a4paper]{article}

% Preamble: Required Packages
\usepackage[margin=1in]{geometry}
\usepackage{pifont} % For checkmarks and crosses
\usepackage{booktabs} % For professional tables
\usepackage{hyperref} % For clickable links and ToC
\usepackage{url} % For formatting URLs
\usepackage{seqsplit} % To split long strings in tt font
\usepackage{xcolor} % For colors
\usepackage{graphicx} % For logo
\usepackage{fancyhdr} % For header/footer

% --- Document Setup ---

% Color definitions for risk levels
\definecolor{critical}{HTML}{940000}
\definecolor{high}{HTML}{D63300}
\definecolor{medium}{HTML}{FFBF00}
\definecolor{low}{HTML}{006400}

% Hyperref setup
\hypersetup{
    colorlinks=true,
    linkcolor=blue,
    filecolor=magenta,      
    urlcolor=cyan,
    pdftitle={Cybersecurity Assessment Report},
    pdfauthor={Cybersecurity Analyst},
    pdfsubject={Security Assessment},
    pdfkeywords={Security, Report, LaTeX},
    bookmarks=true
}

% Header and Footer
\pagestyle{fancy}
\fancyhf{} % clear all header and footer fields
\fancyhead[L]{\textbf{Cybersecurity Assessment Report}}
\fancyhead[R]{\textbf{[Organization Name]}}
\fancyfoot[C]{\thepage}
\renewcommand{\headrulewidth}{0.4pt}
\renewcommand{\footrulewidth}{0.4pt}

% --- Document Start ---
\begin{document}

% --- Title Page ---
\begin{titlepage}
    \centering
    \vspace*{1cm}
    
    \Huge
    \textbf{Cybersecurity Assessment Report}
    
    \vspace{1.5cm}
    
    \Large
    Prepared for: \\
    \vspace{0.5cm}
    \textbf{[Organization Name]}
    
    \vspace{2cm}
    
    \large
    Date of Report: \today
    
    \vfill
    
    \large
    \textbf{CONFIDENTIAL} \\
    \small
    This document contains sensitive information. Distribution is restricted to authorized personnel only.
    
\end{titlepage}

\tableofcontents
\newpage

% --- Section 1: Executive Summary ---
\section{Executive Summary}

This report details the findings of a cybersecurity assessment conducted for \textbf{[Organization Name]}. The assessment combined an external network scan, a review of existing risk documentation, and an analysis of organizational security controls provided via a questionnaire.

The assessment identified a \textbf{critical risk} related to the direct exposure of Remote Desktop Protocol (RDP) on the public internet at \seqsplit{\texttt{[Client IP]}}. This configuration is a primary target for ransomware gangs and other malicious actors who use it for initial access into corporate networks.

This critical technical vulnerability is further compounded by a significant gap in internal security controls: the absence of Multi-Factor Authentication (MFA) for computer logins. This combination creates a high-impact scenario where a single compromised password could lead to an attacker gaining full remote control of an internal system, bypassing other security measures.

While the organization has implemented several positive security controls, such as MFA for email and a security awareness training program, the identified critical findings require immediate attention to mitigate the substantial risk of a security breach. This report provides prioritized, actionable recommendations to address these vulnerabilities.

% --- Section 2: Organizational Information ---
\section{Organizational Information}

The following information was used as the basis for this assessment. Where data was not provided, placeholders have been used.

\begin{table}[h!]
\centering
\begin{tabular}{@{}ll@{}}
\toprule
\textbf{Attribute} & \textbf{Value} \\ \midrule
Organization Name & \textbf{[Organization Name]} \\
Primary Domain & \seqsplit{\texttt{[Domain]}} \\
External IP Scanned & \seqsplit{\texttt{[Client IP]}} \\ \bottomrule
\end{tabular}
\caption{Client Organizational Data}
\label{tab:org_data}
\end{table}

% --- Section 3: Security Control Review ---
\section{Security Control Review}

An analysis of the organization's security practices was conducted based on the provided questionnaire. The results are summarized below. A checkmark (\ding{51}) indicates a positive control is in place, while a cross (\ding{55}) indicates a potential security gap.

\begin{table}[h!]
\centering
\begin{tabular}{@{}lc@{}}
\toprule
\textbf{Security Control Question} & \textbf{Status} \\ \midrule
Do you require MFA to access email? & \ding{51} \\
\textbf{Do you require MFA to log into computers?} & \textbf{\color{red}\ding{55}} \\
Do you require MFA to access sensitive data systems? & \ding{51} \\
Does your organization have an employee acceptable use policy? & \ding{51} \\
Does your organization do security awareness training for new employees? & \ding{51} \\
Does your organization do security awareness training for all employees annually? & \ding{51} \\ \bottomrule
\end{tabular}
\caption{Security Controls Questionnaire Results}
\label{tab:controls}
\end{table}

\subsection*{Analysis of Findings}
The review shows that the organization has a solid foundation in several key areas, including policy and security awareness. However, the lack of MFA for computer logins is a critical weakness. This gap significantly increases the risk of unauthorized access from compromised credentials, nullifying many of the benefits of other security controls. An attacker who obtains a valid username and password can log in to a corporate machine without any secondary challenge.

% --- Section 4: Technical Scan Results ---
\section{Technical Scan Results}

An external network scan was performed on the target IP address to identify open ports and exposed services.

\subsection*{Scan Details}
\begin{itemize}
    \item \textbf{Target IP:} \seqsplit{\texttt{[Target IP]}}
    \item \textbf{Scan Date:} Scan date not provided in source data.
\end{itemize}

\subsection*{Open Ports Discovered}
The following table details the ports found to be open and accessible from the public internet.

\begin{table}[h!]
\centering
\begin{tabular}{@{}llll@{}}
\toprule
\textbf{Port} & \textbf{State} & \textbf{Service Name} & \textbf{Description} \\ \midrule
3389/tcp & open & ms-wbt-server & Microsoft Remote Desktop Protocol (RDP) \\ \bottomrule
\end{tabular}
\caption{Open Port Findings}
\label{tab:nmap}
\end{table}

\subsection*{Analysis of Findings}
The scan confirms that TCP port 3389 is open, exposing the Remote Desktop Protocol (RDP) service directly to the internet. RDP is a common management protocol but is notoriously targeted by attackers for brute-force password attacks, credential stuffing, and exploitation of known vulnerabilities (e.g., BlueKeep). Exposing RDP is considered a highly dangerous practice and is a leading cause of ransomware incidents. This finding validates the pre-existing risk documented in Input 3.

% --- Section 5: Correlated Risk Assessment ---
\section{Correlated Risk Assessment}

This section synthesizes the findings from the security control review, the technical scan, and pre-existing risk data to provide a holistic view of the primary risks facing the organization.

\begin{table}[h!]
\centering
\begin{tabular}{@{}p{0.25\linewidth}p{0.1\linewidth}p{0.55\linewidth}@{}}
\toprule
\textbf{Risk Title} & \textbf{Severity} & \textbf{Description} \\ \midrule
\textbf{Publicly Exposed RDP Service} & \textcolor{critical}{\textbf{Critical}} & The technical scan confirmed that RDP (port 3389) is open on \seqsplit{\texttt{[Target IP]}}. This provides a direct vector for attackers to attempt unauthorized access to the internal network. This is a common tactic used in ransomware attacks. \\
\addlinespace
\textbf{Lack of Endpoint Multi-Factor Authentication} & \textcolor{high}{\textbf{High}} & The questionnaire revealed that MFA is not required for computer logins. This weakness is critically amplified by the exposed RDP service, as a single stolen or brute-forced password would be sufficient for an attacker to gain full remote access to a corporate device. \\
\bottomrule
\end{tabular}
\caption{Summary of Key Risks}
\label{tab:risks}
\end{table}

% --- Section 6: Recommendations ---
\section{Recommendations}

The following recommendations are prioritized to address the most critical risks identified during the assessment.

\subsection*{Priority 1: Remediate RDP Exposure (Immediate)}
The exposed RDP service presents an immediate and severe threat and must be addressed urgently.

\begin{itemize}
    \item \textbf{Immediate Action:} Implement a firewall rule to \textbf{block all inbound traffic} to TCP port 3389 on the external interface of your firewall for IP \seqsplit{\texttt{[Client IP]}}.
    \item \textbf{Long-Term Solution:} If remote access is a business requirement, deploy a secure Virtual Private Network (VPN) solution. Configure the VPN to require MFA for access. All RDP connections should only be permitted through the secure VPN tunnel, never directly from the internet.
\end{itemize}

\subsection*{Priority 2: Implement Endpoint MFA (High)}
Addressing the lack of MFA on endpoints is crucial for a defense-in-depth strategy and to protect against credential-based attacks.

\begin{itemize}
    \item \textbf{Action:} Deploy a mandatory MFA solution for all user and administrator logins to both workstations and servers. Solutions like Windows Hello for Business, Duo Security, or other third-party tools can fulfill this requirement. This ensures that even if an attacker obtains a password, they cannot access the system without the second factor.
\end{itemize}

\subsection*{Positive Acknowledgement}
We commend \textbf{[Organization Name]} for implementing important security controls, including MFA for email and sensitive systems, and for maintaining a consistent security awareness training program. These measures are vital for a strong security posture. By addressing the critical recommendations in this report, the organization can significantly reduce its risk profile.

\end{document}
```