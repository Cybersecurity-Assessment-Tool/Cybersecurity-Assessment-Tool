```latex
\documentclass[12pt]{article}

% ----------------------------------------------------------------------
% PREAMBLE
% ----------------------------------------------------------------------

% --- Package Imports ---
\usepackage[margin=1in]{geometry} % Set page margins
\usepackage{pifont}               % For checkmark and X symbols (\ding)
\usepackage{booktabs}             % For professional-looking tables
\usepackage{hyperref}             % For hyperlinks and document metadata
\usepackage{url}                  % For formatting URLs
\usepackage{seqsplit}             % For splitting long strings in text
\usepackage{graphicx}             % For including images
\usepackage{xcolor}               % For custom colors
\usepackage{fancyhdr}             % For custom headers and footers
\usepackage{lastpage}             % To get the total number of pages

% --- Color Definitions ---
\definecolor{darkblue}{rgb}{0.0, 0.0, 0.55}
\definecolor{darkred}{rgb}{0.55, 0.0, 0.0}
\definecolor{darkgreen}{rgb}{0.0, 0.39, 0.0}

% --- Hyperref Setup ---
\hypersetup{
    colorlinks=true,
    linkcolor=darkblue,
    filecolor=darkblue,      
    urlcolor=darkblue,
    citecolor=darkblue,
    pdftitle={Cybersecurity Posture Assessment Report},
    pdfauthor={Cybersecurity Analyst},
    pdfsubject={Security Assessment},
    pdfkeywords={Security, Assessment, Report},
    bookmarks=true,
    bookmarksopen=true
}

% --- Header and Footer Setup ---
\pagestyle{fancy}
\fancyhf{} % Clear all header and footer fields
\fancyhead[L]{Cybersecurity Posture Assessment}
\fancyhead[R]{\textbf{[Organization Name]}}
\fancyfoot[C]{Page \thepage\ of \pageref{LastPage}}
\renewcommand{\headrulewidth}{0.4pt}
\renewcommand{\footrulewidth}{0.4pt}

% --- Custom Commands ---
\newcommand{\yes}{\textcolor{darkgreen}{\ding{51}}}
\newcommand{\no}{\textcolor{darkred}{\ding{55}}}

% ----------------------------------------------------------------------
% DOCUMENT START
% ----------------------------------------------------------------------
\begin{document}

% --- Title Page ---
\begin{titlepage}
    \centering
    \vspace*{2cm}
    
    {\Huge \textbf{Cybersecurity Posture Assessment Report}\par}
    \vspace{1.5cm}
    
    {\Large \textbf{Prepared For:}}
    \vspace{0.5cm}
    
    {\Huge \textbf{[Organization Name]}}
    \vspace{2cm}
    
    {\Large \textbf{Date of Report:}}
    \vspace{0.5cm}
    
    {\large \today\par}
    
    \vfill
    
    \textit{This report contains sensitive information. Distribution should be limited to authorized personnel only.}
    
\end{titlepage}

\tableofcontents
\newpage

% ----------------------------------------------------------------------
% 1. EXECUTIVE OVERVIEW
% ----------------------------------------------------------------------
\section{Executive Overview}

This report details the findings of a cybersecurity posture assessment conducted for \textbf{[Organization Name]}. The assessment combined an analysis of organizational security controls via a questionnaire, a technical network scan of external-facing assets, and a review of previously identified risks.

The overall security posture is determined to be at a \textbf{high-risk level}, requiring immediate attention. Several critical vulnerabilities and policy gaps were identified that expose the organization to significant threats, including unauthorized data access, malware infection, and system compromise.

\textbf{Key Critical Findings Include:}
\begin{itemize}
    \item \textbf{Exposed Vulnerable FTP Server:} A publicly accessible FTP server was discovered running a dangerously outdated version of \texttt{vsftpd} (2.3.4), which is known to contain a critical backdoor vulnerability (CVE-2011-2523). The server is also misconfigured to allow anonymous logins, presenting a direct and immediate path for an attacker to compromise the system.
    \item \textbf{Lack of Endpoint Multi-Factor Authentication (MFA):} The organization does not require MFA for logging into employee computers. This significantly increases the risk of unauthorized access from stolen or weak credentials.
    \item \textbf{Absence of Acceptable Use Policy:} The lack of a formal Acceptable Use Policy (AUP) represents a critical governance gap. Without an AUP, there are no established rules for employees regarding the use of company assets, which can lead to unintentional security incidents.
\end{itemize}

This report provides a detailed breakdown of these findings and offers prioritized, actionable recommendations to mitigate the identified risks and strengthen the organization's overall security posture.

\newpage

% ----------------------------------------------------------------------
% 2. ORGANIZATIONAL INFORMATION
% ----------------------------------------------------------------------
\section{Organizational Information}

This section provides the baseline information used for this assessment. The data was gathered from provided organizational details.

\begin{table}[h!]
\centering
\begin{tabular}{@{}ll@{}}
\toprule
\textbf{Attribute} & \textbf{Value} \\ \midrule
Organization Name & \textbf{[Organization Name]} \\
Primary Email Domain & \texttt{[Domain]} \\
External IP Address Scanned & \texttt{[Client IP]} \\
Target of Network Scan & \texttt{[Target IP]} \\
\bottomrule
\end{tabular}
\caption{Organizational and Assessment Scope Details.}
\end{table}

% ----------------------------------------------------------------------
% 3. SECURITY CONTROL REVIEW
% ----------------------------------------------------------------------
\section{Security Control Review (Questionnaire Analysis)}

The following table summarizes the organization's responses to a security controls questionnaire. The "Assessment" column highlights areas where current practices deviate from established security best practices, indicating a control gap.

\begin{table}[h!]
\centering
\begin{tabular}{@{}p{0.6\linewidth} c p{0.2\linewidth}@{}}
\toprule
\textbf{Control Question} & \textbf{Response} & \textbf{Assessment} \\ \midrule
Do you require MFA to access email? & \yes & Compliant \\
\addlinespace
Do you require MFA to log into computers? & \no & \textbf{Critical Gap} \\
\addlinespace
Do you require MFA to access sensitive data systems? & \yes & Compliant \\
\addlinespace
Does your organization have an employee acceptable use policy? & \no & \textbf{Critical Gap} \\
\addlinespace
Does your organization do security awareness training for new employees? & \yes & Compliant \\
\addlinespace
Does your organization do security awareness training for all employees at least once per year? & \yes & Compliant \\
\bottomrule
\end{tabular}
\caption{Analysis of Security Control Questionnaire.}
\end{table}

\subsection*{Analysis of Gaps}
\begin{itemize}
    \item \textbf{No MFA on Computers:} The absence of MFA on workstations is a significant weakness. If an employee's password is compromised, an attacker can gain direct access to their computer and potentially the internal network.
    \item \textbf{No Acceptable Use Policy (AUP):} An AUP is a foundational policy that defines how employees may use company IT assets. Without one, the organization lacks a formal mechanism to enforce security standards and manage user-related risk.
\end{itemize}

\newpage

% ----------------------------------------------------------------------
% 4. TECHNICAL SCAN RESULTS
% ----------------------------------------------------------------------
\section{Technical Scan Results}

An Nmap scan was performed on the target host \texttt{[Target IP]} to identify open ports and exposed services. The results are detailed below.

\begin{table}[h!]
\centering
\begin{tabular}{@{}llllll@{}}
\toprule
\textbf{Port} & \textbf{State} & \textbf{Service} & \textbf{Product} & \textbf{Version} & \textbf{Notes} \\ \midrule
21/tcp & Open & ftp & vsftpd & 2.3.4 & \parbox{4cm}{\textbf{CRITICAL FINDING:}\\ Anonymous FTP login allowed. Version 2.3.4 is vulnerable to a backdoor (CVE-2011-2523).} \\
\bottomrule
\end{tabular}
\caption{Open Ports and Services on \texttt{[Target IP]}.}
\end{table}

\subsection*{Analysis of Technical Findings}
The scan revealed a single open port (21/tcp) running a File Transfer Protocol (FTP) service. The specific software, \texttt{vsftpd} version 2.3.4, is critically outdated (released in 2011) and contains a well-known, high-severity backdoor vulnerability. This vulnerability allows an unauthenticated attacker to execute arbitrary commands on the server, leading to a full system compromise.

The misconfiguration allowing anonymous FTP login exacerbates this risk, as it permits any attacker on the internet to connect to the vulnerable service without credentials. This finding represents an immediate and severe threat to the organization.

% ----------------------------------------------------------------------
% 5. CONSOLIDATED RISK ASSESSMENT
% ----------------------------------------------------------------------
\section{Consolidated Risk Assessment}

This section synthesizes findings from the technical scan, questionnaire analysis, and pre-existing risk data into a consolidated list of security risks.

\begin{table}[h!]
\centering
\begin{tabular}{@{}lp{0.55\linewidth}ll@{}}
\toprule
\textbf{Risk ID} & \textbf{Risk Description} & \textbf{Source} & \textbf{Severity} \\ \midrule
RISK-001 & A public-facing FTP server allows anonymous login and runs a version with a known remote code execution backdoor. & Technical Scan & \textbf{Critical} \\
\addlinespace
RISK-002 & Lack of Multi-Factor Authentication (MFA) on employee workstations allows for credential-based takeovers. & Questionnaire & \textbf{High} \\
\addlinespace
RISK-003 & Absence of an Acceptable Use Policy creates ambiguity in security responsibilities and acceptable employee behavior. & Questionnaire & \textbf{High} \\
\addlinespace
RISK-004 & Workstations are running an outdated and unsupported operating system (Windows 7), which no longer receives security updates. & Pre-existing Risk & \textbf{Medium} \\
\bottomrule
\end{tabular}
\caption{Summary of Identified Security Risks.}
\end{table}

\newpage

% ----------------------------------------------------------------------
% 6. RECOMMENDATIONS
% ----------------------------------------------------------------------
\section{Recommendations}
The following actionable recommendations are provided to address the identified risks. They are prioritized based on severity and potential impact.

\subsection*{Immediate Priority (Address within 72 hours)}
\begin{enumerate}
    \item \textbf{Remediate Vulnerable FTP Server (RISK-001):}
    \begin{itemize}
        \item \textbf{Option A (Preferred):} If the FTP service is not essential for business operations, \textbf{disable and uninstall the service immediately}.
        \item \textbf{Option B:} If the service is required, take the following steps:
            \begin{enumerate}
                \item Immediately disable anonymous FTP access.
                \item Upgrade the \texttt{vsftpd} software to the latest stable version.
                \item Place the server behind a firewall and implement strict Access Control Lists (ACLs) to only allow connections from trusted IP addresses.
            \end{enumerate}
    \end{itemize}
\end{enumerate}

\subsection*{High Priority (Address within 30 days)}
\begin{enumerate}
    \setcounter{enumi}{1}
    \item \textbf{Implement MFA on Endpoints (RISK-002):}
    \begin{itemize}
        \item Procure and deploy an MFA solution for all employee computer logins.
        \item Prioritize deployment for administrative accounts and users with access to sensitive data.
    \end{itemize}
    \item \textbf{Develop and Implement an Acceptable Use Policy (RISK-003):}
    \begin{itemize}
        \item Draft a formal AUP that clearly defines rules for using company technology, data handling, and internet usage.
        \item Require all current and new employees to read and formally acknowledge the policy.
    \end{itemize}
\end{enumerate}

\subsection*{Medium Priority (Address within 90 days)}
\begin{enumerate}
    \setcounter{enumi}{3}
    \item \textbf{Upgrade Outdated Operating Systems (RISK-004):}
    \begin{itemize}
        \item Continue and prioritize the project to upgrade all workstations from Windows 7 to a modern, supported operating system (e.g., Windows 10/11).
        \item Isolate any remaining Windows 7 machines from the main network until they can be upgraded.
    \end{itemize}
\end{enumerate}

% ----------------------------------------------------------------------
% DOCUMENT END
% ----------------------------------------------------------------------
\end{document}
```