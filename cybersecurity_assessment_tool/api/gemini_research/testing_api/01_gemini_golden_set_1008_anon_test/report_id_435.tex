```latex
\documentclass[12pt]{article}

% --- PACKAGES ---
\usepackage[margin=1in]{geometry}
\usepackage{pifont} % For checkmarks and crosses (\ding{51} and \ding{55})
\usepackage{booktabs} % For professional tables
\usepackage{hyperref} % For links and document metadata
\usepackage{url} % For formatting URLs
\usepackage{seqsplit} % For splitting long strings in \texttt
\usepackage{graphicx}
\usepackage{xcolor}
\usepackage{fancyhdr}
\usepackage{array}

% --- DOCUMENT & HYPERLINK SETUP ---
\hypersetup{
    colorlinks=true,
    linkcolor=black,
    urlcolor=blue,
    pdftitle={Cybersecurity Posture Assessment Report},
    pdfauthor={Cybersecurity Analysis Division},
    pdfsubject={Security Assessment}
}

% --- HEADER & FOOTER ---
\pagestyle{fancy}
\fancyhf{} % Clear all header and footer fields
\lhead{Cybersecurity Assessment Report}
\rhead{\textbf{[Organization Name]}}
\cfoot{Page \thepage}
\renewcommand{\headrulewidth}{0.4pt}
\renewcommand{\footrulewidth}{0.4pt}

% --- CUSTOM COMMANDS & DEFINITIONS ---
\definecolor{darkred}{rgb}{0.55, 0.0, 0.0}
\definecolor{darkorange}{rgb}{0.8, 0.3, 0.0}
\definecolor{darkgreen}{rgb}{0.0, 0.3, 0.0}
\newcommand{\yes}{\textcolor{darkgreen}{\ding{51}}}
\newcommand{\no}{\textcolor{darkred}{\ding{55}}}
\newcolumntype{L}[1]{>{\raggedright\let\newline\\\arraybackslash\hspace{0pt}}m{#1}}
\newcolumntype{C}[1]{>{\centering\let\newline\\\arraybackslash\hspace{0pt}}m{#1}}

% --- DOCUMENT START ---
\begin{document}

\title{
    \vspace{-1.5cm}
    \includegraphics[width=0.3\textwidth]{example-image-a} \\ % Placeholder for a logo
    \vspace{0.5cm}
    \textbf{Cybersecurity Posture Assessment Report} \\
    \large For \textbf{[Organization Name]}
}
\author{Cybersecurity Analysis Division}
\date{\today}
\maketitle
\thispagestyle{empty}

\newpage

% --- TABLE OF CONTENTS ---
\tableofcontents
\newpage

% --- EXECUTIVE SUMMARY ---
\section{Executive Summary}

This report details the findings of a cybersecurity posture assessment conducted for \textbf{[Organization Name]}. The assessment combined an external network scan, a review of existing risks, and an analysis of organizational security controls based on a questionnaire.

The overall security posture is considered to be at a \textbf{high-risk level}. Several critical and high-severity issues were identified that require immediate attention to mitigate the risk of unauthorized access, data breach, and system compromise.

\paragraph{Key Findings:}
\begin{itemize}
    \item \textbf{Critical - Publicly Exposed Database Service:} A MySQL database server on port 3306 was found to be directly accessible from the internet at \texttt{[Target IP]}. This configuration exposes the database to brute-force attacks and exploitation attempts.
    \item \textbf{Critical - End-of-Life Software:} The exposed MySQL service is running version 5.7.33. The entire MySQL 5.7 branch reached its official End of Life (EOL) in October 2023 and no longer receives security patches, leaving it vulnerable to known exploits.
    \item \textbf{High - Gaps in Access Control:} Multi-Factor Authentication (MFA) is not required for employee computer logins, significantly weakening a primary defense against credential theft and unauthorized access to internal resources.
    \item \textbf{High - Deficiencies in Security Governance:} The organization lacks a formal employee Acceptable Use Policy (AUP) and does not provide security awareness training for new employees. These foundational controls are essential for establishing a security-conscious culture and minimizing human error.
\end{itemize}

Urgent remediation of these findings is strongly recommended. This report provides specific, actionable recommendations to address each identified risk.

% --- ORGANIZATIONAL INFORMATION ---
\section{Organizational Information}

This section provides the context for the assessment based on the information provided.
\begin{table}[h!]
    \centering
    \begin{tabular}{@{}ll@{}}
        \toprule
        \textbf{Attribute} & \textbf{Value} \\
        \midrule
        Organization Name & \textbf{[Organization Name]} \\
        Primary Email Domain & \texttt{[Domain]} \\
        External IP Address Scanned & \texttt{[Target IP]} \\
        \bottomrule
    \end{tabular}
    \caption{Client Organizational Data}
\end{table}

% --- SECURITY CONTROL REVIEW ---
\section{Security Control Review}

The following table summarizes the organization's current security controls based on the provided questionnaire. "No" answers indicate significant control gaps that increase organizational risk.

\begin{table}[h!]
    \centering
    \begin{tabular}{@{}L{0.5\textwidth} C{0.15\textwidth} L{0.25\textwidth}@{}}
        \toprule
        \textbf{Control Question} & \textbf{Response} & \textbf{Assessment} \\
        \midrule
        Do you require MFA to access email? & \yes & Best Practice Met \\
        \addlinespace
        Do you require MFA to log into computers? & \no & \textbf{Critical Gap Identified} \\
        \addlinespace
        Do you require MFA to access sensitive data systems? & \yes & Best Practice Met \\
        \addlinespace
        Does your organization have an employee acceptable use policy? & \no & \textbf{High-Risk Gap} \\
        \addlinespace
        Does your organization do security awareness training for new employees? & \no & \textbf{High-Risk Gap} \\
        \addlinespace
        Does your organization do security awareness training for all employees at least once per year? & \yes & Best Practice Met \\
        \bottomrule
    \end{tabular}
    \caption{Security Controls Questionnaire Analysis}
\end{table}

% --- TECHNICAL SCAN RESULTS ---
\section{Technical Scan Results}

An external network scan was performed on the target IP address \texttt{[Target IP]}. The scan identified the following open ports and services.

\begin{table}[h!]
    \centering
    \begin{tabular}{@{}lllll@{}}
        \toprule
        \textbf{Port} & \textbf{State} & \textbf{Service} & \textbf{Product \& Version} & \textbf{Finding} \\
        \midrule
        3306 & Open & mysql & MySQL 5.7.33 & \begin{tabular}[t]{@{}l@{}}\textbf{Critical Finding:} Publicly exposed database.\\ \textbf{Critical Finding:} Version is End-of-Life.\end{tabular} \\
        \bottomrule
    \end{tabular}
    \caption{Open Ports Detected on \texttt{[Target IP]}}
\end{table}

\paragraph{Analysis of Findings:} The exposure of a MySQL database (port 3306) is a severe security risk. Furthermore, the detected version, 5.7.33, is part of a product line that is no longer supported by its vendor. This means no new security patches will be released, even for newly discovered critical vulnerabilities, making this system an easy target for attackers.

% --- RISK ASSESSMENT SUMMARY ---
\section{Risk Assessment Summary}

The following table synthesizes findings from the security control review, technical scan, and pre-existing risk data into a consolidated risk register.

\begin{table}[h!]
    \centering
    \begin{tabular}{@{}L{0.15\textwidth} L{0.25\textwidth} L{0.4\textwidth} C{0.1\textwidth}@{}}
        \toprule
        \textbf{Risk ID} & \textbf{Risk Name} & \textbf{Description} & \textbf{Severity} \\
        \midrule
        RISK-001 & End-of-Life Software Exposure & The publicly accessible MySQL service is running version 5.7.33, which is End-of-Life and unpatched against new vulnerabilities. & \textbf{Critical} \\
        \addlinespace
        RISK-002 & Public Database Exposure & The MySQL database port (3306) is open to the internet, exposing it to reconnaissance, brute-force attacks, and direct exploitation. & High \\
        \addlinespace
        RISK-003 & Lack of Endpoint MFA & The absence of MFA for computer logins creates a significant risk of unauthorized access via compromised credentials. & High \\
        \addlinespace
        RISK-004 & Missing Foundational Policies & The lack of an Acceptable Use Policy and security training for new hires indicates a weakness in security governance and culture. & Medium \\
        \bottomrule
    \end{tabular}
    \caption{Consolidated Risk Register}
\end{table}

% --- RECOMMENDATIONS ---
\section{Recommendations}

The following actions are recommended to mitigate the identified risks. Recommendations are prioritized based on severity and potential impact.

\subsection{RISK-001 \& RISK-002: Database Exposure and EOL Software}
\begin{itemize}
    \item \textbf{Immediate Action (Containment):} Implement strict firewall rules to deny all public access to TCP port 3306 on \texttt{[Target IP]}. Access should only be permitted from specific, trusted IP addresses.
    \item \textbf{Short-Term (Remediation):} Initiate a project to upgrade the MySQL database from the EOL 5.7 version to a fully supported version (e.g., MySQL 8.x). This is critical for receiving future security patches.
    \item \textbf{Long-Term (Architectural Improvement):} Re-architect the network so that the database server is not on a public IP address. Access should be granted only through a secure baseline, such as a VPN or an authenticated bastion host.
\end{itemize}

\subsection{RISK-003: Lack of Endpoint MFA}
\begin{itemize}
    \item \textbf{Immediate Action (Planning):} Evaluate and select an MFA solution compatible with your organization's computer operating systems (e.g., Windows Hello for Business, Duo, Okta).
    \item \textbf{Short-Term (Implementation):} Begin a phased rollout of MFA for all computer logins, starting with IT administrators and executive leadership, then expanding to all employees.
\end{itemize}

\subsection{RISK-004: Missing Foundational Policies}
\begin{itemize}
    \item \textbf{Immediate Action (Policy Development):} Draft and ratify a formal Acceptable Use Policy (AUP) that defines the rules for using company IT assets. Require all current employees to read and acknowledge the policy.
    \item \textbf{Short-Term (Training Implementation):} Develop or procure a security awareness training module and integrate it into the onboarding process for all new employees. This ensures a baseline of security knowledge from day one.
\end{itemize}

\end{document}
```