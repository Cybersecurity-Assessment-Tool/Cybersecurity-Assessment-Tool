```latex
\documentclass[12pt]{article}

% --- PACKAGES ---
\usepackage[margin=1in]{geometry}
\usepackage{pifont} % Required for \ding{51} (checkmark) and \ding{55} (cross)
\usepackage{booktabs} % For professional-looking tables
\usepackage{hyperref} % For clickable links and references
\usepackage{url}      % For formatting URLs
\usepackage{seqsplit} % To split long strings without breaking words
\usepackage{xcolor}   % For custom colors
\usepackage{fancyhdr} % For headers and footers
\usepackage{graphicx} % To include images (e.g., a logo)

% --- DOCUMENT SETUP ---

% Define colors for severity levels
\definecolor{sevcritical}{HTML}{990000}
\definecolor{sevhigh}{HTML}{D9534F}
\definecolor{sevmedium}{HTML}{F0AD4E}

% Setup Hyperref
\hypersetup{
    colorlinks=true,
    linkcolor=blue,
    filecolor=magenta,      
    urlcolor=cyan,
    pdftitle={Cybersecurity Posture Assessment Report},
    pdfpagemode=FullScreen,
}

% Setup Headers & Footers
\pagestyle{fancy}
\fancyhf{} % Clear all header and footer fields
\fancyhead[L]{\textbf{Cybersecurity Posture Assessment}}
\fancyhead[R]{\textbf{[Organization Name]}}
\fancyfoot[C]{\thepage}
\renewcommand{\headrulewidth}{0.4pt}
\renewcommand{\footrulewidth}{0.4pt}

% --- DOCUMENT START ---
\begin{document}

\begin{titlepage}
    \centering
    \vspace*{1cm}
    \Huge{\textbf{Cybersecurity Posture Assessment Report}}
    \vspace{1.5cm}
    \large{Prepared for:} \\
    \vspace{0.5cm}
    \huge{\textbf{[Organization Name]}}
    \vspace{2cm}
    \large{Date of Report: \today} \\
    \vspace{1cm}
    \large{Report ID: CSA-2024-001}
    \vfill
    \large{\textit{This document contains sensitive information and should be handled with care.}}
\end{titlepage}

\tableofcontents
\newpage

% --- EXECUTIVE SUMMARY ---
\section{Executive Summary}

This report provides a comprehensive assessment of the cybersecurity posture for \textbf{[Organization Name]}, based on an analysis of network scan data, organizational security controls, and pre-existing risk information.

The overall security posture is determined to be at a \textbf{critical risk level}. The assessment identified several severe vulnerabilities that require immediate attention. Key findings include:

\begin{itemize}
    \item \textbf{Publicly Exposed Database:} A MySQL database (Port 3306) is directly exposed to the internet. The running version, MySQL 5.7.33, is \textbf{End-of-Life (EOL)} as of October 2023 and no longer receives security updates, making it an easy target for known exploits.
    \item \textbf{Systemic Lack of Multi-Factor Authentication (MFA):} MFA is not enforced for accessing email or for computer logins. This represents a critical control gap, significantly increasing the risk of account compromise and unauthorized access.
    \item \textbf{Insufficient Security Awareness:} The organization lacks a formal security awareness training program for new or existing employees. This deficiency makes the organization highly susceptible to social engineering and phishing attacks.
\end{itemize}

These findings, when correlated, paint a picture of an environment vulnerable to data breaches, ransomware, and unauthorized system access. Immediate and decisive action is required to mitigate these risks. This report outlines specific, prioritized recommendations to address these deficiencies.

% --- ORGANIZATIONAL INFORMATION ---
\section{Organizational Information}

The following details were used as the basis for this assessment. As the provided data was anonymized, placeholders are used.

\begin{itemize}
    \item \textbf{Organization Name:} \textbf{[Organization Name]}
    \item \textbf{Primary Email Domain:} \texttt{[Domain]}
    \item \textbf{Client External IP:} \texttt{[Client IP]}
\end{itemize}

% --- SECURITY CONTROL REVIEW ---
\section{Security Control Review}

A review of the organization's security controls was conducted via a questionnaire. The responses reveal significant gaps in foundational security practices. "No" answers indicate a failure to meet baseline security standards and are flagged as high-impact risks.

\begin{table}[h!]
\centering
\caption{Security Controls Questionnaire Analysis}
\label{tab:controls}
\begin{tabular}{p{0.6\linewidth} c l}
\toprule
\textbf{Control Question} & \textbf{Response} & \textbf{Status / Analysis} \\
\midrule
Do you require MFA to access email? & \ding{55} & \textcolor{sevcritical}{\textbf{Critical Gap}} \\
Do you require MFA to log into computers? & \ding{55} & \textcolor{sevcritical}{\textbf{Critical Gap}} \\
Do you require MFA to access sensitive data systems? & \ding{51} & Implemented \\
Does your organization have an employee acceptable use policy? & \ding{51} & Implemented \\
Does your organization do security awareness training for new employees? & \ding{55} & \textcolor{sevhigh}{\textbf{High Risk}} \\
Does your organization do security awareness training for all employees at least once per year? & \ding{55} & \textcolor{sevhigh}{\textbf{High Risk}} \\
\bottomrule
\end{tabular}
\end{table}

% --- TECHNICAL SCAN RESULTS ---
\section{Technical Scan Results}

An external network scan was performed to identify exposed services. The scan targeted the perimeter IP address provided by the client.

\begin{itemize}
    \item \textbf{Target IP Address:} \texttt{[Target IP]}
    \item \textbf{Scan Status:} Host is UP.
\end{itemize}

The scan identified one open port with a publicly accessible service, as detailed in Table \ref{tab:scanresults}.

\begin{table}[h!]
\centering
\caption{Open Ports and Services Detected}
\label{tab:scanresults}
\begin{tabular}{l l l l}
\toprule
\textbf{Port} & \textbf{State} & \textbf{Service} & \textbf{Product \& Version} \\
\midrule
3306/tcp & open & mysql & MySQL 5.7.33 \\
\bottomrule
\end{tabular}
\end{table}

\subsection{Analysis of Technical Findings}
The primary finding is the exposed MySQL database on port 3306. This configuration is highly discouraged as it allows attackers worldwide to attempt to connect to and compromise the database.

More critically, the detected version, \textbf{MySQL 5.7.33}, is \textbf{End-of-Life (EOL)}. EOL software no longer receives security patches from the vendor. This means that any vulnerabilities discovered after its EOL date (October 2023) will remain unpatched, leaving the system perpetually vulnerable to exploitation. This finding confirms and elevates the severity of the pre-existing risk "Database Exposure".

% --- RISK ASSESSMENT ---
\section{Risk Assessment Summary}

The following table synthesizes findings from the security questionnaire, the technical scan, and the pre-existing risk register. Risks are prioritized based on their potential impact on the organization.

\begin{table}[h!]
\centering
\caption{Consolidated Risk Register}
\label{tab:risks}
\begin{tabular}{p{0.05\linewidth} p{0.4\linewidth} p{0.15\linewidth} p{0.3\linewidth}}
\toprule
\textbf{ID} & \textbf{Risk Description} & \textbf{Severity} & \textbf{Source / Correlation} \\
\midrule
R-01 & Publicly exposed and End-of-Life (EOL) MySQL database allows for remote exploitation and data breach. & \textcolor{sevcritical}{\textbf{Critical}} & Network Scan (Input 1), Risk Register (Input 3), EOL Analysis \\
\addlinespace
R-02 & Lack of MFA on email and computer logins enables account takeover and unauthorized access if credentials are stolen. & \textcolor{sevcritical}{\textbf{Critical}} & Questionnaire (Input 2) \\
\addlinespace
R-03 & Absence of security awareness training increases susceptibility to phishing and social engineering, which could lead to credential theft. & \textcolor{sevhigh}{\textbf{High}} & Questionnaire (Input 2) \\
\bottomrule
\end{tabular}
\end{table}

% --- RECOMMENDATIONS ---
\section{Prioritized Recommendations}

The following actionable recommendations are provided to mitigate the identified risks. They are prioritized based on urgency and impact.

\begin{enumerate}
    \item \textbf{Immediately Remediate Database Exposure (Risk R-01):}
    \begin{itemize}
        \item \textbf{Short-Term Fix:} Implement strict firewall rules to block all public access to TCP port 3306. Access should only be permitted from specific, trusted IP addresses.
        \item \textbf{Long-Term Fix:} Place the database in a private network segment, accessible only via a secure VPN or bastion host. Public exposure of a database is never a recommended practice.
    \end{itemize}
    \vspace{0.5cm}
    \item \textbf{Urgently Upgrade End-of-Life Software (Risk R-01):}
    \begin{itemize}
        \item Plan and execute an upgrade of the MySQL 5.7.33 database to a currently supported version (e.g., MySQL 8.x). This is critical to ensure the system receives ongoing security patches.
    \end{itemize}
    \vspace{0.5cm}
    \item \textbf{Implement Mandatory MFA (Risk R-02):}
    \begin{itemize}
        \item Enforce MFA across all user accounts for email access (e.g., Office 365, Google Workspace) and VPN/remote access solutions.
        \item Deploy MFA for all computer logins, especially for administrative accounts.
    \end{itemize}
    \vspace{0.5cm}
    \item \textbf{Establish a Security Awareness Program (Risk R-03):}
    \begin{itemize}
        \item Procure or develop a security awareness training module and make it mandatory for all new hires during their onboarding process.
        \item Implement a recurring, annual training program for all employees to keep them informed about the latest threats, such as phishing and ransomware.
    \end{itemize}
\end{enumerate}

\end{document}
```