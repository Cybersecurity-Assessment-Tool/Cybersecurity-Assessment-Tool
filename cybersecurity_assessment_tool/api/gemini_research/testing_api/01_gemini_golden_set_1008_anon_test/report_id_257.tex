```latex
\documentclass[12pt]{article}

% Preamble: Required Packages
\usepackage[margin=1in]{geometry}
\usepackage{pifont} % For \ding
\usepackage{booktabs} % For professional tables (\toprule, \midrule, \bottomrule)
\usepackage[hidelinks]{hyperref}
\usepackage{url}
\usepackage{seqsplit} % For splitting long text strings without breaking
\usepackage{graphicx}
\usepackage{xcolor}
\usepackage{array}

% Define custom colors
\definecolor{severitycritical}{HTML}{940000}
\definecolor{severityhigh}{HTML}{D14000}
\definecolor{severitymedium}{HTML}{EFA900}
\definecolor{severitylow}{HTML}{3E8026}

% Document Metadata
\title{Cybersecurity Posture Assessment Report}
\author{Cybersecurity Analysis Division}
\date{\today}

\begin{document}

\maketitle
\thispagestyle{empty}
\newpage

\tableofcontents
\newpage

% --- Section 1: Executive Summary ---
\section{Executive Summary}
This report provides a cybersecurity assessment for \textbf{[Organization Name]}, based on an analysis of network scan data, organizational security controls, and pre-existing risk information. The assessment identified a critical-risk vulnerability that requires immediate attention.

An external network scan confirmed that Remote Desktop Protocol (RDP) on port 3389 is directly exposed to the internet on a key asset. This finding corroborates a known risk and carries a CVSS score of 9.0 (Critical). This exposure creates a significant risk of unauthorized access, brute-force attacks, and ransomware deployment.

The risk is further exacerbated by two significant gaps identified in the organization's security controls:
\begin{itemize}
    \item \textbf{Lack of Endpoint Multi-Factor Authentication (MFA):} Computers are not protected by MFA, meaning a compromised password is all an attacker needs to gain access.
    \item \textbf{Inadequate New Hire Training:} New employees do not receive security awareness training, making them highly susceptible to phishing and social engineering attacks aimed at stealing credentials.
\end{itemize}

While the organization has implemented some positive security controls, such as MFA for email, the identified vulnerabilities create a direct and severe threat. Immediate remediation of the RDP exposure and implementation of the high-priority recommendations in this report are crucial to mitigating these risks.

% --- Section 2: Organizational Information ---
\section{Organizational Information}
This section provides the key identifying information for the organization under review. As the provided data was anonymized, placeholders are used.

\begin{tabular}{@{}ll}
    \toprule
    \textbf{Attribute} & \textbf{Value} \\
    \midrule
    Organization Name & \textbf{[Organization Name]} \\
    Primary Email Domain & \texttt{[Domain]} \\
    Scanned External IP & \texttt{[Client IP]} \\
    \bottomrule
\end{tabular}

% --- Section 3: Security Control Review ---
\section{Security Control Review}
The following table summarizes the organization's responses to a security controls questionnaire. Items marked with a cross (\textcolor{red}{\ding{55}}) indicate a deviation from security best practices and represent a gap in the defensive posture.

\begin{table}[h!]
\centering
\begin{tabular}{>{\raggedright\arraybackslash}p{0.7\linewidth} c}
    \toprule
    \textbf{Control Question} & \textbf{Response} \\
    \midrule
    Do you require MFA to access email? & \ding{51} \\
    Do you require MFA to log into computers? & \textcolor{red}{\ding{55}} \\
    Do you require MFA to access sensitive data systems? & \ding{51} \\
    Does your organization have an employee acceptable use policy? & \ding{51} \\
    Does your organization do security awareness training for new employees? & \textcolor{red}{\ding{55}} \\
    Does your organization do security awareness training for all employees at least once per year? & \ding{51} \\
    \bottomrule
\end{tabular}
\caption{Security Controls Questionnaire Results}
\end{table}

% --- Section 4: Technical Scan Results ---
\section{Technical Scan Results}
An external network vulnerability scan was conducted to identify open ports and exposed services.

\subsection{Nmap Scan Findings}
The scan was performed against the primary external IP address provided by the client.
\begin{itemize}
    \item \textbf{Target IP:} \texttt{[Target IP]}
    \item \textbf{Host Status:} Up
\end{itemize}

The following table details the open ports discovered on the target system.

\begin{table}[h!]
\centering
\begin{tabular}{l l l}
    \toprule
    \textbf{Port} & \textbf{State} & \textbf{Service} \\
    \midrule
    3389/tcp & open & ms-wbt-server (Microsoft Remote Desktop Protocol) \\
    \bottomrule
\end{tabular}
\caption{Open Ports Detected on \texttt{[Target IP]}}
\end{table}

\paragraph{Analysis:} The discovery of an open RDP port is a critical finding. Exposing RDP directly to the internet is a common vector for ransomware attacks and unauthorized access. This technical finding validates the pre-existing risk documented in Input 3.

% --- Section 5: Correlated Risk Assessment ---
\section{Correlated Risk Assessment}
This section synthesizes the findings from the security control review, technical scans, and pre-existing risk data into a prioritized list of security risks.

\begin{table}[h!]
\centering
\begin{tabular}{p{0.25\linewidth} p{0.15\linewidth} p{0.5\linewidth}}
    \toprule
    \textbf{Risk Title} & \textbf{Severity} & \textbf{Description \& Business Impact} \\
    \midrule
    \textbf{Critical RDP Exposure} & \textcolor{severitycritical}{\textbf{Critical (9.0)}} & Port 3389 (RDP) is open to the internet, confirmed by network scans. This poses a severe threat of unauthorized remote access, brute-force credential attacks, and ransomware deployment. A compromise could lead to a full network breach and major business disruption. \\
    \addlinespace
    \textbf{Lack of Endpoint Multi-Factor Authentication} & \textcolor{severityhigh}{\textbf{High}} & The absence of MFA for computer logins means that a single compromised password provides an attacker with direct access to an endpoint. This weakness critically magnifies the risk posed by the exposed RDP service. \\
    \addlinespace
    \textbf{Inadequate Security Awareness Training} & \textcolor{severityhigh}{\textbf{High}} & New employees do not receive security awareness training, creating a significant vulnerability to phishing and social engineering. This increases the likelihood of credential theft, which attackers could use to exploit the exposed RDP service. \\
    \bottomrule
\end{tabular}
\caption{Summary of Identified Risks}
\end{table}

% --- Section 6: Recommendations ---
\section{Recommendations}
The following actionable recommendations are provided to mitigate the identified risks. They are prioritized based on severity and ease of implementation.

\begin{enumerate}
    \item \textbf{Remediate RDP Exposure (Immediate Priority)}
    \begin{itemize}
        \item \textbf{Action:} Immediately configure the perimeter firewall to \textbf{block all inbound traffic to TCP port 3389} for the asset at \texttt{[Target IP]}.
        \item \textbf{Long-Term Solution:} If remote access is required, implement a Virtual Private Network (VPN) with strong authentication (MFA). All RDP access should occur exclusively through the secure VPN tunnel.
    \end{itemize}
    \vspace{1em}

    \item \textbf{Implement Endpoint MFA (High Priority)}
    \begin{itemize}
        \item \textbf{Action:} Deploy a robust Multi-Factor Authentication solution for all employee computer and server logins. This is a critical compensating control that protects against credential theft and unauthorized access, even if a password is compromised.
    \end{itemize}
    \vspace{1em}
    
    \item \textbf{Enhance Security Awareness Program (High Priority)}
    \begin{itemize}
        \item \textbf{Action:} Integrate mandatory security awareness training into the new employee onboarding process. This training must cover, at a minimum: phishing identification, password security, and the organization's acceptable use policy.
    \end{itemize}
\end{enumerate}

\end{document}
```