```latex
\documentclass[12pt]{article}

% --- PACKAGES ---
\usepackage[margin=1in]{geometry}
\usepackage{pifont} % For checkmarks and crosses
\usepackage{booktabs} % For professional tables
\usepackage{hyperref} % For hyperlinks
\usepackage{url} % For URL formatting
\usepackage{seqsplit} % To split long strings in texttt
\usepackage[T1]{fontenc}

% --- DOCUMENT METADATA ---
\title{Cybersecurity Posture Assessment Report}
\author{Cybersecurity Analysis Division}
\date{\today}

% --- DOCUMENT START ---
\begin{document}

\maketitle
\thispagestyle{empty}
\newpage
\tableofcontents
\newpage

% ==============================================================================
\section{Executive Overview}
% ==============================================================================
This report provides a comprehensive cybersecurity assessment for \textbf{[Organization Name]}. The analysis is based on a correlation of external network scan data, a review of internal security controls via a questionnaire, and an evaluation of previously identified risks.

The assessment reveals several critical and high-risk vulnerabilities that require immediate attention. Key findings include a publicly accessible FTP server with a known critical vulnerability (CVE-2011-2523) and anonymous login enabled. Furthermore, significant gaps in access control policies were identified, most notably the lack of Multi-Factor Authentication (MFA) for computer and sensitive data system access. While some foundational controls are in place, such as an acceptable use policy and annual security training, the identified weaknesses create a significant risk of unauthorized access, data exfiltration, and system compromise.

Prioritized recommendations are provided in Section \ref{sec:recommendations} to mitigate these risks and strengthen the organization's overall security posture.

% ==============================================================================
\section{Organizational Information}
% ==============================================================================
The following information was used as the basis for this assessment. Due to the anonymized nature of the provided data, placeholders have been used where necessary.

\begin{itemize}
    \item \textbf{Organization Name:} \textbf{[Organization Name]}
    \item \textbf{Primary Domain:} \texttt{[Domain]}
    \item \textbf{External IP Scanned:} \texttt{[Client IP]}
    \item \textbf{Target IP Scanned:} \texttt{[Target IP]}
\end{itemize}

% ==============================================================================
\section{Security Control Review}
% ==============================================================================
The following table summarizes the organization's responses to a security controls questionnaire. Items marked with a cross (\ding{55}) represent significant gaps in the security framework and are discussed in the Risk Assessment section.

\begin{table}[h!]
\centering
\caption{Security Controls Questionnaire Results}
\label{tab:controls}
\begin{tabular}{@{}p{0.8\linewidth}c@{}}
\toprule
\textbf{Control Question} & \textbf{Status} \\
\midrule
Do you require MFA to access email? & \ding{51} \\
Do you require MFA to log into computers? & \color{red}\ding{55} \\
Do you require MFA to access sensitive data systems? & \color{red}\ding{55} \\
Does your organization have an employee acceptable use policy? & \ding{51} \\
Does your organization do security awareness training for new employees? & \color{red}\ding{55} \\
Does your organization do security awareness training for all employees at least once per year? & \ding{51} \\
\bottomrule
\end{tabular}
\end{table}

\subsection{Analysis of Gaps}
The review identified three primary control gaps:
\begin{enumerate}
    \item \textbf{Lack of MFA for Computer Logins:} This exposes workstations to compromise via stolen or weak credentials, potentially leading to lateral movement within the network.
    \item \textbf{Lack of MFA for Sensitive Data Systems:} This is a critical deficiency. It significantly increases the risk of a data breach, as access to the organization's most valuable data is protected only by a single factor (a password).
    \item \textbf{No Security Training for New Employees:} New hires are often targeted by social engineering attacks. Without immediate training upon joining, they represent a high-risk group unaware of corporate security policies and common threats.
\end{enumerate}

% ==============================================================================
\section{Technical Scan Results}
% ==============================================================================
An external network scan was performed on the target IP address \texttt{[Target IP]}. The scan identified the following open port and associated service.

\begin{table}[h!]
\centering
\caption{Open Ports and Services}
\label{tab:nmap}
\begin{tabular}{@{}lllll@{}}
\toprule
\textbf{Port} & \textbf{State} & \textbf{Service} & \textbf{Product / Version} & \textbf{Notes} \\
\midrule
21/tcp & Open & ftp & vsftpd 2.3.4 & Anonymous FTP login allowed \\
\bottomrule
\end{tabular}
\end{table}

\subsection{Technical Findings Analysis}
The technical scan revealed two critical vulnerabilities associated with the open FTP port:
\begin{enumerate}
    \item \textbf{Anonymous FTP Login Allowed:} This configuration permits any user on the internet to connect to the FTP server without authentication. This could allow attackers to exfiltrate sensitive data stored on the server or use it as a drop point for malicious files.
    \item \textbf{Vulnerable Service Version (vsftpd 2.3.4):} This specific version of vsftpd is associated with a critical backdoor vulnerability (\textbf{CVE-2011-2523}). If a malicious actor connects to the server with a username ending in `:)`, a command shell is opened on port 6200, granting the attacker remote command execution on the server.
\end{enumerate}
The combination of these two findings presents an extreme risk to the organization.

% ==============================================================================
\section{Consolidated Risk Assessment}
% ==============================================================================
The following table synthesizes findings from the security control review, technical scan, and pre-existing risk data into a consolidated list of identified risks.

\begin{table}[h!]
\centering
\caption{Summary of Identified Risks}
\label{tab:risks}
\begin{tabular}{@{}p{0.3\linewidth}p{0.5\linewidth}l@{}}
\toprule
\textbf{Risk Name} & \textbf{Overview} & \textbf{Severity} \\
\midrule
\textbf{Vulnerable FTP Service} & The vsftpd 2.3.4 service has a known remote code execution backdoor (CVE-2011-2523). & \textbf{Critical} \\
\addlinespace
\textbf{Anonymous FTP Access} & The FTP server allows unauthenticated access, enabling potential data exfiltration or malware staging. & \textbf{Critical} \\
\addlinespace
\textbf{No MFA for Sensitive Systems} & Lack of MFA on critical systems means a single password compromise could lead to a major data breach. & \textbf{Critical} \\
\addlinespace
\textbf{No MFA for Computer Logins} & User workstations are protected only by passwords, increasing the risk of unauthorized access and lateral movement. & High \\
\addlinespace
\textbf{No Onboarding Security Training} & New employees are not trained on security policies upon hiring, making them susceptible to social engineering. & High \\
\addlinespace
\textbf{Outdated Windows Policy} & Workstations are running Windows 7, which is End-of-Life and no longer receives security updates. & Medium \\
\bottomrule
\end{tabular}
\end{table}

% ==============================================================================
\section{Recommendations}
\label{sec:recommendations}
% ==============================================================================
Based on the risk assessment, the following actions are recommended to improve the security posture of \textbf{[Organization Name]}. They are prioritized by severity.

\subsection{Immediate Actions (Critical Risks)}
\begin{enumerate}
    \item \textbf{Remediate FTP Server Vulnerabilities:}
        \begin{itemize}
            \item \textbf{Immediately} disable the vsftpd service on the public-facing server.
            \item If FTP is a business requirement, migrate to a secure file transfer protocol (like SFTP) and deploy a modern, patched server version.
            \item If FTP must be used, upgrade vsftpd to the latest stable version and disable anonymous access immediately.
        \end{itemize}
    \item \textbf{Implement MFA for Sensitive Systems:}
        \begin{itemize}
            \item Prioritize and enforce MFA on all systems identified as containing sensitive or critical data. This should be treated as an emergency change.
        \end{itemize}
\end{enumerate}

\subsection{High-Priority Actions}
\begin{enumerate}
    \item \textbf{Enforce MFA for All Computer Logins:}
        \begin{itemize}
            \item Roll out MFA for all employee workstation and laptop logins, including remote access solutions (e.g., VPN).
        \end{itemize}
    \item \textbf{Implement Onboarding Security Training:}
        \begin{itemize}
            \item Develop a mandatory security awareness training module that is a required part of the new employee onboarding process.
        \end{itemize}
\end{enumerate}

\subsection{Medium-Priority Actions}
\begin{enumerate}
    \item \textbf{Upgrade End-of-Life Operating Systems:}
        \begin{itemize}
            \item Develop and execute a plan to upgrade all workstations from Windows 7 to a currently supported operating system (e.g., Windows 10/11) to ensure they receive security patches.
        \end{itemize}
\end{enumerate}

\end{document}
```