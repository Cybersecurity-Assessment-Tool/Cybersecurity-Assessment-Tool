```latex
\documentclass[12pt]{article}

% --- PACKAGES ---
\usepackage[margin=1in]{geometry}
\usepackage{pifont} % For checkmarks and crosses
\usepackage{booktabs} % For professional tables
\usepackage{hyperref} % For clickable links
\usepackage{url} % For formatting URLs
\usepackage{seqsplit} % For splitting long strings in tt font
\usepackage{xcolor} % For colors

% --- DOCUMENT METADATA ---
\title{Cybersecurity Posture Assessment Report}
\author{Cybersecurity Analysis Division}
\date{\today}

% --- HYPERREF SETUP ---
\hypersetup{
    colorlinks=true,
    linkcolor=blue,
    filecolor=magenta,      
    urlcolor=cyan,
    pdftitle={Cybersecurity Posture Assessment Report},
    pdfpagemode=FullScreen,
}

% --- DOCUMENT START ---
\begin{document}

\maketitle
\thispagestyle{empty}
\newpage

\tableofcontents
\newpage

% --- EXECUTIVE SUMMARY ---
\section{Executive Summary}
This report provides a comprehensive cybersecurity assessment for \textbf{[Organization Name]}, based on an analysis of network scan data, organizational security controls, and pre-existing risk information. The assessment synthesizes these data points to provide a holistic view of the organization's current security posture.

The analysis identified several areas of significant concern requiring immediate attention. A critical pre-existing vulnerability, "Localhost Exposed," with a CVSS score of 10.0, represents an extreme and immediate threat. Furthermore, a critical gap was identified in endpoint security: the absence of Multi-Factor Authentication (MFA) for computer logins. This weakness, combined with an externally exposed SSH service on host \texttt{[Target IP]}, creates a viable attack path for threat actors who manage to compromise user credentials.

The findings in this report are categorized by severity to assist in prioritizing remediation efforts. We strongly recommend that \textbf{[Organization Name]} address the critical and high-risk items outlined in the Recommendations section as a matter of urgency to mitigate potential security breaches.

% --- ORGANIZATIONAL INFORMATION ---
\section{Organizational Information}
The following details were used as the basis for this assessment. Due to the anonymized nature of the provided data, placeholders have been used where necessary.

\begin{itemize}
    \item \textbf{Organization Name:} \textbf{[Organization Name]}
    \item \textbf{Primary Domain:} \texttt{[Domain]}
    \item \textbf{External IP Address Scanned:} \texttt{[Client IP]}
\end{itemize}

% --- SECURITY CONTROL REVIEW ---
\section{Security Control Review}
A review of the organization's self-reported security controls was conducted via a questionnaire. The responses are summarized below. A checkmark (\ding{51}) indicates a positive control is in place, while a cross (\ding{55}) indicates a control gap.

\begin{table}[h!]
\centering
\caption{Security Controls Questionnaire Results}
\begin{tabular}{p{0.7\textwidth}c}
\toprule
\textbf{Control Question} & \textbf{Response} \\
\midrule
Do you require MFA to access email? & \ding{51} \\
\textbf{Do you require MFA to log into computers?} & \textcolor{red}{\ding{55}} \\
Do you require MFA to access sensitive data systems? & \ding{51} \\
Does your organization have an employee acceptable use policy? & \ding{51} \\
Does your organization do security awareness training for new employees? & \ding{51} \\
Does your organization do security awareness training for all employees at least once per year? & \ding{51} \\
\bottomrule
\end{tabular}
\end{table}

\subsection*{Analysis}
The organization has implemented several key security controls, including MFA for email and sensitive systems, and maintains a security awareness program. However, the lack of MFA for computer logins is a \textbf{critical security gap}. This exposes the organization to significant risk from stolen credentials, as a threat actor could gain direct access to an endpoint with only a username and password, bypassing other security measures.

% --- TECHNICAL SCAN RESULTS ---
\section{Technical Scan Results}
An external network scan was performed on the target IP address. The scan identified the following open ports and services.

\begin{itemize}
    \item \textbf{Target IP:} \texttt{[Target IP]}
    \item \textbf{Scan Date:} Not Provided
\end{itemize}

\begin{table}[h!]
\centering
\caption{Open Port Scan Results for \texttt{[Target IP]}}
\begin{tabular}{llll}
\toprule
\textbf{Port} & \textbf{State} & \textbf{Service (Inferred)} & \textbf{Product / Version} \\
\midrule
22/tcp & open & SSH & Not Detected \\
\bottomrule
\end{tabular}
\end{table}

\subsection*{Analysis}
The scan revealed that port 22 (SSH - Secure Shell) is open to the public internet. SSH is a common protocol for remote administration. Exposing this service directly increases the network's attack surface and makes it a target for:
\begin{itemize}
    \item \textbf{Brute-force attacks:} Automated attempts to guess usernames and passwords.
    \item \textbf{Credential stuffing:} Using credentials stolen from other data breaches to attempt logins.
    \item \textbf{Exploitation of vulnerabilities:} If the SSH server software is outdated, it may be vulnerable to known exploits. The inability to detect the software version hinders passive vulnerability assessment.
\end{itemize}

% --- CONSOLIDATED RISK ASSESSMENT ---
\section{Consolidated Risk Assessment}
The following table correlates findings from the security control review, technical scan, and pre-existing risk data to provide a unified view of the primary risks facing the organization.

\begin{table}[h!]
\centering
\caption{Summary of Identified Risks}
\begin{tabular}{p{0.1\textwidth}p{0.25\textwidth}p{0.4\textwidth}l}
\toprule
\textbf{Risk ID} & \textbf{Risk Name} & \textbf{Description} & \textbf{Severity} \\
\midrule
RISK-001 & Localhost Exposed & A pre-existing critical vulnerability was identified on host \texttt{[Target IP]} with a CVSS score of 10.0. & \textbf{Critical} \\
\addlinespace
RISK-002 & Lack of Endpoint MFA & User computers do not require MFA for login. A compromised password could lead to direct endpoint and internal network access. & \textbf{High} \\
\addlinespace
RISK-003 & Exposed SSH Service & The SSH service on port 22 is publicly accessible, making it a target for brute-force and other credential-based attacks. This risk is amplified by the lack of endpoint MFA (RISK-002). & \textbf{High} \\
\bottomrule
\end{tabular}
\end{table}

% --- RECOMMENDATIONS ---
\section{Recommendations}
Based on the consolidated risk assessment, we recommend the following actions, prioritized by severity.

\subsection{Immediate Priority (Critical)}
\begin{enumerate}
    \item \textbf{Remediate "Localhost Exposed" Vulnerability (RISK-001):} Immediately investigate and remediate the critical vulnerability identified as "Localhost Exposed" on \texttt{[Target IP]}. Given its CVSS score of 10.0, this should be considered the absolute highest priority.
\end{enumerate}

\subsection{High Priority}
\begin{enumerate}
    \setcounter{enumi}{1} % Continue numbering
    \item \textbf{Implement MFA for All Computer Logins (RISK-002):} Deploy a mandatory Multi-Factor Authentication solution for all employee and privileged user computer logins. This single control will significantly mitigate the risk of unauthorized access from compromised credentials.
    \item \textbf{Restrict SSH Access (RISK-003):} Limit access to the SSH service on \texttt{[Target IP]}. The preferred method is to place the service behind a Virtual Private Network (VPN). If direct access is required, implement a strict IP allow-list, permitting connections only from trusted administrative networks.
\end{enumerate}

\subsection{General Recommendations}
\begin{enumerate}
    \setcounter{enumi}{3} % Continue numbering
    \item \textbf{Enhance Service Discovery:} Configure network services, including SSH, to provide version information where appropriate. This aids in ongoing vulnerability management and automated scanning.
    \item \textbf{Conduct Regular Vulnerability Scanning:} Implement a program of regular, authenticated and unauthenticated vulnerability scanning across all external and internal assets to proactively identify and remediate security weaknesses.
\end{enumerate}

\end{document}
```