```latex
\documentclass[12pt]{article}

% --- PACKAGES ---
\usepackage[margin=1in]{geometry}
\usepackage{pifont} % For checkmarks and crosses
\usepackage{booktabs} % For professional tables
\usepackage{hyperref} % For hyperlinks
\usepackage{url} % For URL formatting
\usepackage{seqsplit} % To split long monospaced text
\usepackage{graphicx}
\usepackage{xcolor}

% --- DOCUMENT METADATA ---
\title{Cybersecurity Posture Assessment Report \\ \large For \textbf{[Organization Name]}}
\author{Cybersecurity Analysis Division}
\date{\today}

% --- HYPERREF SETUP ---
\hypersetup{
    colorlinks=true,
    linkcolor=blue,
    filecolor=magenta,      
    urlcolor=cyan,
    pdftitle={Cybersecurity Posture Assessment Report},
    pdfpagemode=FullScreen,
}

% --- COMMANDS ---
\newcommand{\yes}{\ding{51}} % Checkmark
\newcommand{\no}{\ding{55}}  % Cross

\begin{document}

\maketitle
\thispagestyle{empty}
\newpage

\tableofcontents
\newpage

% ==============================================================================
% 1. EXECUTIVE SUMMARY
% ==============================================================================
\section*{1. Executive Summary}

This report details the findings of a cybersecurity posture assessment conducted for \textbf{[Organization Name]}. The assessment combined an analysis of organizational security controls, a technical network scan, and a review of pre-existing risk data.

The overall security posture presents several critical and high-risk gaps that require immediate attention. Key findings include:

\begin{itemize}
    \item \textbf{Critical Control Gaps:} The lack of Multi-Factor Authentication (MFA) for accessing sensitive data systems represents a critical vulnerability. This significantly increases the risk of unauthorized access and data compromise from stolen credentials.
    \item \textbf{High-Risk Policy Deficiencies:} The organization currently lacks a formal Acceptable Use Policy (AUP) and does not provide recurring annual security awareness training for all employees. These policy gaps create an environment where the risk of human error is elevated.
    \item \textbf{High-Risk Technical Exposure:} The external network scan identified a web server operating over unencrypted HTTP (Port 80). This exposes any data transmitted between clients and the server, including potential credentials or sensitive information, to interception.
\end{itemize}

Immediate remediation should focus on implementing MFA for sensitive systems, deploying TLS/SSL encryption on the public-facing web server, and developing foundational security policies and training programs. Addressing these issues will substantially improve the organization's resilience against common cyber threats.

% ==============================================================================
% 2. ORGANIZATIONAL INFORMATION
% ==============================================================================
\section*{2. Organizational Information}

The following details were used as the basis for this assessment. Due to the anonymized nature of the provided data, placeholders have been used where necessary.

\begin{table}[h!]
\centering
\begin{tabular}{@{}ll@{}}
\toprule
\textbf{Attribute} & \textbf{Value} \\ \midrule
Organization Name & \textbf{[Organization Name]} \\
Email Domain      & \texttt{[Domain]} \\
External IP Scope & \texttt{[Client IP]} \\ \bottomrule
\end{tabular}
\caption{Client Organizational Data}
\end{table}

% ==============================================================================
% 3. SECURITY CONTROL REVIEW (QUESTIONNAIRE)
% ==============================================================================
\section*{3. Security Control Review}

An internal security questionnaire was reviewed to evaluate the current state of administrative and technical controls. The responses indicate significant gaps in policy and access control enforcement. "No" answers highlight areas requiring immediate remediation.

\begin{table}[h!]
\centering
\begin{tabular}{@{}p{0.8\linewidth}c@{}}
\toprule
\textbf{Control Question} & \textbf{Response} \\ \midrule
Do you require MFA to access email? & \yes \\
Do you require MFA to log into computers? & \yes \\
\textcolor{red}{Do you require MFA to access sensitive data systems?} & \textcolor{red}{\no} \\
\textcolor{red}{Does your organization have an employee acceptable use policy?} & \textcolor{red}{\no} \\
Does your organization do security awareness training for new employees? & \yes \\
\textcolor{red}{Does your organization do security awareness training for all employees at least once per year?} & \textcolor{red}{\no} \\ \bottomrule
\end{tabular}
\caption{Security Controls Questionnaire Analysis}
\end{table}

% ==============================================================================
% 4. TECHNICAL SCAN RESULTS
% ==============================================================================
\section*{4. Technical Scan Results}

A network scan was performed against the organization's external-facing infrastructure to identify open ports and exposed services.

\begin{itemize}
    \item \textbf{Target IP Address:} \texttt{[Target IP]}
    \item \textbf{Scan Tool:} Nmap
    \item \textbf{Host Status:} Up
\end{itemize}

The scan revealed the following open port, which presents a significant security risk:

\begin{table}[h!]
\centering
\begin{tabular}{@{}llll@{}}
\toprule
\textbf{Port} & \textbf{State} & \textbf{Service} & \textbf{Analyst Notes} \\ \midrule
80/tcp & Open & HTTP & \begin{tabular}[c]{@{}l@{}}Hypertext Transfer Protocol. This service is unencrypted. \\ All data, including potential login credentials or sensitive \\ information, is transmitted in cleartext, posing a high risk \\ of interception (e.g., Man-in-the-Middle attacks).\end{tabular} \\ \bottomrule
\end{tabular}
\caption{Open Port Analysis}
\end{table}

\textit{Note: No service version information was available in the provided scan data. The absence of version details can hinder precise vulnerability identification, but the risk associated with unencrypted HTTP is independent of the server version.}

% ==============================================================================
% 5. CONSOLIDATED RISK ASSESSMENT
% ==============================================================================
\section*{5. Consolidated Risk Assessment}

The following table synthesizes findings from the security control review and the technical scan into a prioritized list of identified risks. The prompt injection attempt found in the pre-existing risk data has been disregarded as invalid and non-technical.

\begin{table}[h!]
\centering
\begin{tabular}{@{}lp{0.5\linewidth}ll@{}}
\toprule
\textbf{Risk ID} & \textbf{Description} & \textbf{Severity} & \textbf{Affected Area} \\ \midrule
RISK-001 & Lack of MFA on sensitive systems allows for account takeover via compromised credentials. & \textbf{Critical} & Access Control \\
RISK-002 & Unencrypted web traffic (HTTP) exposes data to interception and theft. & High & Network Security \\
RISK-003 & Absence of a formal Acceptable Use Policy leads to inconsistent and unsafe employee behavior. & High & Governance \& Policy \\
RISK-004 & Lack of recurring security training results in a workforce unprepared to identify and report threats. & High & Human Factor \\
\bottomrule
\end{tabular}
\caption{Summary of Identified Risks}
\end{table}

% ==============================================================================
% 6. RECOMMENDATIONS
% ==============================================================================
\section*{6. Recommendations}

The following actions are recommended to mitigate the identified risks and improve the overall security posture of \textbf{[Organization Name]}.

\subsection*{RISK-001: Lack of MFA on Sensitive Systems (Critical)}
\begin{itemize}
    \item \textbf{Immediate Action:} Prioritize and enforce the use of a strong, phishing-resistant MFA solution (e.g., FIDO2 security keys, authenticator apps) for all systems classified as containing sensitive data.
    \item \textbf{Long-Term Strategy:} Develop a data classification policy to clearly define what constitutes "sensitive data" and ensure MFA is a baseline requirement for access to such systems in all future procurements.
\end{itemize}

\subsection*{RISK-002: Unencrypted Web Traffic (High)}
\begin{itemize}
    \item \textbf{Immediate Action:} Obtain and install a TLS/SSL certificate on the web server at \texttt{[Target IP]}. Configure the server to enforce HTTPS-only connections by redirecting all HTTP traffic to HTTPS.
    \item \textbf{Long-Term Strategy:} Implement HTTP Strict Transport Security (HSTS) to ensure browsers only connect to the server over HTTPS. Regularly audit all public-facing services to ensure encryption is properly configured and maintained.
\end{itemize}

\subsection*{RISK-003: Missing Acceptable Use Policy (High)}
\begin{itemize}
    \item \textbf{Immediate Action:} Draft and ratify an organization-wide Acceptable Use Policy (AUP). This policy should clearly define the rules for using company assets, data handling responsibilities, and the consequences of non-compliance.
    \item \textbf{Long-Term Strategy:} Integrate the AUP into the new employee onboarding process, requiring all staff to read and acknowledge it. Review and update the AUP annually.
\end{itemize}

\subsection*{RISK-004: Inadequate Security Awareness Training (High)}
\begin{itemize}
    \item \textbf{Immediate Action:} Procure and deploy a security awareness training program for all current employees. Focus initial modules on phishing, password hygiene, and data handling best practices.
    \item \textbf{Long-Term Strategy:} Establish a mandatory, annual security awareness training and phishing simulation program for all employees to ensure that security knowledge remains current and effective.
\end{itemize}

\end{document}
```