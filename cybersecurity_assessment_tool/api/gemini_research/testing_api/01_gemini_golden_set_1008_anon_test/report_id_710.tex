```latex
\documentclass[12pt]{article}

% Preamble: Required Packages
\usepackage[margin=1in]{geometry} % For setting page margins
\usepackage{pifont} % For checkmarks and crosses (dingbats)
\usepackage{booktabs} % For professional-looking tables
\usepackage{hyperref} % For hyperlinks
\usepackage{url} % For formatting URLs
\usepackage{seqsplit} % For splitting long strings without spaces
\usepackage{graphicx} % For potential logos/images
\usepackage{xcolor} % For colors in text

% Document Information
\title{Cybersecurity Posture Assessment Report}
\author{Cybersecurity Analysis Division}
\date{\today}

% Hyperref Setup
\hypersetup{
    colorlinks=true,
    linkcolor=blue,
    filecolor=magenta,      
    urlcolor=cyan,
    pdftitle={Cybersecurity Posture Assessment Report},
    pdfpagemode=FullScreen,
}

\begin{document}

\maketitle
\thispagestyle{empty}
\newpage

\tableofcontents
\newpage

% --- 1. Executive Summary ---
\section{Executive Summary}

This report provides a comprehensive analysis of the cybersecurity posture for \textbf{[Organization Name]}. The assessment is based on a synthesis of external network scan data, a review of internal security controls via a questionnaire, and an evaluation of pre-existing documented risks.

The analysis reveals several critical and high-risk vulnerabilities that expose the organization to significant threats, including unauthorized access, data breach, and ransomware. Key findings include:

\begin{itemize}
    \item \textbf{Critically Vulnerable External Service:} An externally facing FTP server was identified running a severely outdated and vulnerable version of \texttt{vsftpd} (2.3.4), which is known to contain a backdoor (CVE-2011-2523). Furthermore, this server is misconfigured to allow anonymous public access.
    \item \textbf{Insufficient Access Controls:} Multi-Factor Authentication (MFA) is not enforced for access to email or computer logins. This represents a fundamental gap in identity and access management, drastically increasing the risk of account compromise.
    \item \textbf{Outdated Operating Systems:} The organization is aware of and continues to operate on outdated Windows 7 workstations, which are no longer supported and do not receive security updates.
\end{itemize}

Immediate remediation of the external FTP server and the rapid implementation of MFA are paramount to mitigating the most severe risks. The overall security posture is considered poor and requires urgent and decisive action.

% --- 2. Organizational Information ---
\section{Organizational Information}

This assessment pertains to the digital assets and security controls of the following entity. The information provided has been anonymized for this report template.

\begin{itemize}
    \item \textbf{Organization Name:} \textbf{[Organization Name]}
    \item \textbf{Primary Email Domain:} \texttt{[Domain]}
    \item \textbf{Scanned External IP:} \texttt{[Client IP]}
\end{itemize}

% --- 3. Security Control Review ---
\section{Security Control Review (Questionnaire Analysis)}

An internal security questionnaire was reviewed to evaluate the maturity of existing administrative and technical controls. The responses are summarized below. A checkmark (\ding{51}) indicates a positive control, while a cross (\ding{55}) indicates a control gap.

\begin{table}[h!]
\centering
\caption{Security Controls Questionnaire Responses}
\begin{tabular}{p{0.75\linewidth} c}
\toprule
\textbf{Control Question} & \textbf{Response} \\
\midrule
Do you require MFA to access email? & \textcolor{red}{\ding{55}} \\
Do you require MFA to log into computers? & \textcolor{red}{\ding{55}} \\
Do you require MFA to access sensitive data systems? & \textcolor{green}{\ding{51}} \\
Does your organization have an employee acceptable use policy? & \textcolor{green}{\ding{51}} \\
Does your organization do security awareness training for new employees? & \textcolor{green}{\ding{51}} \\
Does your organization do security awareness training for all employees at least once per year? & \textcolor{green}{\ding{51}} \\
\bottomrule
\end{tabular}
\end{table}

\subsection*{Analysis of Control Gaps}
The lack of MFA on email and computer logins are \textbf{critical security gaps}. Email is a primary target for phishing attacks, and a compromised account can lead to widespread system access, data exfiltration, and financial fraud. Similarly, the absence of MFA on workstations removes a vital layer of defense against unauthorized access should an employee's credentials be stolen. While it is positive that MFA is used for sensitive data systems, the primary entry points (email and workstations) remain dangerously exposed.

% --- 4. Technical Scan Results ---
\section{Technical Scan Results}

An external network scan was performed on the target IP address \texttt{[Target IP]} to identify open ports and exposed services.

\begin{table}[h!]
\centering
\caption{Open Port Analysis for Target: \texttt{[Target IP]}}
\begin{tabular}{l l l l p{0.3\linewidth}}
\toprule
\textbf{Port} & \textbf{State} & \textbf{Service} & \textbf{Version} & \textbf{Details} \\
\midrule
21/tcp & open & ftp & vsftpd 2.3.4 & \textbf{Critical Vulnerability.} Version 2.3.4 is susceptible to a backdoor (CVE-2011-2523). Anonymous FTP login is also allowed, permitting unauthenticated access. \\
\bottomrule
\end{tabular}
\end{table}

\subsection*{Analysis of Technical Findings}
The scan identified a single, yet extremely high-risk, finding. The \texttt{vsftpd 2.3.4} service is a decade-old version with a well-documented remote code execution vulnerability. An attacker can easily gain a shell on the server by exploiting this flaw. The allowance of anonymous FTP login compounds this risk by providing an easy, unauthenticated entry point for an attacker to begin their reconnaissance or upload malicious files. \textbf{This service should be considered fully compromised.}

% --- 5. Correlated Risk Assessment ---
\section{Correlated Risk Assessment}

By correlating the security control gaps, technical findings, and pre-existing risks, we have compiled a summary of the most significant threats to the organization.

\begin{table}[h!]
\centering
\caption{Summary of Identified Risks}
\begin{tabular}{p{0.2\linewidth} p{0.5\linewidth} l}
\toprule
\textbf{Risk Name} & \textbf{Description} & \textbf{Severity} \\
\midrule
\textbf{Vulnerable Public FTP Server} & An outdated, vulnerable FTP server (\texttt{vsftpd 2.3.4}) with anonymous login enabled is exposed to the internet, allowing for trivial remote compromise. & \textbf{Critical} \\
\addlinespace
\textbf{Insufficient MFA Implementation} & Lack of MFA on core services like email and workstations exposes the organization to account takeover, phishing, and subsequent lateral movement. & \textbf{High} \\
\addlinespace
\textbf{Outdated Operating Systems} & The continued use of Windows 7, an unsupported OS, on workstations increases the attack surface and makes the environment vulnerable to known exploits. & \textbf{Medium} \\
\bottomrule
\end{tabular}
\end{table}

% --- 6. Recommendations ---
\section{Recommendations}

The following actions are recommended to mitigate the identified risks. They are prioritized based on severity and exploitability.

\subsection*{Priority 1: Remediate Vulnerable FTP Server (Critical)}
\begin{itemize}
    \item \textbf{Immediate (0-24 hours):} Take the FTP server offline immediately by shutting down the service or implementing a firewall rule to block all access to port 21 from the internet.
    \item \textbf{Short-Term (1-3 days):} If the service is business-critical, migrate to a secure file transfer protocol like SFTP (SSH File Transfer Protocol). If FTP must be used, upgrade the \texttt{vsftpd} software to the latest stable version, disable anonymous access, and enforce strong password policies.
    \item \textbf{Long-Term:} Decommission legacy file transfer protocols in favor of modern, managed file transfer solutions or secure cloud storage with robust access controls.
\end{itemize}

\subsection*{Priority 2: Implement Comprehensive MFA (High)}
\begin{itemize}
    \item \textbf{Immediate (0-7 days):} Enforce MFA for all user accounts, especially administrative ones, on the primary email system (e.g., Microsoft 365, Google Workspace).
    \item \textbf{Short-Term (1-4 weeks):} Deploy an MFA solution for all workstation and remote access (VPN) logins. Prioritize users with access to sensitive data and system administrators.
\end{itemize}

\subsection*{Priority 3: Upgrade Outdated Operating Systems (Medium)}
\begin{itemize}
    \item \textbf{Short-Term (1-3 months):} Execute the existing plan to upgrade all remaining Windows 7 workstations to a currently supported version of Windows (e.g., Windows 10/11).
    \item \textbf{Long-Term:} Establish a formal asset management and lifecycle policy to ensure that all hardware and software are retired before they reach end-of-life.
\end{itemize}

\end{document}
```