```latex
\documentclass[12pt]{article}

% --- PACKAGES ---
\usepackage[margin=1in]{geometry}
\usepackage{pifont} % For checkmarks and crosses
\usepackage{booktabs} % For professional tables
\usepackage{hyperref} % For hyperlinks and metadata
\usepackage{url} % For URL formatting
\usepackage{seqsplit} % For splitting long strings in tt font
\usepackage{graphicx}
\usepackage{xcolor}

% --- DOCUMENT SETUP ---
\hypersetup{
    colorlinks=true,
    linkcolor=black,
    urlcolor=blue,
    pdftitle={Cybersecurity Posture Assessment Report},
    pdfauthor={Cybersecurity Analysis Cell},
    pdfsubject={Security Assessment},
    pdfkeywords={Security, Risk, Assessment, RDP, MFA},
}

\newcommand{\yes}{\ding{51}} % Checkmark
\newcommand{\no}{\ding{55}}  % Cross

% --- DOCUMENT START ---
\begin{document}

% --- TITLE PAGE ---
\begin{titlepage}
    \centering
    \vspace*{2cm}
    \Huge{\textbf{Cybersecurity Posture Assessment Report}}
    \vspace{1.5cm}
    \Large{\textbf{Prepared for:}}
    \vspace{0.5cm}
    \huge{\textbf{[Organization Name]}}
    \vfill
    \large{\textbf{Date of Report:}}
    \vspace{0.2cm}
    \Large{\today}
    \vspace{2cm}
    \normalsize{This report contains sensitive information regarding the security posture of the organization. Distribution should be limited to authorized personnel only.}
\end{titlepage}

\tableofcontents
\newpage

% --- EXECUTIVE SUMMARY ---
\section*{Executive Summary}

This report details the findings of a cybersecurity posture assessment conducted for \textbf{[Organization Name]}. The analysis combines a review of organizational security controls, an external network scan, and pre-existing risk data.

A critical risk was identified: the direct exposure of the Remote Desktop Protocol (RDP) service on port 3389 at the external IP address \texttt{[Target IP]}. This vulnerability, rated with a CVSS severity score of 9.0, presents a significant and immediate threat of unauthorized access, potentially leading to a full network compromise or ransomware deployment.

This technical vulnerability is severely compounded by critical gaps in organizational security controls. Specifically, the organization does not enforce Multi-Factor Authentication (MFA) for email, computer logins, or access to sensitive data systems. Furthermore, a formal security awareness training program for employees is absent. This combination of an exposed, high-value service and weak authentication and human-layer defenses creates an environment highly susceptible to attack.

Immediate remediation of the exposed RDP service is strongly recommended, followed by the rapid implementation of MFA and the establishment of a security awareness training program.

% --- ORGANIZATIONAL INFORMATION ---
\section*{Organizational Information}

The following details were used as the basis for this assessment. Due to the anonymized nature of the provided data, placeholders have been used.

\begin{itemize}
    \item \textbf{Organization Name:} \textbf{[Organization Name]}
    \item \textbf{Primary Domain:} \texttt{[Domain]}
    \item \textbf{Assessed External IP:} \texttt{[Client IP]}
\end{itemize}

% --- SECURITY CONTROL REVIEW ---
\section*{Security Control Review}

A review of foundational security controls was conducted via a questionnaire. The results indicate significant gaps in identity and access management and employee security training. "No" answers represent a failure to meet baseline security best practices and are highlighted as high-risk deficiencies.

\begin{table}[h!]
\centering
\caption{Organizational Security Controls Questionnaire}
\label{tab:controls}
\begin{tabular}{@{}lc@{}}
\toprule
\textbf{Control Question} & \textbf{Status} \\
\midrule
Do you require MFA to access email? & \textcolor{red}{\no} \\
Do you require MFA to log into computers? & \textcolor{red}{\no} \\
Do you require MFA to access sensitive data systems? & \textcolor{red}{\no} \\
Does your organization have an employee acceptable use policy? & \textcolor{green}{\yes} \\
Does your organization do security awareness training for new employees? & \textcolor{red}{\no} \\
Does your organization do security awareness training for all employees annually? & \textcolor{red}{\no} \\
\bottomrule
\end{tabular}
\end{table}

% --- TECHNICAL SCAN RESULTS ---
\section*{Technical Scan Results}

An external network scan was performed on the target IP address to identify exposed services. The scan revealed one open port, which is detailed below.

\begin{table}[h!]
\centering
\caption{Open Ports Detected on \texttt{[Target IP]}}
\label{tab:nmap}
\begin{tabular}{@{}llll@{}}
\toprule
\textbf{Target IP} & \textbf{Port/Protocol} & \textbf{State} & \textbf{Service Name} \\
\midrule
\texttt{[Target IP]} & 3389/tcp & open & ms-wbt-server (Remote Desktop) \\
\bottomrule
\end{tabular}
\end{table}

\paragraph{Analysis:} The presence of an open RDP port (3389) is a critical finding. This protocol is a primary target for attackers who use brute-force password attacks, credential stuffing, and exploits for known vulnerabilities (e.g., BlueKeep) to gain initial access to a network.

% --- RISK ASSESSMENT & CORRELATION ---
\section*{Risk Assessment \& Correlation}

The following table synthesizes the findings from the security control review, technical scan, and pre-existing risk data into a prioritized list of identified risks.

\begin{table}[h!]
\centering
\caption{Consolidated Risk Register}
\label{tab:risks}
\begin{tabular}{@{}p{0.1\linewidth} p{0.25\linewidth} p{0.4\linewidth} p{0.15\linewidth}@{}}
\toprule
\textbf{Risk ID} & \textbf{Finding} & \textbf{Description \& Impact} & \textbf{Severity} \\
\midrule
\textbf{RISK-001} & \textbf{Exposed RDP Service} & Port 3389 is open to the public internet. This allows threat actors to directly attack a critical remote access service, potentially leading to unauthorized access, data breach, or ransomware. & \textbf{Critical (9.0)} \\
\addlinespace
\textbf{RISK-002} & \textbf{Systemic Lack of MFA} & Multi-Factor Authentication is not enforced. This makes user accounts highly vulnerable to compromise via phishing or password guessing, directly increasing the risk of a successful attack against the exposed RDP service. & \textbf{Critical} \\
\addlinespace
\textbf{RISK-003} & \textbf{Insufficient Security Training} & The absence of a security awareness program leaves employees unprepared to identify and report threats like phishing, which is a common vector for stealing credentials. & \textbf{High} \\
\bottomrule
\end{tabular}
\end{table}

% --- RECOMMENDATIONS ---
\section*{Recommendations}

The following prioritized recommendations are provided to mitigate the identified risks and improve the overall security posture of \textbf{[Organization Name]}.

\subsection*{Priority 1: Immediate Remediation}

\begin{description}
    \item[Action:] \textbf{Remediate RDP Exposure on \texttt{[Target IP]}.}
    \item[Details:] The RDP service on port 3389 must be immediately removed from public internet exposure.
    \begin{itemize}
        \item \textbf{Short-Term Fix:} Block all inbound traffic to TCP port 3389 on the network firewall. If access is essential, restrict it to a whitelist of known, trusted IP addresses.
        \item \textbf{Long-Term Fix:} Implement a Virtual Private Network (VPN) or Zero Trust Network Access (ZTNA) solution for all remote access. Access to RDP should only be possible after a user has authenticated to the VPN/ZTNA solution.
    \end{itemize}
\end{description}

\subsection*{Priority 2: High-Impact Controls}

\begin{description}
    \item[Action:] \textbf{Implement and Enforce Multi-Factor Authentication (MFA).}
    \item[Details:] MFA is the single most effective control to prevent unauthorized access resulting from compromised credentials.
    \begin{itemize}
        \item Deploy MFA across all external-facing services, especially email (e.g., Office 365, Google Workspace) and any VPN solution.
        \item Enforce MFA for all privileged user accounts and for logins to systems containing sensitive data.
        \item Develop a roadmap to enforce MFA for all standard user computer logins.
    \end{itemize}
\end{description}

\subsection*{Priority 3: Foundational Improvements}

\begin{description}
    \item[Action:] \textbf{Establish a Security Awareness Training Program.}
    \item[Details:] A well-trained workforce is a critical layer of defense.
    \begin{itemize}
        \item Implement a mandatory training module for all new employees that covers phishing, password security, and the organization's acceptable use policy.
        \item Conduct annual refresher training for all staff.
        \item Consider periodic phishing simulations to test and reinforce employee awareness.
    \end{itemize}
\end{description}

\end{document}
```