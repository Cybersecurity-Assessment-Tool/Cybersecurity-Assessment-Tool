```latex
\documentclass[12pt]{article}

% ----------------------------------------------------------------------
% PREAMBLE
% ----------------------------------------------------------------------
\usepackage[margin=1in]{geometry}
\usepackage{pifont} % For checkmarks and crosses
\usepackage{booktabs} % For professional tables
\usepackage{hyperref} % For clickable links
\usepackage{url} % For URL formatting
\usepackage{seqsplit} % To split long strings without spaces
\usepackage{xcolor} % For colors
\usepackage{graphicx} % For potential logos
\usepackage{datetime} % For report date

% --- Hyperref Setup ---
\hypersetup{
    colorlinks=true,
    linkcolor=blue,
    filecolor=magenta,      
    urlcolor=cyan,
    pdftitle={Cybersecurity Posture Report},
    pdfpagemode=FullScreen,
}

% --- Custom Commands ---
\newcommand{\yes}{\ding{51}}
\newcommand{\no}{\ding{55}}
\newcommand{\severitycritical}{\textcolor{red!80!black}{\textbf{Critical}}}
\newcommand{\severityhigh}{\textcolor{orange!90!black}{\textbf{High}}}
\newcommand{\severitymedium}{\textcolor{yellow!90!black}{\textbf{Medium}}}
\newcommand{\severitylow}{\textcolor{green!80!black}{\textbf{Low}}}

% ----------------------------------------------------------------------
% DOCUMENT START
% ----------------------------------------------------------------------
\begin{document}

% --- Title Page ---
\begin{titlepage}
    \centering
    \vspace*{1cm}
    \Huge\textbf{Cybersecurity Posture Report}
    \vspace{1.5cm}
    \Large
    \textbf{Prepared for:} \\
    \vspace{0.5cm}
    \textbf{[Organization Name]}
    \vspace{2.5cm}
    
    \textbf{Date of Report:} \today \\
    \vspace{0.5cm}
    \textbf{Report ID:} CSR-2024-001
    
    \vfill
    
    \large
    \textit{This report contains sensitive information regarding the security posture of the organization. Distribution should be limited to authorized personnel only.}
    
\end{titlepage}

\tableofcontents
\newpage

% ----------------------------------------------------------------------
% SECTION 1: EXECUTIVE SUMMARY
% ----------------------------------------------------------------------
\section{Executive Summary}

This report provides a comprehensive analysis of the cybersecurity posture for \textbf{[Organization Name]}, based on a combination of network scanning, a security controls questionnaire, and a review of pre-existing risks. The assessment reveals several critical and high-risk vulnerabilities that require immediate attention to mitigate potential threats to the organization's data and operations.

Key findings indicate a significant risk of data breach and unauthorized access. An externally facing MySQL database was identified running an End-of-Life (EOL) version, which no longer receives security updates. This finding is compounded by critical gaps in internal security controls, most notably the lack of Multi-Factor Authentication (MFA) for computer logins and the absence of mandatory annual security awareness training for all employees.

The overall security posture is assessed as \textbf{High Risk}. This report outlines specific, actionable recommendations to address these findings, strengthen security controls, and improve the organization's resilience against cyber threats.

% ----------------------------------------------------------------------
% SECTION 2: ORGANIZATIONAL INFORMATION
% ----------------------------------------------------------------------
\section{Organizational Information}

The following details were used as the basis for this assessment. Anonymized data has been replaced with placeholders.

\begin{itemize}
    \item \textbf{Organization Name:} \textbf{[Organization Name]}
    \item \textbf{Primary Domain:} \texttt{[Domain]}
    \item \textbf{Assessed External IP:} \texttt{[Client IP]}
\end{itemize}

% ----------------------------------------------------------------------
% SECTION 3: SECURITY CONTROL REVIEW
% ----------------------------------------------------------------------
\section{Security Control Review}

A review of the organization's security controls was conducted via a questionnaire. The responses highlight key areas of strength and weakness in the current security policies and their implementation. Gaps identified here often represent significant organizational risk.

\begin{table}[h!]
\centering
\caption{Security Controls Questionnaire Results}
\begin{tabular}{p{0.7\linewidth} c c}
\toprule
\textbf{Control Question} & \textbf{Response} & \textbf{Status} \\
\midrule
Do you require MFA to access email? & Yes & \yes \\
Do you require MFA to log into computers? & No & \no \\
Do you require MFA to access sensitive data systems? & Yes & \yes \\
Does your organization have an employee acceptable use policy? & Yes & \yes \\
Does your organization do security awareness training for new employees? & Yes & \yes \\
Does your organization do security awareness training for all employees at least once per year? & No & \no \\
\bottomrule
\end{tabular}
\end{table}

\subsection*{Analysis of Gaps}
Two significant gaps were identified:
\begin{itemize}
    \item \textbf{No MFA for Computer Logins:} This is a critical vulnerability. Without MFA, user endpoints are susceptible to compromise through weak, stolen, or brute-forced passwords, which can lead to unauthorized access and lateral movement within the network.
    \item \textbf{No Annual Security Awareness Training:} The lack of recurring training for all employees is a high-risk issue. The threat landscape evolves continuously, and without regular training, staff are more likely to fall victim to phishing, social engineering, and other common attack vectors.
\end{itemize}

% ----------------------------------------------------------------------
% SECTION 4: TECHNICAL SCAN RESULTS
% ----------------------------------------------------------------------
\section{Technical Scan Results}

A network scan was performed to identify open ports and exposed services on the target system.

\begin{itemize}
    \item \textbf{Target IP Address:} \texttt{[Target IP]}
\end{itemize}

\begin{table}[h!]
\centering
\caption{Open Ports and Services Identified}
\begin{tabular}{l l l l l}
\toprule
\textbf{Port} & \textbf{State} & \textbf{Service} & \textbf{Product} & \textbf{Version} \\
\midrule
3306/tcp & Open & mysql & MySQL & 5.7.33 \\
\bottomrule
\end{tabular}
\end{table}

\subsection*{Technical Analysis}
The scan identified that port 3306 is open, exposing a MySQL database service directly to the network. This configuration is highly discouraged as it makes the database a direct target for attackers.

Furthermore, the detected version, \textbf{MySQL 5.7.33}, is a critical concern. MySQL version 5.7 reached its official \textbf{End of Life (EOL) in October 2023}. This means it no longer receives security patches from the vendor, and any newly discovered vulnerabilities will remain unpatched, leaving the system perpetually vulnerable to exploitation. This finding confirms and elevates the severity of the pre-existing risk "Database Exposure."

% ----------------------------------------------------------------------
% SECTION 5: CONSOLIDATED RISK ASSESSMENT
% ----------------------------------------------------------------------
\section{Consolidated Risk Assessment}

The following table synthesizes findings from the security control review, technical scan, and pre-existing risk data into a consolidated list of prioritized risks.

\begin{table}[h!]
\centering
\caption{Prioritized Risk Summary}
\begin{tabular}{p{0.25\linewidth} p{0.5\linewidth} l}
\toprule
\textbf{Risk Name} & \textbf{Description} & \textbf{Severity} \\
\midrule
\textbf{Exposed End-of-Life Database Service} & A MySQL database (v5.7.33) is publicly exposed on port 3306. This version is past its End of Life and no longer receives security updates, making it highly vulnerable to exploitation. & \severitycritical \\
\addlinespace
\textbf{Lack of Endpoint Multi-Factor Authentication} & User computers can be accessed with only a password. A compromised password could lead to direct access to an employee's machine and the internal network. & \severitycritical \\
\addlinespace
\textbf{Insufficient Security Awareness Training} & The absence of mandatory, annual security training for all staff increases the likelihood of successful phishing and social engineering attacks, leading to credential theft or malware infection. & \severityhigh \\
\bottomrule
\end{tabular}
\end{table}

% ----------------------------------------------------------------------
% SECTION 6: RECOMMENDATIONS
% ----------------------------------------------------------------------
\section{Recommendations}

The following actionable recommendations are provided to address the identified risks. They are categorized by priority to guide remediation efforts.

\subsection{Critical Risk: Exposed End-of-Life Database}
\begin{itemize}
    \item \textbf{Immediate Action (Containment):} Implement strict firewall rules to block all public access to TCP port 3306. Access should be restricted to a minimal set of trusted internal IP addresses.
    \item \textbf{Short-Term Action (Remediation):} Develop and execute a plan to upgrade the MySQL 5.7 instance to a currently supported version (e.g., MySQL 8.x). This is essential to ensure the service receives ongoing security patches.
    \item \textbf{Long-Term Action (Architectural):} Re-architect the application environment to place the database in a private network segment, inaccessible from the public internet. Access should only be granted through a secure channel like a VPN or a bastion host.
\end{itemize}

\subsection{Critical Risk: Lack of Endpoint MFA}
\begin{itemize}
    \item \textbf{Immediate Action (Planning):} Begin evaluating and selecting an MFA solution that integrates with your current endpoint operating systems (e.g., Windows Hello for Business, Duo, Okta).
    \item \textbf{Short-Term Action (Implementation):} Roll out MFA for computer logins, starting with a pilot group of IT staff and executives, before expanding to the entire organization.
    \item \textbf{Long-Term Action (Policy):} Update the organization's access control policy to mandate the use of MFA for all endpoint, application, and remote access logins.
\end{itemize}

\subsection{High Risk: Insufficient Security Awareness Training}
\begin{itemize}
    \item \textbf{Immediate Action (Procurement/Scheduling):} Select a reputable security awareness training platform or provider and schedule mandatory annual training for all employees.
    \item \textbf{Short-Term Action (Execution):} Conduct the first organization-wide training session, ensuring completion is tracked. Focus on key topics like phishing, password hygiene, and acceptable use.
    \item \textbf{Long-Term Action (Program Development):} Establish a continuous security awareness program that includes regular training, simulated phishing campaigns, and security newsletters to maintain a high level of vigilance among staff.
\end{itemize}

% ----------------------------------------------------------------------
% DOCUMENT END
% ----------------------------------------------------------------------
\end{document}
```