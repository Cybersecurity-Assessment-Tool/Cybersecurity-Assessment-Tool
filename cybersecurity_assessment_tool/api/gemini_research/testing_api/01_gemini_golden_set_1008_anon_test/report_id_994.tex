```latex
\documentclass[12pt, a4paper]{article}

% Preamble: Required Packages
\usepackage[margin=1in]{geometry}
\usepackage{pifont} % For checkmarks and crosses (\ding)
\usepackage{booktabs} % For professional tables
\usepackage{hyperref} % For clickable links
\usepackage{url} % For URL formatting
\usepackage{seqsplit} % For splitting long strings to prevent overflow
\usepackage{xcolor} % For colors

% Document Metadata
\title{Cybersecurity Posture Assessment Report}
\author{Cybersecurity Analysis Division}
\date{\today}

% Hyperref Setup
\hypersetup{
    colorlinks=true,
    linkcolor=blue,
    filecolor=magenta,      
    urlcolor=cyan,
    pdftitle={Cybersecurity Posture Assessment Report},
    pdfpagemode=FullScreen,
}

\begin{document}

\maketitle
\thispagestyle{empty}
\newpage

\tableofcontents
\newpage

% --- 1. Executive Summary ---
\section{Executive Summary}

This report provides a comprehensive analysis of the cybersecurity posture for \textbf{[Organization Name]}. The assessment is based on a correlation of organizational data, a security controls questionnaire, and an external network vulnerability scan.

The external network scan conducted on the target IP address \texttt{[Target IP]} did not identify any open ports or services. While this is a positive finding from a network perimeter perspective, it does not indicate a secure internal environment.

The primary findings of this assessment stem from the security controls review, which revealed several critical gaps in foundational security practices. Specifically, the lack of mandatory Multi-Factor Authentication (MFA) for computer logins, the absence of an employee Acceptable Use Policy (AUP), and a complete deficiency in security awareness training represent significant risks. These procedural and policy-based vulnerabilities expose the organization to a high likelihood of security incidents, including unauthorized access, data breaches, and social engineering attacks.

Urgent remediation is required to address these fundamental control gaps. The recommendations section of this report outlines actionable steps to mitigate the identified risks and establish a baseline of security hygiene.

% --- 2. Organizational Information ---
\section{Organizational Information}

The following information was used as the basis for this assessment. Due to the anonymized nature of the provided data, placeholders have been used where necessary.

\begin{table}[h!]
\centering
\begin{tabular}{@{}ll@{}}
\toprule
\textbf{Attribute} & \textbf{Value} \\ \midrule
Client Name & \textbf{[Organization Name]} \\
Email Domain & \texttt{[Domain]} \\
Primary External IP & \texttt{[Client IP]} \\
Scanned Target IP & \texttt{[Target IP]} \\
Assessment Date & \today \\ \bottomrule
\end{tabular}
\caption{Client and Assessment Details}
\end{table}

% --- 3. Security Control Review ---
\section{Security Control Review}

A review of the organization's security controls was conducted via a questionnaire. The responses indicate several areas of non-compliance with cybersecurity best practices. A summary of the findings is presented below.

\begin{table}[h!]
\centering
\begin{tabular}{@{}p{0.6\textwidth}cc@{}}
\toprule
\textbf{Control Question} & \textbf{Response} & \textbf{Assessment} \\ \midrule
Do you require MFA to access email? & \ding{51} & Compliant \\
Do you require MFA to log into computers? & \textcolor{red}{\ding{55}} & \textbf{Critical Gap} \\
Do you require MFA to access sensitive data systems? & \ding{51} & Compliant \\
Does your organization have an employee acceptable use policy? & \textcolor{red}{\ding{55}} & \textbf{Critical Gap} \\
Does your organization do security awareness training for new employees? & \textcolor{red}{\ding{55}} & \textbf{Critical Gap} \\
Does your organization do security awareness training for all employees at least once per year? & \textcolor{red}{\ding{55}} & \textbf{Critical Gap} \\ \bottomrule
\end{tabular}
\caption{Security Controls Questionnaire Analysis}
\end{table}

% --- 4. Technical Scan Results ---
\section{Technical Scan Results}

An external network scan was performed on the designated target IP address to identify potential vulnerabilities accessible from the public internet.

\begin{itemize}
    \item \textbf{Target IP Address:} \texttt{[Target IP]}
    \item \textbf{Scan Date:} \today
    \item \textbf{Summary:} The scan completed successfully and found \textbf{no open TCP or UDP ports}. This indicates a strong network perimeter configuration, as no services were exposed to the external scanner. While positive, this result does not provide insight into the security of internal systems or vulnerabilities that could be exploited via other vectors (e.g., phishing).
\end{itemize}

% --- 5. Overall Risk Assessment ---
\section{Overall Risk Assessment}

The following table synthesizes findings from all data sources into a prioritized list of risks. The most severe risks are related to policy and procedural controls, which significantly outweigh the low risk observed at the network perimeter.

\begin{table}[h!]
\centering
\begin{tabular}{@{}lp{0.5\textwidth}ll@{}}
\toprule
\textbf{Risk ID} & \textbf{Risk Description} & \textbf{Source} & \textbf{Severity} \\ \midrule
RISK-001 & \textbf{Inadequate Security Awareness Training:} Lack of new hire and annual training leaves employees vulnerable to phishing, social engineering, and malware. This is a primary vector for initial compromise. & Questionnaire & \textcolor{red}{\textbf{Critical}} \\
\addlinespace
RISK-002 & \textbf{Lack of Endpoint MFA:} The absence of MFA on computer logins allows an attacker with stolen credentials to gain direct access to endpoint devices, company data, and the internal network. & Questionnaire & \textcolor{orange}{\textbf{High}} \\
\addlinespace
RISK-003 & \textbf{Missing Acceptable Use Policy (AUP):} Without a formal AUP, there are no clear guidelines for employees on the safe and acceptable use of company assets, leading to inconsistent security practices and potential insider threats. & Questionnaire & \textcolor{orange}{\textbf{High}} \\ \bottomrule
\end{tabular}
\caption{Synthesized Risk Summary}
\end{table}

% --- 6. Recommendations ---
\section{Recommendations}

Based on the analysis, the following actions are recommended to mitigate the identified risks and improve the overall security posture of \textbf{[Organization Name]}. Recommendations are prioritized based on the severity of the associated risk.

\begin{enumerate}
    \item \textbf{Implement a Comprehensive Security Awareness Program (RISK-001):}
    \begin{itemize}
        \item Immediately establish a mandatory security awareness training module for all new employees as part of the onboarding process.
        \item Procure and deploy a security awareness training platform to conduct annual training for all staff.
        \item Conduct regular phishing simulations to test and reinforce employee knowledge.
    \end{itemize}
    \vspace{1em}
    \item \textbf{Enforce Multi-Factor Authentication on All Endpoints (RISK-002):}
    \begin{itemize}
        \item Procure and deploy an MFA solution that integrates with your operating systems (Windows, macOS).
        \item Develop a phased rollout plan to enforce MFA for all employee and contractor computer logins within the next 90 days.
        \item Ensure that MFA is also required for all administrative access.
    \end{itemize}
    \vspace{1em}
    \item \textbf{Develop and Enforce an Acceptable Use Policy (RISK-003):}
    \begin{itemize}
        \item Draft a formal AUP that clearly defines the rules for using company networks, computers, email, and data.
        \item The policy should be reviewed by legal and HR departments before being finalized.
        \item Require all current employees and new hires to read and formally acknowledge the policy.
    \end{itemize}
\end{enumerate}

\end{document}
```