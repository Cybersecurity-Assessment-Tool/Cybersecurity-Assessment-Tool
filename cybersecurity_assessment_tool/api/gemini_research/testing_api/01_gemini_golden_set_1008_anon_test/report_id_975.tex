```latex
\documentclass[12pt, a4paper]{article}

% --- PACKAGES ---
\usepackage[margin=1in]{geometry} % Set page margins
\usepackage{pifont}                 % For checkmarks and crosses (\ding)
\usepackage{booktabs}               % For professional-looking tables
\usepackage{xcolor}                 % For custom colors
\usepackage{hyperref}               % For hyperlinks and document metadata
\usepackage{url}                    % For formatting URLs
\usepackage{seqsplit}               % To split long strings in \texttt
\usepackage{graphicx}               % For including logos (optional)
\usepackage{fancyhdr}               % For headers and footers

% --- DOCUMENT METADATA ---
\hypersetup{
    colorlinks=true,
    linkcolor=blue,
    filecolor=magenta,      
    urlcolor=cyan,
    pdftitle={Cybersecurity Assessment Report},
    pdfauthor={Automated Security Analysis System},
    pdfsubject={Security Posture Analysis},
    pdfkeywords={Cybersecurity, Risk, Assessment, LaTeX},
    bookmarks=true
}

% --- CUSTOM COMMANDS & DEFINITIONS ---
\newcommand{\yes}{\ding{51}} % Green checkmark
\newcommand{\no}{\ding{55}}  % Red X

% Define severity colors
\definecolor{sevCritical}{HTML}{940000}
\definecolor{sevHigh}{HTML}{D14000}
\definecolor{sevMedium}{HTML}{E0C400}
\definecolor{sevLow}{HTML}{008037}

% --- HEADER & FOOTER ---
\pagestyle{fancy}
\fancyhf{} % Clear all header and footer fields
\fancyhead[L]{Cybersecurity Assessment Report}
\fancyhead[R]{\textbf{[Organization Name]}}
\fancyfoot[C]{\thepage}
\renewcommand{\headrulewidth}{0.4pt}
\renewcommand{\footrulewidth}{0.4pt}

% --- DOCUMENT START ---
\begin{document}

% --- TITLE PAGE ---
\begin{titlepage}
    \centering
    \vspace*{1cm}
    
    \Huge
    \textbf{Cybersecurity Assessment Report}
    
    \vspace{1.5cm}
    
    \Large
    Prepared for: \\
    \vspace{0.5cm}
    \textbf{[Organization Name]}
    
    \vspace{2cm}
    
    \large
    Date of Report: \today
    
    \vfill
    
    \normalsize
    \textit{This report contains sensitive information and should be handled with the utmost confidentiality. Access is restricted to authorized personnel only.}
    
\end{titlepage}

\newpage
\tableofcontents
\newpage

% ===================================================================
% SECTION 1: EXECUTIVE OVERVIEW
% ===================================================================
\section{Executive Overview}

This report details the findings of a cybersecurity assessment conducted for \textbf{[Organization Name]}. The analysis combines a review of organizational security controls, an external network scan, and a summary of known risks.

The assessment identified several critical and high-severity risks that require immediate attention. Key findings include:
\begin{itemize}
    \item \textbf{Critical Database Exposure:} An external scan revealed a MySQL database (port 3306) is directly exposed to the public internet. Furthermore, the database is running MySQL version 5.7.33, which reached its official End-of-Life (EOL) in October 2023 and no longer receives security updates.
    \item \textbf{Critical Control Gaps:} The organization does not enforce Multi-Factor Authentication (MFA) for accessing sensitive data systems. This significantly increases the risk of unauthorized access and data breach, especially when combined with the exposed database.
    \item \textbf{High-Risk Training Deficiencies:} The organization does not provide security awareness training for new or existing employees. This lack of training makes the organization highly susceptible to social engineering and phishing attacks, which are primary vectors for credential theft.
\end{itemize}

The combination of an exposed, unpatched database, a lack of MFA on sensitive systems, and an untrained workforce creates a synergistic risk profile that could easily be exploited by attackers. Immediate remediation of these issues is strongly recommended to reduce the likelihood of a significant security incident.

% ===================================================================
% SECTION 2: ORGANIZATIONAL INFORMATION
% ===================================================================
\section{Organizational Information}

This section provides the high-level details of the organization under review. The information was provided prior to the assessment.

\begin{table}[h!]
\centering
\begin{tabular}{@{}ll@{}}
\toprule
\textbf{Attribute} & \textbf{Value} \\ \midrule
Organization Name    & \textbf{[Organization Name]} \\
Primary Domain       & \texttt{[Domain]} \\
External IP Scanned  & \texttt{[Client IP]} \\
Scan Date            & \today \\ \bottomrule
\end{tabular}
\caption{Client Information}
\end{table}

% ===================================================================
% SECTION 3: SECURITY CONTROL REVIEW
% ===================================================================
\section{Security Control Review}

The following table summarizes the organization's responses to a security controls questionnaire. Answers marked with a \no\ represent significant gaps in the organization's defensive posture.

\begin{table}[h!]
\centering
\begin{tabular}{@{}p{0.7\textwidth}cc@{}}
\toprule
\textbf{Control Question} & \textbf{Response} & \textbf{Status} \\ \midrule
Do you require MFA to access email? & Yes & \yes \\
Do you require MFA to log into computers? & Yes & \yes \\
\textbf{Do you require MFA to access sensitive data systems?} & \textbf{No} & \textcolor{red}{\no} \\
Does your organization have an employee acceptable use policy? & Yes & \yes \\
\textbf{Does your organization do security awareness training for new employees?} & \textbf{No} & \textcolor{red}{\no} \\
\textbf{Does your organization do security awareness training for all employees at least once per year?} & \textbf{No} & \textcolor{red}{\no} \\ \bottomrule
\end{tabular}
\caption{Security Controls Questionnaire Results}
\end{label{tab:controls}}
\end{table}

\subsection*{Analysis of Control Gaps}
\begin{itemize}
    \item \textbf{MFA on Sensitive Systems:} The absence of MFA on systems containing sensitive data is a critical vulnerability. Should an attacker compromise a user's credentials, there is no secondary control to prevent them from accessing and exfiltrating critical information.
    \item \textbf{Security Awareness Training:} Without a formal training program, employees are unlikely to recognize or properly respond to phishing attempts, malware, or other social engineering tactics. This makes them the weakest link in the security chain.
\end{itemize}

% ===================================================================
% SECTION 4: TECHNICAL SCAN RESULTS
% ===================================================================
\section{Technical Scan Results}

An external network scan was performed using Nmap against the target IP address. The following table details the open ports and services discovered.

\begin{table}[h!]
\centering
\begin{tabular}{@{}llllll@{}}
\toprule
\textbf{Target IP} & \textbf{Port} & \textbf{State} & \textbf{Service} & \textbf{Version} & \textbf{Notes} \\ \midrule
\texttt{[Target IP]} & 3306/tcp & open & mysql & MySQL 5.7.33 & \textcolor{sevCritical}{\textbf{End-of-Life}} \\ \bottomrule
\end{tabular}
\caption{Nmap Scan Findings}
\end{label{tab:nmap}}
\end{table}

\subsection*{Analysis of Technical Findings}
The scan identified that port 3306 is open, exposing a MySQL database server directly to the internet. This is a highly dangerous configuration, as it allows attackers worldwide to attempt to connect, brute-force credentials, or exploit vulnerabilities.

The version detected, \textbf{MySQL 5.7.33}, is particularly concerning. MySQL 5.7 reached its official End-of-Life (EOL) in October 2023. This means it no longer receives security patches from the vendor, and any newly discovered vulnerabilities will remain unpatched, leaving the system perpetually at risk.

% ===================================================================
% SECTION 5: CONSOLIDATED RISK ASSESSMENT
% ===================================================================
\section{Consolidated Risk Assessment}

This section correlates the findings from the security control review, technical scan, and pre-existing risk data to provide a consolidated view of the organization's security posture.

\begin{table}[h!]
\centering
\begin{tabular}{@{}p{0.3\textwidth}p{0.5\textwidth}l@{}}
\toprule
\textbf{Risk Name} & \textbf{Description} & \textbf{Severity} \\ \midrule
\textbf{Exposed End-of-Life Database} & A MySQL 5.7.33 database is publicly accessible on port 3306. The software is no longer supported with security patches, making it a prime target for exploitation. & \textcolor{sevCritical}{\textbf{Critical (9.8)}} \\
\addlinespace
\textbf{Lack of MFA on Sensitive Systems} & No MFA is enforced for accessing sensitive data. This eliminates a critical security layer and allows for single-factor authentication takeovers. & \textcolor{sevCritical}{\textbf{Critical (9.1)}} \\
\addlinespace
\textbf{Inadequate Security Awareness Training} & Employees are not trained to identify or respond to security threats. This elevates the risk of successful phishing attacks leading to credential compromise. & \textcolor{sevHigh}{\textbf{High (7.2)}} \\ \bottomrule
\end{tabular}
\caption{Summary of Identified Risks}
\label{tab:risks}}
\end{table}

% ===================================================================
% SECTION 6: RECOMMENDATIONS
% ===================================================================
\section{Recommendations}

Based on the consolidated risk assessment, the following prioritized actions are recommended to mitigate the identified vulnerabilities and improve the overall security posture of \textbf{[Organization Name]}.

\subsection*{Priority 1: Immediate Actions (Within 72 Hours)}
\begin{enumerate}
    \item \textbf{Restrict Access to Database:} Immediately implement firewall rules to block all public access to TCP port 3306. Access should be restricted to a whitelist of trusted IP addresses only. If remote access is required, it should be facilitated through a secure VPN connection.
    \item \textbf{Plan Database Upgrade:} Initiate an emergency project to upgrade the MySQL 5.7.33 database to a currently supported version (e.g., MySQL 8.x). This is critical to ensure the system can receive security patches.
\end{enumerate}

\subsection*{Priority 2: High-Priority Actions (Within 30 Days)}
\begin{enumerate}
    \item \textbf{Implement MFA on All Sensitive Systems:} Procure and deploy an MFA solution for all systems classified as containing sensitive data. This control must be enforced for all users, including administrators, without exception.
    \item \textbf{Establish Security Awareness Training Program:} Contract with a reputable provider for security awareness training. Implement a mandatory training module for all new hires during their onboarding process. Schedule and require annual refresher training for all existing employees.
\end{enumerate}

\subsection*{Priority 3: Medium-Priority Actions (Within 90 Days)}
\begin{enumerate}
    \item \textbf{Conduct a Comprehensive Vulnerability Scan:} Perform an authenticated and unauthenticated vulnerability scan across all external and internal assets to identify other potential security weaknesses.
    \item \textbf{Review and Update Policies:} Review the organization's Acceptable Use Policy to ensure it explicitly addresses requirements for MFA and the handling of sensitive data.
\end{enumerate}

\end{document}
```