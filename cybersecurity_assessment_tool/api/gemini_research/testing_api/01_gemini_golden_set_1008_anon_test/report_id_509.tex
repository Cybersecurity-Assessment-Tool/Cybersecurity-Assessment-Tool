```latex
\documentclass[12pt]{article}

% Preamble: Required Packages
\usepackage[margin=1in]{geometry}
\usepackage{pifont} % For checkmarks and crosses (\ding)
\usepackage{booktabs} % For professional tables
\usepackage{hyperref} % For clickable links
\usepackage{url} % For URL formatting
\usepackage{seqsplit} % To split long strings in \texttt
\usepackage{graphicx}
\usepackage{xcolor}
\usepackage{fancyhdr}

% --- Document Setup ---
\hypersetup{
    colorlinks=true,
    linkcolor=blue,
    filecolor=magenta,      
    urlcolor=cyan,
    pdftitle={Cybersecurity Posture Assessment},
    pdfpagemode=FullScreen,
}

\pagestyle{fancy}
\fancyhf{}
\lhead{Cybersecurity Assessment Report}
\rhead{\textbf{[Organization Name]}}
\cfoot{\thepage}

% --- Custom Commands & Colors ---
\definecolor{criticalred}{HTML}{D7263D}
\definecolor{highorange}{HTML}{F49D40}
\definecolor{mediumyellow}{HTML}{F4D440}
\definecolor{lowblue}{HTML}{54A0C4}
\definecolor{infogray}{HTML}{808080}

\newcommand{\severitybox}[2]{%
  \colorbox{#1}{\textcolor{white}{\textbf{\strut #2}}}%
}

% --- Document Start ---
\begin{document}

% --- Title Page ---
\begin{titlepage}
    \centering
    \vspace*{1cm}
    \Huge{\textbf{Cybersecurity Posture Assessment Report}}
    \vspace{1.5cm}
    
    \includegraphics[width=0.4\textwidth]{example-image-a} % Placeholder logo
    
    \vfill
    
    \large
    \textbf{Prepared for:}\\
    \vspace{0.2cm}
    \textbf{[Organization Name]}
    
    \vspace{1.5cm}
    
    \textbf{Date of Report:}\\
    \vspace{0.2cm}
    \today
    
\end{titlepage}

\tableofcontents
\newpage

% --- Section 1: Executive Summary ---
\section{Executive Summary}

This report provides a comprehensive analysis of the cybersecurity posture for \textbf{[Organization Name]}, based on a synthesis of network scan data, a security controls questionnaire, and a review of pre-existing risks. The assessment, conducted on \today, reveals several critical and high-risk vulnerabilities that require immediate attention to mitigate the threat of unauthorized access, data breach, and potential ransomware attacks.

The most critical finding is the direct exposure of Remote Desktop Protocol (RDP) on the external network at \texttt{[Target IP]}. This was validated by our technical scan and correlates with a known critical risk. Exposed RDP is a primary vector for malicious actors. This technical vulnerability is compounded by significant policy and procedural gaps, most notably the lack of Multi-Factor Authentication (MFA) for email access. This combination creates a high-probability attack path where a compromised email account could lead directly to network infiltration.

Other high-risk findings include the absence of an annual security awareness training program for all staff and the lack of a formal Acceptable Use Policy. These foundational elements are essential for cultivating a security-conscious culture and reducing human-centric risk.

This report outlines actionable recommendations to address these findings, prioritizing the immediate remediation of critical vulnerabilities to significantly improve the organization's defensive posture.

% --- Section 2: Organizational Information ---
\section{Organizational Information}

The following details were used as the basis for this assessment. Due to the anonymized nature of the provided data, placeholders have been used.

\begin{itemize}
    \item \textbf{Organization Name:} \textbf{[Organization Name]}
    \item \textbf{Primary Email Domain:} \texttt{[Domain]}
    \item \textbf{Monitored External IP:} \texttt{[Client IP]}
\end{itemize}

% --- Section 3: Security Control Review ---
\section{Security Control Review}

A review of the organization's security controls was conducted via a questionnaire. The responses highlight critical gaps in access control and employee security policies. A "No" response indicates a deviation from security best practices and represents a significant area of risk.

\begin{table}[h!]
\centering
\caption{Security Controls Questionnaire Analysis}
\begin{tabular}{p{0.6\linewidth} c p{0.2\linewidth}}
\toprule
\textbf{Control Question} & \textbf{Response} & \textbf{Assessment} \\
\midrule
Do you require MFA to access email? & \ding{55} & \textbf{Critical Gap} \\
Do you require MFA to log into computers? & \ding{51} & Best Practice Met \\
Do you require MFA to access sensitive data systems? & \ding{51} & Best Practice Met \\
Does your organization have an employee acceptable use policy? & \ding{55} & \textbf{High Risk} \\
Does your organization do security awareness training for new employees? & \ding{51} & Best Practice Met \\
Does your organization do security awareness training for all employees at least once per year? & \ding{55} & \textbf{High Risk} \\
\bottomrule
\end{tabular}
\end{table}

% --- Section 4: Technical Scan Results ---
\section{Technical Scan Results}

An external network scan was performed to identify exposed services and potential vulnerabilities. The scan confirmed the presence of an open port associated with a high-risk service.

\subsection*{Nmap Scan of Target: \texttt{[Target IP]}}
The scan identified one open port on the target system. The details are listed below.

\begin{table}[h!]
\centering
\caption{Open Port Analysis for \texttt{[Target IP]}}
\begin{tabular}{l l l l l}
\toprule
\textbf{Port} & \textbf{Protocol} & \textbf{State} & \textbf{Service Name} & \textbf{Version} \\
\midrule
3389 & TCP & open & ms-wbt-server & N/A \\
\bottomrule
\end{tabular}
\end{table}

\paragraph{Finding Detail:} The service \texttt{ms-wbt-server} on port 3389 is the Microsoft Remote Desktop Protocol (RDP). Direct exposure of RDP to the public internet is a severe security risk. It is one of the most common attack vectors used by threat actors to gain initial access to a network for deploying ransomware or exfiltrating data. This technical finding directly validates the pre-existing risk documented in the risk register.

% --- Section 5: Consolidated Risk Assessment ---
\section{Consolidated Risk Assessment}
This section synthesizes findings from the security control review, technical scan, and pre-existing risk data into a prioritized list.

\begin{table}[h!]
\centering
\caption{Summary of Identified Risks}
\begin{tabular}{p{0.2\linewidth} p{0.25\linewidth} p{0.45\linewidth}}
\toprule
\textbf{Severity} & \textbf{Risk Title} & \textbf{Description} \\
\midrule
\severitybox{criticalred}{Critical (9.0)} & \textbf{Public RDP Exposure} & Port 3389 (RDP) is open on \texttt{[Target IP]}, confirmed by an Nmap scan. This allows attackers to attempt brute-force or credential-stuffing attacks directly against a network entry point. \\
\addlinespace
\severitybox{criticalred}{Critical} & \textbf{Lack of MFA on Email} & The absence of MFA on the \texttt{[Domain]} email system makes user accounts highly susceptible to takeover via phishing or password spraying. Compromised email is a primary pivot point for larger network intrusions. \\
\addlinespace
\severitybox{highorange}{High} & \textbf{Inadequate Security Training} & While new hires receive training, the lack of an annual refresher for all employees leads to knowledge decay and increases susceptibility to social engineering and phishing attacks. \\
\addlinespace
\severitybox{highorange}{High} & \textbf{Missing Acceptable Use Policy (AUP)} & The absence of a formal AUP means there are no clearly defined rules for employees regarding the use of company assets, which can lead to unintentional security incidents and policy enforcement challenges. \\
\bottomrule
\end{tabular}
\end{table}

% --- Section 6: Recommendations ---
\section{Recommendations}
The following actions are recommended to remediate the identified risks. They are prioritized based on severity and potential impact.

\subsection*{Immediate Actions (To Be Completed Within 72 Hours)}
\begin{enumerate}
    \item \textbf{Remediate RDP Exposure:}
    \begin{itemize}
        \item Immediately implement a firewall rule to block all inbound traffic to TCP port 3389 on \texttt{[Target IP]} from the internet.
        \item If remote access is required, it must be placed behind a secure gateway.
    \end{itemize}
    
    \item \textbf{Enable MFA for Email:}
    \begin{itemize}
        \item Enforce mandatory Multi-Factor Authentication (MFA) for all user accounts accessing the \texttt{[Domain]} email system. This is the single most effective control to prevent email account takeovers.
    \end{itemize}
\end{enumerate}

\subsection*{Long-Term Strategic Actions (To Be Completed Within 90 Days)}
\begin{enumerate}
    \item \textbf{Implement a Secure Remote Access Solution:}
    \begin{itemize}
        \item Deploy a Virtual Private Network (VPN) solution with MFA for all remote access to the internal network. This provides an encrypted and authenticated tunnel, eliminating the need for direct RDP exposure.
    \end{itemize}
    
    \item \textbf{Establish a Security Awareness Program:}
    \begin{itemize}
        \item Procure and implement a security awareness training platform.
        \item Mandate that all employees complete a comprehensive training module annually, with regular phishing simulations to reinforce learning.
    \end{itemize}
    
    \item \textbf{Develop and Implement an AUP:}
    \begin{itemize}
        \item Draft a formal Acceptable Use Policy (AUP) that clearly outlines the rules and responsibilities for all users of corporate IT resources.
        \item Ensure all employees read and acknowledge the policy.
    \end{itemize}
\end{enumerate}

\end{document}
```