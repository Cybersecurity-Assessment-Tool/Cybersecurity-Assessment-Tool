```latex
\documentclass[12pt]{article}

% --- PACKAGES ---
\usepackage[margin=1in]{geometry} % Set page margins
\usepackage{pifont}               % For checkmark and X symbols (\ding)
\usepackage{booktabs}             % For professional-looking tables
\usepackage{hyperref}             % For clickable links
\usepackage{url}                  % For formatting URLs
\usepackage{seqsplit}             % To split long monospaced text
\usepackage{graphicx}             % For logos (optional)
\usepackage{xcolor}               % For colors

% --- DOCUMENT SETUP ---
\hypersetup{
    colorlinks=true,
    linkcolor=blue,
    filecolor=magenta,      
    urlcolor=cyan,
    pdftitle={Cybersecurity Posture Assessment},
    pdfpagemode=FullScreen,
}

\newcommand{\yes}{\ding{51}} % Green checkmark
\newcommand{\no}{\ding{55}}  % Red X

\begin{document}

% --- TITLE PAGE ---
\begin{titlepage}
    \centering
    \vspace*{1cm}
    \Huge\textbf{Cybersecurity Posture Assessment Report}
    \vspace{1.5cm}
    \Large
    \textbf{Prepared for:} \\
    \vspace{0.5cm}
    \textbf{[Organization Name]}
    \vspace{2cm}
    \large
    \textbf{Date of Report:} \today \\
    \vspace{0.5cm}
    \textbf{Analysis Period:} Based on data provided.
    \vfill
    \large
    \textbf{Generated by:} \\
    Cybersecurity Analyst
\end{titlepage}

\tableofcontents
\newpage

% --- EXECUTIVE SUMMARY ---
\section*{1. Executive Summary}
This report provides a comprehensive cybersecurity assessment for \textbf{[Organization Name]}, synthesizing data from a network vulnerability scan, a security controls questionnaire, and a list of pre-existing risks.

The assessment reveals several critical and high-risk security gaps that require immediate attention. While the organization has implemented foundational controls such as Multi-Factor Authentication (MFA) for email and computers, and maintains security awareness training, significant vulnerabilities exist.

Key findings include:
\begin{itemize}
    \item \textbf{Critical Vulnerability:} An externally facing FTP server is running a dangerously outdated version of \texttt{vsftpd} (2.3.4), which is known to contain a critical backdoor vulnerability (CVE-2011-2523).
    \item \textbf{High-Risk Configuration:} The same FTP server is configured to allow anonymous, unauthenticated access, exposing the system to unauthorized data access and potential compromise.
    \item \textbf{High-Risk Policy Gap:} MFA is not enforced for accessing sensitive data systems, creating a significant risk of unauthorized access to critical information.
    \item \textbf{Medium-Risk Environment:} The organization continues to operate on an unsupported Windows 7 platform, which no longer receives security updates, increasing susceptibility to known exploits.
\end{itemize}

This report outlines these findings in detail and provides prioritized, actionable recommendations to mitigate the identified risks and strengthen the organization's overall security posture.

% --- ORGANIZATIONAL INFORMATION ---
\section*{2. Organizational Information}
This section details the information provided about the organization. Note that several key identifiers were anonymized in the source data and are represented by placeholders.

\begin{table}[h!]
\centering
\begin{tabular}{@{}ll@{}}
\toprule
\textbf{Attribute} & \textbf{Value} \\ \midrule
Organization Name & \textbf{[Organization Name]} \\
Primary Domain & \texttt{[Domain]} \\
External IP Scanned & \texttt{[Client IP]} \\
Target of Scan & \texttt{[Target IP]} \\ \bottomrule
\end{tabular}
\caption{Client Information.}
\end{table}

% --- SECURITY CONTROL REVIEW ---
\section*{3. Security Control Review (Questionnaire Analysis)}
The following table summarizes the organization's self-reported security controls. "Yes" answers indicate a control is in place, while "No" answers represent a potential security gap.

\begin{table}[h!]
\centering
\begin{tabular}{@{}lc@{}}
\toprule
\textbf{Security Control Question} & \textbf{Status} \\ \midrule
Do you require MFA to access email? & \yes \\
Do you require MFA to log into computers? & \yes \\
\textbf{Do you require MFA to access sensitive data systems?} & \textbf{\no} \\
Does your organization have an employee acceptable use policy? & \yes \\
Does your organization do security awareness training for new employees? & \yes \\
Does your organization do security awareness training for all employees annually? & \yes \\ \bottomrule
\end{tabular}
\caption{Security Controls Questionnaire Results.}
\end{table}

\paragraph{Analysis:}
The organization has successfully implemented several important security controls, including MFA for email and local computer access, as well as a robust security training program. However, the lack of MFA for sensitive data systems is a \textbf{high-risk gap}. This omission significantly weakens data protection for the organization's most critical assets, leaving them vulnerable to compromise if an attacker obtains valid user credentials.

% --- TECHNICAL SCAN RESULTS ---
\section*{4. Technical Scan Results}
An external network scan was performed on the target IP address \texttt{[Target IP]}. The scan identified one open port with a critically vulnerable service.

\begin{table}[h!]
\centering
\begin{tabular}{@{}llllll@{}}
\toprule
\textbf{Port} & \textbf{State} & \textbf{Service} & \textbf{Product} & \textbf{Version} & \textbf{Notes} \\ \midrule
21/tcp & Open & ftp & vsftpd & 2.3.4 & \begin{tabular}[t]{@{}l@{}}\textbf{Critical Vulnerability (CVE-2011-2523)} \\ Anonymous FTP Login Allowed\end{tabular} \\ \bottomrule
\end{tabular}
\caption{Open Ports and Services Detected.}
\end{table}

\paragraph{Analysis:}
The presence of an open FTP port is a concern, but the specific version detected, \texttt{vsftpd 2.3.4}, elevates this to a \textbf{critical emergency}. This version contains a well-documented backdoor that was inserted into the source code, allowing an attacker to gain a command shell on the server by sending a specific string as the username.

Furthermore, the server is configured to allow \textbf{anonymous FTP login}. This is a severe misconfiguration that permits any external entity to connect to the server and potentially access, upload, or download files without any authentication. This could lead to data leakage, malware distribution, or serve as an initial access point for a wider network compromise.

% --- CONSOLIDATED RISK ASSESSMENT ---
\section*{5. Consolidated Risk Assessment}
This section correlates findings from all data sources into a prioritized list of identified risks.

\begin{table}[h!]
\centering
\begin{tabular}{@{}llll@{}}
\toprule
\textbf{ID} & \textbf{Risk Description} & \textbf{Severity} & \textbf{Source} \\ \midrule
RISK-001 & \begin{tabular}[t]{@{}l@{}}Vulnerable FTP server (\texttt{vsftpd 2.3.4}) exposed \\ to the internet, containing a known backdoor.\end{tabular} & \textbf{Critical} & Network Scan \\
\addlinespace
RISK-002 & \begin{tabular}[t]{@{}l@{}}Lack of Multi-Factor Authentication (MFA) \\ on systems containing sensitive data.\end{tabular} & High & Questionnaire \\
\addlinespace
RISK-003 & \begin{tabular}[t]{@{}l@{}}Anonymous FTP login is enabled, allowing \\ unauthenticated access to the file system.\end{tabular} & High & Network Scan \\
\addlinespace
RISK-004 & \begin{tabular}[t]{@{}l@{}}Workstations are running Windows 7, an \\ unsupported OS that no longer receives security updates.\end{tabular} & Medium & Existing Risks \\ \bottomrule
\end{tabular}
\caption{Summary of Identified Risks.}
\end{table}

% --- RECOMMENDATIONS ---
\section*{6. Recommendations}
The following actions are recommended to mitigate the identified risks, prioritized by severity.

\subsection*{Immediate Actions (Critical Priority)}
\begin{enumerate}
    \item \textbf{Remediate Vulnerable FTP Server (RISK-001):}
    \begin{itemize}
        \item Immediately take the FTP server at \texttt{[Target IP]} offline.
        \item If the service is not business-critical, it should be permanently decommissioned.
        \item If the service is required, it must be updated to the latest, patched version of \texttt{vsftpd} or replaced with a secure file transfer alternative like SFTP (SSH File Transfer Protocol).
    \end{itemize}
    
    \item \textbf{Disable Anonymous FTP Access (RISK-003):}
    \begin{itemize}
        \item As part of the remediation for RISK-001, ensure that anonymous login is disabled. All access must require strong, unique user credentials.
    \end{itemize}
\end{enumerate}

\subsection*{High Priority Actions}
\begin{enumerate}
    \setcounter{enumi}{2} % Continue numbering
    \item \textbf{Implement MFA on Sensitive Systems (RISK-002):}
    \begin{itemize}
        \item Conduct an inventory of all systems classified as containing sensitive data.
        \item Prioritize and implement a robust MFA solution for access to these systems for all users, including administrators.
    \end{itemize}
\end{enumerate}

\subsection*{Medium Priority Actions}
\begin{enumerate}
    \setcounter{enumi}{3} % Continue numbering
    \item \textbf{Upgrade Outdated Operating Systems (RISK-004):}
    \begin{itemize}
        \item Develop and execute a plan to upgrade all remaining Windows 7 workstations to a modern, supported operating system (e.g., Windows 10/11).
        \item This will ensure that devices receive critical security patches and are protected against known vulnerabilities.
    \end{itemize}
\end{enumerate}

\end{document}
```