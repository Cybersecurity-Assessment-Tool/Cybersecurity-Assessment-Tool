```latex
\documentclass[12pt]{article}

% Preamble: Required Packages
\usepackage[margin=1in]{geometry}
\usepackage{pifont} % For checkmarks and crosses
\usepackage{booktabs} % For professional tables
\usepackage{hyperref} % For clickable links and better PDF navigation
\usepackage{url}      % For formatting URLs
\usepackage{seqsplit} % To split long strings in texttt
\usepackage[T1]{fontenc}

% Document Metadata
\hypersetup{
    colorlinks=true,
    linkcolor=black,
    filecolor=magenta,      
    urlcolor=blue,
    pdftitle={Cybersecurity Posture Assessment Report},
    pdfauthor={Cybersecurity Analyst},
    pdfsubject={Security Analysis},
    pdfkeywords={Security, Report, Analysis},
    bookmarks=true
}

\author{Cybersecurity Analyst}
\title{Cybersecurity Posture Assessment Report \\ \large For: \textbf{[Organization Name]}}
\date{\today}

\begin{document}

\maketitle

\begin{abstract}
\noindent This report provides a comprehensive cybersecurity assessment for \textbf{[Organization Name]}. The analysis is based on a synthesis of network scan data, a review of organizational security controls via a questionnaire, and an evaluation of pre-existing risks. The assessment identifies several critical and high-risk gaps in the current security posture, alongside areas of strength. Key findings include the lack of Multi-Factor Authentication (MFA) for critical systems, deficiencies in the employee security training program, and a publicly exposed administrative service. This document outlines these risks and provides prioritized, actionable recommendations to mitigate them and enhance the organization's overall resilience against cyber threats.
\end{abstract}

\newpage

\tableofcontents

\newpage

\section{Overview and Scope}

The primary objective of this assessment is to provide a clear and concise overview of the organization's current security posture. The scope of this report is limited to the analysis of the data provided:
\begin{itemize}
    \item \textbf{Organizational Data:} A security questionnaire assessing policies and controls.
    \item \textbf{Technical Network Scan:} An external Nmap scan targeting the organization's public-facing infrastructure.
    \item \textbf{Current Risk Register:} A list of known, pre-existing vulnerabilities.
\end{itemize}
This report correlates findings from these sources to create a unified view of risk and offers strategic recommendations for remediation.

\section{Organizational Information}
The following details were used as the basis for this assessment. Due to the anonymized nature of the input data, placeholders have been used where necessary.

\begin{itemize}
    \item \textbf{Organization Name:} \textbf{[Organization Name]}
    \item \textbf{Primary Domain:} \texttt{[Domain]}
    \item \textbf{Target External IP:} \texttt{[Client IP]}
\end{itemize}

\section{Security Control Review}
The following table summarizes the organization's responses to the security controls questionnaire. Each response is assessed against industry best practices. "No" answers indicate significant gaps in the security framework.

\begin{table}[h!]
\centering
\caption{Security Controls Questionnaire Analysis}
\label{tab:controls}
\begin{tabular}{@{}p{0.6\linewidth} c p{0.2\linewidth}@{}}
\toprule
\textbf{Control Question} & \textbf{Response} & \textbf{Assessment} \\
\midrule
Do you require MFA to access email? & \ding{51} Yes & Positive Control \\
\addlinespace
Do you require MFA to log into computers? & \ding{55} No & \textbf{Critical Gap} \\
\addlinespace
Do you require MFA to access sensitive data systems? & \ding{55} No & \textbf{Critical Gap} \\
\addlinespace
Does your organization have an employee acceptable use policy? & \ding{51} Yes & Positive Control \\
\addlinespace
Does your organization do security awareness training for new employees? & \ding{55} No & \textbf{High Risk} \\
\addlinespace
Does your organization do security awareness training for all employees at least once per year? & \ding{51} Yes & Positive Control \\
\bottomrule
\end{tabular}
\end{table}

\subsection*{Analysis of Control Gaps}
The questionnaire reveals critical weaknesses in identity and access management and employee onboarding procedures.
\begin{itemize}
    \item \textbf{Lack of MFA:} The absence of MFA on computer logins and sensitive data systems exposes the organization to significant risk from credential theft, phishing, and brute-force attacks. A single compromised password could lead to widespread system access and data breach.
    \item \textbf{Onboarding Training Gap:} Failing to provide security awareness training to new employees leaves a vulnerable window where new hires, who are often targeted by social engineering attacks, are unaware of organizational policies and common threats.
\end{itemize}

\section{Technical Scan Results}
An external network scan was performed to identify open ports and exposed services on the organization's public-facing infrastructure.

\begin{itemize}
    \item \textbf{Target IP Address:} \texttt{[Target IP]}
    \item \textbf{Scan Date:} Scan data processed on \today.
\end{itemize}

\begin{table}[h!]
\centering
\caption{Open Ports Detected on \texttt{[Target IP]}}
\label{tab:ports}
\begin{tabular}{@{}llll@{}}
\toprule
\textbf{Port} & \textbf{State} & \textbf{Service (Presumed)} & \textbf{Notes} \\
\midrule
22/tcp & open & SSH (Secure Shell) & Administrative access protocol. \\
\bottomrule
\end{tabular}
\end{table}

\subsection*{Analysis of Technical Findings}
The scan identified that port 22 (SSH) is open to the public internet. While SSH is a standard protocol for secure remote administration, its public exposure is a security risk. It provides a direct target for automated brute-force attacks that attempt to guess usernames and passwords. Without compensating controls such as IP address whitelisting, strong password policies, key-based authentication, and intrusion detection/prevention systems (e.g., fail2ban), this service is a significant entry point for attackers.

\section{Consolidated Risk Assessment}
This section synthesizes the findings from the security control review, technical scan, and the provided list of current risks. The provided risk list was empty, so all risks below are new findings from this assessment.

\begin{table}[h!]
\centering
\caption{Summary of Identified Risks}
\label{tab:risks}
\begin{tabular}{@{}p{0.1\linewidth} p{0.5\linewidth} l l@{}}
\toprule
\textbf{Risk ID} & \textbf{Description} & \textbf{Severity} & \textbf{Source} \\
\midrule
R-01 & Lack of Multi-Factor Authentication (MFA) on employee computers and sensitive data systems, enabling credential-based attacks. & \textbf{High} & Questionnaire \\
\addlinespace
R-02 & No mandatory security awareness training for new employees during onboarding, increasing susceptibility to social engineering. & \textbf{High} & Questionnaire \\
\addlinespace
R-03 & Publicly exposed SSH service (Port 22), increasing the risk of unauthorized access via brute-force attacks. & \textbf{Medium} & Network Scan \\
\bottomrule
\end{tabular}
\end{table}

\section{Recommendations}
The following prioritized recommendations are provided to address the identified risks and strengthen the organization's security posture.

\subsection*{Priority 1: Remediate Critical Gaps}
\begin{enumerate}
    \item \textbf{Implement Comprehensive MFA (Risk R-01):}
    \begin{itemize}
        \item \textbf{Action:} Deploy a robust MFA solution across all endpoints (employee computer logins) and for all access to systems storing or processing sensitive data.
        \item \textbf{Impact:} Drastically reduces the risk of unauthorized access from compromised credentials. This is the single most effective control to mitigate a wide range of common cyberattacks.
    \end{itemize}
    \item \textbf{Establish Onboarding Security Training (Risk R-02):}
    \begin{itemize}
        \item \textbf{Action:} Integrate a mandatory security awareness training module into the new employee onboarding process. This training should cover the acceptable use policy, phishing identification, password hygiene, and incident reporting procedures.
        \item \textbf{Impact:} Reduces the likelihood of new employees falling victim to social engineering attacks and ensures they are aware of their security responsibilities from day one.
    \end{itemize}
\end{enumerate}

\subsection*{Priority 2: Harden External Services}
\begin{enumerate}
    \setcounter{enumi}{2}
    \item \textbf{Secure the Exposed SSH Service (Risk R-03):}
    \begin{itemize}
        \item \textbf{Action:} Review the business necessity for public SSH access.
            \begin{itemize}
                \item If not required, disable the firewall rule allowing public access.
                \item If required, implement compensating controls: restrict access to known, trusted IP addresses via firewall rules, enforce public key authentication (disabling password-based logins), and deploy a tool like `fail2ban` to block malicious IPs.
            \end{itemize}
        \item \textbf{Impact:} Minimizes the attack surface and protects a key administrative entry point from automated attacks.
    \end{itemize}
\end{enumerate}

\section{Conclusion}
This assessment concludes that while \textbf{[Organization Name]} has implemented some foundational security controls, such as MFA for email and an annual training program, there are critical and high-risk vulnerabilities that require immediate attention. The lack of MFA on computers and sensitive systems, combined with gaps in the onboarding process, creates significant exposure. The publicly accessible SSH port further expands the attack surface.

By implementing the prioritized recommendations outlined in this report, the organization can substantially improve its security posture, reduce its risk profile, and build a more resilient defense against modern cyber threats.

\end{document}
```