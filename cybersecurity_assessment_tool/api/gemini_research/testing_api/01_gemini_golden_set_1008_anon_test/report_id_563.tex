```latex
\documentclass[12pt]{article}

% Preamble: Required Packages
\usepackage[margin=1in]{geometry}
\usepackage{pifont} % For checkmarks and crosses (\ding)
\usepackage{booktabs} % For professional tables
\usepackage{hyperref} % For clickable links and metadata
\usepackage{url} % For typesetting URLs
\usepackage{seqsplit} % For splitting long strings to prevent overflow
\usepackage{xcolor} % For custom colors

% --- Document Metadata and Hyperlink Setup ---
\hypersetup{
    colorlinks=true,
    linkcolor=blue,
    filecolor=magenta,
    urlcolor=cyan,
    pdftitle={Cybersecurity Posture Assessment Report},
    pdfauthor={Cybersecurity Analysis Division},
    pdfsubject={Security Analysis},
    pdfkeywords={Cybersecurity, Nmap, Risk Assessment}
}

% --- Custom Commands for Readability ---
\newcommand{\yes}{\textcolor{green}{\ding{51}}} % Green checkmark for "Yes"
\newcommand{\no}{\textcolor{red}{\ding{55}}}   % Red X for "No"

% --- Document Start ---
\begin{document}

% --- Title Page ---
\title{Cybersecurity Posture Assessment Report \\ \large For \textbf{[Organization Name]}}
\author{Cybersecurity Analysis Division}
\date{\today}
\maketitle

\hrule
\vspace{1em}
\begin{abstract}
This report provides a comprehensive cybersecurity posture assessment for \textbf{[Organization Name]}. The analysis is based on a synthesis of network scan data, a security controls questionnaire, and a review of pre-existing risks. The assessment identifies critical gaps in administrative controls, including the absence of an acceptable use policy and security training for new hires. Furthermore, a technical scan revealed a publicly exposed management port (SSH). These findings, combined with a lack of multi-factor authentication on employee computers and a pre-existing critical vulnerability, present a significant risk to the organization's security and integrity. This document details these risks and provides prioritized, actionable recommendations for remediation.
\end{abstract}
\hrule
\vspace{2em}

\tableofcontents
\newpage

% --- Section 1: Overview and Scope ---
\section{Organizational Information}
This assessment was conducted to evaluate the security posture of \textbf{[Organization Name]}. The information below was used as the basis for this report. Due to the anonymized nature of the provided data, placeholders have been used where necessary.

\begin{table}[h!]
\centering
\caption{Client Information}
\begin{tabular}{@{}ll@{}}
\toprule
\textbf{Attribute} & \textbf{Value} \\ \midrule
Organization Name & \textbf{[Organization Name]} \\
Primary Domain & \texttt{[Domain]} \\
External IP Address Scanned & \seqsplit{\texttt{[Client IP]}} \\
Target IP Address Scanned & \seqsplit{\texttt{[Target IP]}} \\
\bottomrule
\end{tabular}
\end{table}

% --- Section 2: Security Control Review ---
\section{Security Control Review (Questionnaire Analysis)}
A review of the organization's administrative and technical security controls was conducted via a questionnaire. The responses highlight several areas of concern where security best practices are not being met. These gaps significantly increase the organization's attack surface, particularly concerning insider threats and account compromise.

\begin{table}[h!]
\centering
\caption{Security Controls Questionnaire Results}
\begin{tabular}{@{}p{0.75\linewidth}c@{}}
\toprule
\textbf{Control Question} & \textbf{Implemented} \\ \midrule
Do you require MFA to access email? & \yes \\
Do you require MFA to log into computers? & \no \\
Do you require MFA to access sensitive data systems? & \yes \\
Does your organization have an employee acceptable use policy? & \no \\
Does your organization do security awareness training for new employees? & \no \\
Does your organization do security awareness training for all employees at least once per year? & \yes \\
\bottomrule
\end{tabular}
\end{table}

\paragraph{Analysis:} The lack of mandatory MFA for computer logins is a high-risk gap. If an employee's credentials are stolen, an attacker could gain direct access to a company workstation. The absence of an acceptable use policy and security training for new employees represents a critical failure in administrative controls, leaving the organization vulnerable to unintentional policy violations and making new hires prime targets for social engineering attacks.

% --- Section 3: Technical Scan Results ---
\section{Technical Scan Results}
An external network scan was performed on the target IP address \seqsplit{\texttt{[Target IP]}}. The scan identified the following open ports and services accessible from the public internet.

\begin{table}[h!]
\centering
\caption{Open Ports Detected on \seqsplit{\texttt{[Target IP]}}}
\begin{tabular}{@{}lllll@{}}
\toprule
\textbf{Port} & \textbf{Protocol} & \textbf{State} & \textbf{Service} & \textbf{Notes} \\ \midrule
22 & TCP & open & ssh & Secure Shell (SSH) is a critical management \\
& & & & protocol. No version information was available. \\
& & & & Public exposure is not recommended. \\
\bottomrule
\end{tabular}
\end{table}

\paragraph{Analysis:} The presence of an open SSH port on an external-facing IP address is a significant finding. While necessary for remote administration, it is a high-value target for attackers who will attempt to brute-force credentials. This risk is amplified by the lack of MFA on computer logins, as a single compromised password could lead to a system breach.

% --- Section 4: Consolidated Risk Assessment ---
\section{Consolidated Risk Assessment}
The following table synthesizes findings from the security questionnaire, technical scan, and pre-existing risk data. Each identified risk has been assigned a severity level based on its potential impact and likelihood of exploitation.

\begin{table}[h!]
\centering
\caption{Summary of Identified Risks}
\begin{tabular}{@{}p{0.1\linewidth}p{0.4\linewidth}p{0.15\linewidth}p{0.25\linewidth}@{}}
\toprule
\textbf{Risk ID} & \textbf{Description} & \textbf{Severity} & \textbf{Affected Asset(s)} \\ \midrule
\textbf{RISK-001} & \textbf{Pre-existing Critical Vulnerability:} A "Localhost Exposed" vulnerability with a CVSS score of 10.0 was previously identified. & \textbf{Critical} & \seqsplit{\texttt{[Target IP]}} \\
\addlinespace
\textbf{RISK-002} & \textbf{Lack of Foundational Policies:} The organization lacks an employee acceptable use policy, leading to inconsistent security practices. & \textbf{High} & Entire Organization \\
\addlinespace
\textbf{RISK-003} & \textbf{No MFA on Workstations:} User computers are not protected by Multi-Factor Authentication, allowing for takeovers with stolen credentials. & \textbf{High} & All Workstations, User Accounts \\
\addlinespace
\textbf{RISK-004} & \textbf{Inadequate Onboarding Training:} New employees do not receive security awareness training, making them highly susceptible to phishing and social engineering. & \textbf{High} & New Employees, User Accounts \\
\addlinespace
\textbf{RISK-005} & \textbf{Exposed SSH Management Port:} The SSH service on port 22 is publicly accessible, exposing a key administrative interface to attack. & \textbf{Medium} & Server at \seqsplit{\texttt{[Target IP]}} \\
\bottomrule
\end{tabular}
\end{table}

\paragraph{Risk Correlation:} The risks identified are interconnected. For example, an attacker could exploit \textbf{RISK-004} (Inadequate Training) to steal credentials via phishing, then use those credentials to exploit \textbf{RISK-003} (No MFA on Workstations) and gain access to the network, potentially leveraging that access to exploit \textbf{RISK-005} (Exposed SSH) or \textbf{RISK-001} (Critical Vulnerability).

% --- Section 5: Recommendations ---
\section{Recommendations}
Based on the analysis, the following actions are recommended to mitigate the identified risks. Recommendations are prioritized from most to least critical.

\begin{enumerate}
    \item \textbf{Immediately Remediate Critical Vulnerability (RISK-001):} The "Localhost Exposed" vulnerability (CVSS 10.0) must be investigated and remediated as the top priority. This level of risk poses an imminent threat to the organization.

    \item \textbf{Develop and Implement an Acceptable Use Policy (AUP) (RISK-002):} Establish a formal AUP that defines the rules for using company IT assets. Ensure all employees, including new hires, read and acknowledge this policy as part of their onboarding.

    \item \textbf{Enforce Multi-Factor Authentication (MFA) on All Workstations (RISK-003):} Mandate the use of MFA for all computer and remote access logins. This single control dramatically reduces the risk of account compromise from stolen credentials.

    \item \textbf{Implement Security Training for New Hires (RISK-004):} Integrate mandatory security awareness training into the employee onboarding process. This training should cover phishing, password hygiene, and the new AUP.

    \item \textbf{Secure the Exposed SSH Port (RISK-005):} Restrict access to the SSH port. The preferred methods are:
    \begin{itemize}
        \item Implementing a firewall rule to allow access only from trusted IP addresses (IP whitelisting).
        \item Placing the service behind a Virtual Private Network (VPN) or a bastion host.
        \item Disabling password-based authentication in favor of public key cryptography.
    \end{itemize}
\end{enumerate}

\end{document}
```