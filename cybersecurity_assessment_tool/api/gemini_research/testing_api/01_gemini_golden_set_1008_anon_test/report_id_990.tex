```latex
\documentclass[12pt]{article}

% Preamble: Required Packages
\usepackage[a4paper, margin=1in]{geometry} % Page layout
\usepackage{pifont} % For checkmarks and crosses
\usepackage{booktabs} % For professional tables
\usepackage{hyperref} % For hyperlinks
\usepackage{url} % For URL formatting
\usepackage{seqsplit} % For splitting long strings without spaces
\usepackage{graphicx} % For logo (placeholder)
\usepackage{fancyhdr} % For header/footer
\usepackage{lastpage} % To get total page count
\usepackage{xcolor} % For colors

% --- Document Metadata and Hyperlink Setup ---
\hypersetup{
    colorlinks=true,
    linkcolor=blue,
    filecolor=magenta,      
    urlcolor=cyan,
    pdftitle={Cybersecurity Posture Assessment Report},
    pdfauthor={Cybersecurity Analyst},
    pdfsubject={Security Analysis},
    pdfkeywords={Cybersecurity, Risk Assessment, Nmap, LaTeX},
    bookmarks=true
}

% --- Custom Commands ---
\newcommand{\yes}{\ding{51}} % Green checkmark
\newcommand{\no}{\ding{55}}  % Red X

% --- Header and Footer ---
\pagestyle{fancy}
\fancyhf{} % Clear all header and footer fields
\fancyhead[L]{\textbf{Cybersecurity Posture Assessment}}
\fancyhead[R]{\textbf{[Organization Name]}}
\fancyfoot[C]{Page \thepage\ of \pageref{LastPage}}
\fancyfoot[R]{\today}
\renewcommand{\headrulewidth}{0.4pt}
\renewcommand{\footrulewidth}{0.4pt}

\begin{document}

% --- Title Page ---
\begin{titlepage}
    \centering
    \vspace*{2cm}
    
    {\Huge \textbf{Cybersecurity Posture Assessment Report}\par}
    \vspace{1.5cm}
    
    {\Large \textbf{Prepared for:}} \\
    \vspace{0.5cm}
    {\Huge \textbf{[Organization Name]}}\par
    
    \vfill % Pushes content to the bottom
    
    {\large \textbf{Date of Report:} \today}\par
    {\large \textbf{Author:} Expert Cybersecurity Analyst}\par
    
    \vspace{1cm}
    \rule{\linewidth}{0.5pt}
    \par
    \small \textit{This document contains sensitive information and is intended for the exclusive use of the recipient. Unauthorized distribution is prohibited.}
\end{titlepage}

\tableofcontents
\newpage

% --- 1. Executive Summary ---
\section{Executive Summary}

This report provides a comprehensive cybersecurity posture assessment for \textbf{[Organization Name]}, based on an analysis of organizational security controls, an external network scan, and a review of previously identified risks.

The assessment reveals several \textbf{critical gaps} in administrative and access controls that present a high level of risk to the organization. The most significant findings are the lack of Multi-Factor Authentication (MFA) for email and sensitive data systems, and the complete absence of an employee security awareness training program and an acceptable use policy. These deficiencies significantly increase the organization's vulnerability to social engineering, phishing attacks, and potential data breaches.

On a positive note, the technical network scan of the external IP address \texttt{[Client IP]} did not identify any open ports. This indicates that a previously identified risk concerning an unencrypted web server on port 80 has been successfully remediated.

Immediate action is required to address the identified policy and access control weaknesses. Prioritized recommendations are provided in Section \ref{sec:recommendations} to guide remediation efforts and strengthen the overall security posture.

% --- 2. Organizational Information ---
\section{Organizational Information}
This section details the organizational data used as a basis for this assessment. As the provided information was anonymized, placeholders have been used.

\begin{tabular}{@{}ll}
    \toprule
    \textbf{Attribute} & \textbf{Value} \\
    \midrule
    Organization Name & \textbf{[Organization Name]} \\
    Primary Email Domain & \texttt{[Domain]} \\
    External IP Address Scanned & \texttt{[Client IP]} \\
    \bottomrule
\end{tabular}

% --- 3. Security Control Review ---
\section{Security Control Review}
The following table summarizes the organization's responses to a security controls questionnaire. Each "No" response represents a significant gap in the defensive posture and has been flagged with an analyst note.

\begin{table}[h!]
\centering
\caption{Security Controls Questionnaire Analysis}
\label{tab:controls}
\begin{tabular}{@{}p{0.6\linewidth} c p{0.2\linewidth}@{}}
    \toprule
    \textbf{Control Question} & \textbf{Status} & \textbf{Analyst Note} \\
    \midrule
    Do you require MFA to access email? & \no & \textbf{Critical Gap} \\
    Do you require MFA to log into computers? & \yes & Good Practice \\
    Do you require MFA to access sensitive data systems? & \no & \textbf{Critical Gap} \\
    Does your organization have an employee acceptable use policy? & \no & High Risk \\
    Does your organization do security awareness training for new employees? & \no & \textbf{Critical Gap} \\
    Does your organization do security awareness training for all employees at least once per year? & \no & \textbf{Critical Gap} \\
    \bottomrule
\end{tabular}
\end{table}

The analysis indicates severe deficiencies in foundational security controls. The lack of MFA on email and sensitive systems exposes the organization to account takeover and data exfiltration. The absence of policies and training leaves the organization highly susceptible to human error and social engineering attacks.

% --- 4. Technical Scan Results ---
\section{Technical Scan Results}
An external network scan was performed using Nmap against the target IP address. The objective was to identify open ports and exposed services that could be exploited by an attacker.

\begin{itemize}
    \item \textbf{Target IP:} \texttt{[Target IP]} (Placeholder for \texttt{[Client IP]})
    \item \textbf{Scan Type:} TCP Port Scan
    \item \textbf{Scan Date:} \today
\end{itemize}

\subsection{Findings}
The scan results were positive, indicating a strong network perimeter configuration for the scanned host. \textbf{No open ports were discovered.} The status of common web ports is detailed below.

\begin{table}[h!]
\centering
\caption{Nmap Scan Results for \texttt{[Target IP]}}
\label{tab:nmap}
\begin{tabular}{@{}llll@{}}
    \toprule
    \textbf{Port} & \textbf{Protocol} & \textbf{State} & \textbf{Service} \\
    \midrule
    80 & tcp & closed & http \\
    \bottomrule
\end{tabular}
\end{table}

\textbf{Analyst Note:} A previously documented risk, "Unencrypted Web Server," indicated that port 80 was open. This scan validates that the finding is no longer present and the risk has been remediated. This is a positive security improvement.

% --- 5. Risk Assessment Summary ---
\section{Risk Assessment Summary}
This section synthesizes findings from the security control review, technical scan, and pre-existing risk data into a consolidated list of current risks.

\begin{table}[h!]
\centering
\caption{Consolidated Risk Register}
\label{tab:risks}
\begin{tabular}{@{}p{0.3\linewidth} p{0.4\linewidth} l l@{}}
    \toprule
    \textbf{Risk Name} & \textbf{Description} & \textbf{Severity} & \textbf{Status} \\
    \midrule
    \textbf{No MFA on Critical Systems} & Email and sensitive data systems lack Multi-Factor Authentication, exposing them to account compromise. & \textbf{Critical} & \textbf{Active} \\
    \addlinespace
    \textbf{Lack of Security Training} & Employees receive no security awareness training, increasing susceptibility to phishing and social engineering. & \textbf{Critical} & \textbf{Active} \\
    \addlinespace
    \textbf{No Acceptable Use Policy} & The absence of a formal AUP creates ambiguity regarding secure employee conduct and use of IT assets. & High & \textbf{Active} \\
    \addlinespace
    Unencrypted Web Server & Port 80 was believed to be open, exposing unencrypted traffic. & Medium & \textbf{Remediated} \\
    \bottomrule
\end{tabular}
\end{table}

% --- 6. Recommendations ---
\section{Recommendations}
\label{sec:recommendations}
The following actions are recommended to mitigate the identified risks. They are prioritized based on severity and potential impact on the organization.

\subsection{Priority 1: Critical Risks}
\begin{enumerate}
    \item \textbf{Implement Multi-Factor Authentication (MFA):}
    \begin{itemize}
        \item \textbf{Action:} Immediately enforce MFA for all user access to email systems (e.g., Microsoft 365, Google Workspace) and any system identified as containing sensitive data.
        \item \textbf{Impact:} Drastically reduces the risk of account takeover via stolen credentials, which is a primary vector in data breaches.
    \end{itemize}
    \vspace{0.5cm}
    \item \textbf{Establish Security Awareness Training Program:}
    \begin{itemize}
        \item \textbf{Action:} Procure and implement a security awareness training solution. Mandate foundational training for all new employees upon hiring and annual refresher training for all staff.
        \item \textbf{Impact:} Creates a human firewall, empowering employees to recognize and report threats like phishing, thereby reducing the likelihood of a successful attack.
    \end{itemize}
\end{enumerate}

\subsection{Priority 2: High Risks}
\begin{enumerate}
    \setcounter{enumi}{2}
    \item \textbf{Develop and Enforce an Acceptable Use Policy (AUP):}
    \begin{itemize}
        \item \textbf{Action:} Draft a formal AUP that clearly defines the rules for using company IT assets, data handling, and internet usage. Require all employees to read and acknowledge the policy.
        \item \textbf{Impact:} Establishes a clear baseline for secure behavior, reduces insider risk, and provides a basis for disciplinary action in case of policy violations.
    \end{itemize}
\end{enumerate}

\subsection{Priority 3: Informational}
\begin{enumerate}
    \setcounter{enumi}{3}
    \item \textbf{Acknowledge and Maintain Network Hardening:}
    \begin{itemize}
        \item \textbf{Action:} Formally acknowledge the successful remediation of the open port 80 vulnerability. Implement a regular, automated vulnerability scanning schedule to ensure the external perimeter remains secure.
        \item \textbf{Impact:} Validates security efforts and promotes a proactive approach to vulnerability management.
    \end{itemize}
\end{enumerate}

% --- 7. Conclusion ---
\section{Conclusion}
While \textbf{[Organization Name]} has demonstrated a positive security improvement by securing its external network perimeter, significant and critical weaknesses exist in its administrative and access control policies. The current state leaves the organization highly exposed to common cyber threats.

By implementing the critical and high-priority recommendations outlined in this report—specifically enforcing MFA, establishing a security training program, and formalizing an AUP—the organization can substantially reduce its risk profile and build a more resilient security foundation.

\end{document}
```