```latex
\documentclass[12pt]{article}

% Preamble: Required Packages
\usepackage[margin=1in]{geometry}
\usepackage{pifont} % For checkmarks and crosses
\usepackage{booktabs} % For professional tables
\usepackage{hyperref} % For clickable links and ToC
\usepackage{url} % For formatting URLs
\usepackage{seqsplit} % For splitting long strings in tt font
\usepackage{graphicx}
\usepackage{xcolor}
\usepackage{fancyhdr}

% --- Document Setup ---
\hypersetup{
    colorlinks=true,
    linkcolor=blue,
    filecolor=magenta,      
    urlcolor=cyan,
    pdftitle={Cybersecurity Posture Assessment Report},
    pdfpagemode=FullScreen,
}

\pagestyle{fancy}
\fancyhf{}
\lhead{Cybersecurity Posture Assessment}
\rhead{\textbf{[Organization Name]}}
\cfoot{\thepage}

% --- Document Body ---
\begin{document}

% --- Title Page ---
\begin{titlepage}
    \centering
    \vspace*{1cm}
    \Huge\textbf{Cybersecurity Posture Assessment Report}
    \vspace{1.5cm}
    \vfill
    \large
    \textbf{Prepared for:}\\
    \vspace{0.2cm}
    \textbf{[Organization Name]}
    \vspace{2cm}
    \textbf{Date of Report:}\\
    \vspace{0.2cm}
    \today
    \vfill
    \textit{This report contains sensitive information and should be handled with care.}
\end{titlepage}

\tableofcontents
\newpage

% --- Section 1: Executive Summary ---
\section{Executive Summary}
This report provides a comprehensive analysis of the cybersecurity posture for \textbf{[Organization Name]}, based on a review of organizational security controls, an external network scan, and pre-existing risk data. The assessment was conducted to identify key vulnerabilities, policy gaps, and technical misconfigurations that could expose the organization to cyber threats.

The analysis reveals a mixed security posture. The organization has implemented critical controls such as Multi-Factor Authentication (MFA) for email and sensitive data systems, alongside a security awareness training program. These are commendable and foundational security measures.

However, several critical and high-risk gaps were identified that require immediate attention:
\begin{itemize}
    \item \textbf{Critical Pre-existing Risk:} A vulnerability identified as "Localhost Exposed" with a CVSS score of 10.0 represents an immediate and severe threat that must be prioritized.
    \item \textbf{Critical Policy Gap:} The absence of a formal Employee Acceptable Use Policy (AUP) creates significant ambiguity and risk regarding the proper use of company assets, increasing the likelihood of insider threats and unintentional security incidents.
    \item \textbf{High-Risk Endpoint Weakness:} The lack of mandatory MFA for computer logins exposes endpoints to compromise via stolen or weak credentials, which could grant an attacker a significant foothold within the internal network.
    \item \textbf{Exposed Management Service:} The external network scan identified an open SSH port (22), which is a common target for brute-force and credential-based attacks.
\end{itemize}

This report details these findings and provides actionable recommendations to mitigate the identified risks, strengthen the organization's defenses, and improve the overall security posture.

% --- Section 2: Organizational Information ---
\section{Organizational Information}
This section outlines the basic information used as the basis for this assessment. Due to the anonymized nature of the provided data, placeholders are used.

\begin{tabular}{@{}ll}
    \toprule
    \textbf{Attribute} & \textbf{Value} \\
    \midrule
    Organization Name & \textbf{[Organization Name]} \\
    Email Domain & \texttt{[Domain]} \\
    External IP Address Scanned & \texttt{[Client IP]} \\
    \bottomrule
\end{tabular}

% --- Section 3: Security Control Review ---
\section{Security Control Review}
A review of the organization's security controls was conducted via a questionnaire. The responses highlight both strengths and weaknesses in the current security program. "No" answers indicate significant gaps that increase organizational risk.

\begin{table}[h!]
\centering
\caption{Security Controls Questionnaire Results}
\begin{tabular}{@{}p{0.6\linewidth}cc@{}}
    \toprule
    \textbf{Control Question} & \textbf{Response} & \textbf{Status} \\
    \midrule
    Do you require MFA to access email? & Yes & {\color{green}\ding{51}} \\
    Do you require MFA to log into computers? & No & {\color{red}\ding{55}} \\
    Do you require MFA to access sensitive data systems? & Yes & {\color{green}\ding{51}} \\
    Does your organization have an employee acceptable use policy? & No & {\color{red}\ding{55}} \\
    Does your organization do security awareness training for new employees? & Yes & {\color{green}\ding{51}} \\
    Does your organization do security awareness training for all employees at least once per year? & Yes & {\color{green}\ding{51}} \\
    \bottomrule
\end{tabular}
\end{table}

\subsection*{Analysis of Control Gaps}
\begin{itemize}
    \item \textbf{No MFA for Computer Logins:} This is a high-risk gap. If an employee's password is compromised, an attacker can gain direct access to their workstation, potentially bypassing other network controls. This significantly increases the risk of lateral movement and ransomware deployment.
    \item \textbf{No Employee Acceptable Use Policy (AUP):} This is a foundational governance failure. An AUP defines the rules for using company technology and data. Without one, there is no formal basis for preventing risky employee behavior or for taking corrective action when misuse occurs.
\end{itemize}

% --- Section 4: Technical Scan Results ---
\section{Technical Scan Results}
An external network scan was performed to identify open ports and exposed services on the organization's public-facing infrastructure.

\begin{itemize}
    \item \textbf{Scan Target:} \texttt{[Target IP]}
    \item \textbf{Scan Date:} Scan data processed on \today
\end{itemize}

\begin{table}[h!]
\centering
\caption{Open Ports Detected on \texttt{[Client IP]}}
\begin{tabular}{@{}lllll@{}}
    \toprule
    \textbf{Port} & \textbf{State} & \textbf{Service} & \textbf{Version} & \textbf{Notes} \\
    \midrule
    22/tcp & open & ssh & (Not provided) & Secure Shell (SSH) access. \\
    \bottomrule
\end{tabular}
\end{table}

\subsection*{Analysis of Technical Findings}
The scan identified that port 22 (SSH) is open to the public internet. SSH is a common protocol for remote administration. While necessary for management, its public exposure presents a significant attack surface. Attackers frequently scan for open SSH ports to perform automated brute-force attacks or exploit known vulnerabilities. The security of this service is paramount and depends on strong authentication (preferably key-based), up-to-date software, and diligent monitoring.

% --- Section 5: Consolidated Risk Assessment ---
\section{Consolidated Risk Assessment}
The following table synthesizes findings from the security control review, technical scan, and pre-existing risk data into a consolidated list of key risks facing the organization.

\begin{table}[h!]
\centering
\caption{Summary of Identified Risks}
\begin{tabular}{@{}p{0.1\linewidth}p{0.3\linewidth}p{0.4\linewidth}l@{}}
    \toprule
    \textbf{Risk ID} & \textbf{Risk Name} & \textbf{Description} & \textbf{Severity} \\
    \midrule
    RISK-001 & Localhost Exposed & A critical service intended for local access only is exposed to the external network. & \textbf{Critical} \\
    \addlinespace
    RISK-002 & Lack of Endpoint MFA & The absence of MFA on computer logins allows for account takeover if passwords are compromised. & \textbf{High} \\
    \addlinespace
    RISK-003 & Missing Acceptable Use Policy & A foundational policy gap that increases the risk of insider threat and data misuse. & \textbf{High} \\
    \addlinespace
    RISK-004 & Exposed SSH Management Port & The SSH service is publicly accessible, making it a target for brute-force and credential stuffing attacks. & \textbf{Medium} \\
    \bottomrule
\end{tabular}
\end{table}

% --- Section 6: Recommendations ---
\section{Recommendations}
The following actions are recommended to mitigate the identified risks and improve the overall security posture of \textbf{[Organization Name]}.

\subsection{RISK-001: Remediate "Localhost Exposed" (Critical)}
\begin{itemize}
    \item \textbf{Immediate Action:} Conduct an urgent investigation to identify the specific service and configuration error causing this exposure. The affected element, \texttt{[Target IP]}, must be analyzed to determine how a local service is being made public (e.g., via a misconfigured proxy, firewall rule, or tunnel).
    \item \textbf{Long-Term Fix:} Reconfigure the responsible service to bind only to the localhost interface (\texttt{127.0.0.1}). Implement firewall rules to explicitly deny any external access to services not intended for public consumption.
\end{itemize}

\subsection{RISK-002: Implement Endpoint MFA (High)}
\begin{itemize}
    \item \textbf{Immediate Action:} Begin planning the rollout of MFA for all employee computer and server logins. Evaluate solutions compatible with your existing environment (e.g., Windows Hello for Business, Duo Security, or other third-party identity providers).
    \item \textbf{Long-Term Fix:} Mandate MFA for all workstation and server logins as a corporate policy. This control should apply to both on-premise and remote access scenarios.
\end{itemize}

\subsection{RISK-003: Develop and Implement an AUP (High)}
\begin{itemize}
    \item \textbf{Immediate Action:} Draft a comprehensive Acceptable Use Policy (AUP) that clearly defines the rules for using company networks, devices, software, and data.
    \item \textbf{Long-Term Fix:} Formally ratify the AUP and integrate it into the new employee onboarding process. Conduct annual training and require all employees to formally acknowledge they have read and understood the policy.
\end{itemize}

\subsection{RISK-004: Secure Exposed SSH Port (Medium)}
\begin{itemize}
    \item \textbf{Immediate Action:} If public access is not required, restrict access to port 22 to a whitelist of trusted IP addresses (e.g., office or VPN network ranges).
    \item \textbf{Long-Term Fix:} Enforce public key authentication and disable password-based authentication for SSH entirely. Implement an intrusion prevention tool like \texttt{fail2ban} to automatically block IPs that exhibit malicious behavior.
\end{itemize}

\end{document}
```