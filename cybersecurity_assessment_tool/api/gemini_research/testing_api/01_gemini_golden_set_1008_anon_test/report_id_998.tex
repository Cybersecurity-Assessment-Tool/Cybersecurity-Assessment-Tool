```latex
\documentclass[12pt]{article}

% 1. Document Setup & Packages
\usepackage[margin=1in]{geometry}
\usepackage{pifont} % For checkmarks and crosses
\usepackage{booktabs} % For professional tables
\usepackage{hyperref} % For clickable links
\usepackage{url}      % For URL formatting
\usepackage{seqsplit} % For splitting long strings
\usepackage{xcolor}   % For colors

% Define colors for severity
\definecolor{criticalred}{HTML}{D7263D}
\definecolor{highorange}{HTML}{F49D42}
\definecolor{mediumyellow}{HTML}{F4D03F}

% Hyperref setup
\hypersetup{
    colorlinks=true,
    linkcolor=blue,
    filecolor=magenta,      
    urlcolor=cyan,
    pdftitle={Cybersecurity Assessment Report},
    pdfpagemode=FullScreen,
}

% 2. Title Section
\title{Cybersecurity Assessment Report \\ \large For \textbf{[Organization Name]}}
\author{Cybersecurity Analyst}
\date{\today}

\begin{document}
\maketitle
\thispagestyle{empty}
\newpage

\tableofcontents
\newpage

% 3. Executive Summary
\section{Executive Summary}
This report provides a comprehensive cybersecurity assessment for \textbf{[Organization Name]}, based on an analysis of network scan data, organizational security controls, and pre-existing risk documentation. The assessment was conducted on \today.

The analysis revealed several high-priority risks that require immediate attention. A critical vulnerability was identified on an external-facing server, \texttt{[Target IP]}, which is running an outdated and misconfigured FTP service (\texttt{vsftpd 2.3.4}) that allows anonymous access. This version is known to be susceptible to a remote code execution backdoor (CVE-2011-2523).

Furthermore, significant gaps were identified in the organization's security policies. The lack of mandatory Multi-Factor Authentication (MFA) for email access represents a critical weakness, exposing the organization to significant risk from phishing and account compromise. Additionally, the absence of security awareness training for new employees creates a recurring vulnerability within the workforce.

While some security controls are in place, the combination of a directly exploitable external service and critical internal policy gaps creates a high-risk security posture. This report outlines actionable recommendations to mitigate these identified risks and strengthen the organization's overall defenses.

% 4. Organizational & Scan Information
\section{Assessment Scope & Information}
This assessment synthesizes data from three sources to provide a holistic view of the organization's security posture.

\subsection{Organizational Information}
\begin{itemize}
    \item \textbf{Organization Name:} \textbf{[Organization Name]}
    \item \textbf{Primary Email Domain:} \texttt{[Domain]}
    \item \textbf{External IP Scanned:} \texttt{[Client IP]}
\end{itemize}

\subsection{Technical Scan Information}
\begin{itemize}
    \item \textbf{Target IP Address:} \texttt{[Target IP]}
    \item \textbf{Scan Date:} Scan data processed on \today.
\end{itemize}

% 5. Security Control Review (from Questionnaire)
\section{Security Control Review}
The following table summarizes the organization's current security controls based on the provided questionnaire data. Gaps in these controls often represent significant security risks.

\begin{table}[h!]
\centering
\caption{Security Controls Questionnaire Analysis}
\label{tab:controls}
\begin{tabular}{@{}p{0.6\linewidth}ccp{0.2\linewidth}@{}}
\toprule
\textbf{Control Question} & \textbf{Response} & \textbf{Status} & \textbf{Analyst Note} \\
\midrule
Do you require MFA to access email? & No & \ding{55} & \textcolor{criticalred}{\textbf{Critical Gap}} \\
Do you require MFA to log into computers? & Yes & \ding{51} & Good Practice \\
Do you require MFA to access sensitive data systems? & Yes & \ding{51} & Good Practice \\
Does your organization have an employee acceptable use policy? & Yes & \ding{51} & Foundational \\
Does your organization do security awareness training for new employees? & No & \ding{55} & \textcolor{highorange}{\textbf{High Risk}} \\
Does your organization do security awareness training for all employees at least once per year? & Yes & \ding{51} & Good Practice \\
\bottomrule
\end{tabular}
\end{table}

% 6. Technical Scan Results
\section{Technical Scan Results}
An external network scan was performed on the target IP address. The following key findings were identified.

\subsection{Host: \texttt{[Target IP]}}
The host was found to be online and responsive. One critical vulnerability was discovered.

\subsubsection{Open Port: 21/TCP (FTP)}
\begin{itemize}
    \item \textbf{Service:} FTP (File Transfer Protocol)
    \item \textbf{Product:} \texttt{vsftpd}
    \item \textbf{Version:} \texttt{2.3.4}
    \item \textbf{Finding 1 (Critical):} The running version, \texttt{vsftpd 2.3.4}, is extremely outdated (released in 2011) and contains a well-known critical backdoor vulnerability (\textbf{CVE-2011-2523}). An attacker can gain a command shell on the server by sending a specific sequence of characters as a username.
    \item \textbf{Finding 2 (Critical):} The service is configured to allow \textbf{anonymous FTP login}. This permits any unauthenticated user on the internet to access, upload, or download files from the server, posing a severe data breach and malware risk.
\end{itemize}

% 7. Consolidated Risk Assessment
\section{Consolidated Risk Assessment}
The following table consolidates findings from the technical scan, the security control review, and pre-existing risk documentation into a prioritized list.

\begin{table}[h!]
\centering
\caption{Prioritized Risk Register}
\label{tab:risks}
\begin{tabular}{@{}p{0.3\linewidth}p{0.15\linewidth}p{0.45\linewidth}@{}}
\toprule
\textbf{Risk Name} & \textbf{Severity} & \textbf{Overview} \\
\midrule
\textbf{Vulnerable FTP Server} & \textcolor{criticalred}{\textbf{Critical}} & An outdated \texttt{vsftpd 2.3.4} server with anonymous login enabled is exposed to the internet, allowing for remote code execution and unauthorized file access. \\
\addlinespace
\textbf{No MFA on Email} & \textcolor{criticalred}{\textbf{Critical}} & The lack of Multi-Factor Authentication on email accounts makes them highly susceptible to phishing attacks and unauthorized takeover, which can lead to a full corporate breach. \\
\addlinespace
\textbf{No New Employee Security Training} & \textcolor{highorange}{\textbf{High}} & New hires are not receiving security awareness training, making them more likely to fall victim to social engineering and violate security policies unknowingly. \\
\addlinespace
Outdated Windows Policy & \textcolor{mediumyellow}{Medium} & (Pre-existing) Workstations are running Windows 7, which is an unsupported operating system and does not receive security updates. \\
\bottomrule
\end{tabular}
\end{table}

% 8. Recommendations
\section{Recommendations}
The following actions are recommended to mitigate the identified risks. They are prioritized based on severity.

\subsection{Remediate Vulnerable FTP Server (Critical)}
\begin{enumerate}
    \item \textbf{Immediate Action:} If FTP is not essential for business operations, \textbf{disable the service immediately} and block port 21 at the network firewall.
    \item \textbf{If FTP is Required:}
        \begin{itemize}
            \item Immediately disable anonymous FTP login in the server's configuration file (\texttt{vsftpd.conf}).
            \item Upgrade the \texttt{vsftpd} package to the latest stable version provided by your system's package manager.
            \item Consider replacing FTP with a more secure file transfer protocol, such as SFTP (SSH File Transfer Protocol).
        \end{itemize}
\end{enumerate}

\subsection{Enforce MFA for Email Access (Critical)}
\begin{enumerate}
    \item \textbf{Immediate Action:} Begin the process of enabling and enforcing MFA across all user email accounts.
    \item \textbf{Implementation Plan:} Develop a phased rollout plan, starting with administrative and executive accounts.
    \item \textbf{User Communication:} Provide clear instructions and support to all employees to ensure a smooth transition.
\end{enumerate}

\subsection{Enhance Security Awareness Program (High)}
\begin{enumerate}
    \item \textbf{Immediate Action:} Integrate mandatory security awareness training into the onboarding process for all new employees.
    \item \textbf{Content:} Ensure the training covers key topics such as phishing identification, password security, and the organization's acceptable use policy.
\end{enumerate}

\subsection{Address Outdated Operating Systems (Medium)}
\begin{enumerate}
    \item \textbf{Action:} Continue with the existing remediation plan to upgrade or replace all workstations running Windows 7.
    \item \textbf{Interim Control:} Ensure these machines are isolated from critical network segments if they cannot be upgraded immediately.
\end{enumerate}

\end{document}
```