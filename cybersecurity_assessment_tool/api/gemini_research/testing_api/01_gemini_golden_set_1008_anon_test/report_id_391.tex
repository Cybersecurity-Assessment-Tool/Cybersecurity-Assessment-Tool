```latex
\documentclass[12pt]{article}

% Preamble: Required Packages
\usepackage[a4paper, margin=1in]{geometry} % Page layout
\usepackage{pifont}                       % For checkmarks and crosses (\ding)
\usepackage{booktabs}                     % For professional-looking tables
\usepackage{hyperref}                     % For hyperlinks, metadata
\usepackage{url}                          % For URL formatting
\usepackage{seqsplit}                     % For splitting long strings in \texttt
\usepackage{graphicx}                     % For logos, etc.
\usepackage{fancyhdr}                     % For headers and footers
\usepackage{xcolor}                       % For colors

% --- Document Setup ---
\hypersetup{
    colorlinks=true,
    linkcolor=blue,
    filecolor=magenta,      
    urlcolor=cyan,
    pdftitle={Cybersecurity Posture Assessment Report},
    pdfauthor={Cybersecurity Analysis Division},
    pdfsubject={Security Assessment},
    pdfkeywords={Security, Assessment, Report},
}

% --- Header and Footer Customization ---
\pagestyle{fancy}
\fancyhf{} % Clear all header and footer fields
\fancyhead[L]{Cybersecurity Posture Assessment}
\fancyhead[R]{\textbf{[Organization Name]}}
\fancyfoot[C]{\thepage}
\renewcommand{\headrulewidth}{0.4pt}
\renewcommand{\footrulewidth}{0.4pt}

% --- Document Body ---
\begin{document}

% --- Title Page ---
\begin{titlepage}
    \centering
    \vspace*{1cm}
    \Huge\textbf{Cybersecurity Posture Assessment Report}
    \vspace{1.5cm}
    \Large
    \textbf{For:} \\
    \vspace{0.5cm}
    \huge\textbf{[Organization Name]}
    \vfill
    \large
    \textbf{Date of Report:} \today \\
    \textbf{Analysis Period:} October 2023 % Placeholder date as not provided in scan
    \vspace{1.5cm}
    \normalsize
    \textit{This report is confidential and intended solely for the use of \textbf{[Organization Name]}. Unauthorized distribution is prohibited.}
\end{titlepage}

\tableofcontents
\newpage

% --- Executive Summary ---
\section*{Executive Summary}
This report details the findings of a cybersecurity posture assessment conducted for \textbf{[Organization Name]}. The assessment combined a review of organizational security controls, an external network scan, and an analysis of pre-existing risks.

The analysis revealed several high-impact security gaps that require immediate attention. Critically, Multi-Factor Authentication (MFA) is not enforced for accessing email or other sensitive data systems. This oversight, combined with an externally exposed Secure Shell (SSH) service on host \texttt{[Target IP]}, creates a significant risk of unauthorized access and potential data breach.

Furthermore, the absence of a formal Employee Acceptable Use Policy (AUP) indicates a gap in foundational cybersecurity governance. While the organization has implemented security awareness training, the lack of clear policy undermines its effectiveness.

Immediate remediation of these findings is strongly recommended to reduce the organization's risk exposure. Detailed recommendations are provided in Section 5.0 of this report.

% --- Organizational Information ---
\section*{1.0 Organizational Information}
This section provides the organizational details used as the basis for this assessment.
\begin{center}
\begin{tabular}{@{}ll}
\toprule
\textbf{Detail} & \textbf{Value} \\
\midrule
\textbf{Organization Name:} & \textbf{[Organization Name]} \\
\textbf{Primary Domain:} & \texttt{[Domain]} \\
\textbf{Assessed External IP:} & \texttt{[Client IP]} \\
\bottomrule
\end{tabular}
\end{center}

% --- Security Control Review ---
\section*{2.0 Security Control Review}
The following table summarizes the organization's responses to a security controls questionnaire. A checkmark (\ding{51}) indicates a positive control is in place, while a cross (\ding{55}) indicates a control gap.
\begin{center}
\begin{tabular}{p{0.8\textwidth}c}
\toprule
\textbf{Control Question} & \textbf{Status} \\
\midrule
Do you require MFA to access email? & \ding{55} \\
Do you require MFA to log into computers? & \ding{51} \\
Do you require MFA to access sensitive data systems? & \ding{55} \\
Does your organization have an employee acceptable use policy? & \ding{55} \\
Does your organization do security awareness training for new employees? & \ding{51} \\
Does your organization do security awareness training for all employees at least once per year? & \ding{51} \\
\bottomrule
\end{tabular}
\end{center}

\subsection*{Analysis of Control Gaps}
The questionnaire reveals three critical control gaps:
\begin{itemize}
    \item \textbf{Lack of MFA on Email:} Email is a primary target for phishing and account takeover attacks. Without MFA, a compromised password is all an attacker needs to gain access.
    \item \textbf{Lack of MFA on Sensitive Systems:} Failure to protect sensitive data systems with MFA exposes critical business information to a high risk of unauthorized access.
    \item \textbf{Absence of Acceptable Use Policy (AUP):} An AUP is a foundational governance document that sets clear expectations for employee behavior regarding company assets and data. Its absence can lead to inconsistent security practices and insider threats.
\end{itemize}

% --- Technical Scan Results ---
\section*{3.0 Technical Scan Results}
An external network scan was performed to identify open ports and services visible on the public internet.
\subsection*{Host: \texttt{[Target IP]}}
The scan identified the following open port on the target host.
\begin{center}
\begin{tabular}{lllll}
\toprule
\textbf{Port} & \textbf{State} & \textbf{Service} & \textbf{Product} & \textbf{Version} \\
\midrule
22/tcp & open & ssh & \textit{(Not provided)} & \textit{(Not provided)} \\
\bottomrule
\end{tabular}
\end{center}
\subsection*{Analysis of Technical Findings}
The presence of an open SSH port (22/tcp) indicates that a remote management service is exposed to the public internet. SSH is a common target for brute-force password attacks and exploitation of known vulnerabilities. Without specific version information, it is impossible to rule out the presence of known exploits. This finding, when correlated with the lack of MFA on sensitive systems, elevates the overall risk profile significantly.

% --- Risk Assessment ---
\section*{4.0 Risk Assessment}
This section synthesizes the findings from the security control review, technical scan, and pre-existing risk data into a consolidated risk summary. No pre-existing vulnerabilities were reported.

\begin{center}
\begin{tabular}{p{0.3\textwidth}p{0.15\textwidth}p{0.5\textwidth}}
\toprule
\textbf{Risk / Finding} & \textbf{Severity} & \textbf{Description} \\
\midrule
\textbf{Lack of MFA on Critical Systems} & \textbf{Critical} & The absence of MFA on email and sensitive data systems exposes the organization to a high likelihood of account takeover, data breach, and business email compromise. A single compromised password could lead to a major incident. \\
\addlinespace
\textbf{Exposed SSH Management Port} & \textbf{High} & An externally accessible SSH port on \texttt{[Target IP]} serves as a direct vector for attackers. It is susceptible to brute-force attacks and exploitation, potentially granting an adversary direct command-line access to a key system. \\
\addlinespace
\textbf{Missing Acceptable Use Policy (AUP)} & \textbf{High} & This governance gap means there are no formally documented rules for employees regarding the use of technology and data. This increases the risk of insider threat, data mishandling, and non-compliance with regulations. \\
\bottomrule
\end{tabular}
\end{center}

% --- Recommendations ---
\section*{5.0 Recommendations}
The following actions are recommended to mitigate the identified risks and improve the overall security posture of \textbf{[Organization Name]}.

\subsection*{5.1 Immediate Actions (Critical/High Risks)}
\begin{enumerate}
    \item \textbf{Implement Multi-Factor Authentication (MFA):}
    \begin{itemize}
        \item \textbf{Action:} Immediately enable and enforce MFA for all users on all email accounts (e.g., Office 365, Google Workspace) and all systems identified as containing sensitive data.
        \item \textbf{Impact:} Drastically reduces the risk of unauthorized access via compromised credentials.
    \end{itemize}
    \vspace{0.5cm}
    \item \textbf{Secure the Exposed SSH Service:}
    \begin{itemize}
        \item \textbf{Action:}
            \begin{enumerate}
                \item Validate the business need for external SSH access to \texttt{[Target IP]}.
                \item If not required, block port 22 at the firewall immediately.
                \item If required, restrict access to a whitelist of trusted source IP addresses. Furthermore, enforce public key authentication and disable password-based logins. Disable root login over SSH.
            \end{enumerate}
        \item \textbf{Impact:} Reduces the attack surface and protects a critical entry point from external threats.
    \end{itemize}
    \vspace{0.5cm}
    \item \textbf{Develop and Implement an Acceptable Use Policy (AUP):}
    \begin{itemize}
        \item \textbf{Action:} Draft a formal AUP that clearly defines the rules for using company networks, devices, and data. Ensure all current and new employees read and formally acknowledge the policy.
        \item \textbf{Impact:} Establishes a baseline for secure behavior, improves security culture, and provides a basis for enforcement actions.
    \end{itemize}
\end{enumerate}

\end{document}
```