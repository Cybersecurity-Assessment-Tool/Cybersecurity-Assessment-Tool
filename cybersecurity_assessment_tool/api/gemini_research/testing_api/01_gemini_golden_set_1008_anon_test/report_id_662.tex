```latex
\documentclass[12pt]{article}

% --- PACKAGES ---
\usepackage[margin=1in]{geometry}
\usepackage{pifont} % For \ding{51} (checkmark) and \ding{55} (X)
\usepackage{booktabs} % For professional-looking tables
\usepackage{hyperref} % For creating hyperlinks
\usepackage{url}      % For formatting URLs
\usepackage{seqsplit} % For splitting long strings in \texttt
\usepackage[utf8]{inputenc}

% --- HYPERREF SETUP ---
\hypersetup{
    colorlinks=true,
    linkcolor=black,
    filecolor=magenta,      
    urlcolor=blue,
    pdftitle={Cybersecurity Posture Assessment Report},
    pdfpagemode=FullScreen,
}

% --- DOCUMENT START ---
\begin{document}

% --- TITLE PAGE ---
\title{Cybersecurity Posture Assessment Report}
\author{Cybersecurity Analysis Division}
\date{\today}
\maketitle
\thispagestyle{empty}
\newpage

\tableofcontents
\newpage

% --- SECTION 1: EXECUTIVE OVERVIEW ---
\section{Executive Overview}

This report provides a comprehensive cybersecurity assessment for \textbf{[Organization Name]}, based on the analysis of network scan data, organizational security controls, and pre-existing risk documentation. The assessment reveals several critical and high-risk vulnerabilities that require immediate attention to mitigate potential threats.

The primary findings include a \textbf{critical vulnerability} on an external-facing server: an outdated and misconfigured FTP service (\texttt{vsftpd 2.3.4}) is exposed, which is known to contain a backdoor (CVE-2011-2523) and allows for anonymous user access.

Furthermore, significant gaps were identified in administrative and access controls. The lack of mandatory Multi-Factor Authentication (MFA) for computer logins presents a high risk to endpoint security. Foundational security policies, such as an Acceptable Use Policy and a recurring Security Awareness Training program, are not in place, leaving the organization vulnerable to both insider threats and social engineering attacks.

These new findings, combined with the pre-existing risk of outdated Windows 7 workstations, create a security posture that must be addressed proactively. This report outlines these risks in detail and provides prioritized, actionable recommendations to enhance the organization's defenses.

% --- SECTION 2: ORGANIZATIONAL INFORMATION ---
\section{Organizational Information}

This assessment was conducted for the following entity and associated assets. Due to the anonymized nature of the input data, placeholders are used where specific information was not provided.

\begin{itemize}
    \item \textbf{Organization Name:} \textbf{[Organization Name]}
    \item \textbf{Primary Domain:} \texttt{[Domain]}
    \item \textbf{Scanned External IP:} \texttt{[Client IP]}
\end{itemize}

% --- SECTION 3: SECURITY CONTROL REVIEW ---
\section{Security Control Review}

The following table summarizes the organization's responses to a security controls questionnaire. Items marked with \ding{55} indicate a deviation from security best practices and represent a gap in the defensive posture.

\begin{table}[h!]
\centering
\caption{Security Controls Questionnaire Analysis}
\begin{tabular}{@{}lcc@{}}
\toprule
\textbf{Control Question} & \textbf{Response} & \textbf{Assessment} \\
\midrule
Do you require MFA to access email? & \ding{51} & Compliant \\
Do you require MFA to log into computers? & \ding{55} & \textbf{High Risk Gap} \\
Do you require MFA to access sensitive data systems? & \ding{51} & Compliant \\
Does your organization have an employee acceptable use policy? & \ding{55} & \textbf{Critical Policy Gap} \\
Does your organization do security awareness training for new employees? & \ding{55} & \textbf{Critical Policy Gap} \\
Does your organization do security awareness training for all employees annually? & \ding{55} & \textbf{Critical Policy Gap} \\
\bottomrule
\end{tabular}
\end{table}

The absence of MFA on computer logins significantly increases the risk of unauthorized access via compromised credentials. The lack of an Acceptable Use Policy and a formal security training program are critical administrative deficiencies that heighten the risk of human error and insider threats.

% --- SECTION 4: TECHNICAL SCAN RESULTS ---
\section{Technical Scan Results}

An external network scan was performed on the target IP address. The scan identified the following open ports and services.

\begin{itemize}
    \item \textbf{Target IP Address:} \texttt{[Target IP]}
\end{itemize}

\begin{table}[h!]
\centering
\caption{Open Port Analysis}
\begin{tabular}{@{}lllll@{}}
\toprule
\textbf{Port} & \textbf{Service} & \textbf{Product / Version} & \textbf{Finding} \\
\midrule
21/tcp & ftp & vsftpd 2.3.4 & \begin{tabular}[t]{@{}l@{}}\textbf{CRITICAL VULNERABILITY} \\ - Anonymous FTP login is allowed. \\ - This version is vulnerable to a known \\ \phantom{-} backdoor (CVE-2011-2523).\end{tabular} \\
\bottomrule
\end{tabular}
\end{table}

The presence of an outdated and misconfigured FTP server represents a direct and immediate threat. An attacker could exploit the known backdoor to gain remote command execution on the server or use the anonymous access to exfiltrate or store malicious data.

% --- SECTION 5: CONSOLIDATED RISK ASSESSMENT ---
\section{Consolidated Risk Assessment}

The following table synthesizes findings from the technical scan, control review, and pre-existing risk documentation into a prioritized list.

\begin{table}[h!]
\centering
\caption{Summary of Identified Risks}
\begin{tabular}{@{}p{0.3\textwidth}p{0.5\textwidth}l@{}}
\toprule
\textbf{Risk Name} & \textbf{Description} & \textbf{Severity} \\
\midrule
\textbf{Vulnerable FTP Server} & An outdated version of vsftpd (2.3.4) with a known backdoor is exposed to the internet and allows anonymous login. & \textbf{Critical} \\
\addlinespace
\textbf{Lack of Endpoint MFA} & User workstations and laptops do not require Multi-Factor Authentication for login, making them susceptible to credential theft. & \textbf{High} \\
\addlinespace
\textbf{Missing Administrative Controls} & The organization lacks a formal Acceptable Use Policy and a security awareness training program for employees. & \textbf{High} \\
\addlinespace
\textbf{Outdated Windows Policy} & As per existing documentation, workstations are running Windows 7, which is an unsupported and vulnerable operating system. & \textbf{Medium} \\
\bottomrule
\end{tabular}
\end{table}

% --- SECTION 6: RECOMMENDATIONS ---
\section{Recommendations}

Based on the consolidated risk assessment, the following prioritized actions are recommended to improve the security posture of \textbf{[Organization Name]}.

\subsection{Priority 1: Critical Risk Mitigation}
\begin{enumerate}
    \item \textbf{Remediate Vulnerable FTP Server (Immediate):}
    \begin{itemize}
        \item Immediately disable the FTP service on port 21 on host \texttt{[Target IP]} or restrict access to trusted IP addresses only via firewall rules.
        \item If the service is business-critical, upgrade \texttt{vsftpd} to the latest stable version and disable anonymous access.
        \item For long-term security, migrate all file transfer functionality to a secure protocol such as SFTP (SSH File Transfer Protocol).
    \end{itemize}
\end{enumerate}

\subsection{Priority 2: High Risk Mitigation}
\begin{enumerate}
    \setcounter{enumi}{1}
    \item \textbf{Implement Endpoint MFA:} Deploy a mandatory Multi-Factor Authentication solution for all employee computer logins to protect against unauthorized access.
    \item \textbf{Develop an Acceptable Use Policy (AUP):} Create and enforce a formal AUP that clearly defines the rules for using company IT assets, data, and internet access.
    \item \textbf{Establish Security Awareness Training:} Implement a mandatory training program for all new hires and conduct annual refresher training for all staff. Topics should include phishing, password security, and data handling.
\end{enumerate}

\subsection{Priority 3: Medium Risk Mitigation}
\begin{enumerate}
    \setcounter{enumi}{4}
    \item \textbf{Upgrade Outdated Workstations:} Continue with the existing plan to upgrade all Windows 7 machines to a modern, supported operating system like Windows 10 or 11 to ensure they receive critical security patches.
\end{enumerate}

% --- DOCUMENT END ---
\end{document}
```