```latex
\documentclass[12pt]{article}

% Required Packages
\usepackage[margin=1in]{geometry} % Set page margins
\usepackage{pifont}               % For checkmarks and crosses (\ding)
\usepackage{booktabs}             % For professional-looking tables
\usepackage{hyperref}             % For clickable links and references
\usepackage{url}                  % For formatting URLs
\usepackage{seqsplit}             % To split long strings in tt font
\usepackage{graphicx}             % For logos (optional)
\usepackage{xcolor}               % For colors

% Document Metadata
\title{Cybersecurity Posture and Risk Assessment Report}
\author{Cybersecurity Analysis Division}
\date{\today}

% Hyperref Setup
\hypersetup{
    colorlinks=true,
    linkcolor=blue,
    filecolor=magenta,      
    urlcolor=cyan,
    pdftitle={Cybersecurity Posture and Risk Assessment Report},
    pdfpagemode=FullScreen,
}

\begin{document}

\maketitle
\thispagestyle{empty}
\newpage

\tableofcontents
\newpage

% --- 1. Executive Summary ---
\section{Executive Summary}

This report provides a comprehensive analysis of the cybersecurity posture for \textbf{[Organization Name]}. The assessment is based on a synthesis of external network scans, a review of internal security controls via a questionnaire, and an evaluation of previously identified risks.

The analysis has uncovered several high-impact vulnerabilities that require immediate attention. A critical risk was identified on the external network: an outdated and misconfigured FTP server (\texttt{vsftpd 2.3.4}) which allows anonymous access. This version is known to contain a critical backdoor vulnerability (CVE-2011-2523).

Furthermore, a significant gap in the organization's security processes was noted: the absence of mandatory security awareness training for new employees. This exposes the organization to a heightened risk of social engineering and phishing attacks, as new hires are often prime targets.

These findings, combined with the pre-existing risk of outdated Windows 7 workstations, indicate a security posture that requires significant remediation efforts. This report outlines specific, actionable recommendations to mitigate these risks and strengthen the overall defensive capabilities of the organization.

% --- 2. Organizational Information ---
\section{Organizational Information}

The following information was used as the basis for this assessment. Due to the anonymized nature of the provided data, placeholders are used where specific details were unavailable.

\begin{table}[h!]
\centering
\begin{tabular}{@{}ll@{}}
\toprule
\textbf{Attribute} & \textbf{Value} \\ \midrule
Organization Name  & \textbf{[Organization Name]} \\
Primary Domain     & \texttt{[Domain]} \\
External IP Scanned & \texttt{[Client IP]} \\ \bottomrule
\end{tabular}
\caption{Client Organizational Details.}
\label{tab:org_info}
\end{table}

% --- 3. Security Control Review ---
\section{Security Control Review}

An internal security questionnaire was reviewed to assess the maturity of existing administrative and technical controls. While the organization demonstrates strong practices in multi-factor authentication (MFA), a critical gap was identified in the employee onboarding process.

\begin{table}[h!]
\centering
\begin{tabular}{@{}p{0.8\linewidth}c@{}}
\toprule
\textbf{Control Question} & \textbf{Response} \\ \midrule
Do you require MFA to access email? & \textcolor{green}{\ding{51}} \\
Do you require MFA to log into computers? & \textcolor{green}{\ding{51}} \\
Do you require MFA to access sensitive data systems? & \textcolor{green}{\ding{51}} \\
Does your organization have an employee acceptable use policy? & \textcolor{green}{\ding{51}} \\
\textbf{Does your organization do security awareness training for new employees?} & \textcolor{red}{\ding{55}} \\
Does your organization do security awareness training for all employees at least once per year? & \textcolor{green}{\ding{51}} \\ \bottomrule
\end{tabular}
\caption{Security Controls Questionnaire Analysis. A red cross (\textcolor{red}{\ding{55}}) indicates a significant control gap.}
\label{tab:controls}
\end{table}

The lack of security training for new hires represents a high-risk gap. This period is when employees are most vulnerable to social engineering and are still learning organizational policies.

% --- 4. Technical Network Scan Results ---
\section{Technical Network Scan Results}

An external network scan was performed against the target IP address \texttt{[Target IP]} to identify open ports and exposed services. The scan revealed a critically vulnerable service accessible from the public internet.

\begin{table}[h!]
\centering
\begin{tabular}{@{}lllll@{}}
\toprule
\textbf{Port} & \textbf{State} & \textbf{Service} & \textbf{Product \& Version} & \textbf{Notes} \\ \midrule
21/tcp & open & ftp & vsftpd 2.3.4 & \begin{tabular}[c]{@{}l@{}}Anonymous FTP login allowed.\\ Version is vulnerable to a known\\ backdoor (CVE-2011-2523).\end{tabular} \\ \bottomrule
\end{tabular}
\caption{Open Ports and Services Detected on \texttt{[Target IP]}.}
\label{tab:scan_results}
\end{table}

\subsection{Analysis of Findings}
The File Transfer Protocol (FTP) service is inherently insecure as it transmits credentials in cleartext. The specific version detected, \texttt{vsftpd 2.3.4}, is over a decade old and contains a well-documented remote code execution vulnerability. Compounding this issue, the service is configured to allow anonymous login, permitting any external attacker to connect and potentially upload malicious files or exfiltrate sensitive data.

% --- 5. Consolidated Risk Assessment ---
\section{Consolidated Risk Assessment}

The following table synthesizes findings from the technical scan, the controls review, and pre-existing risk data into a prioritized list.

\begin{table}[h!]
\centering
\begin{tabular}{@{}lp{0.5\linewidth}l@{}}
\toprule
\textbf{Risk Name} & \textbf{Description} & \textbf{Severity} \\ \midrule
\textbf{Insecure FTP Service} & An outdated, vulnerable version of vsftpd (2.3.4) is exposed to the internet with anonymous login enabled, allowing for potential remote code execution and data breach. & \textbf{Critical} \\
\addlinespace
\textbf{No Security Training for New Hires} & The absence of a security awareness program during employee onboarding significantly increases susceptibility to phishing, malware, and social engineering attacks. & \textbf{High} \\
\addlinespace
\textbf{Outdated Windows Policy} & Workstations are running Windows 7, which is End-of-Life and no longer receives security updates, leaving them vulnerable to known exploits. & \textbf{Medium} \\ \bottomrule
\end{tabular}
\caption{Summary of Identified Cybersecurity Risks.}
\label{tab:risks}
\end{table}

% --- 6. Recommendations ---
\section{Recommendations}

The following actions are recommended to mitigate the identified risks and improve the organization's overall security posture.

\subsection{Risk: Insecure FTP Service (Critical)}
\begin{itemize}
    \item \textbf{Immediate Action:} Take the FTP server offline immediately by disabling the service or implementing a firewall rule to block all access to TCP port 21.
    \item \textbf{Short-Term Fix:} If FTP is a business necessity, upgrade the \texttt{vsftpd} software to the latest stable version, disable anonymous login, and enforce strong, unique passwords for all FTP accounts.
    \item \textbf{Long-Term Strategy:} Migrate all file transfer workflows to a secure protocol such as SFTP (SSH File Transfer Protocol) or FTPS (FTP over SSL/TLS) and decommission the legacy FTP service entirely.
\end{itemize}

\subsection{Risk: No Security Training for New Hires (High)}
\begin{itemize}
    \item \textbf{Immediate Action:} Develop and mandate a security awareness training module as a required component of the new employee onboarding process. This module should cover phishing, acceptable use, password hygiene, and incident reporting.
    \item \textbf{Long-Term Strategy:} Reinforce security concepts through a continuous awareness program, including periodic simulated phishing campaigns and regular security reminders, to build a strong security culture.
\end{itemize}

\subsection{Risk: Outdated Windows Policy (Medium)}
\begin{itemize}
    \item \textbf{Immediate Action:} If upgrades are not immediately possible, ensure all Windows 7 machines are isolated on a separate network segment with restricted access to critical internal resources.
    \item \textbf{Short-Term Fix:} Execute the existing plan to upgrade all Windows 7 workstations to a currently supported operating system, such as Windows 10 or Windows 11. Prioritize machines used by employees with access to sensitive data.
\end{itemize}

\end{document}
```