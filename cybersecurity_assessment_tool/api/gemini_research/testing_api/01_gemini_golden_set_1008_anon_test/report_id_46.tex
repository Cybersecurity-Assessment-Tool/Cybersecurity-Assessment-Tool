```latex
\documentclass[12pt]{article}

% ----------------------------------------------------------------------
% METADATA & PACKAGES
% ----------------------------------------------------------------------
\usepackage[a4paper, margin=1in]{geometry}
\usepackage{pifont} % For checkmarks and crosses
\usepackage{booktabs} % For professional tables
\usepackage{hyperref} % For clickable links
\usepackage{url}      % For URL formatting
\usepackage{seqsplit} % To split long strings like hashes or tokens
\usepackage[utf8]{inputenc}

\hypersetup{
    colorlinks=true,
    linkcolor=blue,
    filecolor=magenta,      
    urlcolor=cyan,
    pdftitle={Cybersecurity Posture Report},
    pdfpagemode=FullScreen,
}

\newcommand{\yes}{\ding{51}}
\newcommand{\no}{\ding{55}}

% ----------------------------------------------------------------------
% DOCUMENT START
% ----------------------------------------------------------------------
\begin{document}

\title{
    Cybersecurity Posture Report \\
    \large For \textbf{[Organization Name]}
}
\author{Cybersecurity Analyst}
\date{\today}
\maketitle

\hrule\vspace{1em}

% ----------------------------------------------------------------------
% 1. EXECUTIVE SUMMARY
% ----------------------------------------------------------------------
\section*{Executive Summary}

This report provides a comprehensive analysis of the cybersecurity posture for \textbf{[Organization Name]}. The assessment is based on a correlation of data from an external network scan, a review of organizational security controls, and an analysis of pre-existing risks.

The primary findings indicate significant gaps in fundamental security controls. The absence of Multi-Factor Authentication (MFA) for computer and sensitive data access represents a critical vulnerability. Furthermore, the lack of a formal employee acceptable use policy and a structured security awareness training program exposes the organization to substantial human-factor risks, such as phishing and insider threats.

The external network scan of the target IP address \texttt{[Client IP]} did not identify any exposed services. While this may indicate a strong firewall configuration, it does not reduce the severity of the internal policy and procedural gaps identified.

Immediate action is recommended to implement MFA across all critical systems, develop and enforce key security policies, and establish a mandatory security awareness training program to mitigate these high-priority risks.

\vspace{1em}

% ----------------------------------------------------------------------
% 2. ORGANIZATIONAL INFORMATION
% ----------------------------------------------------------------------
\section*{Organizational Information}

The following details were used as the basis for this assessment. In cases where information was not provided, placeholders have been used.

\begin{itemize}
    \item \textbf{Organization Name:} \textbf{[Organization Name]}
    \item \textbf{Email Domain:} \texttt{[Domain]}
    \item \textbf{External IP Scanned:} \texttt{[Client IP]}
\end{itemize}

\vspace{1em}

% ----------------------------------------------------------------------
% 3. SECURITY CONTROL REVIEW
% ----------------------------------------------------------------------
\section*{Security Control Review}

A review of the organization's security controls was conducted via a questionnaire. The responses reveal critical areas requiring immediate attention. "No" answers indicate a lack of a specific control and are considered significant security gaps.

\begin{table}[h!]
\centering
\caption{Organizational Security Control Questionnaire}
\begin{tabular}{p{0.75\textwidth} c}
\toprule
\textbf{Control Question} & \textbf{Response} \\
\midrule
Do you require MFA to access email? & \yes \\
Do you require MFA to log into computers? & \no \\
Do you require MFA to access sensitive data systems? & \no \\
Does your organization have an employee acceptable use policy? & \no \\
Does your organization do security awareness training for new employees? & \no \\
Does your organization do security awareness training for all employees at least once per year? & \no \\
\bottomrule
\end{tabular}
\end{table}

\paragraph{Analysis:}
The responses highlight a critical deficiency in identity and access management, particularly the failure to enforce MFA on workstations and sensitive data systems. This significantly increases the risk of unauthorized access from compromised credentials. Additionally, the absence of an acceptable use policy and security awareness training creates a high-risk environment where employees may be unaware of security best practices, making the organization more susceptible to social engineering and malware.

\vspace{1em}

% ----------------------------------------------------------------------
% 4. TECHNICAL SCAN RESULTS
% ----------------------------------------------------------------------
\section*{Technical Scan Results}

An external network vulnerability scan was performed to identify exposed services and potential vulnerabilities on the organization's public-facing infrastructure.

\begin{itemize}
    \item \textbf{Target IP:} \texttt{[Target IP]}
    \item \textbf{Scan Date:} \today
\end{itemize}

\paragraph{Findings:}
The scan completed successfully but did not identify any open TCP or UDP ports on the target system. This suggests that the network perimeter is likely protected by a well-configured firewall that drops or rejects unsolicited incoming traffic. No vulnerabilities associated with exposed services could be identified.

\vspace{1em}

% ----------------------------------------------------------------------
% 5. RISK ASSESSMENT SUMMARY
% ----------------------------------------------------------------------
\section*{Risk Assessment Summary}

The following table summarizes the key risks identified through the analysis of the security control questionnaire. No pre-existing vulnerabilities were reported, and no technical vulnerabilities were discovered during the scan. The risks below are derived directly from the identified policy and procedure gaps.

\begin{table}[h!]
\centering
\caption{Identified Risks and Severity}
\begin{tabular}{p{0.15\textwidth} p{0.55\textwidth} l}
\toprule
\textbf{Risk Name} & \textbf{Overview} & \textbf{Severity} \\
\midrule
\textbf{Lack of MFA on Endpoints and Systems} & The absence of MFA for computer logins and access to sensitive data systems allows an attacker with stolen credentials to gain unauthorized access. & \textbf{Critical} \\
\addlinespace
\textbf{Inadequate Security Training} & Without initial and ongoing security awareness training, employees are more likely to fall victim to phishing, social engineering, and other common cyberattacks. & \textbf{High} \\
\addlinespace
\textbf{Absence of Acceptable Use Policy} & The lack of a formal policy defining acceptable use of company assets creates ambiguity and increases the risk of insider threats, data leakage, and misuse of resources. & \textbf{High} \\
\bottomrule
\end{tabular}
\end{table}

\vspace{1em}

% ----------------------------------------------------------------------
% 6. RECOMMENDATIONS
% ----------------------------------------------------------------------
\section*{Recommendations}

To mitigate the identified risks and improve the overall security posture, the following actions are recommended with high priority:

\begin{enumerate}
    \item \textbf{Implement Comprehensive MFA:}
    Deploy and enforce Multi-Factor Authentication (MFA) for all employees for logging into computers and accessing any systems containing sensitive or critical data. This is the single most effective control to prevent unauthorized access from compromised credentials.
    
    \item \textbf{Develop and Enforce an Acceptable Use Policy (AUP):}
    Create a formal AUP that clearly outlines the rules and responsibilities for all employees when using company networks, systems, and data. This policy should be signed by all employees upon hire and reviewed annually.
    
    \item \textbf{Establish a Security Awareness Training Program:}
    Implement a mandatory security awareness training program for all new hires and conduct annual refresher training for all staff. The program should cover key topics such as phishing, password security, data handling, and social engineering.
\end{enumerate}

\vspace{1em}

% ----------------------------------------------------------------------
% 7. CONCLUSION
% ----------------------------------------------------------------------
\section*{Conclusion}

While the external network posture of \textbf{[Organization Name]} appears secure, significant internal security gaps present a high level of risk to the organization. The findings from the security control review indicate a need for foundational improvements in identity management, employee education, and corporate policy.

By implementing the recommendations outlined in this report, \textbf{[Organization Name]} can substantially reduce its attack surface and build a more resilient and secure operational environment.

% ----------------------------------------------------------------------
% DOCUMENT END
% ----------------------------------------------------------------------
\end{document}
```