```latex
\documentclass[12pt, a4paper]{article}

% Preamble: Required packages and document setup
\usepackage[margin=1in]{geometry}
\usepackage{pifont} % For checkmarks and crosses
\usepackage{booktabs} % For professional-looking tables
\usepackage{hyperref} % For clickable links
\usepackage{url} % For formatting URLs
\usepackage{seqsplit} % For splitting long strings without spaces
\usepackage{graphicx}
\usepackage{xcolor}

% --- Document Metadata ---
\title{Cybersecurity Posture Assessment Report}
\author{Cybersecurity Analysis Division}
\date{\today}

% --- Hyperref Setup ---
\hypersetup{
    colorlinks=true,
    linkcolor=blue,
    filecolor=magenta,      
    urlcolor=cyan,
    pdftitle={Cybersecurity Posture Assessment Report},
    pdfpagemode=FullScreen,
}

\begin{document}

\maketitle
\thispagestyle{empty}
\newpage

\tableofcontents
\newpage

% ==============================================================================
% 1. EXECUTIVE SUMMARY
% ==============================================================================
\section*{1. Executive Summary}

This report provides a comprehensive analysis of the cybersecurity posture for \textbf{[Organization Name]}. The assessment is based on a correlation of data from an external network scan, a security controls questionnaire, and a review of pre-existing risk documentation.

The analysis reveals a mixed security posture. The organization has implemented strong foundational controls, particularly in the area of Multi-Factor Authentication (MFA) for email, computer logins, and access to sensitive systems. These controls significantly reduce the risk of unauthorized access through compromised credentials.

However, several critical and high-risk issues were identified that require immediate attention:
\begin{itemize}
    \item \textbf{Critical Risk - Exposed End-of-Life Database:} An external scan identified a MySQL database (version 5.7.33) directly accessible from the internet. This version reached its End-of-Life (EOL) in October 2023 and no longer receives security updates, making it a prime target for exploitation.
    \item \textbf{High Risk - Lack of Annual Security Training:} The organization does not conduct mandatory annual security awareness training for all employees. This procedural gap increases the organization's susceptibility to social engineering attacks, such as phishing.
\end{itemize}

This report details these findings and provides actionable recommendations to mitigate the identified risks and improve the overall security maturity of the organization.

% ==============================================================================
% 2. ORGANIZATIONAL INFORMATION
% ==============================================================================
\section*{2. Organizational Information}

This section contains the high-level information used as the basis for this assessment. Due to the anonymized nature of the input data, placeholders have been used where necessary.

\begin{tabular}{@{}ll}
    \toprule
    \textbf{Attribute} & \textbf{Value} \\
    \midrule
    Organization Name & \textbf{[Organization Name]} \\
    Primary Domain & \texttt{[Domain]} \\
    External IP Scanned & \texttt{[Client IP]} \\
    Target IP Scanned & \texttt{[Target IP]} \\
    \bottomrule
\end{tabular}

% ==============================================================================
% 3. SECURITY CONTROL REVIEW (QUESTIONNAIRE)
% ==============================================================================
\section*{3. Security Control Review}

The following table summarizes the organization's self-reported security controls. A green checkmark (\textcolor{green}{\ding{51}}) indicates a positive control is in place, while a red cross (\textcolor{red}{\ding{55}}) indicates a potential control gap.

\begin{table}[h!]
\centering
\begin{tabular}{@{}p{0.8\linewidth}c@{}}
    \toprule
    \textbf{Control Question} & \textbf{Status} \\
    \midrule
    Do you require MFA to access email? & \textcolor{green}{\ding{51}} \\
    Do you require MFA to log into computers? & \textcolor{green}{\ding{51}} \\
    Do you require MFA to access sensitive data systems? & \textcolor{green}{\ding{51}} \\
    Does your organization have an employee acceptable use policy? & \textcolor{green}{\ding{51}} \\
    Does your organization do security awareness training for new employees? & \textcolor{green}{\ding{51}} \\
    \midrule
    \textbf{Does your organization do security awareness training for all employees at least once per year?} & \textcolor{red}{\ding{55}} \\
    \bottomrule
\end{tabular}
\caption{Security Controls Questionnaire Results.}
\end{table}

\paragraph{Analysis:} The organization has effectively implemented MFA across key access points, which is a commendable best practice. The primary gap identified is the lack of ongoing, annual security awareness training for all staff. Human error remains a leading cause of security breaches, and regular training is a critical defense against evolving threats like phishing and social engineering.

% ==============================================================================
% 4. TECHNICAL SCAN RESULTS
% ==============================================================================
\section*{4. Technical Scan Results}

An external network scan was performed using Nmap against the target IP address \texttt{[Target IP]}. The scan identified the following open port and service.

\begin{table}[h!]
\centering
\begin{tabular}{@{}lllll@{}}
    \toprule
    \textbf{Port} & \textbf{State} & \textbf{Service} & \textbf{Product} & \textbf{Version} \\
    \midrule
    3306/tcp & open & mysql & MySQL & 5.7.33 \\
    \bottomrule
\end{tabular}
\caption{Nmap Scan Results for \texttt{[Target IP]}.}
\end{table}

\paragraph{Analysis:} The scan revealed that TCP port 3306 is open to the internet. This port is the default for the MySQL database service. Exposing a database directly to the public internet is a significant security risk, as it allows attackers to directly target the service with brute-force attacks, credential stuffing, and exploits for known vulnerabilities. Furthermore, the identified version, MySQL 5.7.33, is an End-of-Life (EOL) product and is no longer supported with security patches.

% ==============================================================================
% 5. RISK ASSESSMENT SUMMARY
% ==============================================================================
\section*{5. Risk Assessment Summary}

The following table synthesizes findings from the security questionnaire, the technical scan, and pre-existing risk data. Each risk is assigned a severity level to guide prioritization.

\begin{table}[h!]
\centering
\begin{tabular}{@{}p{0.1\linewidth}p{0.25\linewidth}p{0.4\linewidth}p{0.15\linewidth}@{}}
    \toprule
    \textbf{Risk ID} & \textbf{Risk Name} & \textbf{Description} & \textbf{Severity} \\
    \midrule
    \textbf{RISK-001} & Outdated Database Software (EOL) & The public-facing MySQL service is version 5.7.33, which reached its End-of-Life in October 2023. It no longer receives security updates, leaving it vulnerable to newly discovered exploits. & \textbf{Critical} \\
    \addlinespace
    \textbf{RISK-002} & Public Database Exposure & The MySQL database on port 3306 is directly accessible from the internet. This violates the principle of least privilege and exposes a critical data asset to external threats. & \textbf{High} \\
    \addlinespace
    \textbf{RISK-003} & Lack of Annual Security Awareness Training & The organization does not conduct mandatory annual security training for all employees. This increases the likelihood of a successful social engineering attack (e.g., phishing). & \textbf{Medium} \\
    \bottomrule
\end{tabular}
\caption{Consolidated Risk Register.}
\end{table}

% ==============================================================================
% 6. RECOMMENDATIONS
% ==============================================================================
\section*{6. Recommendations}

The following actionable recommendations are provided to address the identified risks.

% --- RISK-001 ---
\subsection*{RISK-001: Outdated Database Software (EOL)}
\begin{itemize}
    \item \textbf{Immediate Action:} Develop a migration plan to upgrade the MySQL 5.7.33 instance to a currently supported version, such as MySQL 8.0 or a managed cloud database equivalent. This is the only way to ensure the service receives critical security patches.
    \item \textbf{Long-Term Strategy:} Implement a formal patch and vulnerability management program that includes regular software inventory and end-of-life tracking to prevent future occurrences of running unsupported software.
\end{itemize}

% --- RISK-002 ---
\subsection*{RISK-002: Public Database Exposure}
\begin{itemize}
    \item \textbf{Immediate Action:} Implement strict firewall rules to block all public access to TCP port 3306. Access should be restricted to only trusted internal IP addresses or specific application servers that require it.
    \item \textbf{Long-Term Strategy:} If remote access to the database is required, it should be facilitated through a secure channel, such as a Virtual Private Network (VPN) with multi-factor authentication. Direct exposure should never be the default configuration.
\end{itemize}

% --- RISK-003 ---
\subsection*{RISK-003: Lack of Annual Security Awareness Training}
\begin{itemize}
    \item \textbf{Immediate Action:} Procure and implement a security awareness training program for all current employees. The training should cover key topics such as phishing identification, password hygiene, and acceptable use policies.
    \item \textbf{Long-Term Strategy:} Establish a formal policy requiring all employees to complete security awareness training on an annual basis. Incorporate phishing simulation exercises to measure effectiveness and provide targeted follow-up training.
\end{itemize}

\end{document}
```