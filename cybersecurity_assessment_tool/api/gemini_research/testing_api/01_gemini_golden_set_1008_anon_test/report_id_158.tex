```latex
\documentclass[12pt]{article}

% --- PREAMBLE ---
\usepackage[margin=1in]{geometry}
\usepackage{pifont} % For checkmarks and crosses
\usepackage{booktabs} % For professional tables
\usepackage{hyperref} % For clickable links
\usepackage{url} % For URL formatting
\usepackage{seqsplit} % For splitting long strings in texttt
\usepackage{graphicx}
\usepackage{fancyhdr}
\usepackage{xcolor}

% --- DOCUMENT SETUP ---
\definecolor{darkblue}{rgb}{0.0, 0.0, 0.55}
\hypersetup{
    colorlinks=true,
    linkcolor=darkblue,
    filecolor=darkblue,      
    urlcolor=darkblue,
    citecolor=darkblue,
}

\pagestyle{fancy}
\fancyhf{} % clear all header and footers
\renewcommand{\headrulewidth}{0.4pt}
\renewcommand{\footrulewidth}{0.4pt}
\lhead{\textbf{[Organization Name]} Cybersecurity Report}
\rhead{\today}
\cfoot{\thepage}

% --- DOCUMENT START ---
\begin{document}

% --- TITLE PAGE ---
\begin{titlepage}
    \centering
    \vspace*{2cm}
    
    \Huge
    \textbf{Cybersecurity Posture Assessment Report}
    
    \vspace{1.5cm}
    
    \Large
    Prepared for:
    
    \vspace{0.5cm}
    
    \textbf{[Organization Name]}
    
    \vspace{2cm}
    
    \large
    \today
    
    \vfill
    
    \normalsize
    \textit{This report contains sensitive information and should be handled with care. Access is restricted to authorized personnel only.}
    
\end{titlepage}

\tableofcontents
\newpage

% --- EXECUTIVE SUMMARY ---
\section*{Executive Summary}

This report provides a cybersecurity assessment for \textbf{[Organization Name]}, synthesizing findings from a technical network scan, a review of organizational security controls, and an analysis of pre-existing risks.

The organization demonstrates a strong foundation in administrative security controls. The security questionnaire reveals consistent implementation of Multi-Factor Authentication (MFA) across critical systems, alongside established policies for acceptable use and security awareness training. These are commendable practices that significantly reduce risks associated with user-based threats.

However, a critical vulnerability was identified on the external network perimeter. An exposed FTP server was found running a dangerously outdated version of \texttt{vsftpd} (2.3.4), which is widely known to contain a critical backdoor vulnerability (CVE-2011-2523). This misconfiguration is exacerbated by the allowance of anonymous FTP logins, creating a direct and immediate path for an attacker to compromise the server and potentially pivot to the internal network.

This technical finding, combined with the pre-existing medium-risk issue of outdated Windows 7 workstations, indicates a disparity between strong policy and weaker technical security implementation.

Immediate action is required to remediate the vulnerable FTP server to prevent a likely compromise. Recommendations are detailed in the final section of this report.

% --- ORGANIZATIONAL INFORMATION ---
\section*{1. Organizational Information}
This section details the high-level information used as the basis for this assessment. Due to the anonymized nature of the provided data, placeholders are used.

\begin{itemize}
    \item \textbf{Organization Name:} \textbf{[Organization Name]}
    \item \textbf{Primary Email Domain:} \texttt{[Domain]}
    \item \textbf{External IP Target:} \texttt{[Target IP]}
\end{itemize}

% --- SECURITY CONTROL REVIEW ---
\section*{2. Security Control Review}
The following table summarizes the organization's responses to a security controls questionnaire. The results indicate a positive commitment to fundamental security policies and user access controls.

\vspace{1em}

\begin{tabular}{p{0.75\textwidth} c}
    \toprule
    \textbf{Control Question} & \textbf{Response} \\
    \midrule
    Do you require MFA to access email? & \ding{51} \\
    Do you require MFA to log into computers? & \ding{51} \\
    Do you require MFA to access sensitive data systems? & \ding{51} \\
    Does your organization have an employee acceptable use policy? & \ding{51} \\
    Does your organization do security awareness training for new employees? & \ding{51} \\
    Does your organization do security awareness training for all employees at least once per year? & \ding{51} \\
    \bottomrule
\end{tabular}

\vspace{1em}
\noindent \textbf{Analysis:} The consistent "Yes" (\ding{51}) responses are indicative of a mature security posture from a policy and administrative standpoint. These controls are effective at mitigating common attack vectors such as phishing and credential theft.

% --- TECHNICAL SCAN RESULTS ---
\section*{3. Technical Scan Results}
An external network scan was performed on the target IP address. The scan identified one open port with a critically vulnerable service.

\begin{itemize}
    \item \textbf{Target IP:} \texttt{[Target IP]}
    \item \textbf{Scan Date:} Date not specified in scan data.
\end{itemize}

\subsection*{Open Ports and Services}
\begin{tabular}{l l l l p{0.3\textwidth}}
    \toprule
    \textbf{Port/Proto} & \textbf{State} & \textbf{Service} & \textbf{Product \& Version} & \textbf{Notes} \\
    \midrule
    21/tcp & open & ftp & vsftpd 2.3.4 & Anonymous FTP login is enabled. \textbf{This version contains a critical backdoor vulnerability (CVE-2011-2523).} \\
    \bottomrule
\end{tabular}

\vspace{1em}
\noindent \textbf{Analysis:} The presence of an open FTP port is generally discouraged in favor of more secure protocols like SFTP. The version of \texttt{vsftpd} (2.3.4) detected is extremely dangerous. It was compromised in 2011, and a backdoor was inserted into the source code, allowing an attacker to gain a remote command shell. The additional finding that anonymous login is permitted further lowers the bar for exploitation, making this a severe and urgent issue.

% --- RISK ASSESSMENT ---
\section*{4. Risk Assessment}
This section correlates the findings from the security control review, the technical scan, and pre-existing risk data into a consolidated list. Risks are prioritized based on severity.

\begin{tabular}{p{0.2\textwidth} p{0.12\textwidth} p{0.4\textwidth} p{0.2\textwidth}}
    \toprule
    \textbf{Risk Name} & \textbf{Severity} & \textbf{Overview} & \textbf{Affected Elements} \\
    \midrule
    \textbf{Vulnerable Public FTP Server} & \textbf{Critical (9.8)} & An externally accessible FTP server is running \texttt{vsftpd 2.3.4}, which contains a critical backdoor vulnerability. Anonymous login is also enabled, presenting an immediate threat of compromise. & The server at \texttt{[Target IP]} \\
    \addlinespace
    Outdated Windows Policy & Medium (5.0) & Computers are running the unsupported Windows 7 operating system, which no longer receives security updates and is vulnerable to known exploits. & Workstations \\
    \bottomrule
\end{tabular}

% --- RECOMMENDATIONS ---
\section*{5. Recommendations}
The following actionable recommendations are provided to mitigate the identified risks. They are prioritized based on the severity and potential impact on the organization.

\subsection*{Priority 1: Remediate Critical FTP Vulnerability (Immediate Action)}
The vulnerable FTP server at \texttt{[Target IP]} represents a clear and present danger to the organization and must be addressed immediately.
\begin{enumerate}
    \item \textbf{Isolate the Server:} Immediately take the FTP service offline or use a firewall to block all access to port 21 on \texttt{[Target IP]} from the internet.
    \item \textbf{Investigate for Compromise:} Assume the server has been compromised. Conduct a thorough forensic investigation to look for indicators of unauthorized access, malware, or data exfiltration.
    \item \textbf{Remediate:}
        \begin{itemize}
            \item \textbf{If FTP is not needed:} Decommission the service permanently. Ensure the service is disabled and the port is closed on the host and at the network firewall.
            \item \textbf{If FTP is required:} Rebuild the server from a known-good state. Install the latest stable version of an FTP server (e.g., a modern version of \texttt{vsftpd} or ProFTPD). Critically, disable anonymous access and configure the service to use secure protocols like FTPS or, preferably, migrate users to SFTP (which uses SSH).
        \end{itemize}
\end{enumerate}

\subsection*{Priority 2: Address Outdated Windows Policy (High Priority)}
The risk associated with running unsupported operating systems is significant and should be addressed as a high priority.
\begin{enumerate}
    \item \textbf{Upgrade Operating Systems:} Execute the plan to upgrade all remaining Windows 7 workstations to a supported version, such as Windows 10 or Windows 11.
    \item \textbf{Implement Asset Management:} Establish a formal hardware and software lifecycle management policy to ensure that systems are replaced or upgraded before they reach end-of-life.
\end{enumerate}

\end{document}
```