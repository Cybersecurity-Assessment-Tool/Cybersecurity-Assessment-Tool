```latex
\documentclass[12pt]{article}

% Preamble: Required Packages
\usepackage[a4paper, margin=1in]{geometry}
\usepackage{pifont} % For checkmarks and crosses (\ding)
\usepackage{booktabs} % For professional-looking tables
\usepackage{hyperref} % For hyperlinks, metadata
\usepackage{url} % For typesetting URLs
\usepackage{seqsplit} % To split long strings without breaking
\usepackage{xcolor} % For colors
\usepackage{fancyhdr} % For headers/footers
\usepackage{graphicx} % For images (e.g., logo)

% --- Document Metadata ---
\hypersetup{
    colorlinks=true,
    linkcolor=blue,
    filecolor=magenta,      
    urlcolor=cyan,
    pdftitle={Cybersecurity Posture Assessment Report},
    pdfauthor={Cybersecurity Analysis Division},
    pdfsubject={Security Analysis},
    pdfkeywords={Cybersecurity, Risk, Assessment},
}

% --- Page Style ---
\pagestyle{fancy}
\fancyhf{} % Clear all header and footer fields
\fancyhead[L]{Cybersecurity Assessment Report}
\fancyhead[R]{\textbf{[Organization Name]}}
\fancyfoot[C]{\thepage}
\renewcommand{\headrulewidth}{0.4pt}
\renewcommand{\footrulewidth}{0.4pt}

% --- Custom Commands ---
\newcommand{\yes}{\ding{51}}
\newcommand{\no}{\textcolor{red}{\ding{55}}}

% ==============================================================================
% --- BEGIN DOCUMENT ---
% ==============================================================================
\begin{document}

% --- Title Page ---
\begin{titlepage}
    \centering
    \vspace*{1cm}
    \Huge\textbf{Cybersecurity Posture Assessment Report}
    \vspace{1.5cm}
    \
    \large
    \textbf{Prepared for:}\\
    \vspace{0.5cm}
    \Huge\textbf{[Organization Name]}
    \
    \vspace{2cm}
    \large
    \textbf{Date of Report:}\\
    \vspace{0.5cm}
    \Large{\today}
    \
    \vfill
    \large
    \textbf{Generated by:}\\
    \vspace{0.5cm}
    Cybersecurity Analysis Division
\end{titlepage}

\tableofcontents
\newpage

% --- Section 1: Executive Summary ---
\section*{1. Executive Summary}

This report provides a cybersecurity posture assessment for \textbf{[Organization Name]}, based on an analysis of organizational security controls, external network scan results, and a review of pre-existing risks. The assessment was conducted on \today.

The analysis reveals a mixed security posture. The organization has implemented critical controls, such as requiring Multi-Factor Authentication (MFA) for email and sensitive data access. However, several significant gaps were identified that present a high level of risk. These include the lack of MFA for computer logins, the absence of a formal Acceptable Use Policy (AUP), and no security awareness training for new employees during their onboarding process.

From a technical perspective, a network scan identified an openly accessible Secure Shell (SSH) service on port 22. While necessary for remote administration, if not properly hardened, this service can be a primary target for brute-force attacks and unauthorized access attempts.

No pre-existing vulnerabilities were provided for review. This report concludes with prioritized, actionable recommendations to mitigate the identified risks and strengthen the overall security posture of the organization.

% --- Section 2: Organizational & Assessment Scope ---
\section*{2. Organizational Information and Assessment Scope}

This assessment synthesizes information from a security controls questionnaire and a technical network scan to provide a holistic view of the organization's security posture.

\subsection*{2.1. Organizational Details}
\begin{tabular}{@{}ll}
    \textbf{Organization Name:} & \textbf{[Organization Name]} \\
    \textbf{Primary Email Domain:} & \texttt{[Domain]} \\
    \textbf{Known External IP:} & \texttt{[Client IP]} \\
\end{tabular}

\subsection*{2.2. Assessment Scope}
\begin{tabular}{@{}ll}
    \textbf{Target IP Scanned:} & \texttt{[Target IP]} \\
    \textbf{Scan Date:} & \today \\
    \textbf{Data Sources:} & 1. Security Controls Questionnaire \\
                         & 2. External Nmap Network Scan \\
                         & 3. Pre-existing Risk Register (No risks provided) \\
\end{tabular}

% --- Section 3: Security Control Review ---
\section*{3. Security Control Review}

The following table summarizes the organization's responses to the security controls questionnaire. Items marked with a \no\ represent significant gaps in the current security framework and are discussed in the Risk Assessment section.

\begin{table}[h!]
\centering
\caption{Security Controls Questionnaire Results}
\begin{tabular}{@{}lp{0.6\textwidth}c@{}}
\toprule
\textbf{ID} & \textbf{Control Question} & \textbf{Status} \\
\midrule
C-01 & Do you require MFA to access email? & \yes \\
C-02 & Do you require MFA to log into computers? & \no \\
C-03 & Do you require MFA to access sensitive data systems? & \yes \\
C-04 & Does your organization have an employee acceptable use policy? & \no \\
C-05 & Does your organization do security awareness training for new employees? & \no \\
C-06 & Does your organization do security awareness training for all employees at least once per year? & \yes \\
\bottomrule
\end{tabular}
\end{table}

\paragraph{Analysis:} The organization has successfully implemented MFA for high-value targets like email and sensitive data systems. However, the lack of MFA on endpoint computers (C-02) leaves a critical entry point for attackers who compromise user credentials. Furthermore, the absence of foundational policies and training (C-04, C-05) indicates a reactive rather than proactive approach to security culture and governance.

% --- Section 4: Technical Scan Results ---
\section*{4. Technical Scan Results}

An external network scan was performed on the target IP address \texttt{[Target IP]} to identify accessible services.

\subsection*{4.1. Open Ports and Services}
The scan revealed the following open port:

\begin{table}[h!]
\centering
\caption{Nmap Scan Results for \texttt{[Target IP]}}
\begin{tabular}{@{}llll@{}}
\toprule
\textbf{Port} & \textbf{State} & \textbf{Service} & \textbf{Details} \\
\midrule
22/tcp & open & ssh & Service version information was not available from the scan. \\
\bottomrule
\end{tabular}
\end{table}

\paragraph{Analysis:} Port 22 is used for the Secure Shell (SSH) protocol, a standard for secure remote administration. While its use is common, exposing SSH to the entire internet increases the risk of brute-force password attacks and exploitation of potential vulnerabilities in the SSH server software. Without detailed version information, it is not possible to determine if the running service is vulnerable to known exploits.

% --- Section 5: Risk Assessment ---
\section*{5. Risk Assessment}

The following risks were identified by correlating the findings from the security control review and the technical scan. Each risk has been assigned a severity level based on its potential impact and likelihood of exploitation.

\begin{table}[h!]
\centering
\caption{Summary of Identified Risks}
\begin{tabular}{@{}lp{0.55\textwidth}l@{}}
\toprule
\textbf{ID} & \textbf{Risk Description} & \textbf{Severity} \\
\midrule
R-01 & \textbf{Lack of Endpoint MFA:} User computers do not require MFA for login. A compromised password could grant an attacker direct access to an endpoint and the internal network. & \textbf{High} \\
\addlinespace
R-02 & \textbf{Foundational Policy Gaps:} The absence of an Acceptable Use Policy and security training for new hires creates an inconsistent security culture and increases the likelihood of human error. & \textbf{High} \\
\addlinespace
R-03 & \textbf{Publicly Exposed SSH Service:} The SSH service on \texttt{[Target IP]} is open to the internet, making it a target for automated brute-force attacks and potential exploitation. & \textbf{Medium} \\
\bottomrule
\end{tabular}
\end{table}

% --- Section 6: Recommendations ---
\section*{6. Recommendations}

The following actions are recommended to mitigate the identified risks and improve the organization's security posture.

\subsection*{R-01: Lack of Endpoint MFA (High)}
\begin{itemize}
    \item \textbf{Implement MFA for All Computer Logins:} Deploy a solution (e.g., Windows Hello for Business, Duo, YubiKey) to enforce MFA for all users logging into their workstations and laptops. This adds a critical layer of defense against credential theft.
    \item \textbf{Prioritize Privileged Accounts:} If a phased rollout is necessary, begin immediately with all administrative and executive accounts.
\end{itemize}

\subsection*{R-02: Foundational Policy Gaps (High)}
\begin{itemize}
    \item \textbf{Develop and Implement an Acceptable Use Policy (AUP):} Create a formal AUP that clearly defines the rules for using company IT assets, data, and internet access. Require all employees to read and acknowledge the policy.
    \item \textbf{Integrate Security Training into Onboarding:} Develop a mandatory security awareness training module for all new employees. This training should be completed within their first week and cover topics such as phishing, password security, and the new AUP.
\end{itemize}

\subsection*{R-03: Publicly Exposed SSH Service (Medium)}
\begin{itemize}
    \item \textbf{Restrict Access via Firewall:} If possible, configure firewall rules to allow SSH access only from trusted IP addresses (e.g., administrator home or office networks).
    \item \textbf{Enforce Key-Based Authentication:} Disable password-based authentication for SSH and require the use of strong cryptographic keys. This is significantly more resistant to brute-force attacks.
    \item \textbf{Implement Intrusion Prevention:} Deploy a tool like Fail2ban to automatically block IP addresses that exhibit malicious behavior, such as repeated failed login attempts.
\end{itemize}

% --- Section 7: Conclusion ---
\section*{7. Conclusion}
\textbf{[Organization Name]} has established a baseline of security by protecting critical assets like email with MFA. However, this assessment highlights urgent risks at the endpoint and policy levels that currently undermine these protections. The identified gaps in endpoint MFA, employee onboarding, and policy enforcement represent the most immediate threats to the organization.

By implementing the recommendations outlined in this report, the organization can significantly reduce its attack surface, mitigate the risk of a security breach, and foster a more resilient security culture. We advise prioritizing the High-severity risks for immediate remediation.

% ==============================================================================
% --- END DOCUMENT ---
% ==============================================================================
\end{document}
```