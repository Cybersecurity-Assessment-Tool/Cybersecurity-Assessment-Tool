```latex
\documentclass[12pt, a4paper]{article}

% Preamble: Required Packages
\usepackage[margin=1in]{geometry}
\usepackage{pifont} % For checkmarks and crosses
\usepackage{booktabs} % For professional tables
\usepackage{hyperref} % For clickable links
\usepackage{url} % For URL formatting
\usepackage{seqsplit} % For splitting long strings without spaces
\usepackage{graphicx}
\usepackage[table]{xcolor}
\usepackage{fancyhdr}

% --- Document Setup ---
\definecolor{tablehead}{gray}{0.9}
\definecolor{criticalred}{RGB}{217, 83, 79}
\definecolor{highorange}{RGB}{240, 173, 78}
\definecolor{mediumyellow}{RGB}{255, 215, 0}
\definecolor{lowblue}{RGB}{91, 192, 222}

\hypersetup{
    colorlinks=true,
    linkcolor=blue,
    filecolor=magenta,      
    urlcolor=cyan,
    pdftitle={Cybersecurity Posture Assessment},
    pdfpagemode=FullScreen,
}

\pagestyle{fancy}
\fancyhf{}
\lhead{Cybersecurity Postuadfasdfre Assessment}
\rhead{\textbf{[Organization Name]}}
\cfoot{\thepage}

% --- Document Start ---
\begin{document}

% --- Title Page ---
\begin{titlepage}
    \centering
    \vspace*{1cm}
    \includegraphics[width=0.4\textwidth]{example-image-a} % Placeholder logo
    \vfill
    \huge\textbf{Cybersecurity Posture Assessment Report}
    \vspace{1cm}
    \Large\textbf{For: \textbf{[Organization Name]}}
    \vspace{2cm}
    \large
    \begin{tabular}{ll}
        \textbf{Report Date:} & \today \\
        \textbf{Author:} & Cybersecurity Analyst \\
        \textbf{Status:} & Final \\
    \end{tabular}
    \vfill
    \small
    \textit{This document contains sensitive information. Distribution is restricted to authorized personnel within \textbf{[Organization Name]}. Unauthorized disclosure is prohibited.}
\end{titlepage}

\tableofcontents
\newpage

% --- Section 1: Executive Summary ---
\section{Executive Summary}
This report provides a comprehensive assessment of the cybersecurity posture for \textbf{[Organization Name]}, based on an analysis of organizational security controls, external network scans, and pre-existing risk data. The assessment was conducted to identify key vulnerabilities, policy gaps, and technical misconfigurations that could expose the organization to significant cyber threats.

\paragraph{Key Findings:}
Several critical and high-risk issues were identified that require immediate attention. The most significant findings include:
\begin{itemize}
    \item \textbf{Critical Pre-existing Vulnerability:} A risk documented as "Localhost Exposed" with a CVSS score of 10.0 represents an immediate and severe threat.
    \item \textbf{Lack of Multi-Factor Authentication (MFA):} MFA is not enforced for logging into computers or accessing sensitive data systems. This gap dramatically increases the risk of unauthorized access and lateral movement following a credential compromise.
    \item \textbf{Missing Foundational Policies:} The absence of a formal Employee Acceptable Use Policy (AUP) creates ambiguity regarding security responsibilities and acceptable user behavior, weakening the overall security culture.
    \item \textbf{Exposed Network Services:} The external network scan identified an open SSH port (22), a common target for brute-force attacks and exploitation if not securely configured.
\end{itemize}

\paragraph{Overall Posture:}
The current security posture is considered high-risk due to the combination of policy deficiencies and critical technical vulnerabilities. While the organization has implemented security awareness training and MFA for email, foundational controls for protecting workstations and sensitive data are missing. Immediate remediation of the identified risks is strongly recommended to reduce the likelihood of a security breach.

\newpage

% --- Section 2: Organizational Information ---
\section{Organizational Information}
This section details the organizational data used as the basis for this assessment. The information was provided by the client.

\begin{table}[h!]
\centering
\rowcolors{2}{gray!10}{white}
\begin{tabular}{p{0.4\linewidth} p{0.5\linewidth}}
\toprule
\rowcolor{tablehead}
\textbf{Attribute} & \textbf{Value} \\
\midrule
Organization Name & \textbf{[Organization Name]} \\
Primary Domain & \texttt{[Domain]} \\
External IP Address Scanned & \texttt{[Client IP]} \\
Target IP Address Scanned & \texttt{[Target IP]} \\
\bottomrule
\end{tabular}
\caption{Client Organizational Details.}
\label{tab:org_info}
\end{table}

% --- Section 3: Security Control Review ---
\section{Security Control Review (Questionnaire Analysis)}
The following table summarizes the organization's responses to a security controls questionnaire. Each response is assessed against industry best practices. "No" answers indicate significant gaps in the security framework.

\begin{table}[h!]
\centering
\rowcolors{2}{gray!10}{white}
\begin{tabular}{p{0.5\linewidth} c p{0.3\linewidth}}
\toprule
\rowcolor{tablehead}
\textbf{Control Question} & \textbf{Response} & \textbf{Assessment} \\
\midrule
Do you require MFA to access email? & \ding{51} & \textbf{Good:} Protects a primary communication channel. \\
\addlinespace
Do you require MFA to log into computers? & \textbf{\color{criticalred}\ding{55}} & \textbf{High Risk:} Lack of MFA on endpoints allows for easy lateral movement if credentials are stolen. \\
\addlinespace
Do you require MFA to access sensitive data systems? & \textbf{\color{criticalred}\ding{55}} & \textbf{Critical Risk:} Core assets are not protected by a critical access control, leaving sensitive data vulnerable. \\
\addlinespace
Does your organization have an employee acceptable use policy? & \textbf{\color{criticalred}\ding{55}} & \textbf{High Risk:} Absence of a formal policy leads to inconsistent security practices and lack of enforceability. \\
\addlinespace
Does your organization do security awareness training for new employees? & \ding{51} & \textbf{Good:} Establishes a security baseline for new hires. \\
\addlinesdpace
Does your organization do security awareness training for all employees at least once per year? & \ding{51} & \textbf{Good:} Reinforces security concepts and addresses evolving threats. \\
\bottomrule
\end{tabular}
\caption{Security Controls Questionnaire Analysis.}
\label{tab:controls}
\end{table}

\newpage

% --- Section 4: Technical Scan Results ---
\section{Technical Scan Results}
An external network scan was performed against the target IP address \texttt{[Target IP]} to identify open ports and exposed services.

\subsection{Nmap Scan Findings}
The scan revealed the following open port, indicating a service accessible from the public internet.

\begin{table}[h!]
\centering
\rowcolors{2}{gray!10}{white}
\begin{tabular}{c c p{0.6\linewidth}}
\toprule
\rowcolor{tablehead}
\textbf{Port} & \textbf{Service} & \textbf{Analysis} \\
\midrule
22/TCP & SSH & The Secure Shell (SSH) protocol is open. This service is commonly used for remote administration. If not securely configured (e.g., weak passwords, outdated version), it is a primary target for brute-force attacks and remote code execution exploits. Version information was not available from this scan. \\
\bottomrule
\end{tabular}
\caption{Open Ports Detected on \texttt{[Target IP]}.}
\label{tab:nmap}
\end{table}

% --- Section 5: Consolidated Risk Assessment ---
\section{Consolidated Risk Assessment}
This section synthesizes findings from the questionnaire, technical scans, and pre-existing risk data into a prioritized list of security risks.

\begin{table}[h!]
\centering
\begin{tabular}{p{0.25\linewidth} p{0.5\linewidth} p{0.15\linewidth}}
\toprule
\rowcolor{tablehead}
\textbf{Risk Title} & \textbf{Description} & \textbf{Severity} \\
\midrule
\rowcolor{criticalred!20}
Localhost Exposed & A pre-existing critical vulnerability (CVSS 10.0) indicates a severe misconfiguration that could lead to a complete system compromise. & \textbf{Critical} \\
\addlinespace
\rowcolor{criticalred!20}
No MFA on Sensitive Data Systems & The absence of MFA on systems housing sensitive data exposes the organization's most valuable assets to unauthorized access via credential theft. & \textbf{Critical} \\
\addlinespace
\rowcolor{highorange!20}
No MFA on Workstations & Lack of MFA on computer logins allows an attacker with a single stolen password to gain network access and move laterally to other systems. & \textbf{High} \\
\addlinespace
\rowcolor{highorange!20}
Exposed SSH Service & The publicly accessible SSH port is a constant target for automated attacks. A successful compromise would grant an attacker remote administrative access. & \textbf{High} \\
\addlinespace
\rowcolor{highorange!20}
Missing Acceptable Use Policy (AUP) & Without a formal AUP, there is no enforceable standard for employee behavior, increasing the risk of insider threat and non-compliance. & \textbf{High} \\
\bottomrule
\end{tabular}
\caption{Summary of Identified Security Risks.}
\label{tab:risks}
\end{table}

\newpage

% --- Section 6: Recommendations ---
\section{Recommendations}
The following actionable recommendations are provided to mitigate the identified risks. They are prioritized based on severity.

\subsection{Immediate Actions (Critical Risks)}

\subsubsection{Remediate "Localhost Exposed" Vulnerability}
\begin{itemize}
    \item \textbf{Action:} Immediately investigate the "Localhost Exposed" finding. This typically refers to a service that should only be accessible locally (on 127.0.0.1) but is incorrectly bound to a public network interface.
    \item \textbf{Guidance:} Reconfigure the affected service(s) to bind only to the localhost interface or apply strict firewall rules to block external access.
\end{itemize}

\subsubsection{Implement MFA for Sensitive Systems}
\begin{itemize}
    \item \textbf{Action:} Deploy a robust MFA solution (e.g., TOTP, FIDO2/U2F keys) for all applications and systems that process or store sensitive data.
    \item \textbf{Guidance:} Prioritize systems with financial, customer, or proprietary data. Enforce this control for all users, including administrators and service accounts where possible.
\end{itemize}

\subsection{High-Priority Actions}

\subsubsection{Enforce MFA for Workstation Logins}
\begin{itemize}
    \item \textbf{Action:} Enable MFA for all employee computer logins (Windows, macOS, Linux).
    \item \textbf{Guidance:} Solutions like Windows Hello for Business, Duo, or other third-party identity providers can integrate with operating systems to provide this layer of security.
\end{itemize}

\subsubsection{Secure the Exposed SSH Service}
\begin{itemize}
    \item \textbf{Action:} Restrict and harden the public-facing SSH service.
    \item \textbf{Guidance:}
        \begin{enumerate}
            \item \textbf{Restrict Access:} If possible, restrict SSH access to known, trusted IP addresses using a firewall whitelist.
            \item \textbf{Disable Password Authentication:} Enforce public key-based authentication only. This prevents password-guessing and brute-force attacks.
            \item \textbf{Implement Fail2Ban:} Deploy software like Fail2Ban to automatically block IPs that exhibit malicious behavior (e.g., repeated failed login attempts).
            \item \textbf{Disable Root Login:} Prohibit direct root login via SSH by setting \texttt{PermitRootLogin no} in the \texttt{sshd\_config} file.
        \end{enumerate}
\end{itemize}

\subsubsection{Develop and Implement an Acceptable Use Policy (AUP)}
\begin{itemize}
    \item \textbf{Action:} Create a formal AUP that clearly defines the rules and responsibilities for all users of the organization's IT resources.
    \item \textbf{Guidance:} The policy should cover data handling, password security, internet usage, and consequences for non-compliance. Ensure all employees read and acknowledge the policy.
\end{itemize}

\end{document}
```