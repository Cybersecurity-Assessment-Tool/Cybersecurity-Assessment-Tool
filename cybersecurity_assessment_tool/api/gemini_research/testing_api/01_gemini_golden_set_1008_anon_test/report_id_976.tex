```latex
\documentclass[12pt]{article}

% Preamble: Required Packages
\usepackage[margin=1in]{geometry}
\usepackage{pifont} % For checkmarks and crosses
\usepackage{booktabs} % For professional tables
\usepackage{hyperref} % For clickable links
\usepackage{url} % For URL formatting
\usepackage{seqsplit} % For splitting long strings
\usepackage{graphicx}
\usepackage{xcolor}
\usepackage{fancyhdr}

% Document Metadata
\title{Cybersecurity Posture Assessment Report}
\author{Cybersecurity Analysis Division}
\date{\today}

% Header and Footer
\pagestyle{fancy}
\fancyhf{}
\lhead{\textbf{Cybersecurity Report}}
\rhead{\textbf{[Organization Name]}}
\cfoot{\thepage}

% Hyperref Setup
\hypersetup{
    colorlinks=true,
    linkcolor=blue,
    filecolor=magenta,      
    urlcolor=cyan,
    pdftitle={Cybersecurity Posture Assessment Report},
    pdfpagemode=FullScreen,
}

\begin{document}

\maketitle
\thispagestyle{empty}
\newpage
\tableofcontents
\newpage

% --- 1. Executive Summary ---
\section{Executive Summary}

This report provides a cybersecurity posture assessment for \textbf{[Organization Name]}, conducted on \today. The analysis synthesizes data from an external network scan, a review of internal security controls via a questionnaire, and a check of pre-existing risk registers.

The assessment identified a significant strength in the organization's network perimeter defense. An external scan of the public-facing IP address \texttt{[Client IP]} revealed no open ports, indicating a robust firewall configuration that effectively limits the external attack surface.

However, the review of internal security controls uncovered two high-priority gaps that require immediate attention:
\begin{itemize}
    \item \textbf{Critical Risk: Lack of MFA for Email.} The absence of Multi-Factor Authentication (MFA) on the email system presents a critical vulnerability. Email is a primary target for attackers, and a compromised account can lead to severe consequences, including data breaches, financial fraud, and further system compromise.
    \item \textbf{High Risk: No Security Training for New Employees.} New hires are not provided with security awareness training as part of their onboarding process. This creates a window of vulnerability where new staff are more susceptible to social engineering attacks and may be unaware of critical security policies.
\end{itemize}

This report details these findings and provides actionable recommendations to mitigate the identified risks and strengthen the organization's overall security posture.

% --- 2. Organizational Information ---
\section{Organizational Information}

The following details were used as the basis for this assessment. Where information was not provided, placeholders have been used.

\begin{tabular}{@{}ll}
    \toprule
    \textbf{Attribute} & \textbf{Value} \\
    \midrule
    Organization Name & \textbf{[Organization Name]} \\
    Primary Email Domain & \texttt{[Domain]} \\
    External IP Address & \texttt{[Client IP]} \\
    \bottomrule
\end{tabular}

% --- 3. Security Control Review ---
\section{Security Control Review}

A review of organizational security controls was conducted based on a standardized questionnaire. The responses indicate the current state of implemented policies and procedures. Gaps identified here often represent significant organizational risk.

\begin{tabular}{@{}p{0.8\linewidth}c}
    \toprule
    \textbf{Control Question} & \textbf{Response} \\
    \midrule
    Do you require MFA to access email? & \textcolor{red}{\ding{55}} No \\
    Do you require MFA to log into computers? & \textcolor{green}{\ding{51}} Yes \\
    Do you require MFA to access sensitive data systems? & \textcolor{green}{\ding{51}} Yes \\
    Does your organization have an employee acceptable use policy? & \textcolor{green}{\ding{51}} Yes \\
    Does your organization do security awareness training for new employees? & \textcolor{red}{\ding{55}} No \\
    Does your organization do security awareness training for all employees at least once per year? & \textcolor{green}{\ding{51}} Yes \\
    \bottomrule
\end{tabular}

% --- 4. Technical Scan Results ---
\section{Technical Scan Results}

An external network vulnerability scan was performed to identify open ports and services exposed to the internet.

\begin{itemize}
    \item \textbf{Target IP Address:} \texttt{[Target IP]}
    \item \textbf{Scan Date:} \textbf{[Scan Date]}
    \item \textbf{Scanner Used:} Nmap
\end{itemize}

\subsection{Summary of Findings}
The scan concluded with a positive security finding:
\begin{itemize}
    \item \textbf{No Open Ports Detected:} The scan confirmed that all 1000 most common TCP ports were in a \texttt{closed} state. This indicates that the perimeter firewall is correctly configured to deny unsolicited inbound traffic, significantly reducing the external attack surface. No services were exposed.
\end{itemize}

% --- 5. Risk Assessment ---
\section{Risk Assessment}

This section synthesizes findings from the security control review and technical scan. While the technical posture is strong, critical procedural gaps were identified. The pre-existing risk register was empty.

\begin{table}[h!]
\centering
\begin{tabular}{@{}p{0.1\linewidth} p{0.25\linewidth} p{0.45\linewidth} p{0.1\linewidth}@{}}
    \toprule
    \textbf{Risk ID} & \textbf{Risk Name} & \textbf{Description} & \textbf{Severity} \\
    \midrule
    ORG-001 & Lack of MFA for Email Access & Email accounts are protected only by a password. This exposes the organization to a high risk of account compromise via phishing, credential stuffing, or password spraying, which can lead to Business Email Compromise (BEC) and data exfiltration. & \textbf{Critical} \\
    \addlinespace
    ORG-002 & No Security Training for New Hires & New employees do not receive security awareness training during onboarding. This creates a significant vulnerability, as untrained staff are more likely to fall for social engineering attacks or mishandle sensitive data, undermining other security controls. & \textbf{High} \\
    \bottomrule
\end{tabular}
\caption{Summary of Identified Risks}
\end{table}

% --- 6. Recommendations ---
\section{Recommendations}

The following actions are recommended to address the identified risks. Recommendations are prioritized based on severity and potential impact.

\subsection{ORG-001: Implement MFA for Email (Critical)}
\begin{itemize}
    \item \textbf{Action:} Immediately procure, configure, and enforce Multi-Factor Authentication (MFA) for all user accounts with access to the organization's email system.
    \item \textbf{Justification:} MFA is the single most effective control for preventing unauthorized account access. It provides a critical second layer of defense that protects against compromised credentials, drastically reducing the likelihood of a successful email-based attack.
    \item \textbf{Priority:} \textbf{Critical}
\end{itemize}

\subsection{ORG-002: Integrate Security Training into Onboarding (High)}
\begin{itemize}
    \item \textbf{Action:} Develop a mandatory security awareness training module and integrate it into the formal onboarding process for all new employees and contractors. This training should cover, at a minimum, acceptable use policies, phishing and social engineering awareness, and data handling procedures.
    \item \textbf{Justification:} A strong security culture starts on day one. Training new hires ensures a baseline level of security knowledge across the organization, reducing human error and strengthening the human firewall.
    \item \textbf{Priority:} \textbf{High}
\end{itemize}

\end{document}
```