```latex
\documentclass[12pt]{article}

% 1. Required Packages
\usepackage[margin=1in]{geometry}
\usepackage{pifont} % For checkmarks and crosses
\usepackage{booktabs} % For professional tables
\usepackage{hyperref} % For hyperlinks
\usepackage{url} % For URL formatting
\usepackage{seqsplit} % For splitting long strings
\usepackage{graphicx}
\usepackage{xcolor}

% --- Document Setup ---
\hypersetup{
    colorlinks=true,
    linkcolor=blue,
    filecolor=magenta,      
    urlcolor=cyan,
    pdftitle={Cybersecurity Posture Report},
    pdfpagemode=FullScreen,
}

\newcommand{\yes}{\ding{51}}
\newcommand{\no}{\ding{55}}

\begin{document}

% --- Title Page ---
\begin{titlepage}
    \centering
    \vspace*{1cm}
    \Huge\textbf{Cybersecurity Posture Report}
    \vspace{1.5cm}
    \Large
    \textbf{Prepared for:}\\
    \vspace{0.5cm}
    \textbf{[Organization Name]}
    \vspace{2cm}
    \large
    \textbf{Date of Report:}\\
    \vspace{0.5cm}
    \today
    \vfill
    \large
    \textbf{Generated by:}\\
    \vspace{0.5cm}
    Cybersecurity Analysis Engine
\end{titlepage}

\tableofcontents
\newpage

% --- Section 1: Executive Summary ---
\section{Executive Summary}
This report provides a comprehensive analysis of the cybersecurity posture for \textbf{[Organization Name]}, based on data collected from a network scan, a security controls questionnaire, and a review of pre-existing risks. The assessment was conducted on \today.

The analysis reveals several critical and high-risk security deficiencies that require immediate attention. The most significant findings include a complete absence of Multi-Factor Authentication (MFA) across all critical services, including email, computer logins, and access to sensitive data. This represents a critical vulnerability that significantly increases the risk of account compromise and unauthorized access.

Furthermore, the organization lacks fundamental governance controls, such as an employee acceptable use policy and mandatory annual security awareness training. These gaps in policy and training cultivate a high-risk environment susceptible to human error and insider threats.

Technically, an exposed SSH (Secure Shell) service was identified on the external network. When combined with the lack of MFA, this exposed management port presents a tangible target for external attackers. No pre-existing vulnerabilities were documented, indicating that the risks identified in this report are newly discovered and must be prioritized for remediation.

\section{Organizational Information}
The following details were used as the basis for this assessment. As per the provided data, identity information has been anonymized.

\begin{itemize}
    \item \textbf{Organization Name:} \textbf{[Organization Name]}
    \item \textbf{Primary Domain:} \texttt{[Domain]}
    \item \textbf{External IP Scanned:} \texttt{[Client IP]}
\end{itemize}

% --- Section 3: Security Control Review ---
\section{Security Control Review (Questionnaire Analysis)}
A review of the organization's security controls was conducted via a questionnaire. The responses indicate significant gaps in foundational security practices. A summary of the findings is presented in Table \ref{tab:controls}. Items marked with \no{} represent a failure to meet baseline security standards and are considered high-risk findings.

\begin{table}[h!]
\centering
\caption{Security Controls Questionnaire Results}
\label{tab:controls}
\begin{tabular}{@{}p{0.6\linewidth} c l@{}}
\toprule
\textbf{Control Question} & \textbf{Response} & \textbf{Assessment} \\
\midrule
Do you require MFA to access email? & \no & \textcolor{red}{\textbf{Critical Gap}} \\
Do you require MFA to log into computers? & \no & \textcolor{red}{\textbf{Critical Gap}} \\
Do you require MFA to access sensitive data systems? & \no & \textcolor{red}{\textbf{Critical Gap}} \\
Does your organization have an employee acceptable use policy? & \no & \textcolor{orange}{High Risk} \\
Does your organization do security awareness training for new employees? & \yes & Best Practice Met \\
Does your organization do security awareness training for all employees at least once per year? & \no & \textcolor{orange}{High Risk} \\
\bottomrule
\end{tabular}
\end{table}

% --- Section 4: Technical Scan Results ---
\section{Technical Scan Results}
An external network scan was performed to identify exposed services and potential vulnerabilities. The scan was executed against the target IP address \texttt{[Target IP]}.

\subsection{Open Ports}
The scan identified the following open port, which is accessible from the public internet.

\begin{table}[h!]
\centering
\caption{Discovered Open Ports}
\label{tab:ports}
\begin{tabular}{@{}llll@{}}
\toprule
\textbf{Port} & \textbf{State} & \textbf{Service} & \textbf{Product / Version} \\
\midrule
22/tcp & open & ssh & Not Disclosed \\
\bottomrule
\end{tabular}
\end{table}

\subsection{Technical Analysis}
The discovery of an open SSH port (22) is a significant finding. SSH is a common protocol for remote server administration. While essential for system management, its exposure to the public internet creates a substantial attack surface. Attackers frequently scan for open SSH ports to perform brute-force password attacks or exploit known vulnerabilities. Given the lack of MFA enforcement within the organization, a compromised user credential could be leveraged to gain direct administrative access to this system.

% --- Section 5: Consolidated Risk Assessment ---
\section{Consolidated Risk Assessment}
The following table synthesizes findings from the security control review and the technical scan. No previously documented risks were provided. The risks below have been identified and prioritized based on their potential impact on the organization.

\begin{table}[h!]
\centering
\caption{Summary of Identified Risks}
\label{tab:risks}
\begin{tabular}{@{}p{0.15\linewidth} p{0.55\linewidth} l@{}}
\toprule
\textbf{Risk ID} & \textbf{Description} & \textbf{Severity} \\
\midrule
RISK-001 & \textbf{Widespread Lack of Multi-Factor Authentication (MFA):} The absence of MFA on email, computer logins, and sensitive systems exposes the organization to severe risk of account takeover and data breaches. & \textcolor{red}{\textbf{Critical}} \\
\addlinespace
RISK-002 & \textbf{Exposed SSH Management Port:} An open SSH port on the external network provides a direct vector for brute-force attacks and unauthorized access, a risk magnified by the lack of MFA. & \textcolor{orange}{\textbf{High}} \\
\addlinespace
RISK-003 & \textbf{Deficient Security Policies and Training:} The lack of an Acceptable Use Policy and annual security training indicates a weak security culture, increasing the likelihood of security incidents caused by human error. & \textcolor{orange}{\textbf{High}} \\
\bottomrule
\end{tabular}
\end{table}

% --- Section 6: Recommendations ---
\section{Recommendations}
To mitigate the identified risks and improve the overall security posture, the following actions are strongly recommended.

\subsection{Remediation for RISK-001 (Critical)}
\begin{itemize}
    \item \textbf{Immediate Action:} Procure and deploy an MFA solution across the organization.
    \item \textbf{Priority 1:} Enforce MFA for all user access to email systems (e.g., Office 365, Google Workspace).
    \item \textbf{Priority 2:} Enforce MFA for all remote access systems (VPNs) and access to sensitive data repositories.
    \item \textbf{Priority 3:} Implement MFA for all workstation and server logins.
\end{itemize}

\subsection{Remediation for RISK-002 (High)}
\begin{itemize}
    \item \textbf{Immediate Action:} If SSH access is not required from the public internet, block port 22 at the network firewall immediately.
    \item \textbf{If Access is Required:}
        \begin{enumerate}
            \item Implement a firewall rule to restrict access to port 22 to only trusted IP addresses.
            \item Configure the SSH service to disable password-based authentication and require public key authentication only.
            \item If possible, place the SSH service behind a Virtual Private Network (VPN) that requires MFA for access.
        \end{enumerate}
\end{itemize}

\subsection{Remediation for RISK-003 (High)}
\begin{itemize}
    \item \textbf{Policy Development:} Draft and formally approve an Employee Acceptable Use Policy (AUP). This policy should be reviewed and signed by all current and new employees.
    \item \textbf{Training Program:} Implement a mandatory, annual security awareness training program for all employees. This program should cover topics such as phishing, password security, and data handling.
\end{itemize}

\end{document}
```