```latex
\documentclass[12pt]{article}

% Preamble: Required Packages
\usepackage[margin=1in]{geometry}
\usepackage{pifont} % For checkmarks and crosses (\ding)
\usepackage{booktabs} % For professional-looking tables
\usepackage{hyperref} % For clickable links and references
\usepackage{url} % For formatting URLs
\usepackage{seqsplit} % To split long strings without breaking words
\usepackage{graphicx}
\usepackage{fancyhdr}
\usepackage{xcolor}
\usepackage{array}

% --- Document Setup ---

% Define colors for hyperref
\definecolor{darkblue}{rgb}{0.0, 0.0, 0.55}
\hypersetup{
    colorlinks=true,
    linkcolor=darkblue,
    filecolor=darkblue,
    urlcolor=darkblue,
    citecolor=darkblue,
}

% Custom header and footer
\pagestyle{fancy}
\fancyhf{} % Clear all header and footer fields
\lhead{Cybersecurity Assessment Report}
\rhead{\textbf{[Organization Name]}}
\cfoot{Page \thepage}
\renewcommand{\headrulewidth}{0.4pt}
\renewcommand{\footrulewidth}{0.4pt}

% Define a new column type for better table formatting
\newcolumntype{L}[1]{>{\raggedright\let\newline\\\arraybackslash\hspace{0pt}}m{#1}}
\newcolumntype{C}[1]{>{\centering\let\newline\\\arraybackslash\hspace{0pt}}m{#1}}

% --- Document Start ---
\begin{document}

\title{
    \vspace{-2cm}
    \includegraphics[width=0.3\textwidth]{./logo.png}\\[1cm] % Placeholder for a logo
    \textbf{Cybersecurity Posture Assessment Report}\\[0.5cm]
    \large Prepared for: \textbf{[Organization Name]}
}
\author{Cybersecurity Analysis Division}
\date{\today}

\maketitle
\thispagestyle{empty}
\newpage

\tableofcontents
\newpage

% --- Section 1: Executive Summary ---
\section{Executive Summary}

This report details the findings of a cybersecurity assessment conducted for \textbf{[Organization Name]}. The analysis synthesized data from an external network scan, a security controls questionnaire, and a review of pre-existing risks.

A \textbf{critical risk} was identified: the direct exposure of the Remote Desktop Protocol (RDP) on the public internet at \texttt{[Client IP]}. This configuration is a primary target for ransomware gangs and other malicious actors, representing an immediate and severe threat to the organization's network integrity and data security.

This technical vulnerability is significantly amplified by critical gaps in foundational security controls. The lack of Multi-Factor Authentication (MFA) for email and computer logins means that a single compromised password could lead to a full network breach. Furthermore, the absence of annual security awareness training for all employees increases the likelihood of such a credential compromise occurring through phishing or other social engineering attacks.

The overall security posture is assessed as \textbf{High-Risk}. Immediate remediation is required to address the exposed RDP service. Strategic initiatives to implement comprehensive MFA and a robust security training program are strongly recommended to build a more resilient security foundation.

% --- Section 2: Organizational Information ---
\section{Assessment Scope and Information}

This assessment focused on the external network perimeter and internal security policies of the organization. The following details were used as the basis for this report.

\begin{itemize}
    \item \textbf{Organization Name:} \textbf{[Organization Name]}
    \item \textbf{Primary Email Domain:} \texttt{[Domain]}
    \item \textbf{External IP Scanned:} \texttt{[Client IP]}
    \item \textbf{Target IP in Scan Data:} \texttt{[Target IP]}
\end{itemize}

% --- Section 3: Security Control Review ---
\section{Security Control Review}

A review of the organization's security controls was conducted via a questionnaire. The responses highlight key areas of strength and weakness in the current security policy framework. "No" answers indicate significant gaps that increase organizational risk.

\begin{table}[h!]
\centering
\caption{Security Controls Questionnaire Results}
\begin{tabular}{L{11cm} C{2cm} C{1.5cm}}
\toprule
\textbf{Control Question} & \textbf{Response} & \textbf{Status} \\
\midrule
Do you require MFA to access email? & No & \textcolor{red}{\ding{55}} \\
Do you require MFA to log into computers? & No & \textcolor{red}{\ding{55}} \\
Do you require MFA to access sensitive data systems? & Yes & \textcolor{green}{\ding{51}} \\
Does your organization have an employee acceptable use policy? & Yes & \textcolor{green}{\ding{51}} \\
Does your organization do security awareness training for new employees? & Yes & \textcolor{green}{\ding{51}} \\
Does your organization do security awareness training for all employees at least once per year? & No & \textcolor{red}{\ding{55}} \\
\bottomrule
\end{tabular}
\end{table}

\subsection*{Analysis of Control Gaps}
\begin{itemize}
    \item \textbf{Lack of MFA:} The absence of MFA for email and computer logins is a critical weakness. These are the two most common entry points for attackers who have stolen user credentials.
    \item \textbf{Inadequate Security Training:} While new employees receive training, the lack of an annual refresher for all staff allows security knowledge to become outdated, making the organization more susceptible to evolving phishing and social engineering tactics.
\end{itemize}

% --- Section 4: Technical Scan Results ---
\section{Technical Scan Results}

An external network scan was performed to identify open ports and exposed services on the organization's public-facing infrastructure.

\begin{table}[h!]
\centering
\caption{Open Ports Detected on \texttt{[Target IP]}}
\begin{tabular}{l l l L{7cm}}
\toprule
\textbf{Port} & \textbf{State} & \textbf{Service Name} & \textbf{Analysis} \\
\midrule
3389/tcp & open & \texttt{ms-wbt-server} & This is the default port for Microsoft's Remote Desktop Protocol (RDP). Exposing RDP directly to the internet is extremely dangerous and is a common vector for brute-force attacks and ransomware deployment. \\
\bottomrule
\end{tabular}
\end{table}

% --- Section 5: Correlated Risk Assessment ---
\section{Correlated Risk Assessment}

This section synthesizes the findings from the security questionnaire, the technical scan, and pre-existing risk data to provide a holistic view of the organization's risk profile.

\begin{table}[h!]
\centering
\caption{Summary of Identified Risks}
\begin{tabular}{p{4cm} p{1.5cm} p{8.5cm}}
\toprule
\textbf{Risk Title} & \textbf{Severity} & \textbf{Description and Correlation} \\
\midrule
\textbf{Publicly Exposed RDP Service} & \textbf{Critical} & The technical scan and pre-existing risk data confirm that RDP is open on port 3389. This allows attackers to attempt direct remote access to the internal network. This is a well-known and actively exploited vulnerability. \\
\addlinespace
\textbf{Insufficient Multi-Factor Authentication} & \textbf{Critical} & The questionnaire revealed a lack of MFA for email and computer access. This directly correlates with the exposed RDP risk; if an attacker obtains a user's password, they can log in without needing a second factor, gaining a foothold in the network. \\
\addlinespace
\textbf{Inadequate Security Awareness Program} & \textbf{High} & The lack of annual training for all employees makes it more likely that an attacker can successfully steal credentials via a phishing email. This weakness serves as a potential trigger for the other identified risks. \\
\bottomrule
\end{tabular}
\end{table}

% --- Section 6: Recommendations ---
\section{Recommendations}

The following actions are recommended to mitigate the identified risks. They are prioritized based on the severity of the risk and the urgency of the required action.

\subsection*{Priority 1: Immediate Actions (Remediate within 24 hours)}
\begin{itemize}
    \item \textbf{Isolate the Exposed RDP Service:}
    \begin{itemize}
        \item \textbf{Action:} Apply a firewall rule to \textbf{block all inbound traffic to TCP port 3389} on the external IP address \texttt{[Client IP]}. This is the most critical first step.
        \item \textbf{Justification:} This action immediately removes the direct threat of an external attacker accessing the network via RDP.
    \end{itemize}
\end{itemize}

\subsection*{Priority 2: Short-Term Actions (Remediate within 30 days)}
\begin{itemize}
    \item \textbf{Implement Comprehensive MFA:}
    \begin{itemize}
        \item \textbf{Action:} Procure and enforce an MFA solution for all user accounts. Prioritize email (e.g., Office 365, Google Workspace) and all computer logins (local and remote).
        \item \textbf{Justification:} MFA is one of the most effective controls for preventing account takeovers, even if passwords are stolen.
    \end{itemize}
\end{itemize}

\subsection*{Priority 3: Foundational Improvements (Implement within 90 days)}
\begin{itemize}
    \item \textbf{Establish Annual Security Awareness Training:}
    \begin{itemize}
        \item \textbf{Action:} Develop and mandate an annual security awareness training program for all employees, covering topics like phishing, password hygiene, and acceptable use.
        \item \textbf{Justification:} A well-trained workforce is a critical layer of defense, reducing the "human factor" risk.
    \end{itemize}
    \item \textbf{Deploy a Secure Remote Access Solution:}
    \begin{itemize}
        \item \textbf{Action:} For any legitimate need for remote access, replace direct RDP exposure with a secure, modern solution like a Virtual Private Network (VPN) or a Zero Trust Network Access (ZTNA) platform. All such solutions must be protected with MFA.
        \item \textbf{Justification:} This provides a secure, encrypted, and authenticated channel for remote work, adhering to the principle of least privilege.
    \end{itemize}
\end{itemize}

\end{document}
```