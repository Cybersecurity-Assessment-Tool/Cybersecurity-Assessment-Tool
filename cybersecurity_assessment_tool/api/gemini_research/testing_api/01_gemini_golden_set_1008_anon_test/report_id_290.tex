```latex
\documentclass[12pt]{article}

% Required Packages
\usepackage[margin=1in]{geometry}
\usepackage{pifont} % For \ding symbols (checkmark/cross)
\usepackage{booktabs} % For professional-looking tables
\usepackage{hyperref} % For clickable links and references
\usepackage{url}      % For properly formatting URLs
\usepackage{seqsplit} % For splitting long strings in texttt
\usepackage[table]{xcolor} % For coloring table cells and text

% --- Document Setup ---
\hypersetup{
    colorlinks=true,
    linkcolor=blue,
    filecolor=magenta,
    urlcolor=cyan,
    pdftitle={Cybersecurity Posture Assessment Report},
    pdfauthor={Cybersecurity Analysis Division},
}

% --- Custom Commands ---
\newcommand{\yes}{\ding{51}} % Checkmark
\newcommand{\no}{\ding{55}}  % Cross mark
\newcommand{\criticalrisk}{\textcolor{red}{\textbf{Critical}}}
\newcommand{\highrisk}{\textcolor{orange}{\textbf{High}}}
\newcommand{\mediumrisk}{\textcolor{yellow!80!black}{\textbf{Medium}}}

% --- Document ---
\begin{document}

\title{Cybersecurity Posture Assessment Report}
\author{Cybersecurity Analysis Division}
\date{\today}
\maketitle

\begin{abstract}
This report provides a comprehensive analysis of the cybersecurity posture for \textbf{[Organization Name]}. The assessment is based on a correlation of external network scan data, a review of internal security controls via a questionnaire, and an evaluation of pre-existing risks. The analysis has identified several critical and high-risk vulnerabilities that require immediate attention. Key findings include an externally exposed, outdated, and vulnerable FTP server, significant gaps in Multi-Factor Authentication (MFA) deployment, and a lack of a formal security awareness training program. These issues, combined, present a significant risk of unauthorized access and potential data breach.
\end{abstract}

\section*{1. Overview and Scope}
This assessment was conducted to evaluate the security posture of \textbf{[Organization Name]}. The scope of this report includes:
\begin{itemize}
    \item A review of organizational security policies and controls.
    \item A technical analysis of an external network scan performed against the client's perimeter.
    \item A consolidated risk register incorporating technical findings, policy gaps, and known issues.
\end{itemize}
The primary objective is to identify vulnerabilities, assess their potential impact, and provide actionable recommendations for remediation.

\section*{2. Organizational Information}
The following information was used as the basis for this assessment. As per the template mode for this report, placeholders are used where data was not provided.

\begin{tabular}{@{}ll}
    \toprule
    \textbf{Attribute} & \textbf{Value} \\
    \midrule
    Organization Name & \textbf{[Organization Name]} \\
    Primary Domain & \texttt{[Domain]} \\
    External IP Scanned & \texttt{[Client IP]} \\
    Target of Nmap Scan & \texttt{[Target IP]} \\
    \bottomrule
\end{tabular}

\section*{3. Security Control Review}
The following table summarizes the organization's responses to a security controls questionnaire. Items marked with a cross (\no) represent significant gaps in the security framework and are discussed in the Risk Assessment section.

\begin{center}
\begin{tabular}{p{0.7\textwidth}c}
    \toprule
    \textbf{Control Question} & \textbf{Status} \\
    \midrule
    Do you require MFA to access email? & \no \\
    Do you require MFA to log into computers? & \yes \\
    Do you require MFA to access sensitive data systems? & \no \\
    Does your organization have an employee acceptable use policy? & \yes \\
    Does your organization do security awareness training for new employees? & \no \\
    Does your organization do security awareness training for all employees at least once per year? & \no \\
    \bottomrule
\end{tabular}
\end{center}

\section*{4. Technical Scan Results}
An external network scan was performed using Nmap. The scan identified the following open ports and services on the target system.

\subsection*{Open Ports and Services}
\begin{center}
\begin{tabular}{lllll}
    \toprule
    \textbf{Port} & \textbf{State} & \textbf{Service} & \textbf{Product \& Version} \\
    \midrule
    21/tcp & open & ftp & vsftpd 2.3.4 \\
    \bottomrule
\end{tabular}
\end{center}

\subsection*{Analysis of Technical Findings}
The scan revealed a critical vulnerability.
\begin{itemize}
    \item \textbf{Vulnerable FTP Service:} The server is running \textbf{vsftpd version 2.3.4}. This specific version is widely known to contain a critical backdoor vulnerability (CVE-2011-2523). If exploited, this vulnerability allows an attacker to execute arbitrary commands on the server with root-level privileges.
    \item \textbf{Anonymous FTP Enabled:} The scan confirmed that "Anonymous FTP login allowed". This is a severe misconfiguration that permits any external user to connect to the FTP server without authentication, potentially allowing them to upload malicious files or download sensitive information.
\end{itemize}

\section*{5. Consolidated Risk Assessment}
The following table synthesizes findings from the security control review, technical scan, and pre-existing risk data into a prioritized list.

\begin{center}
\begin{tabular}{lp{0.55\textwidth}l}
    \toprule
    \textbf{Severity} & \textbf{Risk / Vulnerability} & \textbf{Source} \\
    \midrule
    \criticalrisk & \textbf{Vulnerable FTP Service} & An outdated version of vsftpd (2.3.4) with a known remote code execution backdoor is exposed to the internet. Anonymous login is also enabled. \\
    \addlinespace
    \criticalrisk & \textbf{Lack of MFA} & Multi-Factor Authentication is not enforced for email or access to sensitive data systems, creating a high risk of account compromise and data breach. \\
    \addlinespace
    \highrisk & \textbf{No Security Awareness Training} & The absence of a training program for new or existing employees increases susceptibility to phishing, social engineering, and other human-targeted attacks. \\
    \addlinespace
    \mediumrisk & \textbf{Outdated Windows Policy} & The continued use of Windows 7, an end-of-life operating system, exposes workstations to numerous unpatched vulnerabilities. \\
    \bottomrule
\end{tabular}
\end{center}

\section*{6. Recommendations}
Based on the analysis, we recommend the following actions, prioritized by severity.

\subsection*{Immediate Actions (Priority 1)}
\begin{enumerate}
    \item \textbf{Remediate FTP Server:}
    \begin{itemize}
        \item Immediately take the FTP server offline.
        \item If the service is business-critical, upgrade \texttt{vsftpd} to the latest stable version and place it behind a firewall with strict access control lists.
        \item If the service is not required, decommission it permanently.
        \item Disable anonymous FTP login under all circumstances.
    \end{itemize}
    \item \textbf{Deploy MFA:}
    \begin{itemize}
        \item Implement and enforce MFA for all email accounts (e.g., Office 365, Google Workspace) immediately.
        \item Prioritize the rollout of MFA for all systems identified as containing sensitive data.
    \end{itemize}
\end{enumerate}

\subsection*{High-Priority Actions (Priority 2)}
\begin{enumerate}
    \setcounter{enumi}{2}
    \item \textbf{Establish Security Awareness Training:}
    \begin{itemize}
        \item Implement a mandatory security awareness training module for all new employees as part of their onboarding process.
        \item Schedule and conduct annual security awareness training for all staff, covering topics like phishing, password hygiene, and acceptable use.
    \end{itemize}
\end{enumerate}

\subsection*{Medium-Priority Actions (Priority 3)}
\begin{enumerate}
    \setcounter{enumi}{3}
    \item \textbf{Address Outdated Systems:}
    \begin{itemize}
        \item Continue with the existing plan to upgrade all Windows 7 workstations to a supported operating system like Windows 10 or 11.
        \item Develop a formal patch and vulnerability management policy to prevent systems from becoming unsupported in the future.
    \end{itemize}
\end{enumerate}

\end{document}
```