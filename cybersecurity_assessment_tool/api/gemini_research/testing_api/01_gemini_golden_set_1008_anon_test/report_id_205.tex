```latex
\documentclass[12pt]{article}

% Preamble: Required Packages
\usepackage[margin=1in]{geometry}
\usepackage{pifont} % For checkmarks and crosses (\ding)
\usepackage{booktabs} % For professional-looking tables
\usepackage{hyperref} % For clickable links
\usepackage{url}      % For URL formatting
\usepackage{seqsplit} % To split long strings without breaking
\usepackage{xcolor}   % For custom colors

% Hyperref Configuration
\hypersetup{
    colorlinks=true,
    linkcolor=blue,
    filecolor=magenta,
    urlcolor=cyan,
}

% Custom Commands for Report Consistency
\newcommand{\cmark}{\ding{51}} % Checkmark
\newcommand{\xmark}{\ding{55}} % Cross
\newcommand{\riskcritical}{\textbf{\textcolor{red}{Critical}}}
\newcommand{\riskhigh}{\textbf{\textcolor{orange}{High}}}
\newcommand{\riskmedium}{\textbf{\textcolor{yellow}{Medium}}}
\newcommand{\risklow}{\textbf{Low}}

% Document Start
\begin{document}

% --- Title Page ---
\begin{titlepage}
    \centering
    \vspace*{1cm}
    \Huge \textbf{Cybersecurity Posture Assessment Report}
    \vspace{1.5cm}
    \Large Prepared for:
    \vspace{0.5cm}
    \huge \textbf{[Organization Name]}
    \vspace{2cm}
    \large \textbf{Date of Report:} \today \\
    \textbf{Date of Scan:} 2025-11-22
    \vfill
    \large \textbf{Generated by:} \\
    Cybersecurity Analysis Division
\end{titlepage}

\tableofcontents
\newpage

% --- Section 1: Executive Summary ---
\section*{1. Executive Summary}
This report details the findings of a cybersecurity assessment for \textbf{[Organization Name]}. The assessment combined a review of organizational security controls via questionnaire, an external network vulnerability scan, and an analysis of known risks.

While the organization has implemented some positive security controls, such as Multi-Factor Authentication (MFA) for computer and sensitive system access, several \riskcritical{} and \riskhigh{} risk vulnerabilities were identified.

Key findings include a \textbf{\riskcritical{} lack of MFA on email services}, the use of \textbf{\riskhigh{} outdated and vulnerable web server software}, and significant gaps in security policy and employee training. These issues collectively expose the organization to a high likelihood of account compromise, data breach, and successful phishing attacks.

Immediate remediation of the identified risks is strongly recommended to improve the overall security posture and reduce the attack surface.

% --- Section 2: Organizational Information ---
\section*{2. Organizational Information}
This section provides the high-level details of the entity under assessment. The data is based on the information provided for this engagement.

\begin{tabular}{@{}ll}
\toprule
\textbf{Attribute} & \textbf{Value} \\
\midrule
Organization Name & \textbf{[Organization Name]} \\
Email Domain & \texttt{[Domain]} \\
Assessed External IP & \texttt{[Client IP]} \\
\bottomrule
\end{tabular}

% --- Section 3: Security Control Review ---
\section*{3. Security Control Review}
The following table summarizes the organization's self-reported security posture based on a standard controls questionnaire. Items marked with an \xmark\ represent significant gaps that increase organizational risk.

\vspace{1em}
\begin{tabular}{@{}p{0.8\linewidth}c@{}}
\toprule
\textbf{Control Question} & \textbf{Status} \\
\midrule
Do you require MFA to access email? & \xmark \\
Do you require MFA to log into computers? & \cmark \\
Do you require MFA to access sensitive data systems? & \cmark \\
Does your organization have an employee acceptable use policy? & \xmark \\
Does your organization do security awareness training for new employees? & \cmark \\
Does your organization do security awareness training for all employees at least once per year? & \xmark \\
\bottomrule
\end{tabular}

% --- Section 4: Technical Scan Results ---
\section*{4. Technical Scan Results}
An external network scan was performed to identify exposed services and potential technical vulnerabilities.

\begin{itemize}
    \item \textbf{Scan Target:} \texttt{[Target IP]}
    \item \textbf{Scan Date:} 2025-11-22T10:00:00Z
\end{itemize}

\subsection*{Open Ports and Services}
The following table details the services discovered on the target system.

\vspace{1em}
\begin{tabular}{@{}lllll@{}}
\toprule
\textbf{Port} & \textbf{State} & \textbf{Service} & \textbf{Product} & \textbf{Version} \\
\midrule
443/tcp & open & https & nginx & 1.18.0 \\
\bottomrule
\end{tabular}

\subsection*{Analysis}
The scan identified an \textbf{Nginx web server, version 1.18.0}, exposed to the internet via port 443 (HTTPS). This version was released in April 2020 and is now considered significantly outdated. It is missing numerous security patches for vulnerabilities discovered since its release, potentially exposing the server to information disclosure, denial-of-service, or other critical attacks. This finding is classified as a \riskhigh{} risk vulnerability.

% --- Section 5: Risk Assessment Summary ---
\section*{5. Risk Assessment Summary}
The following table consolidates and prioritizes risks identified from the security control review and the technical scan. No pre-existing vulnerabilities were reported.

\vspace{1em}
\begin{tabular}{@{}p{0.25\linewidth}p{0.5\linewidth}l@{}}
\toprule
\textbf{Risk Name} & \textbf{Overview} & \textbf{Severity} \\
\midrule
Lack of MFA on Email & The absence of Multi-Factor Authentication on email accounts makes them highly susceptible to compromise via phishing or credential stuffing attacks. Email is a primary target for attackers seeking to gain an initial foothold. & \riskcritical{} \\
\addlinespace
Outdated Web Server Software & The public-facing Nginx server is running version 1.18.0, which is outdated and lacks patches for multiple known vulnerabilities. This could allow an attacker to compromise the web server and the data it hosts. & \riskhigh{} \\
\addlinespace
Missing Annual Security Awareness Training & Without regular, ongoing security training, employees are more likely to fall victim to social engineering and phishing attacks. Security knowledge degrades over time, making annual refreshers essential. & \riskhigh{} \\
\addlinespace
No Acceptable Use Policy (AUP) & The lack of a formal AUP creates ambiguity regarding the proper use of company assets, handling of sensitive data, and employee responsibilities, increasing the risk of insider threats and compliance violations. & \riskhigh{} \\
\bottomrule
\end{tabular}

% --- Section 6: Recommendations ---
\section*{6. Recommendations}
Based on the findings, the following actions are recommended to mitigate the identified risks. Recommendations are prioritized by severity.

\begin{enumerate}
    \item \textbf{Implement MFA for Email (\riskcritical{}):} Immediately enforce mandatory Multi-Factor Authentication for all user access to the email system. This is the single most effective control to prevent unauthorized account access and subsequent business email compromise.
    
    \item \textbf{Upgrade Nginx Server (\riskhigh{}):} Plan and execute an upgrade of the Nginx web server on host \texttt{[Target IP]} to the latest stable version. Before deploying to production, test the new version in a staging environment to ensure application compatibility. Regularly schedule patch reviews for all internet-facing systems.

    \item \textbf{Establish Annual Security Training (\riskhigh{}):} Develop and deploy a mandatory security awareness training program for all employees to be completed annually. The training should cover current threats such as phishing, ransomware, password security, and secure data handling.
    
    \item \textbf{Develop an Acceptable Use Policy (\riskhigh{}):} Draft and implement a comprehensive Acceptable Use Policy (AUP) that clearly defines the rules and responsibilities for all users of corporate IT resources. Ensure all current and new employees read and formally acknowledge the policy.
\end{enumerate}

\end{document}
```