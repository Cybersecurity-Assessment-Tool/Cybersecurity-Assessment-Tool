```latex
\documentclass[12pt]{article}

% Preamble: Required Packages
\usepackage[margin=1in]{geometry}
\usepackage{pifont} % For checkmarks and crosses
\usepackage{booktabs} % For professional tables
\usepackage{hyperref} % For clickable links
\usepackage{url} % For formatting URLs
\usepackage{seqsplit} % To split long strings in tt font
\usepackage{xcolor} % For custom colors
\usepackage{graphicx} % For logos (if any)
\usepackage{fancyhdr} % For headers and footers
\usepackage{lastpage} % To get the total number of pages

% --- Document Setup ---
% Define colors for severity
\definecolor{red}{RGB}{255,0,0}
\definecolor{green}{RGB}{0,128,0}
\definecolor{orange}{RGB}{255,165,0}
\definecolor{blue}{RGB}{0,0,255}

% Hyperlink setup
\hypersetup{
    colorlinks=true,
    linkcolor=blue,
    filecolor=magenta,      
    urlcolor=cyan,
    pdftitle={Cybersecurity Posture Assessment Report},
    pdfpagemode=FullScreen,
}

% Header and Footer Configuration
\pagestyle{fancy}
\fancyhf{} % Clear all header and footer fields
\fancyhead[L]{Cybersecurity Posture Assessment Report}
\fancyhead[R]{\textbf{[Organization Name]}}
\fancyfoot[C]{Page \thepage\ of \pageref{LastPage}}
\renewcommand{\headrulewidth}{0.4pt}
\renewcommand{\footrulewidth}{0.4pt}

% --- Document Body ---
\begin{document}

% --- Title Page ---
\begin{titlepage}
    \centering
    \vspace*{1cm}
    \Huge\textbf{Cybersecurity Posture Assessment Report}
    \vspace{1.5cm}
    \large
    \begin{center}
        \includegraphics[width=0.4\textwidth]{example-image-a} % Placeholder for a logo
    \end{center}
    \vspace{1.5cm}
    \textbf{Prepared for:}\\
    \Large\textbf{[Organization Name]}\\
    \vspace{2cm}
    \textbf{Date of Report:}\\
    \large\today
    \vfill
    \textit{This report contains sensitive information and should be handled with care.}
\end{titlepage}

\tableofcontents
\newpage

% --- Section 1: Executive Summary ---
\section{Executive Summary}
This report details the findings of a cybersecurity posture assessment conducted for \textbf{[Organization Name]}. The assessment combined an analysis of organizational security controls via a questionnaire, a technical network scan of a designated external asset, and a review of pre-existing risk data.

\paragraph{Key Findings:} The organization demonstrates a strong commitment to identity and access management, with Multi-Factor Authentication (MFA) widely implemented across key systems. Furthermore, a technical scan of the target asset \texttt{[Target IP]} revealed a secure configuration, with no open ports detected. This finding indicates that a previously identified risk related to an unencrypted web server on Port 80 has been successfully remediated.

\paragraph{Areas for Improvement:} Despite the positive technical findings, the assessment identified two critical gaps in the organization's foundational security policies and procedures.
\begin{itemize}
    \item \textbf{Lack of an Acceptable Use Policy (AUP):} The absence of a formal AUP creates ambiguity for employees regarding the proper use of company assets and data, increasing the risk of insider threats and non-compliance.
    \item \textbf{No Security Training for New Hires:} New employees are not receiving security awareness training during their onboarding process. This oversight leaves a critical window of vulnerability, as new staff are often prime targets for social engineering attacks.
\end{itemize}

\paragraph{Conclusion:} While the organization's technical controls for the assessed asset are robust, immediate attention must be directed towards strengthening the human and procedural elements of the security program. Establishing a formal AUP and a mandatory security onboarding program for new employees are the highest priority recommendations to mitigate significant organizational risk.

\newpage

% --- Section 2: Organizational Information ---
\section{Organizational Information}
This section provides a summary of the organizational details used as a basis for this assessment. As the provided data was anonymized, placeholders have been used.

\begin{table}[h!]
\centering
\begin{tabular}{@{}ll@{}}
\toprule
\textbf{Attribute} & \textbf{Value} \\ \midrule
Organization Name & \textbf{[Organization Name]} \\
Primary Email Domain & \texttt{[Domain]} \\
External IP Assessed & \texttt{[Client IP]} \\
Target IP Scanned & \texttt{[Target IP]} \\
\bottomrule
\end{tabular}
\caption{Client and Assessment Scope.}
\end{table}

% --- Section 3: Security Control Review ---
\section{Security Control Review (Questionnaire)}
The following table summarizes the organization's responses to a security controls questionnaire. The responses are compared against industry best practices. Gaps identified by a "No" answer are significant and are addressed in the Risk Assessment section.

\begin{table}[h!]
\centering
\begin{tabular}{@{}p{0.7\linewidth}c@{}}
\toprule
\textbf{Control Question} & \textbf{Response} \\ \midrule
Do you require MFA to access email? & \textcolor{green}{\ding{51}} \\
Do you require MFA to log into computers? & \textcolor{green}{\ding{51}} \\
Do you require MFA to access sensitive data systems? & \textcolor{green}{\ding{51}} \\
Does your organization do security awareness training for all employees at least once per year? & \textcolor{green}{\ding{51}} \\
\addlinespace
Does your organization have an employee acceptable use policy? & \textcolor{red}{\ding{55}} \\
Does your organization do security awareness training for new employees? & \textcolor{red}{\ding{55}} \\
\bottomrule
\end{tabular}
\caption{Security Controls Questionnaire Results.}
\end{table}

\paragraph{Analysis:} The organization has successfully implemented MFA across critical access points, which is a commendable and highly effective security control. However, the lack of a formal Acceptable Use Policy and the absence of security training for new hires represent critical procedural deficiencies that undermine the overall security posture.

% --- Section 4: Technical Scan Results ---
\section{Technical Scan Results}
A network scan was performed on the target asset to identify its external exposure and potential vulnerabilities.

\begin{itemize}
    \item \textbf{Target IP:} \texttt{[Target IP]}
    \item \textbf{Scan Date:} \today
    \item \textbf{Scanner Used:} Nmap
\end{itemize}

\paragraph{Summary of Findings:} The scan determined the host is online and responsive. However, no open TCP ports were discovered. The scan specifically confirmed that Port 80 (HTTP) is \textbf{closed}, which is a secure configuration. This result contradicts pre-existing risk data which suggested this port was open.

\begin{table}[h!]
\centering
\begin{tabular}{@{}llll@{}}
\toprule
\textbf{Port} & \textbf{State} & \textbf{Service} & \textbf{Product / Version} \\ \midrule
80/tcp & closed & http & N/A \\
\bottomrule
\end{tabular}
\caption{Nmap Scan Results for Target: \texttt{[Target IP]}.}
\end{table}

% --- Section 5: Consolidated Risk Assessment ---
\section{Consolidated Risk Assessment}
This section synthesizes findings from the questionnaire, technical scan, and pre-existing risk data into a consolidated list of current risks.

\begin{table}[h!]
\centering
\begin{tabular}{@{}p{0.15\linewidth}p{0.55\linewidth}p{0.15\linewidth}@{}}
\toprule
\textbf{Risk Name} & \textbf{Description} & \textbf{Severity} \\ \midrule
\textbf{Lack of Acceptable Use Policy} & Without a formal AUP, there is no enforceable standard for employee behavior regarding IT assets and data. This increases the risk of data misuse, unauthorized software installation, and insider threats. & \textcolor{red}{\textbf{High}} \\
\addlinespace
\textbf{No Security Training for New Hires} & New employees are not trained on security best practices, making them highly susceptible to phishing, social engineering, and other common attacks. This gap exposes the organization from an employee's first day. & \textcolor{red}{\textbf{High}} \\
\addlinespace
\textbf{Unencrypted Web Server (Remediated)} & Pre-existing data indicated Port 80 was open. Our scan confirmed this port is now \textbf{closed}, and the associated risk has been mitigated. This should be updated in the internal risk register. & \textcolor{blue}{Informational} \\
\bottomrule
\end{tabular}
\caption{Summary of Identified Risks.}
\end{table}

% --- Section 6: Recommendations ---
\section{Recommendations}
Based on the consolidated risk assessment, the following actions are recommended to improve the security posture of \textbf{[Organization Name]}.

\subsection{High Priority Recommendations}

\paragraph{1. Develop and Implement an Acceptable Use Policy (AUP)}
An AUP is a foundational policy that governs the use of all company IT assets.
\begin{itemize}
    \item \textbf{Action:} Draft a comprehensive AUP that clearly defines rules for email usage, internet access, data handling, password security, and use of personal devices.
    \item \textbf{Implementation:} The policy should be reviewed by HR and legal counsel, distributed to all current employees for acknowledgement and signature, and integrated into the new hire onboarding process.
    \item \textbf{Impact:} Establishes clear security expectations for all staff and provides a basis for disciplinary action in case of violation.
\end{itemize}

\paragraph{2. Establish a Mandatory Security Onboarding Program}
New hires must be equipped with security knowledge from day one.
\begin{itemize}
    \item \textbf{Action:} Create a mandatory security awareness training module specifically for new employees. This module should cover topics such as phishing identification, password hygiene, social engineering tactics, and the new AUP.
    \item \textbf{Implementation:} Integrate this training into the formal onboarding checklist. Access to sensitive systems should only be granted upon completion of this training.
    \item \textbf{Impact:} Significantly reduces the risk posed by untrained new employees and fosters a security-conscious culture from the start.
\end{itemize}

\subsection{Informational Recommendations}

\paragraph{3. Update Internal Risk Register}
The organization's internal risk register should be a living document.
\begin{itemize}
    \item \textbf{Action:} Mark the risk associated with "Unencrypted Web Server" or "Open Port 80" as closed or remediated, referencing the scan date of this report.
    \item \textbf{Impact:} Ensures that security efforts are focused on current, relevant threats and that risk documentation is accurate.
\end{itemize}

\end{document}
```