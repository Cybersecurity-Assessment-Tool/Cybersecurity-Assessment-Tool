```latex
\documentclass[12pt]{article}

% Preamble: Required Packages
\usepackage[margin=1in]{geometry}
\usepackage{pifont} % For checkmarks and crosses
\usepackage{booktabs} % For professional tables
\usepackage{hyperref} % For clickable links and references
\usepackage{url}      % For formatting URLs
\usepackage{seqsplit} % For splitting long strings to prevent overflow
\usepackage{graphicx} % For potential logos
\usepackage{xcolor}   % For custom colors

% Document Information & Styling
\hypersetup{
    colorlinks=true,
    linkcolor=blue,
    filecolor=magenta,      
    urlcolor=cyan,
    pdftitle={Cybersecurity Posture Assessment Report},
    pdfpagemode=FullScreen,
}

\newcommand{\yes}{\ding{51}}
\newcommand{\no}{\ding{55}}

\begin{document}

% --- Title Page ---
\begin{titlepage}
    \centering
    \vspace*{1cm}
    \Huge\textbf{Cybersecurity Posture Assessment Report}
    \vspace{1.5cm}
    \Large
    \textbf{Prepared for:}\\
    \vspace{0.5cm}
    \textbf{[Organization Name]}\\
    \vspace{2cm}
    \textbf{Date of Report:}\\
    \today\\
    \vspace{2cm}
    \textbf{Scan Date:}\\
    2025-11-22\\
    \vfill
    \large
    \textit{This report contains sensitive information and should be handled with care.}
\end{titlepage}

\tableofcontents
\newpage

% --- 1. Executive Summary ---
\section{Executive Summary}
This report details the findings of a cybersecurity assessment conducted for \textbf{[Organization Name]}. The assessment combined a review of organizational security controls, an external network vulnerability scan, and an analysis of pre-existing risks.

The overall security posture requires immediate attention. Several critical and high-risk gaps were identified, primarily related to identity and access management and employee security awareness. Specifically, the lack of Multi-Factor Authentication (MFA) on employee computers and sensitive data systems represents a critical vulnerability. This is compounded by the absence of a formal security awareness training program, leaving the organization susceptible to phishing and social engineering attacks.

Furthermore, the technical scan revealed an externally facing web server running an outdated version of Nginx (1.18.0), which is known to have multiple security vulnerabilities. While no pre-existing risks were documented, the combination of these new findings presents a significant threat to the confidentiality, integrity, and availability of the organization's data and systems.

Immediate remediation of the identified risks, starting with the implementation of MFA and patching the vulnerable web server, is strongly recommended.

% --- 2. Organizational & Scan Information ---
\section{Assessment Scope & Information}
This assessment provides a point-in-time analysis of the organization's security posture based on the data provided.

\subsection{Organizational Information}
\begin{tabular}{@{}ll}
\toprule
\textbf{Attribute} & \textbf{Value} \\
\midrule
Organization Name & \textbf{[Organization Name]} \\
Email Domain      & \texttt{[Domain]} \\
External IP Range & \texttt{[Client IP]} \\
\bottomrule
\end{tabular}

\subsection{Technical Scan Information}
\begin{tabular}{@{}ll}
\toprule
\textbf{Attribute} & \textbf{Value} \\
\midrule
Scan Date   & 2025-11-22T10:00:00Z \\
Target IP   & \texttt{[Target IP]} \\
Scan Type   & External Network Port Scan (Nmap) \\
\bottomrule
\end{tabular}

% --- 3. Security Control Review ---
\section{Security Control Review}
The following table summarizes the organization's responses to a security controls questionnaire. "No" answers indicate significant gaps in the security framework and are flagged as risks.

\begin{table}[h!]
\centering
\caption{Security Controls Questionnaire Analysis}
\begin{tabular}{p{0.6\linewidth} c l}
\toprule
\textbf{Control Question} & \textbf{Response} & \textbf{Assessment} \\
\midrule
Do you require MFA to access email? & \yes & Control in Place \\
Do you require MFA to log into computers? & \no & \textbf{Critical Gap} \\
Do you require MFA to access sensitive data systems? & \no & \textbf{Critical Gap} \\
Does your organization have an employee acceptable use policy? & \yes & Control in Place \\
Does your organization do security awareness training for new employees? & \no & \textbf{High Risk} \\
Does your organization do security awareness training for all employees at least once per year? & \no & \textbf{High Risk} \\
\bottomrule
\end{tabular}
\end{table}

The identified gaps in MFA and security awareness training significantly increase the risk of a successful breach through credential compromise.

% --- 4. Technical Scan Results ---
\section{Technical Scan Results}
An external network scan was performed against the target IP address \texttt{[Target IP]}. The scan identified the following open port and service.

\begin{table}[h!]
\centering
\caption{Open Ports and Services}
\begin{tabular}{l l l l l}
\toprule
\textbf{Port} & \textbf{State} & \textbf{Service} & \textbf{Product} & \textbf{Version} \\
\midrule
443/tcp & open & https & nginx & 1.18.0 \\
\bottomrule
\end{tabular}
\end{table}

\subsection{Analysis of Findings}
The scan identified an instance of \textbf{Nginx version 1.18.0}. This version was released in April 2020 and is now considered outdated. It is associated with several publicly disclosed vulnerabilities, including but not limited to:
\begin{itemize}
    \item \textbf{CVE-2021-23017:} A DNS resolver vulnerability that could allow an attacker to cause a denial of service or potentially execute arbitrary code.
\end{itemize}
Running outdated software on internet-facing systems presents a high risk of exploitation by automated scanners and malicious actors.

% --- 5. Risk Assessment Summary ---
\section{Risk Assessment Summary}
The following table consolidates the risks identified during this assessment. The severity level is based on the potential impact and likelihood of exploitation.

\begin{table}[h!]
\centering
\caption{Identified Risk Summary}
\begin{tabular}{p{0.1\linewidth} p{0.3\linewidth} p{0.4\linewidth} l}
\toprule
\textbf{ID} & \textbf{Risk Name} & \textbf{Description} & \textbf{Severity} \\
\midrule
RISK-001 & Lack of Comprehensive MFA & MFA is not enforced for computer logins or access to sensitive data systems, making credential theft highly impactful. & \textbf{Critical} \\
\addlinespace
RISK-002 & Outdated Web Server Software & The external web server runs Nginx 1.18.0, which has known, exploitable vulnerabilities. & \textbf{High} \\
\addlinespace
RISK-003 & Inadequate Security Awareness Training & The absence of a formal training program for new or existing employees increases susceptibility to phishing and social engineering. & \textbf{High} \\
\bottomrule
\end{tabular}
\end{table}

% --- 6. Recommendations ---
\section{Recommendations}
To mitigate the identified risks and improve the overall security posture of \textbf{[Organization Name]}, the following actions are recommended in order of priority.

\begin{enumerate}
    \item \textbf{(Critical) Implement Comprehensive Multi-Factor Authentication:}
    \begin{itemize}
        \item \textbf{Priority 1:} Immediately enforce MFA for all access to systems containing sensitive data.
        \item \textbf{Priority 2:} Roll out MFA for all employee computer logins (local and remote).
        \item This single control is the most effective way to mitigate the risk of compromised credentials.
    \end{itemize}
    \vspace{0.5cm}
    \item \textbf{(High) Remediate Web Server Vulnerability:}
    \begin{itemize}
        \item Plan and execute an upgrade of the Nginx server from version 1.18.0 to the latest stable version.
        \item Before deployment, test the new version in a staging environment to ensure application compatibility.
    \end{itemize}
    \vspace{0.5cm}
    \item \textbf{(High) Establish a Security Awareness Training Program:}
    \begin{itemize}
        \item Implement a mandatory security awareness training module for all new employees during their onboarding process.
        \item Conduct annual, mandatory refresher training for all staff to keep them informed about current threats (e.g., phishing, ransomware).
    \end{itemize}
    \vspace{0.5cm}
    \item \textbf{(Medium) Develop a Vulnerability Management Program:}
    \begin{itemize}
        \item Implement a process for regularly scanning internal and external systems for vulnerabilities.
        \item Establish a policy for patching critical and high-severity vulnerabilities within a defined timeframe (e.g., 14 days for critical, 30 days for high).
    \end{itemize}
\end{enumerate}

\end{document}
```