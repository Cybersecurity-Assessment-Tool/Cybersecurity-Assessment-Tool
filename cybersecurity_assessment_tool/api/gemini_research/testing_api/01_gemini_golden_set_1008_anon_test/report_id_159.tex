```latex
\documentclass[12pt]{article}

% Preamble: Required Packages
\usepackage[margin=1in]{geometry}
\usepackage{pifont} % For checkmarks and crosses
\usepackage{booktabs} % For professional tables
\usepackage{hyperref} % For clickable links
\usepackage{url} % For URL formatting
\usepackage{seqsplit} % For splitting long strings in tt font

% Document Metadata
\title{Cybersecurity Posture Assessment Report}
\author{Cybersecurity Analysis Division}
\date{\today}

\begin{document}

\maketitle
\thispagestyle{empty}
\newpage

\tableofcontents
\newpage

% --- 1. Executive Summary ---
\section{Executive Summary}

This report details the findings of a cybersecurity posture assessment for \textbf{[Organization Name]}. The analysis combines a review of self-reported security controls, an external network scan, and a summary of pre-existing risks.

The assessment identified several critical and high-risk vulnerabilities that require immediate attention. Key findings include:

\begin{itemize}
    \item \textbf{Critical Service Exposure:} An externally accessible MySQL database (port 3306) was discovered. The service is running MySQL version 5.7.33, which is officially End-of-Life (EOL) and no longer receives security updates, exposing it to numerous known and unpatchable vulnerabilities.
    \item \textbf{Insufficient Access Controls:} Multi-Factor Authentication (MFA) is not enforced for employee email or computer logins. This represents a critical gap, as compromised credentials could lead to widespread unauthorized access.
    \item \textbf{Foundational Policy Gaps:} The organization lacks a formal Acceptable Use Policy and does not provide security awareness training to new employees. These omissions cultivate a high-risk environment susceptible to insider threats and social engineering attacks.
\end{itemize}

Collectively, these issues indicate a significant risk of data breach, service disruption, and unauthorized system access. This report provides a detailed breakdown of these risks and offers actionable recommendations for remediation.

% --- 2. Organizational Information ---
\section{Organizational Information}

The following details were used as the basis for this assessment. Due to the anonymized nature of the provided data, placeholders have been used where necessary.

\begin{itemize}
    \item \textbf{Organization Name:} \textbf{[Organization Name]}
    \item \textbf{Primary Domain:} \seqsplit{\texttt{[Domain]}}
    \item \textbf{Client External IP:} \seqsplit{\texttt{[Client IP]}}
    \item \textbf{Scan Target IP:} \seqsplit{\texttt{[Target IP]}}
\end{itemize}

% --- 3. Security Control Review (Questionnaire) ---
\section{Security Control Review (Questionnaire)}

The following table summarizes the organization's self-reported security controls. Responses marked with \ding{55} indicate significant gaps in the current security program and are addressed in the Risk Assessment section.

\begin{table}[h!]
\centering
\caption{Security Controls Questionnaire Results}
\begin{tabular}{@{}lc@{}}
\toprule
\textbf{Control Question} & \textbf{Response} \\
\midrule
Do you require MFA to access email? & \ding{55} \\
Do you require MFA to log into computers? & \ding{55} \\
Do you require MFA to access sensitive data systems? & \ding{51} \\
Does your organization have an employee acceptable use policy? & \ding{55} \\
Does your organization do security awareness training for new employees? & \ding{55} \\
Does your organization do security awareness training for all employees at least once per year? & \ding{51} \\
\bottomrule
\end{tabular}
\end{table}

\paragraph{Analysis:} The lack of MFA on core communication (email) and access (computer logins) systems is a critical weakness. Furthermore, the absence of an Acceptable Use Policy and new-hire security training prevents the establishment of a strong security culture from the outset.

% --- 4. Technical Scan Results ---
\section{Technical Scan Results}

An external network scan was conducted against the target IP address \seqsplit{\texttt{[Target IP]}}. The scan identified the following open ports and services.

\begin{table}[h!]
\centering
\caption{Nmap Scan Findings}
\begin{tabular}{@{}lllll@{}}
\toprule
\textbf{Port} & \textbf{State} & \textbf{Service} & \textbf{Product} & \textbf{Version} \\
\midrule
3306/tcp & open & mysql & MySQL & \seqsplit{\texttt{5.7.33}} \\
\bottomrule
\end{tabular}
\end{table}

\paragraph{Analysis:} The scan confirms the pre-existing risk of database exposure. The identified MySQL version 5.7.33 is particularly alarming. MySQL 5.7 reached its End-of-Life (EOL) in October 2023. This means it no longer receives security patches, bug fixes, or updates from Oracle, leaving it perpetually vulnerable to any exploits discovered after that date. Publicly exposing an EOL database service is a critical security risk.

% --- 5. Consolidated Risk Assessment ---
\section{Consolidated Risk Assessment}

The following table synthesizes findings from the security control review, technical scan, and pre-existing risk data into a prioritized list of security risks.

\begin{table}[h!]
\centering
\caption{Summary of Identified Risks}
\begin{tabular}{@{}p{0.1\linewidth} p{0.25\linewidth} p{0.4\linewidth} p{0.15\linewidth}@{}}
\toprule
\textbf{Risk ID} & \textbf{Risk Name} & \textbf{Description} & \textbf{Severity} \\
\midrule
\textbf{R-01} & Exposed End-of-Life Database Service & The MySQL database on port 3306 is publicly accessible and runs an unsupported, EOL version (5.7.33). This exposes the organization to known, unpatchable vulnerabilities and a high risk of data breach. & \textbf{Critical} \\
\addlinespace
\textbf{R-02} & Inadequate Access Control (MFA) & MFA is not enforced for email or computer logins, significantly increasing the risk of account compromise via phishing, credential stuffing, or password spraying attacks. & \textbf{Critical} \\
\addlinespace
\textbf{R-03} & Foundational Policy \& Training Gaps & The lack of an Acceptable Use Policy and security training for new employees creates an uninformed user base, increasing susceptibility to social engineering and insider threats. & \textbf{High} \\
\bottomrule
\end{tabular}
\end{table}

% --- 6. Recommendations ---
\section{Recommendations}

Based on the consolidated risk assessment, the following actions are recommended to mitigate the identified vulnerabilities and improve the overall security posture of \textbf{[Organization Name]}.

\subsection{Immediate Actions (To Be Completed Within 72 Hours)}
\begin{itemize}
    \item \textbf{[R-01] Restrict Database Access:} Immediately apply firewall rules to block all public access to TCP port 3306 on \seqsplit{\texttt{[Target IP]}}. Access should only be permitted from trusted internal IP addresses or through a secure VPN tunnel.
    \item \textbf{[R-02] Enforce MFA on Critical Systems:} Immediately begin the rollout and enforcement of MFA for all user accounts with access to email and for all remote access systems (e.g., VPN, RDP).
\end{itemize}

\subsection{Short-Term Actions (To Be Completed Within 30 Days)}
\begin{itemize}
    \item \textbf{[R-01] Plan Database Upgrade:} Develop a migration plan to upgrade the MySQL 5.7 database to a currently supported version (e.g., MySQL 8.x) to ensure it receives ongoing security updates.
    \item \textbf{[R-02] Expand MFA Coverage:} Complete the rollout of MFA for all endpoint (computer) logins across the organization.
    \item \textbf{[R-03] Develop and Implement Foundational Policies:} Draft and formally adopt an employee Acceptable Use Policy (AUP). Require all current employees to read and acknowledge the policy.
    \item \textbf{[R-03] Institute Onboarding Security Training:} Create a mandatory security awareness training module for all new hires to be completed during their first week of employment.
\end{itemize}

\subsection{Long-Term Strategic Actions}
\begin{itemize}
    \item \textbf{Establish Vulnerability Management Program:} Implement a formal program for regular vulnerability scanning, risk assessment, and patch management to proactively identify and remediate security weaknesses.
    \item \textbf{Conduct Comprehensive Security Assessment:} Engage a third party to perform a full penetration test and security architecture review to identify any further unknown risks.
\end{itemize}

\end{document}
```