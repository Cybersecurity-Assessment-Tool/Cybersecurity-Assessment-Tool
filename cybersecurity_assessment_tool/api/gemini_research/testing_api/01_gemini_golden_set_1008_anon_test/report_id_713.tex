```latex
\documentclass[12pt]{article}

% Preamble: Required Packages
\usepackage[margin=1in]{geometry}
\usepackage{pifont} % For checkmarks and crosses
\usepackage{booktabs} % For professional tables
\usepackage{hyperref} % For hyperlinks
\usepackage{url} % For URL formatting
\usepackage{seqsplit} % For splitting long text strings
\usepackage{xcolor} % For colors in the report

% Document Information
\title{Cybersecurity Posture \& Risk Assessment Report \\ \large For: \textbf{[Organization Name]}}
\author{Cybersecurity Analysis Division}
\date{\today}

\begin{document}

\maketitle
\thispagestyle{empty}
\newpage

\tableofcontents
\newpage

\section{Executive Summary}

This report provides a comprehensive analysis of the current cybersecurity posture for \textbf{[Organization Name]}. The assessment is based on a correlation of external network scan data, a review of internal security controls via a questionnaire, and an analysis of pre-existing risk documentation.

The assessment has identified a \textbf{CRITICAL} risk posture. The primary finding is the direct exposure of the Remote Desktop Protocol (RDP) service on the public internet at IP address \texttt{[Target IP]}. This vulnerability, documented as a known risk, is significantly amplified by critical gaps in internal security controls. Specifically, the absence of mandatory Multi-Factor Authentication (MFA) for computer logins and the lack of a formal Acceptable Use Policy create a high-likelihood path for a malicious actor to gain unauthorized access to the internal network.

Immediate remediation of the exposed RDP service is strongly recommended to prevent a potential security breach, such as a ransomware attack. Further recommendations focus on strengthening foundational security controls to build a more resilient and defensible security posture.

\section{Organizational Information}

This section details the information provided about the organization. As this report was generated in a template mode, placeholders are used for sensitive data.

\begin{itemize}
    \item \textbf{Organization Name:} \textbf{[Organization Name]}
    \item \textbf{Primary Email Domain:} \texttt{[Domain]}
    \item \textbf{External IP Address Scanned:} \texttt{[Client IP]}
\end{itemize}

\section{Security Control Review}

The following table summarizes the organization's responses to a security controls questionnaire. "No" answers indicate significant gaps in the security framework and are flagged as areas of high concern.

\begin{table}[h!]
\centering
\caption{Security Controls Questionnaire Analysis}
\begin{tabular}{p{0.6\textwidth} c l}
\toprule
\textbf{Control Question} & \textbf{Response} & \textbf{Assessment} \\
\midrule
Do you require MFA to access email? & \ding{51} & Good Practice \\
\addlinespace
Do you require MFA to log into computers? & \textbf{\color{red}\ding{55}} & \textbf{Critical Gap} \\
\addlinespace
Do you require MFA to access sensitive data systems? & \ding{51} & Good Practice \\
\addlinespace
Does your organization have an employee acceptable use policy? & \textbf{\color{red}\ding{55}} & \textbf{High Risk} \\
\addlinespace
Does your organization do security awareness training for new employees? & \ding{51} & Good Practice \\
\addlinespace
Does your organization do security awareness training for all employees at least once per year? & \textbf{\color{red}\ding{55}} & \textbf{High Risk} \\
\bottomrule
\end{tabular}
\end{table}

\subsection{Analysis of Control Gaps}
\begin{itemize}
    \item \textbf{No MFA for Computer Logins:} This is a critical weakness. In the event of a password compromise (e.g., via phishing or credential stuffing), there is no secondary control to prevent an attacker from logging into a corporate machine. This risk is severely amplified by the externally exposed RDP service.
    \item \textbf{Lack of Acceptable Use Policy (AUP):} An AUP is a foundational document that sets clear expectations for employee behavior regarding company assets and data. Its absence can lead to inconsistent security practices and a lack of legal and administrative recourse for policy violations.
    \item \textbf{No Annual Security Training:} The threat landscape evolves continuously. Without regular, recurring security awareness training, employees are more likely to fall victim to social engineering attacks, mishandle sensitive data, or be unaware of current security policies.
\end{itemize}

\section{Technical Scan Results}

An external network scan was conducted against the target host \texttt{[Target IP]}. The scan identified the following open ports and services.

\begin{table}[h!]
\centering
\caption{Open Ports Detected on \texttt{[Target IP]}}
\begin{tabular}{l l l l}
\toprule
\textbf{Port} & \textbf{State} & \textbf{Service} & \textbf{Notes} \\
\midrule
3389/tcp & open & ms-wbt-server & Microsoft Remote Desktop Protocol (RDP) \\
\bottomrule
\end{tabular}
\end{table}

\subsection{Analysis of Technical Findings}
The discovery of an open RDP port (3389) is a finding of the highest severity. RDP is a primary target for attackers seeking to gain initial access to a network. It is frequently exploited through brute-force password attacks, credential stuffing, and the exploitation of unpatched vulnerabilities (e.g., BlueKeep). This finding directly corroborates the pre-existing risk documented in the organization's risk register.

\section{Correlated Risk Assessment}

This section synthesizes the findings from the security control review, the technical scan, and pre-existing risk data into a prioritized list of risks.

\begin{table}[h!]
\centering
\caption{Summary of Identified Risks}
\begin{tabular}{p{0.2\textwidth} p{0.5\textwidth} p{0.2\textwidth}}
\toprule
\textbf{Severity} & \textbf{Risk Description} & \textbf{Affected Systems} \\
\midrule
\textbf{CRITICAL} & \textbf{Exposed RDP Service:} Port 3389 is open to the public internet, allowing direct login attempts. This is a common vector for ransomware attacks. This finding correlates the network scan with the existing "RDP Exposure" risk (CVSS 9.0). & \texttt{[Target IP]} and the internal network it connects to. \\
\addlinespace
\textbf{HIGH} & \textbf{Insufficient MFA Controls:} The lack of MFA on computer logins drastically lowers the effort required for an attacker to compromise an account and gain access via the exposed RDP service. & All user workstations and servers accessible via RDP. \\
\addlinespace
\textbf{HIGH} & \textbf{Weak Security Culture \& Policy:} The absence of an Acceptable Use Policy and annual security training indicates a reactive security posture and increases the likelihood of human error leading to a security incident. & Organization-wide. \\
\bottomrule
\end{tabular}
\end{table}

\section{Recommendations}

The following actionable recommendations are provided in order of priority to mitigate the identified risks.

\subsection{Priority 1: Immediate Remediation (Within 24 Hours)}
\begin{enumerate}
    \item \textbf{Remediate RDP Exposure:} The open RDP port on \texttt{[Target IP]} must be secured immediately.
    \begin{itemize}
        \item \textbf{Option A (Preferred):} Implement a Virtual Private Network (VPN) or Zero Trust Network Access (ZTNA) solution. Require all remote access to occur through this secure, authenticated, and encrypted tunnel.
        \item \textbf{Option B (Alternative):} If RDP access is not required from the public internet, block port 3389 at the network firewall immediately.
        \item \textbf{Option C (Interim):} If immediate implementation of A or B is not possible, restrict access to port 3389 at the firewall to a whitelist of known, trusted IP addresses. This is a temporary measure and should be replaced by Option A.
    \end{itemize}
\end{enumerate}

\subsection{Priority 2: High-Impact Hardening (Within 30 Days)}
\begin{enumerate}
    \setcounter{enumi}{1}
    \item \textbf{Enforce MFA for All Logins:} Deploy and enforce Multi-Factor Authentication for all user accounts for computer and server logins. This is especially critical for any accounts that have remote access capabilities or administrative privileges.
    \item \textbf{Develop and Implement an Acceptable Use Policy (AUP):} Draft a formal AUP that clearly defines the rules for using company IT assets, data handling, and internet usage. Ensure all employees read and acknowledge the policy.
\end{enumerate}

\subsection{Priority 3: Foundational Improvements (Within 90 Days)}
\begin{enumerate}
    \setcounter{enumi}{3}
    \item \textbf{Establish an Annual Security Awareness Program:} Implement a mandatory security awareness training program for all employees to be completed at least once per year. The training should cover current threats such as phishing, password security, and social engineering.
\end{enumerate}

\end{document}
```