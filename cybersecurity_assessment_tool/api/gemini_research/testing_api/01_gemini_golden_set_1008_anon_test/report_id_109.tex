```latex
\documentclass[12pt]{article}

% Preamble: Required Packages
\usepackage[margin=1in]{geometry}
\usepackage{pifont} % For checkmarks and crosses
\usepackage{booktabs} % For professional tables
\usepackage{hyperref} % For hyperlinks
\usepackage{url} % For URL formatting
\usepackage{seqsplit} % For splitting long strings
\usepackage{graphicx} % For logo
\usepackage{xcolor} % For colors

% Document Information
\title{Cybersecurity Assessment Report}
\author{Cybersecurity Assessment Team}
\date{November 22, 2025}

% Hyperref Setup
\hypersetup{
    colorlinks=true,
    linkcolor=blue,
    filecolor=magenta,      
    urlcolor=cyan,
    pdftitle={Cybersecurity Assessment Report},
    pdfpagemode=FullScreen,
}

\begin{document}

\maketitle
\hrule
\vspace{1cm}

\begin{abstract}
This report details the findings of a cybersecurity assessment conducted for \textbf{[Organization Name]}. The evaluation combined a review of organizational security controls, an external network vulnerability scan, and an analysis of pre-existing risks. The assessment identified critical gaps in security posture, including deficiencies in multi-factor authentication (MFA) and employee security policies, alongside a significant technical vulnerability in an internet-facing web server. This document provides a detailed risk assessment and actionable recommendations to mitigate the identified threats.
\end{abstract}

\newpage

\tableofcontents

\newpage

\section{Executive Summary}
The primary objective of this assessment was to evaluate the current cybersecurity posture of \textbf{[Organization Name]}. The analysis revealed several areas of high and critical risk that require immediate attention.

Key findings include:
\begin{itemize}
    \item \textbf{Critical Gaps in Access Control:} Multi-factor authentication (MFA) is not enforced for accessing email or sensitive data systems. This exposes the organization to a high risk of account takeover and subsequent data breaches.
    \item \textbf{Outdated Internet-Facing Software:} An external scan identified an outdated version of the Nginx web server (\texttt{1.18.0}) running on a public-facing system. This version is known to have multiple security vulnerabilities, presenting a direct vector for compromise.
    \item \textbf{Deficient Security Governance:} The organization lacks a formal Employee Acceptable Use Policy and does not conduct mandatory annual security awareness training for all staff. These foundational policy gaps increase the likelihood of security incidents caused by human error.
\end{itemize}

These findings, when correlated, paint a picture of an organization vulnerable to both targeted attacks and opportunistic threats. The recommendations provided in this report are designed to address these specific weaknesses and build a more resilient security foundation.

\section{Organizational Information}
\begin{itemize}
    \item \textbf{Organization Name:} \textbf{[Organization Name]}
    \item \textbf{Primary Domain:} \seqsplit{\texttt{[Domain]}}
    \item \textbf{External IP Address:} \seqsplit{\texttt{[Client IP]}}
\end{itemize}

\section{Security Control Review}
The following table summarizes the organization's responses to a security controls questionnaire. A green checkmark (\ding{51}) indicates a positive control is in place, while a red cross (\ding{55}) indicates a control gap.

\begin{table}[h!]
\centering
\caption{Security Controls Questionnaire Results}
\label{tab:controls}
\begin{tabular}{p{0.6\textwidth} c c}
\toprule
\textbf{Control Question} & \textbf{Response} & \textbf{Status} \\
\midrule
Do you require MFA to access email? & No & \textcolor{red}{\ding{55}} \\
Do you require MFA to log into computers? & Yes & \textcolor{green}{\ding{51}} \\
Do you require MFA to access sensitive data systems? & No & \textcolor{red}{\ding{55}} \\
Does your organization have an employee acceptable use policy? & No & \textcolor{red}{\ding{55}} \\
Does your organization do security awareness training for new employees? & Yes & \textcolor{green}{\ding{51}} \\
Does your organization do security awareness training for all employees at least once per year? & No & \textcolor{red}{\ding{55}} \\
\bottomrule
\end{tabular}
\end{table}

\subsection{Analysis of Control Gaps}
The questionnaire reveals critical deficiencies in both technical and administrative controls. The absence of MFA on email and sensitive data systems is a major security flaw. Email is a primary target for phishing attacks, and a compromised account can serve as a gateway to the entire organization. The lack of an Acceptable Use Policy and annual security training for all employees exacerbates this risk, as staff may not be aware of their responsibilities or the latest threats.

\section{Technical Scan Results}
An external network scan was performed to identify open ports and services exposed to the internet.

\begin{itemize}
    \item \textbf{Target IP Address:} \texttt{[Target IP]}
    \item \textbf{Scan Date:} November 22, 2025
\end{itemize}

\begin{table}[h!]
\centering
\caption{Open Ports Detected on \texttt{[Target IP]}}
\label{tab:nmap}
\begin{tabular}{l l l l l}
\toprule
\textbf{Port} & \textbf{State} & \textbf{Service} & \textbf{Product} & \textbf{Version} \\
\midrule
443/tcp & open & https & nginx & 1.18.0 \\
\bottomrule
\end{tabular}
\end{table}

\subsection{Technical Analysis}
The scan identified an Nginx web server, version \textbf{1.18.0}, listening on port 443 (HTTPS). This version was released in April 2020 and is now significantly outdated. It is known to be vulnerable to multiple Common Vulnerabilities and Exposures (CVEs), such as CVE-2021-23017. Running outdated software on an internet-facing system presents a high risk of remote code execution, denial-of-service, or other forms of compromise.

\section{Consolidated Risk Assessment}
The following table synthesizes findings from the security control review, technical scan, and pre-existing risk data. As no pre-existing risks were provided, all items listed are new findings from this assessment.

\begin{table}[h!]
\centering
\caption{Summary of Identified Risks}
\label{tab:risks}
\begin{tabular}{p{0.1\textwidth} p{0.25\textwidth} p{0.4\textwidth} p{0.1\textwidth}}
\toprule
\textbf{Risk ID} & \textbf{Risk Name} & \textbf{Description} & \textbf{Severity} \\
\midrule
RISK-001 & Lack of MFA on Critical Systems & Email and sensitive data systems are not protected by MFA, allowing for account compromise via stolen credentials. & \textbf{Critical} \\
\addlinespace
RISK-002 & Outdated Web Server Software & The public-facing Nginx server is outdated (v1.18.0) and has known vulnerabilities, exposing it to remote exploitation. & \textbf{High} \\
\addlinespace
RISK-003 & Deficient Security Policies \& Training & The absence of an Acceptable Use Policy and mandatory annual training increases the risk of security incidents due to human error. & \textbf{High} \\
\bottomrule
\end{tabular}
\end{table}

\section{Recommendations}
To address the identified risks, we recommend the following actions, prioritized by severity.

\begin{enumerate}
    \item \textbf{Implement Comprehensive MFA (RISK-001):}
    \begin{itemize}
        \item \textbf{Action:} Immediately deploy and enforce MFA for all user access to email systems (e.g., Microsoft 365, Google Workspace) and all systems identified as containing sensitive data.
        \item \textbf{Impact:} Drastically reduces the risk of unauthorized access and data breaches resulting from compromised credentials.
    \end{itemize}
    \vspace{0.5cm}
    \item \textbf{Remediate Vulnerable Web Server (RISK-002):}
    \begin{itemize}
        \item \textbf{Action:} Upgrade the Nginx server running on \texttt{[Target IP]} to the latest stable version. Implement a formal patch management policy to ensure all internet-facing systems are updated in a timely manner.
        \item \textbf{Impact:} Eliminates known vulnerabilities and protects the server from being a point of entry for attackers.
    \end{itemize}
    \vspace{0.5cm}
    \item \textbf{Strengthen Security Governance (RISK-003):}
    \begin{itemize}
        \item \textbf{Action:} Develop and ratify a formal Employee Acceptable Use Policy (AUP) that all employees must read and acknowledge.
        \item \textbf{Action:} Implement a mandatory annual security awareness training program for all employees, covering topics such as phishing, password hygiene, and data handling.
        \item \textbf{Impact:} Creates a stronger security culture and reduces the organization's susceptibility to social engineering and human error.
    \end{itemize}
\end{enumerate}

\section{Conclusion}
This assessment has identified significant and actionable risks to the security of \textbf{[Organization Name]}. While the discovery of these vulnerabilities is concerning, they can be effectively mitigated. By prioritizing the implementation of multi-factor authentication, remediating the outdated web server, and establishing foundational security policies and training, the organization can substantially improve its defensive posture against modern cyber threats.

\end{document}
```