```latex
\documentclass[12pt]{article}

% --- PACKAGES ---
\usepackage[margin=1in]{geometry}
\usepackage{pifont} % Required for \ding
\usepackage{booktabs} % For professional-looking tables
\usepackage{hyperref}
\usepackage{url}
\usepackage{seqsplit} % For splitting long strings in \texttt
\usepackage{xcolor} % For adding color to text

% --- DOCUMENT SETUP ---
\hypersetup{
    colorlinks=true,
    linkcolor=blue,
    filecolor=magenta,
    urlcolor=cyan,
}

% --- CUSTOM COMMANDS ---
\newcommand{\yes}{\ding{51}} % Checkmark
\newcommand{\no}{\textcolor{red}{\ding{55}}} % Red X mark

% --- DOCUMENT START ---
\begin{document}

\title{Cybersecurity Posture Assessment Report}
\author{Cybersecurity Analysis Division}
\date{\today}
\maketitle

\section*{Executive Summary}
This report details the findings of a cybersecurity assessment for \textbf{[Organization Name]}. The analysis reveals a high-risk security posture, primarily driven by the external exposure of a critical database service running an outdated, End-of-Life (EOL) version of MySQL. This technical vulnerability is compounded by significant gaps in foundational security controls, including a lack of Multi-Factor Authentication (MFA) on employee computers and the absence of an Acceptable Use Policy (AUP) and new-hire security training. Immediate remediation is required to mitigate the risk of data breach and unauthorized access.

\section*{1. Organizational Information}
This section provides an overview of the client information used for this assessment.
\vspace{1em}
\begin{tabular}{@{}ll}
\textbf{Organization Name:} & \textbf{[Organization Name]} \\
\textbf{Primary Domain:} & \texttt{[Domain]} \\
\textbf{External IP Scanned:} & \texttt{[Client IP]} \\
\end{tabular}

\section*{2. Security Control Review}
The following table summarizes the organization's current security controls based on the provided questionnaire. Items marked with \no\ represent significant gaps requiring attention.

\begin{center}
\begin{tabular}{p{0.7\textwidth}c}
\toprule
\textbf{Control Question} & \textbf{Status} \\
\midrule
Do you require MFA to access email? & \yes \\
Do you require MFA to log into computers? & \no \\
Do you require MFA to access sensitive data systems? & \yes \\
Does your organization have an employee acceptable use policy? & \no \\
Does your organization do security awareness training for new employees? & \no \\
Does your organization do security awareness training for all employees at least once per year? & \yes \\
\bottomrule
\end{tabular}
\end{center}

\subsection*{Analysis of Control Gaps}
\begin{itemize}
    \item \textbf{No MFA for Computers:} This is a critical weakness. A compromised user password could grant an attacker direct access to an employee's computer, providing a launchpad for further network intrusion.
    \item \textbf{No Acceptable Use Policy (AUP):} The absence of a formal AUP means there are no documented rules for employee behavior regarding company assets and data, leading to inconsistent and potentially insecure practices.
    \item \textbf{No New Hire Security Training:} New employees are a primary target for social engineering attacks. Failing to provide immediate security training leaves the organization vulnerable from day one of a new hire's employment.
\end{itemize}

\section*{3. Technical Scan Results}
An external network scan was performed on the target IP address \texttt{[Target IP]}. The scan identified the following open port and service.

\subsection*{Open Ports}
\begin{center}
\begin{tabular}{lllll}
\toprule
\textbf{Port} & \textbf{State} & \textbf{Service} & \textbf{Product} & \textbf{Version} \\
\midrule
3306/tcp & open & mysql & MySQL & 5.7.33 \\
\bottomrule
\end{tabular}
\end{center}

\subsection*{Technical Analysis}
The scan identified a publicly accessible MySQL database on port 3306. This configuration is highly discouraged as it exposes the database authentication interface to the entire internet, making it a target for brute-force attacks and exploitation of known vulnerabilities.

\vspace{1em}
\noindent\textbf{Critical Finding:} The identified MySQL version, \textbf{5.7.33}, is \textbf{End-of-Life (EOL)} as of October 2023. EOL software no longer receives security updates from the vendor and is known to contain unpatched vulnerabilities. This represents a critical risk of data compromise.

\section*{4. Consolidated Risk Assessment}
The following table correlates findings from the security control review, technical scan, and pre-existing risk data.

\begin{center}
\begin{tabular}{p{0.3\textwidth}p{0.5\textwidth}l}
\toprule
\textbf{Risk Name} & \textbf{Overview} & \textbf{Severity} \\
\midrule
Exposed End-of-Life Database & The primary database (MySQL 5.7.33) is publicly accessible and no longer receives security patches. This is a direct confirmation of the pre-existing "Database Exposure" risk, elevated to Critical due to the EOL status. & \textbf{Critical} \\
\addlinespace
Lack of Endpoint MFA & User computers are secured only by passwords. A single credential compromise could lead to full endpoint and potential network access, bypassing other controls. & \textbf{High} \\
\addlinespace
Missing Foundational Policies \& Training & The absence of an AUP and new hire training creates a weak security culture and increases the likelihood of human error leading to a security incident. & \textbf{High} \\
\bottomrule
\end{tabular}
\end{center}

\section*{5. Recommendations}
Based on the identified risks, the following actions are recommended to improve the organization's security posture.

\subsection*{Immediate Actions (0-7 Days)}
\begin{enumerate}
    \item \textbf{Restrict Database Access:} Immediately implement firewall rules to block all public access to TCP port 3306 on \texttt{[Target IP]}. Access should only be permitted from explicitly trusted IP addresses.
\end{enumerate}

\subsection*{Short-Term Actions (1-3 Months)}
\begin{enumerate}
    \setcounter{enumi}{1} % Continue numbering
    \item \textbf{Upgrade MySQL Database:} Plan and execute the migration of the MySQL 5.7.33 database to a currently supported version (e.g., MySQL 8.x) to ensure security patches are received.
    \item \textbf{Enforce Endpoint MFA:} Deploy a Multi-Factor Authentication solution for all employee computer logins to protect against credential theft.
\end{enumerate}

\subsection*{Mid-Term Actions (3-6 Months)}
\begin{enumerate}
    \setcounter{enumi}{3} % Continue numbering
    \item \textbf{Develop and Implement AUP:} Create a formal Acceptable Use Policy that all employees must read and acknowledge as a condition of employment.
    \item \textbf{Institute New Hire Training:} Develop a security awareness training module that is a mandatory part of the onboarding process for all new employees.
    \item \textbf{Implement Secure Access Architecture:} For long-term secure database access, establish a Virtual Private Network (VPN) and require its use for all remote administration, following the principle of least privilege.
\end{enumerate}

\end{document}
```