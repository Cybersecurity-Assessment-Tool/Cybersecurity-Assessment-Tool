```latex
\documentclass[12pt, a4paper]{article}

% Preamble: Required Packages
\usepackage[margin=1in]{geometry}
\usepackage{pifont} % For checkmarks and crosses
\usepackage{booktabs} % For professional tables
\usepackage{hyperref} % For hyperlinks
\usepackage{url}      % For URL formatting
\usepackage{seqsplit} % To split long strings in tt font
\usepackage[T1]{fontenc}

% Document Metadata
\hypersetup{
    colorlinks=true,
    linkcolor=black,
    urlcolor=blue,
    pdftitle={Cybersecurity Posture Report},
    pdfauthor={Cybersecurity Analysis Unit},
    pdfsubject={Security Assessment},
    pdfkeywords={Cybersecurity, Risk, Assessment, Nmap, Policy}
}

\title{Cybersecurity Posture Report}
\author{Cybersecurity Analysis Unit}
\date{\today}

\begin{document}

\maketitle

\begin{abstract}
\noindent This report provides a comprehensive analysis of the cybersecurity posture for \textbf{[Organization Name]}. The assessment is based on a synthesis of data from an external network scan, a review of organizational security controls via a questionnaire, and an evaluation of pre-existing documented risks. The analysis identifies key strengths, weaknesses, and actionable recommendations to enhance the organization's security resilience.
\end{abstract}

\newpage

\tableofcontents

\newpage

% Section 1: Overview
\section*{Executive Summary}

This assessment reveals a mixed cybersecurity posture for \textbf{[Organization Name]}. On one hand, the organization demonstrates a strong external network security configuration, with no open ports detected on the scanned public-facing asset. This indicates an effective firewall implementation and a minimized external attack surface, which is a significant strength.

However, the review of internal security controls highlights several critical and high-risk gaps in policy and procedure. The most severe finding is the lack of Multi-Factor Authentication (MFA) for accessing sensitive data systems. This exposes critical assets to significant risk from credential compromise. Furthermore, the absence of a formal Acceptable Use Policy (AUP) and a mandatory security awareness training program for new employees creates substantial operational and human-factor risks.

While no pre-existing vulnerabilities were provided for this assessment, the identified policy gaps represent foundational weaknesses that could be exploited by threat actors. Recommendations focus on immediately addressing these procedural and administrative control deficiencies to build a more robust, defense-in-depth security strategy.

% Section 2: Organizational Information
\section*{Organizational Information}

The following details were used as the basis for this assessment. In cases where information was not provided, placeholders have been used.

\begin{itemize}
    \item \textbf{Organization Name:} \textbf{[Organization Name]}
    \item \textbf{Primary Email Domain:} \texttt{[Domain]}
    \item \textbf{External IP Address Scanned:} \texttt{[Client IP]}
\end{itemize}

% Section 3: Security Control Review
\section*{Security Control Review}

An internal security questionnaire was completed to evaluate the implementation of key administrative and technical controls. The responses are summarized below. "No" answers indicate significant gaps in the security framework and are flagged as risks.

\begin{table}[h!]
\centering
\caption{Security Controls Questionnaire Analysis}
\label{tab:controls}
\begin{tabular}{@{}p{0.6\linewidth} c l@{}}
\toprule
\textbf{Control Question} & \textbf{Response} & \textbf{Assessment} \\
\midrule
Do you require MFA to access email? & \ding{51} (Yes) & Best Practice Met \\
Do you require MFA to log into computers? & \ding{51} (Yes) & Best Practice Met \\
\textbf{Do you require MFA to access sensitive data systems?} & \textbf{\ding{55} (No)} & \textbf{Critical Gap} \\
\textbf{Does your organization have an employee acceptable use policy?} & \textbf{\ding{55} (No)} & \textbf{High-Risk Gap} \\
\textbf{Does your organization do security awareness training for new employees?} & \textbf{\ding{55} (No)} & \textbf{High-Risk Gap} \\
Does your organization do security awareness training for all employees at least once per year? & \ding{51} (Yes) & Best Practice Met \\
\bottomrule
\end{tabular}
\end{table}

% Section 4: Technical Scan Results
\section*{Technical Scan Results}

An external network scan was performed to identify listening services and potential vulnerabilities on the organization's public-facing infrastructure.

\begin{itemize}
    \item \textbf{Target IP Address:} \texttt{[Target IP]}
    \item \textbf{Scan Date:} \today
    \item \textbf{Scanner Used:} Nmap
\end{itemize}

\subsection*{Scan Summary}
The target host was found to be online and responsive to network probes. However, the scan determined that \textbf{no TCP or UDP ports were open}. All 65,535 ports on the target system were in a "closed" or "filtered" state.

\subsection*{Analysis}
This is a positive security finding. It indicates that a well-configured firewall or security group is in place, effectively blocking all unsolicited inbound traffic from the internet. This configuration significantly reduces the external attack surface and is a commendable security practice. No vulnerabilities were identified from this external scan.

% Section 5: Risk Assessment
\section*{Risk Assessment}

This section correlates the findings from the security control review and technical scan. The primary risks identified are related to internal policies and procedures rather than technical vulnerabilities on the external perimeter.

\begin{table}[h!]
\centering
\caption{Identified Risks Summary}
\label{tab:risks}
\begin{tabular}{@{}p{0.15\linewidth} p{0.55\linewidth} l@{}}
\toprule
\textbf{Risk ID} & \textbf{Description} & \textbf{Severity} \\
\midrule
RISK-001 & \textbf{No MFA on Sensitive Data Systems:} Lack of MFA on critical systems allows an attacker with compromised credentials (e.g., from phishing) to gain direct access to the organization's most valuable data. & \textbf{Critical} \\
\addlinespace
RISK-002 & \textbf{Lack of Acceptable Use Policy (AUP):} Without a formal AUP, employees lack clear guidelines on the secure and acceptable use of company assets. This increases the risk of insider threat, data leakage, and legal liability. & \textbf{High} \\
\addlinespace
RISK-003 & \textbf{No Onboarding Security Training:} New employees are not trained on security policies and threats upon hiring. This makes them highly susceptible to social engineering and phishing attacks before they receive the annual training. & \textbf{High} \\
\bottomrule
\end{tabular}
\end{table}

% Section 6: Recommendations
\section*{Recommendations}

Based on the identified risks, the following prioritized actions are recommended to improve the organization's security posture.

\subsection*{Immediate Priority (Critical Risk)}

\begin{itemize}
    \item \textbf{Recommendation for RISK-001 (MFA on Sensitive Systems):}
    \begin{itemize}
        \item \textbf{Action:} Immediately develop and execute a plan to enforce MFA for all user accounts (including administrative and service accounts) that have access to systems storing or processing sensitive data (e.g., financial systems, customer databases, intellectual property).
        \item \textbf{Impact:} Drastically reduces the risk of unauthorized access via compromised credentials, protecting the organization's most critical assets.
    \end{itemize}
\end{itemize}

\subsection*{Secondary Priority (High Risks)}

\begin{itemize}
    \item \textbf{Recommendation for RISK-002 (Acceptable Use Policy):}
    \begin{itemize}
        \item \textbf{Action:} Develop a formal Acceptable Use Policy (AUP) that clearly defines the rules for using company networks, devices, and data. This policy should be communicated to all employees and signed as a condition of employment.
        \item \textbf{Impact:} Establishes a clear security baseline for employee behavior, reduces insider risk, and provides a basis for disciplinary action in case of policy violation.
    \end{itemize}
    \vspace{1em}
    \item \textbf{Recommendation for RISK-003 (Onboarding Training):}
    \begin{itemize}
        \item \textbf{Action:} Integrate a mandatory cybersecurity awareness module into the new employee onboarding process. This training should cover key topics such as phishing, password security, data handling, and the new AUP.
        \item \textbf{Impact:} Ensures that all new staff members are equipped with fundamental security knowledge from day one, reducing their susceptibility to common cyber threats.
    \end{itemize}
\end{itemize}

\end{document}
```