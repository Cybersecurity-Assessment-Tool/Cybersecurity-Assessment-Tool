```latex
\documentclass[12pt, a4paper]{article}

% Preamble: Required Packages
\usepackage[margin=1in]{geometry}
\usepackage{pifont} % For checkmarks and crosses
\usepackage{booktabs} % For professional tables
\usepackage{hyperref} % For clickable links
\usepackage{url} % For formatting URLs
\usepackage{seqsplit} % For splitting long strings
\usepackage{graphicx} % For logo
\usepackage{xcolor} % For colors

% Document Information
\title{Cybersecurity Posture Assessment Report}
\author{Cybersecurity Analysis Division}
\date{\today}

% Hyperref Setup
\hypersetup{
    colorlinks=true,
    linkcolor=blue,
    filecolor=magenta,      
    urlcolor=cyan,
    pdftitle={Cybersecurity Posture Assessment Report},
    pdfpagemode=FullScreen,
}

% Custom Commands
\newcommand{\yes}{\ding{51}}
\newcommand{\no}{\ding{55}}
\newcommand{\orgname}{\textbf{[Organization Name]}}
\newcommand{\clientdomain}{\texttt{[Domain]}}
\newcommand{\clientip}{\texttt{[Client IP]}}
\newcommand{\targetip}{\texttt{[Target IP]}}

\begin{document}

\maketitle
\thispagestyle{empty}
\newpage

\tableofcontents
\newpage

% --- 1. Executive Overview ---
\section{Executive Overview}
This report details the findings of a cybersecurity posture assessment conducted for \orgname. The assessment combines an analysis of organizational security controls via a questionnaire, a technical network scan of external-facing assets, and a review of pre-existing risks.

The overall security posture requires immediate attention. Several critical and high-risk gaps were identified. The most significant findings include:
\begin{itemize}
    \item \textbf{Critical Gap in Access Control:} Multi-Factor Authentication (MFA) is not enforced for accessing sensitive data systems. This exposes the organization to a high risk of data breach from compromised credentials.
    \item \textbf{High Risk from Lack of Training:} The organization does not have a formal security awareness training program for new or existing employees. This significantly increases susceptibility to social engineering and phishing attacks.
    \item \textbf{Exposed Management Service:} The technical scan identified an open Secure Shell (SSH) port on an external-facing system, which presents a direct vector for unauthorized access attempts.
\end{itemize}

Immediate remediation of these issues is crucial to reduce the organization's risk profile and protect its critical assets. Detailed recommendations are provided in Section \ref{sec:recommendations}.

% --- 2. Organizational Information ---
\section{Organizational Information}
The following information was used as the basis for this assessment. Due to the anonymized nature of the input data, placeholders have been used.

\begin{tabular}{@{}ll}
    \toprule
    \textbf{Attribute} & \textbf{Value} \\
    \midrule
    Organization Name & \orgname \\
    Email Domain & \clientdomain \\
    External IP Address (Client) & \clientip \\
    \bottomrule
\end{tabular}

% --- 3. Security Control Review (Questionnaire Analysis) ---
\section{Security Control Review (Questionnaire Analysis)}
A review of the organization's security controls was conducted based on a standardized questionnaire. The responses highlight significant gaps in the security program, particularly concerning access control and employee training.

\begin{table}[h!]
\centering
\caption{Security Control Questionnaire Results}
\label{tab:controls}
\begin{tabular}{@{}p{0.6\linewidth}cc@{}}
    \toprule
    \textbf{Control Question} & \textbf{Response} & \textbf{Status} \\
    \midrule
    Do you require MFA to access email? & Yes & \yes \\
    Do you require MFA to log into computers? & Yes & \yes \\
    \color{red}Do you require MFA to access sensitive data systems? & \color{red}No & \color{red}\no \\
    Does your organization have an employee acceptable use policy? & Yes & \yes \\
    \color{red}Does your organization do security awareness training for new employees? & \color{red}No & \color{red}\no \\
    \color{red}Does your organization do security awareness training for all employees at least once per year? & \color{red}No & \color{red}\no \\
    \bottomrule
\end{tabular}
\end{table}

\subsection*{Analysis of Gaps}
\begin{itemize}
    \item \textbf{MFA on Sensitive Systems:} The absence of MFA on systems containing sensitive data is a critical oversight. Should an employee's credentials be compromised, an attacker could gain direct access to the organization's most valuable information without needing a second authentication factor.
    \item \textbf{Security Awareness Training:} The complete lack of a security awareness training program leaves employees, the "human firewall," unequipped to recognize and respond to common cyber threats like phishing, malware, and social engineering. This makes the organization an easy target for initial access attacks.
\end{itemize}

% --- 4. Technical Scan Results ---
\section{Technical Scan Results}
An external network scan was performed to identify open ports and services on the organization's public-facing infrastructure.

\begin{itemize}
    \item \textbf{Target IP Address:} \targetip
    \item \textbf{Scan Date:} \today
    \item \textbf{Scanner Used:} Nmap
\end{itemize}

The scan revealed the following open port:

\begin{table}[h!]
\centering
\caption{Open Port Analysis}
\label{tab:ports}
\begin{tabular}{@{}llll@{}}
    \toprule
    \textbf{Port} & \textbf{State} & \textbf{Service} & \textbf{Notes} \\
    \midrule
    22/TCP & Open & SSH & Secure Shell is a common remote management protocol. \\
    & & & Exposing it to the public internet is a high risk. \\
    & & & No version information was available from the scan. \\
    \bottomrule
\end{tabular}
\end{table}

\subsection*{Analysis of Findings}
The presence of an open SSH port (22) on an external-facing asset (\targetip) is a significant security concern. This service is a primary target for automated brute-force attacks attempting to guess user credentials. Without proper controls, such as IP whitelisting, strong password policies, and ideally MFA, this port provides a direct entry point for attackers into the network.

% --- 5. Risk Assessment ---
\section{Risk Assessment}
This section synthesizes the findings from the security control review and the technical scan. The pre-existing risk register provided was empty; therefore, all identified risks are new findings from this assessment.

\begin{table}[h!]
\centering
\caption{Summary of Identified Risks}
\label{tab:risks}
\begin{tabular}{@{}p{0.1\linewidth}p{0.25\linewidth}p{0.4\linewidth}l@{}}
    \toprule
    \textbf{Risk ID} & \textbf{Risk Name} & \textbf{Description} & \textbf{Severity} \\
    \midrule
    RISK-001 & Lack of MFA on Sensitive Systems & The absence of a second authentication factor allows for unauthorized access to critical data if primary credentials are compromised. & \textbf{Critical} \\
    \addlinespace
    RISK-002 & Inadequate Security Awareness Program & Employees are not trained to identify or report security threats, increasing the likelihood of successful phishing and social engineering attacks. & \textbf{High} \\
    \addlinespace
    RISK-003 & Exposed SSH Management Port & The SSH service is open to the public internet, making it a target for brute-force attacks and unauthorized access attempts. & \textbf{High} \\
    \bottomrule
\end{tabular}
\end{table}

% --- 6. Recommendations ---
\section{Recommendations}
\label{sec:recommendations}
The following actionable recommendations are provided to address the identified risks and improve the overall security posture of \orgname.

\subsection*{RISK-001: Lack of MFA on Sensitive Systems (Critical)}
\begin{itemize}
    \item \textbf{Immediate Action:} Prioritize the implementation of a robust MFA solution for all systems classified as containing sensitive or critical data. This includes databases, file servers, and critical applications.
    \item \textbf{Policy Update:} Update the organization's access control policy to mandate the use of MFA for all privileged access and access to sensitive data.
\end{itemize}

\subsection*{RISK-002: Inadequate Security Awareness Program (High)}
\begin{itemize}
    \item \textbf{Immediate Action:} Implement a security awareness training program for all new employees as part of the onboarding process.
    \item \textbf{Ongoing Action:} Establish a mandatory, annual security awareness training program for all staff. This program should cover key topics such as phishing, password security, and acceptable use.
    \item \textbf{Testing:} Conduct periodic phishing simulation campaigns to test employee awareness and reinforce training concepts.
\end{itemize}

\subsection*{RISK-003: Exposed SSH Management Port (High)}
\begin{itemize}
    \item \textbf{Immediate Action:} Implement firewall rules to restrict access to port 22 on \targetip. Access should be limited to a whitelist of trusted IP addresses of administrators.
    \item \textbf{Best Practice:} If remote access from untrusted networks is required, it should be facilitated through a secure Virtual Private Network (VPN). The VPN should require MFA for access.
    \item \textbf{Hardening:} Ensure the SSH service is configured securely by disabling root login, using key-based authentication instead of passwords, and keeping the SSH server software up-to-date.
\end{itemize}

\end{document}
```