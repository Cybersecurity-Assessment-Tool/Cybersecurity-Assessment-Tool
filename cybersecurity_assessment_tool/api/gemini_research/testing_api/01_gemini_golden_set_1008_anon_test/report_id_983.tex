```latex
\documentclass[12pt]{article}

% Preamble: Required Packages
\usepackage[margin=1in]{geometry}
\usepackage{pifont} % For checkmarks and crosses
\usepackage{booktabs} % For professional tables
\usepackage{hyperref} % For clickable links and TOC
\usepackage{url} % For formatting URLs
\usepackage{seqsplit} % For splitting long strings to prevent overflow
\usepackage{graphicx}
\usepackage[table]{xcolor}
\usepackage{fancyhdr}

% --- Document Setup ---
\hypersetup{
    colorlinks=true,
    linkcolor=blue,
    filecolor=magenta,      
    urlcolor=cyan,
    pdftitle={Cybersecurity Assessment Report},
    pdfpagemode=FullScreen,
}

\pagestyle{fancy}
\fancyhf{}
\fancyhead[L]{Cybersecurity Assessment Report}
\fancyhead[R]{\textbf{[Organization Name]}}
\fancyfoot[C]{\thepage}

% Define severity colors
\definecolor{sev_critical}{HTML}{940000}
\definecolor{sev_high}{HTML}{D13F00}
\definecolor{sev_medium}{HTML}{E8A600}

% --- Document Start ---
\begin{document}

% --- Title Page ---
\begin{titlepage}
    \centering
    \vspace*{1cm}
    \Huge\textbf{Cybersecurity Assessment Report}
    \vspace{1.5cm}
    \Large
    Prepared For: \\
    \vspace{0.5cm}
    \textbf{[Organization Name]}
    \vfill
    \large
    Report Date: \today \\
    \vspace{0.5cm}
    Generated By: \\
    Expert Cybersecurity Analyst
\end{titlepage}

\tableofcontents
\newpage

% --- Section 1: Executive Overview ---
\section{Executive Overview}
This report provides a comprehensive analysis of the security posture for \textbf{[Organization Name]}, based on network scans, a security controls questionnaire, and a review of pre-existing risks. The assessment identified several areas of significant concern that require immediate attention to mitigate potential threats.

The most critical finding is a pre-existing vulnerability identified as \textbf{`Localhost Exposed'} with a CVSS score of 10.0 (Critical). This suggests a service intended for internal use only is accessible from the internet, posing an extreme risk of compromise.

Furthermore, a significant gap was identified in the organization's security culture. The lack of mandatory security awareness training for both new and existing employees creates a high susceptibility to social engineering and phishing attacks. This human-layer vulnerability undermines the effectiveness of other technical controls.

Technical scans revealed a publicly accessible SSH service (Port 22) on the external host \texttt{[Target IP]}. While necessary for remote administration, if not properly configured and hardened, this service presents a prime target for brute-force attacks and unauthorized access attempts.

In summary, the combination of a critical-level technical vulnerability and major deficiencies in security training places the organization at a \textbf{High Risk} of a security breach. The recommendations in this report are prioritized to address the most severe threats first.

\newpage

% --- Section 2: Organizational Information ---
\section{Organizational Information}
This section details the information provided for the assessment. Placeholders are used where data was not available.

\begin{tabular}{@{}ll}
    \toprule
    \textbf{Attribute} & \textbf{Value} \\
    \midrule
    Organization Name & \textbf{[Organization Name]} \\
    Primary Domain & \texttt{[Domain]} \\
    External IP Address Scanned & \texttt{[Client IP]} \\
    \bottomrule
\end{tabular}

% --- Section 3: Security Control Review ---
\section{Security Control Review}
The following table summarizes the organization's responses to a security controls questionnaire. These answers provide insight into the current policies and procedures governing information security.

\begin{table}[h!]
\centering
\begin{tabular}{@{}p{0.75\linewidth}c@{}}
    \toprule
    \textbf{Control Question} & \textbf{Response} \\
    \midrule
    Do you require MFA to access email? & \ding{51} \\
    Do you require MFA to log into computers? & \ding{51} \\
    Do you require MFA to access sensitive data systems? & \ding{51} \\
    Does your organization have an employee acceptable use policy? & \ding{51} \\
    \rowcolor{red!15} Does your organization do security awareness training for new employees? & \ding{55} \\
    \rowcolor{red!15} Does your organization do security awareness training for all employees at least once per year? & \ding{55} \\
    \bottomrule
\end{tabular}
\caption{Security Controls Questionnaire Results (\ding{51}=Yes, \ding{55}=No)}
\end{table}

\subsection*{Analysis}
The organization has successfully implemented Multi-Factor Authentication (MFA) across key systems, which is a commendable and effective control against credential theft. However, the two "No" responses represent a \textbf{critical gap} in the security program. Without mandatory security awareness training, employees are significantly more likely to become victims of phishing, malware, and other social engineering attacks, potentially rendering other technical controls ineffective.

\newpage

% --- Section 4: Technical Scan Results ---
\section{Technical Scan Results}
An external network scan was performed to identify open ports and services visible on the public internet.

\begin{itemize}
    \item \textbf{Scan Target:} \texttt{[Target IP]}
    \item \textbf{Scan Date:} Not specified in scan data.
\end{itemize}

\begin{table}[h!]
\centering
\begin{tabular}{@{}lllll@{}}
    \toprule
    \textbf{Port} & \textbf{State} & \textbf{Service} & \textbf{Product/Version} & \textbf{Notes} \\
    \midrule
    22/tcp & Open & SSH & Not Detected & Secure Shell (SSH) allows for remote \\
           &        &     &              & administration. Public exposure is a \\
           &        &     &              & significant risk if not properly hardened. \\
    \bottomrule
\end{tabular}
\caption{Open Ports Detected on \texttt{[Target IP]}}
\end{table}

\subsection*{Analysis}
The scan identified that port 22 (SSH) is open to the internet. This service is a common target for automated brute-force attacks that attempt to guess credentials. While the specific version of the SSH server was not determined, it is crucial to ensure it is fully patched against known vulnerabilities. The presence of this service, combined with the lack of security training, elevates the risk of a successful compromise.

% --- Section 5: Consolidated Risk Assessment ---
\section{Consolidated Risk Assessment}
This section synthesizes findings from all data sources into a consolidated list of identified risks.

\begin{table}[h!]
\centering
\begin{tabular}{@{}p{0.25\linewidth}p{0.5\linewidth}p{0.15\linewidth}@{}}
    \toprule
    \textbf{Risk Name} & \textbf{Description} & \textbf{Severity} \\
    \midrule
    \rowcolor{sev_critical!25}
    Localhost Exposed & A critical vulnerability (CVSS 10.0) indicates a service intended for internal-only access is exposed to the public internet. This could lead to a complete system compromise. & \textcolor{sev_critical}{\textbf{Critical}} \\
    
    \rowcolor{sev_high!25}
    Lack of Security Awareness Training & The absence of a formal training program leaves employees vulnerable to phishing and social engineering, making them the weakest link in the security chain. & \textcolor{sev_high}{\textbf{High}} \\
    
    \rowcolor{sev_high!25}
    Publicly Exposed SSH Service & The SSH administrative service is open to the internet, creating a target for brute-force attacks and exploitation if any vulnerabilities exist or weak credentials are used. & \textcolor{sev_high}{\textbf{High}} \\
    \bottomrule
\end{tabular}
\caption{Summary of Identified Risks}
\end{table}

\newpage

% --- Section 6: Recommendations ---
\section{Recommendations}
The following actionable recommendations are prioritized based on risk severity to help \textbf{[Organization Name]} improve its security posture.

\begin{enumerate}
    \item \textbf{[Critical] Investigate and Remediate `Localhost Exposed` Vulnerability:}
    \begin{itemize}
        \item \textbf{Action:} Immediately investigate the finding on host \texttt{[Target IP]} to identify the specific service that is improperly exposed.
        \item \textbf{Solution:} Reconfigure network firewalls and service settings to ensure the affected service is only accessible from trusted internal networks (e.g., localhost or a private IP range). This is the highest priority and should be addressed within 24 hours.
    \end{itemize}
    \vspace{0.5cm}
    
    \item \textbf{[High] Implement a Comprehensive Security Awareness Training Program:}
    \begin{itemize}
        \item \textbf{Action:} Procure and implement a security awareness training solution.
        \item \textbf{Solution:} Mandate that all new employees complete training as part of their onboarding process. Require all existing employees to complete the training annually. This program should cover topics like phishing identification, password hygiene, and acceptable use policies.
    \end{itemize}
    \vspace{0.5cm}

    \item \textbf{[High] Harden the Publicly Exposed SSH Service:}
    \begin{itemize}
        \item \textbf{Action:} Secure the SSH service running on port 22 of \texttt{[Target IP]}.
        \item \textbf{Solution:}
            \begin{itemize}
                \item Apply strict firewall rules to only allow SSH connections from known, trusted IP addresses (IP whitelisting).
                \item Disable password-based authentication and enforce the use of strong cryptographic keys (public key authentication).
                \item Ensure the SSH server software is updated to the latest stable version to protect against known vulnerabilities.
            \end{itemize}
    \end{itemize}
\end{enumerate}

\end{document}
```