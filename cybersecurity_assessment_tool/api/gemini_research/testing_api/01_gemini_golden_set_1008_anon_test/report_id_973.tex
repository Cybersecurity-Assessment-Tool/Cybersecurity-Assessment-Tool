```latex
\documentclass[12pt, a4paper]{article}

% Preamble: Required Packages
\usepackage[margin=1in]{geometry}
\usepackage{pifont} % For checkmarks and crosses
\usepackage{booktabs} % For professional tables
\usepackage{hyperref} % For clickable links and better PDF navigation
\usepackage{url} % For formatting URLs
\usepackage{seqsplit} % For splitting long strings without breaking words
\usepackage{graphicx}
\usepackage{xcolor}

% --- Document Setup ---
\hypersetup{
    colorlinks=true,
    linkcolor=blue,
    filecolor=magenta,      
    urlcolor=cyan,
    pdftitle={Cybersecurity Posture Report},
    pdfauthor={Cybersecurity Analyst},
    pdfsubject={Security Assessment},
    pdfkeywords={Cybersecurity, Nmap, Risk Assessment},
    bookmarks=true
}

\newcommand{\yes}{\ding{51}}
\newcommand{\no}{\ding{55}}

% --- Document Start ---
\begin{document}

% --- Title Page ---
\begin{titlepage}
    \centering
    \vspace*{1cm}
    \Huge\textbf{Cybersecurity Posture Report}
    \vspace{1.5cm}
    \Large
    Prepared for: \textbf{[Organization Name]} \\
    \vspace{1cm}
    Date of Report: \today \\
    Scan Date: November 22, 2025
    \vfill
    \large
    \textbf{Author:} Cybersecurity Analyst \\
    \textbf{Classification:} Confidential
\end{titlepage}

\tableofcontents
\newpage

% --- Section 1: Executive Summary ---
\section{Executive Summary}
This report details the findings of a cybersecurity assessment conducted for \textbf{[Organization Name]}. The assessment combined a review of organizational security controls via a questionnaire, an external network scan of public-facing assets, and an analysis of pre-existing risks.

The overall security posture is mixed. The organization demonstrates strong identity and access management practices with consistent enforcement of Multi-Factor Authentication (MFA). However, two high-risk findings were identified that require immediate attention:

\begin{enumerate}
    \item \textbf{Critical Gap in Employee Onboarding:} The lack of mandatory security awareness training for new employees represents a significant vulnerability. New hires are often prime targets for social engineering and phishing attacks, and this gap exposes the organization to an elevated risk of initial compromise.
    \item \textbf{Vulnerable Public-Facing Web Server:} The external network scan identified an outdated version of the Nginx web server (1.18.0) running on a public-facing system. This version is several years old and has multiple known vulnerabilities, making it a direct target for exploitation.
\end{enumerate}

No pre-existing risks were documented in the provided data. Recommendations in this report focus on remediating these critical findings to strengthen the organization's defensive posture against common cyber threats.

% --- Section 2: Organizational Information ---
\section{Organizational Information}
The following details were used as the basis for this assessment. As per the provided data, placeholder information has been used where specific details were not available.

\begin{tabular}{@{}ll}
    \toprule
    \textbf{Attribute} & \textbf{Value} \\
    \midrule
    Organization Name & \textbf{[Organization Name]} \\
    Primary Email Domain & \texttt{[Domain]} \\
    Assessed External IP & \texttt{[Client IP]} \\
    \bottomrule
\end{tabular}

% --- Section 3: Security Control Review ---
\section{Security Control Review}
A review of administrative and procedural security controls was conducted based on a standardized questionnaire. The responses indicate a solid foundation in access control but reveal a critical weakness in the security training program.

\subsection{Questionnaire Results}
\begin{tabular}{@{}p{0.7\linewidth}c@{}}
    \toprule
    \textbf{Control Question} & \textbf{Response} \\
    \midrule
    Do you require MFA to access email? & \yes \\
    Do you require MFA to log into computers? & \yes \\
    Do you require MFA to access sensitive data systems? & \yes \\
    Does your organization have an employee acceptable use policy? & \yes \\
    Does your organization do security awareness training for new employees? & \textcolor{red}{\no} \\
    Does your organization do security awareness training for all employees at least once per year? & \yes \\
    \bottomrule
\end{tabular}

\subsection{Analysis}
The "No" response to providing security awareness training for new employees is a \textbf{High-Risk} finding. The initial period of employment is a critical window where new staff are unfamiliar with company policies and are more susceptible to targeted attacks. Failing to provide immediate training on topics such as phishing identification, acceptable use, and data handling significantly increases the risk of a security incident originating from human error.

% --- Section 4: Technical Scan Results ---
\section{Technical Scan Results}
An external network vulnerability scan was performed against the organization's public-facing infrastructure to identify open ports and potentially vulnerable services.

\begin{itemize}
    \item \textbf{Target IP Address:} \texttt{[Target IP]}
    \item \textbf{Scan Date:} November 22, 2025
\end{itemize}

\subsection{Open Ports and Services}
The following table details the services discovered during the scan.

\begin{tabular}{@{}lllll@{}}
    \toprule
    \textbf{Port} & \textbf{State} & \textbf{Service} & \textbf{Product} & \textbf{Version} \\
    \midrule
    443/tcp & open & https & nginx & 1.18.0 \\
    \bottomrule
\end{tabular}

\subsection{Analysis}
The scan identified an Nginx web server, version \textbf{1.18.0}, exposed to the internet. This version was released in April 2020 and is now considered significantly outdated. It is affected by several publicly disclosed vulnerabilities, including but not limited to CVE-2021-23017. Running outdated software on internet-facing systems is a \textbf{High-Risk} finding, as it provides attackers with a well-known and often easily exploitable entry point into the network.

% --- Section 5: Risk Assessment Summary ---
\section{Risk Assessment Summary}
This section synthesizes findings from the security control review and technical scan. As no pre-existing risks were provided, the table below reflects only the newly identified risks from this assessment.

\begin{tabular}{@{}p{0.1\linewidth}p{0.45\linewidth}p{0.2\linewidth}l@{}}
    \toprule
    \textbf{Risk ID} & \textbf{Description} & \textbf{Source} & \textbf{Severity} \\
    \midrule
    RISK-001 & Lack of security awareness training for new employees increases susceptibility to social engineering and phishing attacks. & Questionnaire & \textbf{High} \\
    \addlinespace
    RISK-002 & An outdated and vulnerable Nginx web server (v1.18.0) is exposed to the internet, risking system compromise. & Network Scan & \textbf{High} \\
    \bottomrule
\end{tabular}

% --- Section 6: Recommendations ---
\section{Recommendations}
The following actions are recommended to mitigate the identified risks and improve the overall security posture of \textbf{[Organization Name]}.

\begin{enumerate}
    \item \textbf{Implement Onboarding Security Training (RISK-001):}
    \begin{itemize}
        \item \textbf{Action:} Develop and mandate a security awareness training module as a required step in the new employee onboarding process.
        \item \textbf{Details:} This training should be completed within the first week of employment and cover core topics such as phishing awareness, password security, acceptable use of company assets, and how to report a security incident.
        \item \textbf{Priority:} Critical
    \end{itemize}
    \vspace{0.5cm}
    \item \textbf{Upgrade Web Server and Implement Patch Management (RISK-002):}
    \begin{itemize}
        \item \textbf{Action:} Immediately plan and execute the upgrade of the Nginx server on \texttt{[Target IP]} from version 1.18.0 to the latest stable release.
        \item \textbf{Details:} Establish a formal patch management policy that requires quarterly reviews and timely application of security patches for all internet-facing systems.
        \item \textbf{Priority:} Critical
    \end{itemize}
\end{enumerate}

% --- Section 7: Conclusion ---
\section{Conclusion}
While \textbf{[Organization Name]} has implemented commendable access controls through MFA, this assessment highlights critical deficiencies in security training and patch management that undermine these strengths. By addressing the high-risk findings related to new employee onboarding and the vulnerable Nginx server, the organization can significantly reduce its attack surface and enhance its resilience against prevalent cyber threats. Continuous monitoring and proactive management of both human and technical vulnerabilities are essential for maintaining a robust security posture.

\end{document}
```