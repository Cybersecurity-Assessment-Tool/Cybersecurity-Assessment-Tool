```latex
\documentclass[12pt]{article}

% Required Packages
\usepackage[margin=1in]{geometry}
\usepackage{pifont} % For checkmarks and crosses
\usepackage{booktabs} % For professional tables
\usepackage{hyperref} % For hyperlinks
\usepackage{url} % For URL formatting
\usepackage{seqsplit} % To split long strings in texttt
\usepackage[utf8]{inputenc}

% Document Metadata
\hypersetup{
    colorlinks=true,
    linkcolor=blue,
    filecolor=magenta,      
    urlcolor=cyan,
    pdftitle={Cybersecurity Assessment Report},
    pdfauthor={Cybersecurity Analyst},
    pdfsubject={Security Analysis},
    pdfkeywords={Security, Risk, Assessment},
}

% Define checkmark and cross symbols for clarity
\newcommand{\cmark}{\ding{51}}
\newcommand{\xmark}{\ding{55}}

\begin{document}

% --- Title Page ---
\begin{titlepage}
    \centering
    \vspace*{\fill}
    \Huge\textbf{Cybersecurity Assessment Report}
    \vspace{1.5cm}
    \Large
    Prepared for: \textbf{[Organization Name]} \\
    \vspace{0.5cm}
    Date: \today
    \vspace*{\fill}
    \textit{This report contains sensitive information and should be handled with care.}
\end{titlepage}

\tableofcontents
\newpage

% --- Section 1: Executive Summary ---
\section{Executive Summary}
This report provides a comprehensive analysis of the security posture of \textbf{[Organization Name]}, based on a review of organizational security controls, an external network scan, and pre-existing risk data.

The assessment identified critical gaps in the organization's security controls, despite a strong external network perimeter. Specifically, the lack of Multi-Factor Authentication (MFA) for computer logins and the absence of security awareness training for new employees represent high-severity risks. These weaknesses could expose the organization to significant threats, such as unauthorized access and social engineering attacks.

The external network scan of the target IP address revealed no open ports, suggesting a well-configured firewall and a minimal external attack surface. No previously documented vulnerabilities were provided for this assessment.

Immediate remediation is recommended to address the identified control gaps to strengthen the organization's defense against common cyber threats.

% --- Section 2: Organizational Information ---
\section{Organizational Information}
This section details the information provided by the organization.
\begin{itemize}
    \item \textbf{Organization Name:} \textbf{[Organization Name]}
    \item \textbf{Primary Domain:} \texttt{[Domain]}
    \item \textbf{External IP Scanned:} \texttt{[Client IP]}
\end{itemize}

% --- Section 3: Security Control Review ---
\section{Security Control Review}
The following table summarizes the organization's responses to a security controls questionnaire. A green checkmark (\cmark) indicates a positive control is in place, while a red cross (\xmark) indicates a potential security gap.

\begin{table}[h!]
\centering
\caption{Security Controls Questionnaire Results}
\begin{tabular}{p{0.75\linewidth} c}
\toprule
\textbf{Control Question} & \textbf{Response} \\
\midrule
Do you require MFA to access email? & \cmark \\
Do you require MFA to log into computers? & \xmark \\
Do you require MFA to access sensitive data systems? & \cmark \\
Does your organization have an employee acceptable use policy? & \cmark \\
Does your organization do security awareness training for new employees? & \xmark \\
Does your organization do security awareness training for all employees at least once per year? & \cmark \\
\bottomrule
\end{tabular}
\end{table}

\subsection{Analysis of Control Gaps}
Two critical control gaps were identified:
\begin{itemize}
    \item \textbf{No MFA for Computer Logins:} The absence of MFA on workstations is a significant vulnerability. If an attacker compromises an employee's password, they can gain direct access to the network without needing a second authentication factor.
    \item \textbf{No Security Training for New Employees:} New hires are a common target for phishing and other social engineering attacks. Without immediate training upon joining, they represent an uninformed and high-risk group within the organization.
\end{itemize}

% --- Section 4: Technical Scan Results ---
\section{Technical Scan Results}
An external network scan was performed to identify open ports and exposed services.

\begin{itemize}
    \item \textbf{Target IP Address:} \texttt{[Target IP]}
    \item \textbf{Scan Date:} Not available in scan data.
\end{itemize}

\subsection{Findings}
The network scan completed successfully but did not identify any open TCP or UDP ports on the target host. This indicates a strong firewall configuration that properly implements a default-deny policy for unsolicited inbound traffic, which is a security best practice.

% --- Section 5: Risk Assessment ---
\section{Risk Assessment}
This section synthesizes findings from the security control review, technical scan, and any pre-existing risks. The overall risk is determined by correlating these data points.

\begin{table}[h!]
\centering
\caption{Identified Risks}
\begin{tabular}{p{0.25\linewidth} p{0.5\linewidth} l}
\toprule
\textbf{Risk Name} & \textbf{Overview} & \textbf{Severity} \\
\midrule
\textbf{Lack of MFA on Workstations} & The absence of Multi-Factor Authentication on employee computers significantly increases the risk of unauthorized access and lateral movement should an attacker compromise user credentials. & \textbf{High} \\
\addlinespace
\textbf{No Onboarding Security Training} & New employees are not provided with security awareness training, making them more susceptible to phishing and social engineering attacks from their first day of employment. & \textbf{High} \\
\addlinespace
\textbf{Previously Known Vulnerabilities} & No pre-existing vulnerabilities were reported or identified during this assessment. & Low \\
\bottomrule
\end{tabular}
\end{table}

% --- Section 6: Recommendations ---
\section{Recommendations}
Based on the analysis, the following actions are recommended to mitigate the identified risks and improve the overall security posture of \textbf{[Organization Name]}.

\begin{enumerate}
    \item \textbf{Implement MFA for All Workstation Logins (High Priority):}
    \begin{itemize}
        \item \textbf{Action:} Deploy a mandatory Multi-Factor Authentication (MFA) solution for all employee computer logins. This is the single most effective control to prevent unauthorized access from compromised credentials.
        \item \textbf{Examples:} Utilize solutions such as Windows Hello for Business, Duo Security, Okta, or other reputable identity providers.
    \end{itemize}
    \vspace{0.5cm}
    \item \textbf{Establish a Security Training Program for New Hires (High Priority):}
    \begin{itemize}
        \item \textbf{Action:} Develop and mandate a security awareness training module as a required part of the new employee onboarding process. This training should be completed within the first week of employment.
        \item \textbf{Content:} The training should cover, at a minimum, phishing identification, password security, acceptable use of company assets, and how to report a security incident.
    \end{itemize}
\end{enumerate}

% --- Section 7: Conclusion ---
\section{Conclusion}
The security assessment of \textbf{[Organization Name]} reveals a mixed posture. The organization demonstrates strength in its external network security, with no exposed services detected. However, significant internal and procedural weaknesses exist. The lack of MFA on workstations and the absence of security training for new hires create high-risk vulnerabilities that must be addressed urgently. By implementing the recommendations outlined in this report, the organization can substantially reduce its risk exposure and build a more resilient security foundation.

\end{document}
```