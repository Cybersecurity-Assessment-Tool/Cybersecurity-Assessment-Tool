```latex
\documentclass[12pt]{article}

% --- PACKAGES ---
\usepackage[margin=1in]{geometry}
\usepackage{pifont} % For checkmarks and crosses
\usepackage{booktabs} % For professional tables
\usepackage{hyperref} % For hyperlinks
\usepackage{url}      % For URL formatting
\usepackage{seqsplit} % For splitting long strings in texttt
\usepackage{xcolor}   % For colors

% --- DOCUMENT METADATA ---
\title{Cybersecurity Posture Assessment Report}
\author{Cybersecurity Analysis Division}
\date{\today}

% --- HYPERREF SETUP ---
\hypersetup{
    colorlinks=true,
    linkcolor=blue,
    filecolor=magenta,      
    urlcolor=cyan,
    pdftitle={Cybersecurity Posture Assessment Report},
    pdfpagemode=FullScreen,
}

\begin{document}

\maketitle
\thispagestyle{empty}
\newpage

\tableofcontents
\newpage

% ===================================================================
% SECTION 1: EXECUTIVE SUMMARY
% ===================================================================
\section{Executive Summary}

This report provides a cybersecurity posture assessment for \textbf{[Organization Name]}, conducted on \today. The analysis is based on a combination of a self-reported security controls questionnaire, an external network vulnerability scan, and a review of previously documented risks.

The assessment identified two significant areas of concern requiring immediate attention. A critical gap exists in the enforcement of Multi-Factor Authentication (MFA) for systems containing sensitive data. Additionally, the organization lacks a formal Acceptable Use Policy (AUP), which represents a high-risk governance gap. These findings indicate a heightened risk of unauthorized access to critical assets and potential insider threats.

The external network scan of the target IP address \texttt{[Target IP]} did not detect any open ports or exposed services. While this can indicate a strong firewall configuration, it does not diminish the severity of the identified internal policy and control weaknesses.

This report details these findings and provides actionable recommendations to mitigate the identified risks and strengthen the organization's overall security posture.

% ===================================================================
% SECTION 2: ORGANIZATIONAL INFORMATION
% ===================================================================
\section{Organizational Information}

The following details were used as the basis for this assessment. Due to the anonymized nature of the provided data, placeholders have been used where necessary.

\begin{itemize}
    \item \textbf{Organization Name:} \textbf{[Organization Name]}
    \item \textbf{Primary Domain:} \texttt{[Domain]}
    \item \textbf{Target IP Address for Scan:} \texttt{[Client IP]}
\end{itemize}

% ===================================================================
% SECTION 3: SECURITY CONTROL REVIEW
% ===================================================================
\section{Security Control Review}

The following table summarizes the organization's responses to a security controls questionnaire. A checkmark (\ding{51}) indicates a positive response (control in place), while a cross (\ding{55}) indicates a negative response, highlighting a potential security gap.

\begin{table}[h!]
\centering
\caption{Security Controls Questionnaire Results}
\begin{tabular}{p{0.7\textwidth}c}
\toprule
\textbf{Control Question} & \textbf{Response} \\
\midrule
Do you require MFA to access email? & \ding{51} \\
Do you require MFA to log into computers? & \ding{51} \\
Do you require MFA to access sensitive data systems? & \textcolor{red}{\ding{55}} \\
Does your organization have an employee acceptable use policy? & \textcolor{red}{\ding{55}} \\
Does your organization do security awareness training for new employees? & \ding{51} \\
Does your organization do security awareness training for all employees at least once per year? & \ding{51} \\
\bottomrule
\end{tabular}
\end{table}

\subsection*{Analysis of Gaps}
The questionnaire reveals two primary areas of concern:
\begin{itemize}
    \item \textbf{MFA on Sensitive Systems:} The lack of mandatory MFA for sensitive data systems is a critical vulnerability. This significantly increases the risk of a data breach resulting from compromised credentials.
    \item \textbf{Acceptable Use Policy (AUP):} The absence of an AUP creates ambiguity for employees regarding the proper use of company resources and data. This can lead to unintentional security incidents, insider threats, and non-compliance with regulations.
\end{itemize}

% ===================================================================
% SECTION 4: TECHNICAL SCAN RESULTS
% ===================================================================
\section{Technical Scan Results}

An external network scan was performed to identify exposed services and potential vulnerabilities on the organization's perimeter.

\begin{itemize}
    \item \textbf{Target IP Address:} \texttt{[Target IP]}
    \item \textbf{Scan Date:} Information not available in scan data.
\end{itemize}

\subsection*{Findings}
The scan completed successfully, but \textbf{no open ports or services were detected} on the target system. This suggests that the host may be offline, not responsive to the scan probes, or protected by a well-configured firewall that drops or rejects unsolicited traffic. While a clean scan is a positive indicator of strong network perimeter defense, it does not provide visibility into internal vulnerabilities or misconfigurations.

% ===================================================================
% SECTION 5: RISK ASSESSMENT
% ===================================================================
\section{Risk Assessment}

This section correlates findings from the security control review, technical scan, and any pre-existing risk data. The pre-existing risk data provided for this assessment was empty. The following table summarizes the new risks identified during this engagement.

\begin{table}[h!]
\centering
\caption{Identified Risk Summary}
\begin{tabular}{p{0.25\textwidth}p{0.5\textwidth}p{0.15\textwidth}}
\toprule
\textbf{Risk Name} & \textbf{Overview} & \textbf{Severity} \\
\midrule
Lack of MFA on Sensitive Systems & The absence of MFA on critical systems makes them susceptible to unauthorized access via compromised credentials. A single password breach could lead to a significant data compromise. & \textbf{Critical} \\
\addlinespace
Missing Acceptable Use Policy & Without a formal AUP, the organization lacks a foundational governance document to enforce secure behavior, manage insider risk, and establish a baseline for employee accountability. & \textbf{High} \\
\bottomrule
\end{tabular}
\end{table}

% ===================================================================
% SECTION 6: RECOMMENDATIONS
% ===================================================================
\section{Recommendations}

Based on the analysis, the following actions are recommended to mitigate the identified risks and improve the overall security posture of \textbf{[Organization Name]}.

\subsection*{Immediate Actions (Critical Priority)}
\begin{enumerate}
    \item \textbf{Implement MFA on All Sensitive Systems:}
    \begin{itemize}
        \item \textbf{Action:} Deploy and enforce a strong Multi-Factor Authentication solution for all applications, databases, and administrative interfaces that process, store, or transmit sensitive information.
        \item \textbf{Justification:} This is the single most effective control to prevent unauthorized access resulting from stolen or weak credentials, directly addressing the critical risk identified.
    \end{itemize}
\end{enumerate}

\subsection*{Strategic Actions (High Priority)}
\begin{enumerate}
    \item \textbf{Develop and Implement an Acceptable Use Policy (AUP):}
    \begin{itemize}
        \item \textbf{Action:} Draft a comprehensive AUP that clearly defines the rules and responsibilities for all employees and contractors when using company-owned technology and data. The policy should be formally approved, distributed to all staff, and integrated into the new employee onboarding process.
        \item \textbf{Justification:} An AUP establishes clear security expectations, reduces the likelihood of insider threats (both malicious and accidental), and provides a legal and administrative framework for enforcing security standards.
    \end{itemize}
\end{enumerate}

\end{document}
```