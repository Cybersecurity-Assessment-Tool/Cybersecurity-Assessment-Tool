```latex
\documentclass[12pt]{article}

% ----------------------------------------------------------------
% PACKAGES
% ----------------------------------------------------------------
\usepackage[margin=1in]{geometry}
\usepackage{pifont} % For checkmarks and crosses
\usepackage{booktabs} % For professional tables
\usepackage{hyperref} % For clickable links
\usepackage{url}      % For URL formatting
\usepackage{seqsplit} % For splitting long strings to prevent overflow

% ----------------------------------------------------------------
% DOCUMENT METADATA
% ----------------------------------------------------------------
\title{Cybersecurity Assessment Report}
\author{Cybersecurity Analysis Division}
\date{\today}

% ----------------------------------------------------------------
% DOCUMENT START
% ----------------------------------------------------------------
\begin{document}

\maketitle
\thispagestyle{empty}
\newpage

\tableofcontents
\newpage

% ----------------------------------------------------------------
% 1. EXECUTIVE SUMMARY
% ----------------------------------------------------------------
\section{Executive Summary}

This report provides a cybersecurity assessment for \textbf{[Organization Name]}, based on an analysis of organizational security controls, an external network scan, and a review of pre-existing risks.

The assessment identified several critical and high-risk gaps in the organization's administrative and access control policies. Specifically, the lack of multi-factor authentication (MFA) on employee computers and sensitive data systems, the absence of a formal Acceptable Use Policy, and the omission of security training during employee onboarding represent significant vulnerabilities. These weaknesses could expose the organization to risks such as unauthorized access, data breaches, and insider threats.

The external network scan of the target IP address \texttt{[Target IP]} did not reveal any open ports. While this suggests a strong network perimeter, it does not mitigate the internal risks identified. No pre-existing vulnerabilities were documented for review.

Immediate remediation should focus on implementing comprehensive MFA, developing foundational security policies, and integrating security awareness into the employee lifecycle. Addressing these gaps is crucial to strengthening the organization's overall security posture.

% ----------------------------------------------------------------
% 2. ORGANIZATIONAL INFORMATION
% ----------------------------------------------------------------
\section{Organizational Information}

This section contains the high-level information used as the basis for this assessment. Due to the anonymized nature of the input data, placeholders are used where necessary.

\begin{itemize}
    \item \textbf{Organization Name:} \textbf{[Organization Name]}
    \item \textbf{Primary Domain:} \texttt{[Domain]}
    \item \textbf{External IP Scanned:} \texttt{[Client IP]}
\end{itemize}

% ----------------------------------------------------------------
% 3. SECURITY CONTROL REVIEW (QUESTIONNAIRE)
% ----------------------------------------------------------------
\section{Security Control Review}

The following table summarizes the organization's responses to a security controls questionnaire. The assessment column highlights alignment with cybersecurity best practices. Answers marked with \ding{55} (No) indicate significant gaps requiring attention.

\begin{table}[h!]
\centering
\caption{Security Controls Questionnaire Analysis}
\begin{tabular}{p{0.5\textwidth} c p{0.3\textwidth}}
\toprule
\textbf{Control Question} & \textbf{Response} & \textbf{Assessment} \\
\midrule
Do you require MFA to access email? & \ding{51} (Yes) & Best Practice Met \\
\addlinespace
Do you require MFA to log into computers? & \ding{55} (No) & \textbf{Critical Risk Gap} \\
\addlinespace
Do you require MFA to access sensitive data systems? & \ding{55} (No) & \textbf{Critical Risk Gap} \\
\addlinespace
Does your organization have an employee acceptable use policy? & \ding{55} (No) & \textbf{High Risk Gap} \\
\addlinespace
Does your organization do security awareness training for new employees? & \ding{55} (No) & \textbf{High Risk Gap} \\
\addlinespace
Does your organization do security awareness training for all employees at least once per year? & \ding{51} (Yes) & Best Practice Met \\
\bottomrule
\end{tabular}
\end{table}

% ----------------------------------------------------------------
% 4. TECHNICAL SCAN RESULTS
% ----------------------------------------------------------------
\section{Technical Scan Results}

An external network vulnerability scan was conducted to identify open ports and exposed services.

\begin{itemize}
    \item \textbf{Target IP Address:} \texttt{[Target IP]}
    \item \textbf{Scan Date:} Not provided in scan data.
    \item \textbf{Findings:} The scan completed successfully but found \textbf{no open ports}.
\end{itemize}

\subsection*{Analysis}
The absence of open ports is a positive finding, suggesting that a firewall is properly configured to block unsolicited inbound traffic from the internet. This significantly reduces the external attack surface. However, this result does not provide insight into the security of internal systems or vulnerabilities that could be exploited by other means, such as phishing or malware.

% ----------------------------------------------------------------
% 5. RISK ASSESSMENT SUMMARY
% ----------------------------------------------------------------
\section{Risk Assessment Summary}

This section synthesizes findings from the security control review and technical scan. The primary risks identified are related to internal policies and access controls rather than external network vulnerabilities.

\begin{table}[h!]
\centering
\caption{Identified Risks}
\begin{tabular}{p{0.25\textwidth} p{0.5\textwidth} l}
\toprule
\textbf{Risk Name} & \textbf{Overview} & \textbf{Severity} \\
\midrule
\addlinespace
Lack of MFA on Endpoints and Systems & The absence of MFA on computers and sensitive systems allows an attacker with stolen credentials to gain unauthorized access, pivot within the network, and exfiltrate data. & \textbf{Critical} \\
\addlinespace
Inadequate Security Policies & Without a formal Acceptable Use Policy (AUP), there is no clear guidance for employees on the proper use of company assets, leading to inconsistent security practices and a lack of accountability. & \textbf{High} \\
\addlinespace
Deficient Employee Onboarding & Failing to provide security awareness training to new employees leaves a critical window of vulnerability. New hires are often targeted by social engineering attacks and are unaware of internal security procedures. & \textbf{High} \\
\addlinespace
\bottomrule
\end{tabular}
\end{table}

% ----------------------------------------------------------------
% 6. RECOMMENDATIONS
% ----------------------------------------------------------------
\section{Recommendations}

The following actions are recommended to mitigate the identified risks and improve the overall security posture of \textbf{[Organization Name]}.

\subsection*{Immediate Actions (0-30 Days)}
\begin{enumerate}
    \item \textbf{Implement MFA on Sensitive Systems:} Prioritize the deployment of multi-factor authentication on all systems containing sensitive or critical data. This is the most effective control to prevent unauthorized access.
    \item \textbf{Develop an Acceptable Use Policy (AUP):} Draft and implement a formal AUP that clearly defines the rules and responsibilities for all employees when using company IT assets. Require all employees to read and acknowledge the policy.
\end{enumerate}

\subsection*{Short-Term Actions (30-90 Days)}
\begin{enumerate}
    \item \textbf{Enforce MFA on all Endpoints:} Expand the MFA requirement to cover all employee computer logins (desktops and laptops). This helps prevent lateral movement within the network should a user's credentials be compromised.
    \item \textbf{Integrate Security Training into Onboarding:} Develop a mandatory security awareness training module for all new hires. This module should be completed as part of the standard onboarding process before full system access is granted.
\end{enumerate}

\subsection*{Ongoing Actions}
\begin{enumerate}
    \item \textbf{Conduct Regular Policy Reviews:} Review and update the AUP and other security policies at least annually to ensure they remain relevant to current threats and business operations.
    \item \textbf{Enhance Security Awareness Program:} Continue the annual security training program and supplement it with periodic phishing simulations and security bulletins to maintain a high level of awareness among all staff.
\end{enumerate}

% ----------------------------------------------------------------
% DOCUMENT END
% ----------------------------------------------------------------
\end{document}
```