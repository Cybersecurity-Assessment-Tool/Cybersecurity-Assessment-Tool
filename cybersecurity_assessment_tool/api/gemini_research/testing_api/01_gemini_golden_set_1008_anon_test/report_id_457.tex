```latex
\documentclass[12pt]{article}

% Required Packages
\usepackage[margin=1in]{geometry}
\usepackage{pifont} % For checkmarks and crosses
\usepackage{booktabs} % For professional tables
\usepackage{hyperref} % For hyperlinks
\usepackage{url} % For URL formatting
\usepackage{seqsplit} % For splitting long strings in tt font

% Document Metadata
\title{Cybersecurity Posture Assessment Report}
\author{Cybersecurity Analysis Division}
\date{November 22, 2025}

\begin{document}

\maketitle
\thispagestyle{empty}
\newpage
\tableofcontents
\newpage

\section{Executive Summary}

This report details the findings of a cybersecurity assessment conducted on November 22, 2025. The evaluation combined a review of organizational security controls, an external network vulnerability scan, and an analysis of pre-existing risks.

The assessment identified several areas of significant concern that elevate the organization's risk profile. Key findings include:
\begin{itemize}
    \item \textbf{Critical Control Gap:} Multi-Factor Authentication (MFA) is not enforced for accessing sensitive data systems. This represents a critical vulnerability, as compromised credentials could lead directly to a major data breach.
    \item \textbf{High-Risk Policy Gap:} The organization does not conduct mandatory annual security awareness training for all employees. This oversight increases susceptibility to social engineering attacks, such as phishing.
    \item \textbf{High-Risk Technical Vulnerability:} The external-facing web server is running an outdated version of Nginx (1.18.0), which has numerous publicly known vulnerabilities. This exposes the organization to potential remote exploitation.
\end{itemize}

Immediate remediation of these issues is strongly recommended to reduce the likelihood of a security incident and strengthen the overall defensive posture.

\section{Organizational Information}

This assessment was conducted for the following entity. As per the provided data, placeholder values are used where specific information was not available.

\begin{itemize}
    \item \textbf{Organization Name:} \textbf{[Organization Name]}
    \item \textbf{Primary Email Domain:} \texttt{[Domain]}
    \item \textbf{External IP Address Scanned:} \texttt{[Client IP]}
\end{itemize}

\section{Security Control Review (Questionnaire Analysis)}

The following table summarizes the organization's responses to a security controls questionnaire. Responses marked with a cross (\ding{55}) indicate a deviation from security best practices and represent a control gap.

\begin{table}[h!]
\centering
\caption{Security Controls Questionnaire Results}
\begin{tabular}{p{0.8\linewidth} c}
\toprule
\textbf{Control Question} & \textbf{Response} \\
\midrule
Do you require MFA to access email? & \ding{51} \\ % Yes
Do you require MFA to log into computers? & \ding{51} \\ % Yes
\textbf{Do you require MFA to access sensitive data systems?} & \textbf{\ding{55}} \\ % No
Does your organization have an employee acceptable use policy? & \ding{51} \\ % Yes
Does your organization do security awareness training for new employees? & \ding{51} \\ % Yes
\textbf{Does your organization do security awareness training for all employees at least once per year?} & \textbf{\ding{55}} \\ % No
\bottomrule
\end{tabular}
\end{table}

\subsection*{Analysis}
The review identified two significant control gaps:
\begin{enumerate}
    \item \textbf{Lack of MFA for Sensitive Systems:} The absence of MFA on critical data systems is a severe security weakness. Should an attacker compromise a user's credentials, they would have direct access to the organization's most valuable data.
    \item \textbf{Lack of Annual Security Training:} While new employees receive training, the lack of an annual refresher course for all staff leaves the organization vulnerable. Threat landscapes evolve, and continuous education is essential to defend against modern phishing and social engineering tactics.
\end{enumerate}

\section{Technical Scan Results}

An external network scan was performed to identify open ports and exposed services.

\begin{itemize}
    \item \textbf{Target IP Address:} \texttt{[Target IP]}
    \item \textbf{Scan Date:} 2025-11-22
\end{itemize}

\begin{table}[h!]
\centering
\caption{Open Ports and Services}
\begin{tabular}{l l l l l}
\toprule
\textbf{Port} & \textbf{State} & \textbf{Service} & \textbf{Product} & \textbf{Version} \\
\midrule
443/tcp & open & https & nginx & 1.18.0 \\
\bottomrule
\end{tabular}
\end{table}

\subsection*{Analysis}
The scan identified a web server running \textbf{Nginx version 1.18.0}. This version was released in April 2020 and is now significantly outdated. It is affected by multiple Common Vulnerabilities and Exposures (CVEs) discovered since its release. Running this version on a public-facing server presents a high risk of compromise through well-known exploits.

\section{Consolidated Risk Assessment}

The following table consolidates all identified risks from the questionnaire, technical scan, and pre-existing risk register. Each risk has been assigned a severity level based on its potential impact and likelihood of exploitation.

\begin{table}[h!]
\centering
\caption{Summary of Identified Risks}
\begin{tabular}{p{0.1\linewidth} p{0.5\linewidth} p{0.15\linewidth} p{0.1\linewidth}}
\toprule
\textbf{Risk ID} & \textbf{Description} & \textbf{Source} & \textbf{Severity} \\
\midrule
RISK-001 & Lack of Multi-Factor Authentication (MFA) on sensitive data systems. & Questionnaire & Critical \\
\addlinespace
RISK-002 & Outdated Nginx web server (v1.18.0) exposed to the internet. & Network Scan & High \\
\addlinespace
RISK-003 & No mandatory annual security awareness training for all employees. & Questionnaire & High \\
\bottomrule
\end{tabular}
\end{table}

\section{Recommendations}

To mitigate the identified risks and improve the organization's security posture, the following actions are recommended for immediate implementation.

\begin{enumerate}
    \item \textbf{RISK-001 (Critical): Implement MFA for Sensitive Systems}
    \begin{itemize}
        \item Prioritize the deployment of a robust MFA solution across all applications and systems that store, process, or transmit sensitive data.
        \item This should be treated as the highest priority remediation effort to prevent unauthorized access in the event of credential compromise.
    \end{itemize}
    
    \item \textbf{RISK-002 (High): Upgrade Web Server and Implement Patch Management}
    \begin{itemize}
        \item Immediately plan and execute an upgrade of the Nginx server from version 1.18.0 to the latest stable release.
        \item Establish a formal patch management policy that ensures all critical systems are reviewed and updated on a regular, frequent schedule (e.g., monthly).
    \end{itemize}
    
    \item \textbf{RISK-003 (High): Establish Annual Security Awareness Training}
    \begin{itemize}
        \item Develop and implement a mandatory security awareness training program for all employees, to be completed annually.
        \item The training should cover current threats, including phishing, ransomware, and social engineering, as well as reinforce organizational security policies.
    \end{itemize}
\end{enumerate}

\end{document}
```