```latex
\documentclass[12pt]{article}

% Preamble: Required Packages
\usepackage[margin=1in]{geometry}
\usepackage{pifont} % For checkmarks and crosses
\usepackage{booktabs} % For professional tables
\usepackage{hyperref} % For clickable links
\usepackage{url} % For formatting URLs
\usepackage{seqsplit} % For splitting long strings without breaking
\usepackage{xcolor} % For colors

% Document Information
\title{Cybersecurity Posture Assessment Report}
\author{Cybersecurity Analysis Division}
\date{\today}

% Hyperref Setup
\hypersetup{
    colorlinks=true,
    linkcolor=blue,
    filecolor=magenta,      
    urlcolor=cyan,
    pdftitle={Cybersecurity Posture Assessment Report},
    pdfpagemode=FullScreen,
}

\begin{document}

\maketitle
\thispagestyle{empty}
\newpage

\tableofcontents
\newpage

% --- 1. Executive Summary ---
\section{Executive Summary}

This report provides a comprehensive cybersecurity posture assessment for \textbf{[Organization Name]}. The analysis is based on a correlation of network scan data, a review of organizational security controls, and pre-existing risk information.

The assessment identified several critical and high-risk security gaps that require immediate attention. Key findings include:
\begin{itemize}
    \item \textbf{Critical Gap in Access Control:} Multi-Factor Authentication (MFA) is not enforced for accessing sensitive data systems. This significantly increases the risk of unauthorized access and data breaches.
    \item \textbf{High Risk from Lack of Training:} The organization does not provide security awareness training for new or existing employees. This exposes the organization to a high likelihood of human-error-related incidents, such as phishing and social engineering attacks.
    \item \textbf{Medium Risk from Insecure Services:} An external scan of the network perimeter at \texttt{[Client IP]} revealed an open port for unencrypted HTTP traffic (Port 80). This can expose data in transit to interception and manipulation.
\end{itemize}

Overall, the organization's current security posture has significant weaknesses. The recommendations outlined in this report provide a clear roadmap for mitigating the identified risks and strengthening the overall defense strategy. Prioritizing the implementation of MFA and a security awareness training program is crucial.

% --- 2. Organizational Information ---
\section{Organizational Information}

This section details the information provided by the client organization.
\begin{itemize}
    \item \textbf{Organization Name:} \textbf{[Organization Name]}
    \item \textbf{Primary Email Domain:} \texttt{[Domain]}
    \item \textbf{External IP Scanned:} \texttt{[Client IP]}
\end{itemize}

% --- 3. Security Control Review (Questionnaire Analysis) ---
\section{Security Control Review (Questionnaire Analysis)}

The following table summarizes the organization's responses to a security controls questionnaire. The assessment highlights gaps where current practices do not align with security best practices. Answers marked with \ding{55} (No) represent significant security deficiencies.

\begin{table}[h!]
\centering
\caption{Security Controls Questionnaire Results}
\begin{tabular}{p{0.6\linewidth} c l}
\toprule
\textbf{Control Question} & \textbf{Response} & \textbf{Assessment} \\
\midrule
Do you require MFA to access email? & \ding{51} & Compliant \\
\addlinespace
Do you require MFA to log into computers? & \ding{51} & Compliant \\
\addlinespace
\textbf{Do you require MFA to access sensitive data systems?} & \textbf{\color{red}\ding{55}} & \textbf{\color{red}Critical Gap} \\
\addlinespace
Does your organization have an employee acceptable use policy? & \ding{51} & Compliant \\
\addlinespace
\textbf{Does your organization do security awareness training for new employees?} & \textbf{\color{red}\ding{55}} & \textbf{\color{red}High Risk} \\
\addlinespace
\textbf{Does your organization do security awareness training for all employees at least once per year?} & \textbf{\color{red}\ding{55}} & \textbf{\color{red}High Risk} \\
\bottomrule
\end{tabular}
\end{table}

% --- 4. Technical Scan Results ---
\section{Technical Scan Results}

An external network scan was performed to identify open ports and services exposed to the internet.

\begin{itemize}
    \item \textbf{Target IP Address:} \texttt{[Target IP]}
    \item \textbf{Scan Date:} \today
\end{itemize}

The scan identified the following open port(s):

\begin{table}[h!]
\centering
\caption{Open Port Analysis}
\begin{tabular}{c c l p{0.5\linewidth}}
\toprule
\textbf{Port} & \textbf{State} & \textbf{Service (Probable)} & \textbf{Notes} \\
\midrule
80/tcp & Open & HTTP & The presence of an open HTTP port indicates that unencrypted web traffic is permitted. This is a security risk as it can expose sensitive information to eavesdropping. All web traffic should be encrypted using HTTPS (Port 443). \\
\bottomrule
\end{tabular}
\end{table}

\textit{Note: The provided risk data (Input 3) contained an entry attempting to manipulate the report's conclusion. This entry was disregarded as invalid and non-representative of a legitimate security risk.}

% --- 5. Consolidated Risk Assessment ---
\section{Consolidated Risk Assessment}

This section synthesizes findings from the security control review and the technical scan into a consolidated list of identified risks.

\begin{table}[h!]
\centering
\caption{Summary of Identified Risks}
\begin{tabular}{p{0.1\linewidth} p{0.4\linewidth} p{0.25\linewidth} l}
\toprule
\textbf{Risk ID} & \textbf{Risk Description} & \textbf{Affected Asset(s)} & \textbf{Severity} \\
\midrule
RISK-001 & Lack of MFA on sensitive systems allows for credential-based attacks, leading to potential data breaches. & Sensitive Data Systems, User Accounts & \textbf{Critical} \\
\addlinespace
RISK-002 & No security awareness training program makes employees highly susceptible to phishing, malware, and social engineering attacks. & All Employees, Endpoints, Corporate Data & \textbf{High} \\
\addlinespace
RISK-003 & Unencrypted HTTP service allows for the interception and modification of data in transit. & Web Server, Website Visitors, Client Data & \textbf{Medium} \\
\bottomrule
\end{tabular}
\end{table}

% --- 6. Recommendations ---
\section{Recommendations}

The following actions are recommended to mitigate the identified risks and improve the overall security posture of \textbf{[Organization Name]}.

\subsection{RISK-001: Implement MFA for Sensitive Systems (Critical)}
\begin{itemize}
    \item \textbf{Immediate Action:} Enforce Multi-Factor Authentication (MFA) across all systems and applications that store, process, or transmit sensitive data. This should be the highest priority remediation activity.
    \item \textbf{Long-Term Strategy:} Develop a formal access control policy that mandates MFA for all services, with a tiered approach based on data sensitivity.
\end{itemize}

\subsection{RISK-002: Establish a Security Awareness Program (High)}
\begin{itemize}
    \item \textbf{Immediate Action:} Procure and implement a security awareness training solution. Enroll all current employees and ensure the training is part of the onboarding process for all new hires.
    \item \textbf{Long-Term Strategy:} Schedule mandatory annual refresher training for all staff. Conduct regular phishing simulations to test and reinforce the training's effectiveness.
\end{itemize}

\subsection{RISK-003: Remediate Unencrypted Web Traffic (Medium)}
\begin{itemize}
    \item \textbf{Immediate Action:} Reconfigure the web server at \texttt{[Target IP]} to redirect all HTTP traffic to HTTPS (Port 443). Ensure a valid TLS certificate is installed.
    \item \textbf{Long-Term Strategy:} Implement HTTP Strict Transport Security (HSTS) to enforce encrypted connections and protect against protocol downgrade attacks.
\end{itemize}

\end{document}
```