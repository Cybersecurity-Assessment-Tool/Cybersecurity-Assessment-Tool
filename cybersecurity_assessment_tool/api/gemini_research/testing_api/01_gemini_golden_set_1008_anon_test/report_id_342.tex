```latex
\documentclass[12pt]{article}

% --- PACKAGES ---
\usepackage[margin=1in]{geometry}
\usepackage{pifont}         % For checkmarks and crosses (\ding)
\usepackage{booktabs}       % For professional-looking tables
\usepackage{hyperref}       % For clickable links and references
\usepackage{url}            % For formatting URLs
\usepackage{seqsplit}       % For splitting long strings without spaces
\usepackage{xcolor}         % For using colors

% --- HYPERREF SETUP ---
\hypersetup{
    colorlinks=true,
    linkcolor=blue,
    filecolor=magenta,
    urlcolor=cyan,
    pdftitle={Cybersecurity Posture Assessment Report},
    pdfauthor={Cybersecurity Analysis Division},
}

% --- CUSTOM COMMANDS ---
\newcommand{\yes}{\textcolor{green}{\ding{51}}} % Green checkmark
\newcommand{\no}{\textcolor{red}{\ding{55}}}   % Red cross

% --- DOCUMENT METADATA ---
\title{Cybersecurity Posture Assessment Report}
\author{Cybersecurity Analysis Division}
\date{\today}

% --- DOCUMENT START ---
\begin{document}

\maketitle

\section*{Executive Summary}
This report provides a cybersecurity posture assessment for \textbf{[Organization Name]}. The analysis is based on a review of organizational security controls (Input 2), a network vulnerability scan (Input 1), and pre-existing risk data (Input 3).

The assessment identified several critical and high-risk security gaps. The most significant findings are the complete absence of Multi-Factor Authentication (MFA) across email, workstations, and sensitive data systems. This represents a critical vulnerability that significantly increases the risk of unauthorized access via compromised credentials. Additionally, security awareness training is not provided to new employees, creating an immediate and ongoing vulnerability to social engineering attacks.

A technical scan of the target IP address \texttt{[Target IP]} did not identify any open ports. This result conflicts with a pre-existing risk report indicating an "Unencrypted Web Server" on an open Port 80. This discrepancy requires immediate investigation to validate the current state of the external perimeter. Actionable recommendations are provided to address these findings and improve the overall security posture.

\section*{Organizational Information}
The following details were used as the basis for this assessment. As per the provided data, placeholder values are used where specific information was not available.

\begin{tabular}{@{}ll}
\textbf{Organization Name:} & \textbf{[Organization Name]} \\
\textbf{Primary Domain:} & \texttt{[Domain]} \\
\textbf{External IP Scanned:} & \texttt{[Client IP]} \\
\end{tabular}

\section*{Security Control Review}
The following table summarizes the organization's responses to a security controls questionnaire. Items marked with \no{} represent significant gaps in the security framework and are primary contributors to the overall risk profile.

\begin{center}
\begin{tabular}{p{0.7\textwidth}c}
\toprule
\textbf{Control Question} & \textbf{Response} \\
\midrule
Do you require MFA to access email? & \no \\
Do you require MFA to log into computers? & \no \\
Do you require MFA to access sensitive data systems? & \no \\
Does your organization have an employee acceptable use policy? & \yes \\
Does your organization do security awareness training for new employees? & \no \\
Does your organization do security awareness training for all employees at least once per year? & \yes \\
\bottomrule
\end{tabular}
\end{center}

\subsection*{Analysis}
The lack of MFA for email, computer logins, and sensitive data access is a critical vulnerability. This control is a foundational defense against account takeover attacks. Furthermore, the absence of security awareness training during employee onboarding leaves the organization highly susceptible to phishing and other social engineering attacks from the moment a new employee joins.

\section*{Technical Scan Results}
A network scan was performed on the target IP address provided in the scan data. It is assumed this target corresponds to the organization's primary external IP, \texttt{[Client IP]}.

\begin{itemize}
    \item \textbf{Target IP:} \texttt{[Target IP]}
    \item \textbf{Scan Date:} Not provided in source data.
    \item \textbf{Host Status:} Up
\end{itemize}

\subsection*{Port Summary}
The scan did not identify any open ports on the target host. Port 80 (HTTP), which was listed in the pre-existing risks, was explicitly checked and found to be in a \textbf{closed} state. This indicates a potentially well-filtered network perimeter at the time of the scan.

\textbf{Important Note:} This finding directly conflicts with the pre-existing risk data (Input 3), which lists a vulnerability ("Unencrypted Web Server") related to an open Port 80. This discrepancy must be investigated to determine if the risk has been remediated or if the scan was performed on an incorrect target.

\section*{Risk Assessment}
The following table synthesizes findings from the security control review, technical scan, and pre-existing risk data into a prioritized list of risks.

\begin{center}
\begin{tabular}{p{0.25\textwidth}p{0.5\textwidth}l}
\toprule
\textbf{Risk Name} & \textbf{Description} & \textbf{Severity} \\
\midrule
Absence of Multi-Factor Authentication (MFA) & No MFA is enforced for email, computer logins, or access to sensitive data. This exposes the organization to account takeover via credential theft. & \textbf{Critical} \\
\addlinespace
Inadequate Onboarding Security Training & New employees do not receive security awareness training, making them immediate targets for phishing and social engineering from their first day. & \textbf{High} \\
\addlinespace
Unconfirmed Web Server Vulnerability & A previously identified risk indicates an unencrypted web server on Port 80. The current scan found this port closed, creating a discrepancy that requires verification. & Medium \\
\bottomrule
\end{tabular}
\end{center}

\section*{Recommendations}
Based on the analysis, the following actions are recommended to mitigate the identified risks and improve the security posture of \textbf{[Organization Name]}.

\begin{enumerate}
    \item \textbf{Implement Multi-Factor Authentication (Critical):} Immediately deploy a robust MFA solution across all critical systems. Prioritize the following:
    \begin{itemize}
        \item All email accounts (e.g., Microsoft 365, Google Workspace).
        \item All remote access solutions (VPNs, RDP).
        \item All systems containing sensitive organizational or client data.
        \item Workstation logins.
    \end{itemize}

    \item \textbf{Enhance Security Awareness Program (High):} The current training program must be expanded to cover critical gaps.
    \begin{itemize}
        \item Integrate mandatory security awareness training into the new employee onboarding process.
        \item Continue the annual refresher training for all staff to ensure ongoing vigilance.
    \end{itemize}

    \item \textbf{Verify and Remediate External Risks (Medium):} Investigate the discrepancy between the current scan results and the previously reported "Unencrypted Web Server" vulnerability.
    \begin{itemize}
        \item Conduct a comprehensive scan of the external IP address (\texttt{[Client IP]}) to confirm the status of all ports, especially Port 80.
        \item If Port 80 is found to be open, immediately implement TLS/SSL to encrypt traffic (HTTPS on Port 443) and close Port 80.
        \item If the port is confirmed closed, update the risk register to formally close the risk and document the remediation.
    \end{itemize}
\end{enumerate}

\end{document}
```