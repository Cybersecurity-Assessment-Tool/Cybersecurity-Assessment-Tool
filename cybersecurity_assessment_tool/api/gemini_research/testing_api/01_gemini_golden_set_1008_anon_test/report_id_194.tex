```latex
\documentclass[12pt]{article}

% Preamble: Required Packages
\usepackage[margin=1in]{geometry}
\usepackage{pifont}                 % For \ding{51} (checkmark) and \ding{55} (cross)
\usepackage{booktabs}               % For professional-looking tables (\toprule, \midrule, \bottomrule)
\usepackage{hyperref}               % For hyperlinks, metadata
\usepackage{url}                    % For typesetting URLs
\usepackage{seqsplit}               % To split long strings in \texttt
\usepackage{xcolor}                 % For colors
\usepackage{fancyhdr}               % For headers and footers
\usepackage{graphicx}               % To include images

% --- Document Setup ---
\hypersetup{
    colorlinks=true,
    linkcolor=blue,
    filecolor=magenta,      
    urlcolor=cyan,
    pdftitle={Cybersecurity Posture Assessment Report},
    pdfauthor={Cybersecurity Analysis Division},
    pdfsubject={Security Assessment},
    pdfkeywords={Cybersecurity, Risk, Assessment},
}

% --- Header and Footer ---
\pagestyle{fancy}
\fancyhf{} % Clear all header and footer fields
\fancyhead[L]{Cybersecurity Post-Assessment Report}
\fancyhead[R]{\textbf{[Organization Name]}}
\fancyfoot[C]{\thepage}
\renewcommand{\headrulewidth}{0.4pt}
\renewcommand{\footrulewidth}{0.4pt}

% --- Document Start ---
\begin{document}

% --- Title Page ---
\begin{titlepage}
    \centering
    \vspace*{1cm}
    \includegraphics[width=0.3\textwidth]{example-image-a} % Placeholder for a logo
    \vfill
    \Huge\bfseries
    Cybersecurity Posture Assessment Report
    \vspace{1cm}
    \Large
    For: \textbf{[Organization Name]}
    \vspace{1.5cm}
    \normalsize
    \begin{tabular}{ll}
        \textbf{Date of Report:} & \today \\
        \textbf{Date of Scan:} & 2023-10-27 \\ % Extracted from Input 1 metadata
        \textbf{Report ID:} & CYBER-2023-Q4-001 \\
    \end{tabular}
    \vfill
    \small
    \textit{This report contains sensitive information and is intended solely for the use of \textbf{[Organization Name]}. Distribution without prior written consent is prohibited.}
\end{titlepage}

\tableofcontents
\newpage

% --- 1. Executive Summary ---
\section{Executive Summary}
This report details the findings of a cybersecurity posture assessment conducted for \textbf{[Organization Name]}. The assessment combined a technical network scan, a review of existing risks, and an analysis of organizational security controls via a questionnaire.

The analysis revealed a \textbf{critical risk posture}. Key findings include a publicly accessible, outdated, and dangerously misconfigured FTP server that allows anonymous user access. This vulnerability (CVE-2011-2523) could allow an attacker to execute arbitrary code and gain complete control of the affected system.

Furthermore, significant gaps were identified in administrative and policy controls. The lack of multi-factor authentication (MFA) on sensitive data systems, the absence of an employee acceptable use policy, and infrequent security awareness training create substantial organizational risk. These procedural weaknesses, combined with the technical vulnerabilities, leave the organization highly susceptible to compromise.

Immediate remediation of the critical technical vulnerability is paramount, followed by the implementation of the administrative and policy recommendations outlined in this report.

% --- 2. Organizational Information ---
\section{Organizational Information}
The following information was used as the basis for this assessment. As per the template mode for anonymized data, placeholders are used where information was not provided.

\begin{tabular}{@{}ll}
    \toprule
    \textbf{Attribute} & \textbf{Value} \\
    \midrule
    Organization Name & \textbf{[Organization Name]} \\
    Primary Domain & \texttt{[Domain]} \\
    External IP Scanned & \texttt{[Client IP]} \\
    \bottomrule
\end{tabular}

% --- 3. Security Control Review (Questionnaire Analysis) ---
\section{Security Control Review}
The following table summarizes the organization's responses to the security controls questionnaire. Items marked with a red 'X' (\textcolor{red}{\ding{55}}) represent significant gaps in the security framework and are discussed in the Risk Assessment section.

\begin{table}[h!]
\centering
\caption{Security Controls Questionnaire Analysis}
\begin{tabular}{@{}p{0.6\linewidth} c l@{}}
    \toprule
    \textbf{Control Question} & \textbf{Response} & \textbf{Assessment} \\
    \midrule
    Do you require MFA to access email? & \textcolor{green}{\ding{51}} & Best practice met. \\
    Do you require MFA to log into computers? & \textcolor{green}{\ding{51}} & Best practice met. \\
    Do you require MFA to access sensitive data systems? & \textcolor{red}{\ding{55}} & \textbf{Critical Gap} \\
    Does your organization have an employee acceptable use policy? & \textcolor{red}{\ding{55}} & \textbf{High Risk Gap} \\
    Does your organization do security awareness training for new employees? & \textcolor{green}{\ding{51}} & Good baseline. \\
    Does your organization do security awareness training for all employees at least once per year? & \textcolor{red}{\ding{55}} & \textbf{High Risk Gap} \\
    \bottomrule
\end{tabular}
\end{table}

% --- 4. Technical Scan Results ---
\section{Technical Scan Results}
An external network scan was performed on the target IP address provided. The target was identified as \texttt{[Target IP]}. The scan revealed the following open ports and services.

\begin{table}[h!]
\centering
\caption{Open Ports and Services on \texttt{[Target IP]}}
\begin{tabular}{@{}lllll@{}}
    \toprule
    \textbf{Port} & \textbf{State} & \textbf{Service} & \textbf{Product \& Version} & \textbf{Notes} \\
    \midrule
    21/tcp & Open & ftp & vsftpd 2.3.4 & Anonymous FTP login allowed \\
    \bottomrule
\end{tabular}
\end{table}

\subsection{Critical Finding: Vulnerable FTP Server}
The scan identified a critical vulnerability on the FTP service (port 21).
\begin{itemize}
    \item \textbf{Vulnerable Software:} The server is running \texttt{vsftpd 2.3.4}. This version, released in 2011, is known to contain a critical backdoor vulnerability (\textbf{CVE-2011-2523}). An attacker can exploit this to gain a command shell on the underlying server.
    \item \textbf{Insecure Configuration:} The server is configured to allow \textbf{anonymous FTP login}. This allows any unauthenticated user on the internet to connect to the server, view, and potentially upload or download files. This configuration dramatically increases the attack surface and risk of data exposure or malware implantation.
\end{itemize}

% --- 5. Consolidated Risk Assessment ---
\section{Consolidated Risk Assessment}
The following table synthesizes findings from the technical scan, the security control review, and pre-existing risk data into a prioritized list.

\begin{table}[h!]
\centering
\caption{Summary of Identified Risks}
\begin{tabular}{@{}p{0.5\linewidth} p{0.2\linewidth} p{0.2\linewidth}@{}}
    \toprule
    \textbf{Risk Description} & \textbf{Source} & \textbf{Severity} \\
    \midrule
    \textbf{Vulnerable FTP Server with Anonymous Access}: An outdated, backdoor-vulnerable FTP server is publicly exposed and allows unauthenticated access. & Technical Scan & \textbf{Critical} \\
    \addlinespace
    \textbf{Lack of MFA on Sensitive Systems}: Critical data repositories are not protected by multi-factor authentication, relying solely on passwords. & Questionnaire & \textbf{Critical} \\
    \addlinespace
    \textbf{Missing Acceptable Use Policy (AUP)}: Lack of a formal AUP leads to inconsistent user behavior and no enforcement mechanism for security policies. & Questionnaire & High \\
    \addlinespace
    \textbf{Inadequate Security Awareness Training}: Failure to conduct annual training for all employees allows security knowledge to become stale, increasing susceptibility to phishing and social engineering. & Questionnaire & High \\
    \addlinespace
    \textbf{Outdated Windows Policy}: Workstations are running Windows 7, which is end-of-life and no longer receives security updates. & Pre-existing Data & Medium \\
    \bottomrule
\end{tabular}
\end{table}

% --- 6. Recommendations ---
\section{Recommendations}
Based on the identified risks, the following remediation actions are recommended, prioritized by severity.

\subsection{Priority 1: Critical Risks}
\begin{enumerate}
    \item \textbf{Remediate Vulnerable FTP Server (Immediate Action Required):}
    \begin{itemize}
        \item \textbf{Option A (Preferred):} If the FTP service is not essential, take the server offline immediately by disabling the service and blocking port 21 at the firewall.
        \item \textbf{Option B:} If the service is required, immediately upgrade \texttt{vsftpd} to the latest stable version. Crucially, disable anonymous login and configure strong user authentication.
    \end{itemize}
    \item \textbf{Implement MFA on Sensitive Systems:}
    \begin{itemize}
        \item Immediately begin a project to enforce MFA on all systems classified as containing sensitive data. This is a critical compensating control to protect against credential theft.
    \end{itemize}
\end{enumerate}

\subsection{Priority 2: High Risks}
\begin{enumerate}
    \setcounter{enumi}{2} % Continue numbering
    \item \textbf{Develop and Implement an Acceptable Use Policy (AUP):}
    \begin{itemize}
        \item Draft an AUP that clearly defines the rules for using company IT assets, data handling, and internet usage.
        \item Require all employees to read and formally acknowledge the policy as a condition of their employment.
    \enditemize}
    \item \textbf{Establish an Annual Security Awareness Training Program:}
    \begin{itemize}
        \item Procure or develop a security awareness training module and make it mandatory for all employees to complete annually.
        \item Track completion rates to ensure 100\% compliance.
    \end{itemize}
\end{enumerate}

\subsection{Priority 3: Medium Risks}
\begin{enumerate}
    \setcounter{enumi}{4} % Continue numbering
    \item \textbf{Accelerate Windows 7 Migration Plan:}
    \begin{itemize}
        \item Continue with the existing recommendation to upgrade all Windows 7 workstations to a modern, supported operating system like Windows 10 or 11 to ensure they receive critical security patches.
    \end{itemize}
\end{enumerate}

\end{document}
```