\documentclass[12pt]{article}
\usepackage[letterpaper, margin=1in]{geometry}
\usepackage{amsmath, amssymb}
\usepackage{pifont} % For the checkmark symbol \ding{51}
\usepackage{booktabs} % For professional-looking tables
\usepackage{hyperref} % For links
\usepackage{fontawesome5}
\usepackage{setspace} % For line spacing
\usepackage{url} % Included and applied to domains
\usepackage{seqsplit} % Applied to long code/IP strings

\hypersetup{
    colorlinks=true,
    urlcolor=blue,
    linkcolor=black
}

\title{\textbf{Cybersecurity Readiness Report for Valier School District}}
\author{Date: July 18, 2025}
\date{}

\begin{document}
\maketitle
\onehalfspacing

\section{Overview}
This report evaluates the cybersecurity posture of Valier School District based on technical scans (DNS, DMARC, and port scanning) and a self-reported security questionnaire. The findings reflect a strong commitment to foundational cybersecurity practices across user access, email protection, network exposure, and staff awareness.

\section{Organizational Information}
\begin{itemize}
    \item \textbf{Organization Name}: Valier School District
    \item \textbf{Email Domain}: valier.k12.mt.us
    \item \textbf{Website Domain}: \url{www.valier.k12.mt.us}
    \item \textbf{External IP (Firewall)}: 216.220.16.170
    \item \textbf{Website Hosting IPs}: \seqsplit{216.239.32.21, 216.239.34.21, 216.239.36.21, 216.239.38.21}
    \item \textbf{DNS Hosting}: Managed by University of Montana (umt.edu nameservers)
\end{itemize}

\section{Security Questionnaire Review}
\begin{table}[h!]
\centering
\caption{Security Control Status}
\label{tab:security_controls}
\begin{tabular}{l c}
\toprule
\textbf{Security Control} & \textbf{Status} \\
\midrule
MFA for Email & \ding{51} Yes \\
MFA for Computer Login & \ding{51} Yes \\
MFA for Sensitive Systems & \ding{51} Yes \\
Acceptable Use Policy & \ding{51} Yes \\
New Employee Security Awareness Training & \ding{51} Yes \\
Annual All-Employee Security Training & \ding{51} Yes \\
\bottomrule
\end{tabular}
\end{table}

\noindent \textbf{Summary}: The district reports complete implementation of basic cyber hygiene practices, especially user authentication (Multi-Factor Authentication) and routine training. This indicates a proactive and policy-driven approach to risk mitigation.

\section{DNS \& Email Security}

\subsection*{DNS Records}
\begin{itemize}
    \item DNS is managed by the University of Montana (\url{cudess1.umt.edu}, \url{cudess2.umt.edu}), suggesting centralized and professionally administered DNS.
    \item A records point to IPs within Google's network (likely Google Sites hosting for web content).
\end{itemize}

\subsection*{MX Records (Email)}
\begin{itemize}
    \item The district uses Google Workspace (Gmail) for email, as shown by multiple \seqsplit{\texttt{\url{aspmx.l.google.com}}} MX records.
    \item \textbf{SPF record} is correctly configured: \seqsplit{\texttt{v=spf1 include:\url{spf.google.com} include:\url{mg.infinitecampus.org -all}}}. This helps mitigate spoofing by defining authorized mail senders.
\end{itemize}

\subsection*{DMARC Record}
\begin{itemize}
    \item A valid DMARC record exists with a \textbf{reject policy}: \seqsplit{\texttt{v=DMARC1; p=reject; rua=\url{mailto:dmarc@valier.k12.mt.us}}}.
    \item This instructs receiving servers to reject unauthenticated mail, providing strong protection against phishing.
\end{itemize}

\noindent \textbf{Conclusion}: DNS and email protections (SPF, DMARC, and hosting security) are configured correctly and follow best practices.

\section{Port Scanning Results}

\subsection*{Website Hosting (Google IP: 216.239.32.21)}
\begin{itemize}
    \item \textbf{Port 80 (HTTP)}: Open
    \item \textbf{Port 443 (HTTPS)}: Open
\end{itemize}
These ports are expected for a publicly accessible website and are typical for Google-hosted services.

\subsection*{Firewall / External IP (216.220.16.170)}
\begin{itemize}
    \item All scanned ports are \textbf{closed}.
\end{itemize}
This is a strong sign of network perimeter hardening and good firewall configuration. No externally exposed services were found on the organization's primary IP.

\section{Risk Assessment \& Readiness Summary}
\begin{table}[h!]
\centering
\caption{Readiness Summary}
\label{tab:readiness_summary}
\begin{tabular}{l c l}
\toprule
\textbf{Category} & \textbf{Status} & \textbf{Notes} \\
\midrule
Authentication Security & Strong & MFA is required across key systems \\
Email Security & Strong & SPF and DMARC with "reject" policy in place \\
Network Exposure & Secure & No exposed services on the external firewall IP \\
Web Hosting & Secure & Google-hosted; limited attack surface \\
Policy \& Training & Comprehensive & Acceptable use policies and regular training in place \\
\bottomrule
\end{tabular}
\end{table}

\section{Recommendations}
Although the cybersecurity readiness is solid, continuous improvement is essential. We recommend the following:
\begin{enumerate}
    \item \textbf{Verify DKIM}: While SPF and DMARC are configured, ensure DKIM is also active for all sending domains.
    \item \textbf{Vulnerability Scanning}: Consider regular internal and external vulnerability assessments of network devices and servers.
    \item \textbf{Incident Response Plan}: Document and regularly test a cybersecurity incident response and disaster recovery plan.
    \item \textbf{Asset Inventory}: Maintain a regularly updated inventory of hardware/software assets and monitor for unauthorized changes.
    \item \textbf{Third-party Risk}: Evaluate vendors (e.g., Infinite Campus) for their security posture, especially since they're included in SPF.
\end{enumerate}

\section{Conclusion}
Valier School District demonstrates a strong cybersecurity foundation, particularly in authentication, email protection, staff training, and perimeter security. Continued vigilance and regular audits will help maintain and improve this strong security posture.

\vspace{1cm}
\noindent Prepared by: \\
Cybersecurity Assessment Team\\
Date: July 18, 2025

\end{document}